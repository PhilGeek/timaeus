%!TEX root = /Users/markelikalderon/Documents/Git/timaeus/timaeus.tex
\chapter{Cosmogony} % (fold)
\label{cha:cosmogony}

\section{The Cosmic Significance of Sense and Sensibilia} % (fold)
\label{sec:the_cosmic_significance_of_sensibilia}

Given that the present essay concerns sense and sensibilia in the \emph{Timaeus}, why devote an entire chapter on the generation of the body of the Cosmos? Of what interest is cosmogony to a philosopher of perception? 

It is hard not to suspect that the anxiety that prompts this question is distinctly modern. Modern cosmology, or those aspects of it that correspond to cosmogony such as the Big Bang theory, have nothing much to teach us about sense and sensibilia. But nothing much follows about ancient cosmogony in general, and Timaeus' cosmogony in particular. As we shall see, the pre-cosmic chaos is sensible, and Timaeus derives the elemental composition of the body of the Cosmos from the fact that it is visible and tangible. Modern thought about sense and sensibilia is animated by the worry that the Scientific Image of Nature is in conflict with the Manifest Image of Nature that experience affords us. Even those that feel that the conflict is not, in the end, genuine, or may at least be partially overcome, theorize about perception and its objects with an eye to this alleged conflict. One lesson of the present chapter is that, for Timaeus at least, there is no conflict between the Scientific and Manifest Images of Nature. This is all the more significant since, though the terminology may be Sellarsian, the roots of the conflict are Eleatic (if not in Parmenides, properly understood, then at least in a Sophistical take on the Parmenides, or in the work of Melissus). So it is not as if Timaeus' speech is delivered in naive ignorance of the potential for such conflict, an ignorance only overcome by scientific modernity. In the face of the Parmenidean explosion, Timaeus denies that the Manifest Image of Nature is upended by its intelligible underpinnings. 

The lack of conflict between the Scientific and Manifest Images of Nature concern the nature of sensibilia. Timaeus' cosmogony has important lessons for the nature of sense as well. Perception is essentially environmental. Perception takes as its object aspects of the environment that circumscribes the perceiver. Without an environment, there is no perception, and no need for sensory apparatus, the instruments of perception. This important lesson figures in Timaeus' argument that the body of the Cosmos is spherical. So not only is cosmogony relevant to the nature of sensibilia, but in this way it is relevant to the environmental nature of sense as well.

Timaeus' narrative is non-linear. He does not, as it were, begin at the beginning and proceed inexorably to the end. Thus, for example, he describes the generation of the body of the Cosmos before describing the generation of the World Soul, even though the World Soul is prior in dignity and birth. Moreover, Timaeus only describes the pre-cosmic chaos after his cosmogony and psychogony. This is no mere happenstance. Timaeus has good reason for the temporal disruption of his narrative. Nor does Timaeus disrupt the temporal order for dramatic reasons the way that Kubrick does in \emph{The Killing}, a device that Tarantino subsequently made much of. Timaeus' reasons are methodological. The first part of his speech concerns the works of Reason, the second concern the works of Necessity, and the third concern the interaction of the Reason and Necessity. The temporal order does not respect these divisions, and so Timaeus narrates them accordingly.

In his cosmogony, Timaeus does four things. Timaeus argues that:
\begin{enumerate}[(1)]
	\item the Cosmos is unique in order to better resemble the Paradigm on which it was modeled
	\item the Cosmos is composed of four primary bodies: fire, air, water, and earth
	\item the Cosmos is comprehensive, containing all the primary bodies that there are, and all their parts and powers, leaving nothing corporeal external to it
	\item the Cosmos is spherical in shape
\end{enumerate}
What does the uniqueness and comprehensiveness of the Cosmos have to do with philosophy of perception? While the second and fourth arguments are directly relevant to the nature of sense and sensibilia, the first and third are indirectly relevant. As we shall see, the uniqueness of the Cosmos is part of Timaeus' reason for its comprehensiveness, and the comprehensiveness of the Cosmos is what establishes that there is nothing corporeal external to it. So these two arguments play a role in establishing the Cosmos is an environment, and it is essential to human perception that the objects of perception be environmental.

As we shall see, there are important parallels between the Demiurge's generation of the body of the Cosmos and the Young Gods' generation of the human body. But more importantly, there will be differences. Mortal human beings are situated within the finite body of the Cosmos. The Cosmos itself, a visible God, is situated in no other thing. In this way it resembles the self-sufficiency of the Paradigm, the Living Being. Mortal human beings, by contrast, live within an environment, the visible God. Mortal beings are not themselves self-sufficient. Mortal beings depend upon the environment even as they must defend themselves against it. That mortal beings are not only embodied but emdedded in an environment will have consequences for their motion, their shape, their bodily parts, their affections, their powers of perception and opinion, their susceptibility to aging and sickness, and, importantly, their salvation.

% section the_cosmic_significance_of_sensibilia (end)

\section{The Uniqueness of the Cosmos} % (fold)
\label{sec:the_uniqueness_of_the_Cosmos}

Timaeus describes the Cosmos as a living being, one and visible, that contains within itself all other sensible living beings (30d), and later echoes this claim, if in more general terms (33a, the generalisation consists in replacing talk of a living being containing living beings with talk of a whole of wholes). The claim is deceptively straightforward. Its implications will be spelled out, and the claim itself elucidated, in the discussion of 31a--34a. Timaeus' claim has a number of distinguishable parts. In this section, we shall begin with the claim that the Cosmos is one.

Timaeus provides an argument for the claim that there is one Heaven (\emph{ouranos}, Timaeus uses \emph{ouranos} and \emph{kosmos} interchangeably, 28b2). \citet[84]{Taylor:1928qb} observes that, unlike the English word ``\underline{uni}verse'', it is no part of the meaning of \emph{ouranos} or \emph{kosmos} that there can be only one. If there is only one, this must be for some reason, and the argument presents that reason (or at least as much of a reason that may be apprehended by mortal intellect). The argument comes in two parts. In the first part, Timaeus argues that the Paradigm, the intelligible Living Being, is one. In the second part, Timaeus argues that the Demiurge uses the Paradigm to create only one sensible living being, Heaven or the Cosmos, so that it may better resemble the Paradigm.

Timaeus first establishes the uniqueness of the Paradigm. That argument depends upon a particular feature of the Paradigm. The Paradigm is an intelligible living being (\emph{noēton zōon}). Moreover, it contains within in itself all other intelligible living beings. So for every potential perceptible animal there is an intelligible type upon which it is modeled and the Living Being contains within itself this intelligible type. The Living Being is, in this sense, all-inclusive or comprehensive (\emph{pantelēs}). Timaeus exploits this feature to establish the uniqueness of the Living Being. If the Living Being really contains within itself all other intelligible living beings, it could not be one of a pair. Consider any pair of intelligible living beings. Any intelligible living being that is comprehensive in the sense of containing all other intelligible living beings would contain within in itself these two. For, if it did not, it would not be comprehensive. And it is the all-inclusive Living Being that the Demiurge uses as His model for the creation of the Cosmos.

While \citet[42--3]{Cornford:1935fk} highlights the comprehensiveness of the Paradigm, his presentation of the argument is curiously generic: ``There cannot be a second all-inclusive model; for then the two models would be duplicate instances of the same Form, and that Form would become the true model.'' Notice that the reasoning here does not exploit the fact that the intelligible Living Being contains within in itself all other intelligible living beings. The reasoning is generic and universal. It establishes that any Form must be unique, whether or not that Form is a Form of a comprehensive living being. Indeed, what Cornford offers us is merely the uniqueness proof of the Forms presented in the \emph{Res Publica} 10 597c1--d3 (for discussion see \citealt{Parry:1985tx}). But that proof makes no appeal to the comprehensiveness of the intelligible Living Being.

How does Timaeus move from the uniqueness of the Paradigm to the uniqueness of the Cosmos that is modeled on it? There is no question of this being a straightforward entailment. Familiarly, a model may have multiple copies. ``From one seal there derive many impressions'', Alcinous observes (\emph{Didaskalikos} 12 167 4--5, \citealt[20]{Dillon:2002aa}). Rather, Timaeus claims that the Demiurge makes one Heaven or Cosmos in order that it may better resemble the Paradigm in its uniqueness. Uniqueness is a perfection and the Demiurge, being benevolent and ungrudging, strives to make the best world possible. In so doing, He makes the Cosmos unique. But why is uniqueness a perfection? There are two resons that differ in kind.

First, uniqueness is in and of itself a perfection. The Paradigm is unique. It must be unique in order to fulfil its paradigmatic role (\emph{Res Publica} 597c1--d3). The Paradigm is good. The Cosmos is better the better it resembles the paradigm. So the Cosmos would be better if it were unique than if it were one of a plurality of cosmoi. So uniqueness is in and of itself a perfection. 

Second, as we shall see, a further reason emerges in subsequent discussion (section~\ref{sec:the_elemental_composition_of_the_corporeal}). If the Cosmos is unique, then there is nothing beyond it. If there is nothing beyond the Cosmos, then there are no strong powers external to it that may weaken or destroy it (34a). Think of how extreme heat or cold may age and destroy mortal beings. On the assumption that the cause of aging and sickness is invariably external, the Cosmos must be unique if it is to be everlasting. And the everlastingness of the Cosmos is an image of the eternity of its Paradigm. Everlastingness is in this way a perfection of the Cosmos, and uniqueness is required for its everlastingness. Uniqueness may be a perfection in and of itself, but it is also the means to the realization of other perfections.

As we shall see, the uniqueness of the Cosmos has important consequences. That the Cosmos is unique is part of Timaeus' reason for thinking that the corporeal composition of the Cosmos is comprehensive, in the sense that there are no primary bodies, nor parts, nor powers of them, that are outside the Cosmos. The Cosmos may be a sensible living being, but it is unlike the sensible living beings that it contains in that it is not situated in an environment that circumscribes it. And this explains why it has need of neither eyes to see with nor ears to hear with. The objects of perception are environmental, and the Cosmos is not situated in an environment that circumscribes it. Instead, the Cosmos circumscribes the bodies that compose it. The Cosmos is the environment in which they reside.

% The Cosmos is a perceptible living being modeled on the intelligible Living Being. 

% section the_uniqueness_of_the_Cosmos (end)

\section{The Elemental Composition of the Cosmos} % (fold)
\label{sec:the_elemental_composition_of_the_corporeal}

Empedocles took it for granted that the corporeal was composed of four ``roots'' (\emph{rhi\-zō\-ma\-ta}). Initially presented in divinized form (DK 31B6)---Zeus, Hera, Aidoneus (Hades), and Nestis---there is some controversy as to the assignment of roots to deities, apart from Nestis who is explicitly associated with water \citep[165--6]{Wright:1981zr}. \citet[165]{Wright:1981zr} endorses the Theophrastean interpretation on which Zeus is fire, Hera is air, Aidon\-eus is earth, and Nestis, of course, is water. Timaeus will deny that the Empedoclean ``roots'' are elements (\emph{stoicheia}) since they are themselves polyhedra that are further composed of elemental triangles (48b8). It is these triangles that are, in Timaeus' system, the \emph{stoicheia} or \emph{elementa}. Without loosing sight of this important Timaean claim, I will sometimes refer to the Empedoclean ``roots'' as elements. That fire, air, earth, and water are not, strictly speaking, elements is not the only way in which Timaeus departs from Empedocles. Timaeus does not simply take it for granted that there are four ``roots'' of all things. Rather, Timaeus derives the elemental composition of the corporeal from its sensible aspect.

The derivation of the elemental composition of the Cosmos is complex. It has a number of distinct steps each of which requires comment. However it is useful to begin with a schematic representation of that derivation before a detailed discussion of it:
\begin{enumerate}[(1)]
	\item That which comes to be is corporeal and so is visible and tangible (31b4--5)
	\item Since the Cosmos is visible, it contains fire (31b5--6)
	\item Since the Cosmos is tangible, it is solid, and since it is solid, it contains earth (31b6--7)
	\item Two things may not be joined without a third (31bd--31c)
	\item And so there must be an intermediary bond that joins them together (31c1)
	\item The fairest bond that the most perfectly unites both itself and that which it joins is proportion (31c2--4)
	\item Whenever of three things the middle is such that as the first is to the middle, so the middle is to the last, and conversely, as the last is to the middle, so the middle is to the first, then the middle becomes both first and last, and the last and first also appear in the middle (31c4--32a8)
	\item If the Cosmos were planar, then one middle term would suffice to bind the two (32a8--32b1)
	\item But the Cosmos is solid, and so two middle terms are required to bind the two (32b1--4)
	\item The Cosmos thus contains fire, air, water, and earth, where air is to water as fire is to air, and water is to earth as air is to water (32b4--9)
\end{enumerate}
The derivation consists in two parts or three, depending on how you look at it. The derivation has two parts in the sense that Timaeus first derives fire and earth (1--3), and then goes on to derive the two other ``roots'', air and water (4--10). The derivation of air and water in the second part is advanced on grounds distinct from those from which fire and earth were themselves derived. The Cosmos must contain fire and earth in order for it to be visible and tangible. But air and water are not derived as conditions on the possibility of forms of perception. Rather, they are derived from the need for proportionate intermediaries in the composition of the corporeal, which is by nature solid. The derivation consists in three parts in the sense that there is a methodological digression (4--7) on the nature of proportional bonds that prefaces the derivation of air and water that may be marked (8--10), thus splitting the second part of the previous scheme in two.

We shall discuss each of the steps of Timaeus' argument in turn, but let us begin with some general observations. First, the Cosmos that comes to be is corporeal. Timaeus argues that the corporeal is composed of the four Empedoclean ``roots''. Notice that this is a preliminary analysis of the composition of the corporeal. The ``roots'' turn out to be no roots at all but are merely regular polyhedra that are themselves composed of, and so rooted in, elemental triangles. Only the elemental triangles are properly regarded as \emph{stoicheia} or \emph{elementa}. Second, Timaeus' argument has universal scope. It is meant to apply to all bodies. It is not restricted, for example, to sublunary bodies as in Aristotle's cosmology in \emph{De caelo}. Finally, and importantly, Timaeus takes it for granted that bodies are sensible. Indeed, being sensible, as opposed to being extended, is the characteristic mark of the corporeal. That bodies are composed of the Empedoclean ``roots'' does not undermine their status as sensible. Rather, the assumption that bodies are sensible is a premise in the derivation of their being composed of these ``roots''. Nor does the further analysis of the four primary bodies into regular polyhedra composed of elemental triangles undermine their sensible character. Indeed, the behavior of these polyhedra explain how a sentient animate body is affected so as to give rise to perception and sensation (61d--68e). Timaean physics thus stands in sharp contrast with Cartesian physics.

(1) First, that which comes to be is corporeal and so is visible and tangible. Timaeus, here, takes for granted what he earlier argued for, that the Cosmos has come to be. Recall, in the \emph{proemium}, Timaeus argued that the Cosmos has come to be (28b7--c1):
\begin{enumerate}[(a)]
	\item The Cosmos is visible, tangible, and corporeal
	\item The Cosmos is sensible since only sensible things are visible, tangible, and corporeal
	\item Since the Cosmos is sensible, it is apprehended by perception and opinion
	\item What is apprehended by perception and opinion has come to be and never is
	\item Therefore, the Cosmos has come to be and never is
\end{enumerate}
The initial claim of Timaeus' derivation of the elemental composition of the Cosmos seems to reverse this line of reasoning. Specifically, Timaeus begins with the idea that the Cosmos has come to be and concludes first that it is corporeal and then that it is visible and tangible. 

% Notice that the argument from the \emph{proemium}, like the present argument, involves the claim that if the Cosmos is corporeal then it is visible and tangible.

Another thing to observe about the initial claim of the derivation is that the Cosmos, that which has come to be, is not merely claimed to be sensible but more specifically visible and tangible. Why begin with these two species, the visible and the tangible, rather than the genus, the sensible? And why these two species, rather than the tastable, smellable, or the audible?  Timaeus may have strategic and non-strategic reasons for this specification. The strategic reason is that from this specification, he may immediately conclude that the Cosmos is composed of fire and earth, thus beginning the derivation. These conclusions would not have been warranted merely from the claim that the Cosmos was sensible. Notice that while this answers are first question---Why begin with species rather than the genus?---it does not answer our second---Why begin with these two species as opposed to some other species? 

There may, however, be non-strategic reasons as well. While it may be plausible to claim that we can see and feel the Cosmos, it is perhaps less plausible to claim that we can taste it, say, even if the Cosmos contains parts that we can taste. Nor is the Cosmos as a whole smellable and audible, even if it contains parts that are smellable and audible. The Cosmos as a whole may be visible and tangible, but the Cosmos as a whole is not tastable, smellable, or audible (Proclus, \emph{In Timaeum} 2 25--6, \citealt{Diehl:1903re}). If that is right, then the visible and the tangible are the only two species of the sensible that qualify the Cosmos as a whole. While the strategic reason addresses our first question, the non-strategic reason addresses our second. 

\citet[93]{Taylor:1928qb} offers another reason. By means of sight and touch we may discern the shape and size of bodies. Taylor claims, by contrast, that we cannot taste, smell, or hear the shape and size of bodies. (This last may be questioned, however. Can we not hear the relative size of a room by its characteristic resonance? Think of the sound of a train as it enters a tunnel.) Not all bodies have a taste, smell, or sound, but all bodies have shape and size. So Taylor's idea is that Timaeus begins with sight and touch since it is by means of these senses alone that universal sensible features of the corporeal may be perceived. Perhaps Taylor rightly understands Timaeus here, but the absence of direct textual evidence, and the way Taylor's idea approximates a primary quality conception of the corporeal makes me apprehensive of the potential for anachronism here.

There is another aspect of this specification that is worth mentioning. Proclus suggests that the visible and the tangible are the extreme terms of the sensible (\emph{In Timaeum} 2 6--7, \citealt{Diehl:1903re}, see also Calcidius' translation of 31c as well as his \emph{In Timaeum} 21). So understood, the visible and the tangible are opposed with the other sensibles arrayed as intermediaries. This coheres well with the order in which Timaeus discusses the special sensibles when discussing the affections of the body that are liable to give rise to perception and sensation (61d--68e). Timaeus begins with the tangible and ends with the visible, with the tastable, the smellable, and the audible arrayed as intermediaries. Though Proclus does not mention it, it also coheres well with the natural distribution of the elements in the Cosmos. Thanks to the winnowing motion of the Receptacle, like tends to like, with the result that while fire naturally tends toward the circumference of the Cosmos, earth naturally tends toward the center. As a result, the elements end up being arrayed in terms of their sensible aspects. And according to Timeaus, circumference and center are naturally opposed (62c8–d4). And so the fire that naturally tends toward the circumference is opposed to the earth that naturally tends toward the center. On the Proclean interpretation, the specification involves no loss of generality. One may speak of the sensible in general in terms of its opposed extremes. Thus one may speak of temperature in general by speaking of the hot and the cold since the hot and the cold are the opposed extremes of temperature and all other temperatures are arrayed between them as proportionate intermediaries.

Proclus' professed reason for this claim is non-Timaean, however. Indeed, it is Peripatetic. Proclus claims that the visible is opposed to the tangible since while the former requires a medium, the latter does not (\emph{In Timaeum} 2 9--11, \citealt{Diehl:1903re}). While Timaeus will claim that the visual body (\emph{opsis}) composed of the fire emanating from within the eye and the daylight that it meets without mediates the action of chromatic fire (45b--46c, 67c--68d), the thought that vision requires a medium is really only made fully explicit by Aristotle, even it may be found, in embryonic form, in Timaeus' speech. I have characterized Proclus' reason for the opposition between the visible and the tangible as Peripatetic. One might reasonably object that, for Aristotle, the flesh is the medium of touch, and so that the tangible involves a medium just as much as the visible does. The observation upon which this objection rests is correct, but there is an important shift in vocabulary, if not doctrine, between books two and three of \emph{De anima}. While in \emph{De anima} 2.11 Aristotle emphasizes that the flesh is the medium by which we feel the tangible, in \emph{De anima} 3.1, he seems to claim what Proclus' is claiming, that while vision requires a medium, touch does not. The claims of book three may be reconciled with the claims of book two, if we understand the denial that the tangible involves a medium as the denial that it involves an external medium, the flesh being, of course, internal to the perceiver (\citealt[chapter 2.1.3]{Kalderon:2015fr}). 

Just because an argument is bad does not mean that its conclusion is false. Similarly, just because Proclus' reason for claiming that the visible and the tangible are the opposed extremes of the sensible is not Timaeus', does not mean that Timaeus does not accept that the visible and the tangible are the opposed extremes of the sensible. Timaeus, after all, might maintain this claim for some other reason. At the very least, I think we should be open to the Proclean suggestion here. As we shall see, the Proclean suggestion can help us understand the general claim about bonds that two things cannot be joined without a third.

The initial claim makes an interesting contrast with Descartes. While Timaeus agrees with Descartes that the corporeal is extended, the corporeal is not identified with extension. Indeed, the mark of the corporeal is not that it is extended but that it is sensible. That the corporeal is sensible figures not only in the derivation of the elemental composition of the Cosmos, but it also figures in the argument, from the \emph{proemium}, that the Cosmos has come to be. (Though this contrast is lessoned somewhat if Taylor is right and Timaeus cites the visible and the tangible since only these senses reveal shape and size common to all bodies. But, again, I am apprehensive of anachronism here.)

(2) Second, from the fact that the Cosmos is visible, Timaeus concludes that it must contain fire. As we shall see, vision arises from the interaction of three fires of two kinds (chapter~\ref{sec:the_end_of_sight}). One kind of fire is a mild light that does not burn (58c5–d1). Daylight is this kind of fire (45b4–6), as is, presumably, the fire that emanates from within the eye as well as the compound that results from these. Timaeus conceives of colors, the proper objects of vision, as a power of bodies to emit a different kind of fire, a flame (\emph{phlox}, 67c–68d). It is because vision is the result of the interaction of these three fires of two kinds, that Timaeus concludes that the Cosmos contains fire from its visibility. Importantly, Timaeus is not restricting the visible to the fiery. We can see things other than fire, such as earth, air, and water. Rather, Timaeus is claiming that if anything at all is visible, then there must be fire that illuminates the Cosmos (Proclus, \emph{In Timaeum} 2 7.32--9.8, \citealt{Diehl:1903re}). And if there is, then the Cosmos contains fire. Fire is a condition on the very possibility of a body's visibility.

Against this Aristotle will object that what visibility requires is light, and light is not fire. While fire is hot and dry, light is not (\emph{De sensu} 2 437b16--19). It is hard to determine to what extent this is merely a verbal dispute. According to Timaeus, daylight is a mild light that does not burn. Aristotle does not dispute this so much as he disputes Timaeus taking this to be a kind of fire. But even so, the mild light of day that does not burn is, according to Aristotle, contingent upon the presence and activity of a fiery substance. Part of what is at issue is Aristotle's insistence that light is a qualitative state (\emph{hexis}) of a transparent medium actualized by the presence and activity of the fiery substance. Fire, by contrast, is not a state but a body, if a rarefied one.

(3) Third, from the fact that the Cosmos is tangible, Timaeus first concludes that it is solid, and then concludes that it must be composed of earth. While the derivation of fire from the visibility of the Cosmos is immediate, the derivation of earth from the tangibility of the Cosmos is not. What immediately follows from the tangibility of the Cosmos is that it is solid. And from its solidity it is further inferred that it must be composed of earth. Let us consider these two inferences in turn. 

Why does solidity of a body follow from its tangibility? Why is solidity a condition on the very possibility of a body's tangibility? And what, exactly, does Timaeus mean by \emph{stereou} here? Does he mean a geometrical solid or does he mean something hard and resistant to touch? While the nature of geometrical solids will play a role in the derivation, a question arises about its connection with tangibility. If there may be geometrical solids that are incorporeal, these will be intangible, and solidity, so understood, would not be sufficient for tangibility. Perhaps, but it may yet remain a condition on the possibility of touch without being a sufficient condition. Or, perhaps, \emph{stereou} means hard or resistant to touch. So understood, \emph{stereou} would be analogous in sense to the Latin \emph{robustus}. While plausible as a condition on the possibility of touch, this needs to be understood in a manner that is consistent with what Timaeus has to say about the hard and the soft. The soft is that which yields to touch whereas the hard is that which is resistant to touch (62b6–c3). So the requirement on touch is not that the tangible is resistant to touch in the way hard bodies are, for then soft bodies would be intangible. But softness is a tangible quality. How, then, is \emph{stereou} to be understood? Perhaps the thought is that only that which is solid may withstand touch, understood as a mode of perception, since if the object of touch does not withstand being touched it is not so much as touched as it is crushed, say. As crushing a body destroys at least some of its tangible qualities, if the aim in touching it were to discern these, crushing would fail to meet this perceptual end (for further discussion see \citealt[59--60]{Kalderon:2018oe}). A surface may be soft and pliant and yield to the percipient's touch all the while withstanding the touch to which it yields. It is not that the tangible is resistant to touch the way that hard bodies are but rather that tangible bodies should withstand being touched. Different ways of withstanding being touched would correspond to different tangible characters.

Why does being composed of earth follow from solidity? Later on, Timaeus will argue that individual bodies of earth are cubic and thus have solid bases (59d4--7). It is the solidity of their bases that determines the solidity of aggregates of earth particles. However, according to Timaeus, this is a sufficient and not a necessary condition on a body's solidity. Bodies may be solid by being composed, in part, of earth, but they may also be solid by by being densely compacted (59b5). Adamant is hard, not because it contains earth, but because of its density (59b5). And when the fluid form of water freezes, it becomes solid, not because it has been tainted with earth, but because escaping fire particles have caused the octahedra to fill in the space they have vacated resulting in a densely compacted and so solid body (59d4–e5). If being composed of earth is, by Timaeus' lights, a sufficient but not necessary condition for a body to be solid, then how is this second inference valid? Notice that the problem raised here goes beyond the terms of Timaeus' preliminary analysis. The account of solidity in terms of density is only really available once the Empedoclean ``roots'' have been identified with regular polyhedra. Perhaps, then, the derivation is merely probable, adequate only to the present stage of inquiry, and thus, in this sense, an \emph{eikōs logos}.

There is, however, an additional problem. While we may grant to Timaeus that it is at the very least odd to describe ourselves as touching air or fire, cannot we not, in a straightforward sense, touch water? If punting, for example, may we not idly touch the passing water and feel its coolness and the force with which it streams past? And if we can, then water is tangible. There is no reasons to suppose that this is only so because the water is impure being tainted with earth (though Londoners' afflicted with limescale may suspect otherwise). And if water untainted with earth is tangible, even in its liquid state, then not only is being composed of earth not necessary, neither is being solid, since water in its liquid state is not solid. Indeed, it is solid in neither the geometrical sense nor in the sense of being resistant to touch in terms of which hardness is defined. Does water, like soft things, genuinely withstand our touching it? Theophrastus claims that soft things yield in the direction in which they are touched and form a depression rather than being scattered so as to flow around the flesh of the body (\emph{De sensibus} 87). In making this claim, Theophrastus is plausibly following Aristotle (\emph{Meterologia} 3.1 3299b11). With the Theophrastian distinction in mind, perhaps one could say that soft things withstand touch in the way that water does not. But Timaeus claims that water is soft in relation to earth (54d4–7). I speculatively suggested that \emph{stereou} might be understood as withstanding touch. Should that reading turn out to be unsustainable, then there is a real problem here. Water would not be solid in any of the available senses. If so, this threatens the validity of the first inference as well, since solidity would not be a necessary condition on tangibility. 

(4) Having derived fire and earth from the visibility and tangibility of the Cosmos, Timaeus goes on to derive the remaining two Empedoclean ``roots'', air and water. Timae\-us does not do so, however, on the basis of their sensible aspects. The presence of air and water in the composition of the body of the Cosmos is not derived on the basis of their being conditions on the possibility of forms of perception. Rather, they are derived from the way in which fire and earth must be proportionally bound together in the composition of the body of the Cosmos. Timaeus will first elucidate the nature of this bond before proceeding to the derivation of air and water.

The first claim that Timaeus makes in his digression on the nature of bonds (\emph{desmō}) is that no two things may be joined together without a third. First, notice that Timaeus has yet to say anything explicitly about bonds. The connection with bonds is only made explicit in the second claim. Second, an observation should be made about the intended generality of the first claim. Timaeus is not claiming that only a third is required. Rather, Timaeus' claim is that at least a third is required. No two things may be joined together without a third or more. After all, Timaeus goes on to argue that air and water, both, are required to join together fire and earth in a solid body (in the geometrical sense of solidity). Third, there is a puzzle about this claim that suggests that a suppressed condition may be in play. If that is right, then, as we shall see, the Proclean interpretation that fire and earth, at least in their sensible aspects, are opposed extremes, may do some explanatory work.

Let us consider the puzzle first. The visual body (\emph{opsis}) by which we see is composite (45b2--d3). It consists, on the one hand, of the fire emanating from within the eye and, on the other hand, the daylight that it encounters without. These two fires are akin. Each is a mild light that does not burn. Since they are akin, they grow together or coalesce in the direction that the perceiver is looking. What explains this is the principle, like's affinity for like. The fire emanating from within the eye and the daylight that it encounters without compose the visual body without the assistance of any third thing since they are alike. So it seems that this initial principle of binding is false, at least by Timaeus' lights.

In the present instance what is being bound is fire and earth in the composition of the body of the Cosmos. The fire emanating from within the eye and the daylight that it encounters without may be alike in being the same kind of fire, but fire and earth are unlike. They are not the same kind of body. Not only are they unlike, they are naturally located in the opposed extremes of the Cosmos. Timaeus holds that the Cosmos is spherical and that the center of the Cosmos is opposed to the points that lie, in extremity, on its circumference (62c8–d4). Fire naturally accumulates in the circumference of the Cosmos, while earth naturally accumulates in the center. Owing to the winnowing movement of the Receptacle, the Empedoclean ``roots'' are naturally distributed, like by like, with fire and earth being distributed at the extreme ends of the Cosmos. Perhaps this is so since fire and earth are opposed extremes at least with respect to their sensible aspects. Notice that this thought requires the Proclean suggestion that the visible and the tangible are themselves the opposed extremes of the sensible. If we accept all this, then our puzzle may be resolved by claiming that Timaeus' initial claim about bonds is tacitly restricted to opposed extremes: No two things that are opposed extremes may be joined together without a third (Proclus, \emph{In Timaeum} 2 17.21--18.7, \citealt{Diehl:1903re}). If fire and earth are opposed extremes, at least in their sensible aspects, then the principle of like's affinity for like would not suffice for their compounding, and it would make sense that a third thing is somehow needed to compound them. But more than that, a third thing is required in order to overcome the opposition of the extremes and so mediate between them.

(5) The second claim about bonds identifies the third thing as the intermediary bond (\emph{desmos}) that joins the two opposed extremes. So a bond is a third (or more) thing distinct from the two opposed extremes whose function is to mediate between opposed extremes so as to join them. While informative about the nature of bonds, as Timaeus conceives of them, these first two claims, taken together, still leave open important questions. Most importantly, one wants to know what it is to mediate between opposed extremes so as to join these together in a unified compound? 

(6) Timaeus' third claim about bonds is meant to answer that question. The fairest bond that most perfectly unites that which it joins is proportion (\emph{analogia}). Putting all three claims together we get the following. An intermediary third (or more) thing between the two opposed extremes may mediate between them so as to join them given the proportions that obtain between the opposed extremes and their intermediaries. As we shall see, in the present instance, what is required are two intermediaries, air and water, that stand in relevant proportions to themselves and to fire and earth so as to perfectly unite these bodies in the fairest bond.

Timaeus' third claim requires some unpacking before it provides a full answer to the question of how a bond mediates between opposed extremes so as to join them in a unity. Specifically, what exactly Timaeus means by \emph{analogia}? Nichomachus of Gerasa defines proportion (\emph{analogia}) as follows:
\begin{quote}
	A proportion (\emph{analogia}), then, is in the proper sense, the combination of two or more ratios (\emph{logos}), but by the more general definition the combination of two or more relations (\emph{schesis}), even if they are not brought under the same ratio, but rather a difference, or something else. (Nichomachus of Gerasa, \emph{Arithmetike eisagoge} 2.24.2; \citealt[264--5]{Dooge:1926aa})
\end{quote}
By ratios, Nichomachus seems to have in mind ratios of \emph{arithmoi}, nevertheless he is prepared to generalize the notion. There are two dimensions of generalization. First, proportions may not just be geometrical, but may be arithmetical and harmonic as well \citep[264 n2]{Dooge:1926aa}. Second, the generalized definition is not restricted to proportions among \emph{arithmoi}. As we shall see, Timaeus, has in mind primarily progressions of geometrical proportions. But he must allow non-\emph{arithmoi} to stand in geometrical proportions. The task, after all, is to establish that there are four primary bodies that compose the body of the Cosmos. It is these bodies, in their sensible aspects, that are said to stand in geometrical proportions.

(7) Timaeus offers a reason for this third claim about bonds that marks also a transition to the next step of the derivation. Timaeus' claim is convoluted, but the basic idea is simple enough: Whenever of three things the middle is such that as the first is to the middle, so the middle is to the last, and conversely, as the last is to the middle, so the middle is to the first, then the middle becomes both first and last, and the last and first also appear in the middle. (``Whenever of three things the middle is such that \ldots'' I am being deliberately non-specific here, for, as we shall see, there is controversy attaching to Timaeus' specific wording.) There are three distinguishable components to Timaeus' claim:
\begin{enumerate}[(a)]
	\item Whenever of three things the middle is such that as the first is to the middle, so the middle is to the last \ldots\ :
	\begin{quote}
		F:M :: M:L (for example, 2:4 :: 4:8)
	\end{quote}
	\item \ldots\ and conversely, as the last is to the middle, so the middle is to the first \dots\ :
	\begin{quote}
		L:M :: M:F (for example, 8:4 :: 4:2)
	\end{quote}
	\item \ldots\ then the middle becomes both first and last, and the last and first also appear in the middle:
	\begin{quote}
		M:L :: F:M (for example, 4:8 :: 2:4)
	\end{quote}
\end{enumerate}
So understood, Timaeus is describing progressions arrayed in geometrical proportions. (Proclus, due in part to lexical ambiguities that we shall come to, mistakenly takes this passage to describe, not only geometric proportions, but arithmetical and harmonic proportions as well, \emph{In Timaeum} 2, \citealt{Diehl:1903re}.) The fairness of the bond is undoubtedly due to the symmetry of the proportions. And it is these like proportions that perfectly unite even opposed extremes.  

Unfortunately, the Greek is beset by syntactic and lexical ambiguities that divide interpreters \citep{Prtichard:1990aa}. Let us consider the syntactic ambiguity first. It besets the following opening phrase:
\begin{quote}
{\sbl ὁπόταν γὰρ ἀριθμῶν τριῶν εἴτε ὄγκων εἴτε δυνάμεων ὡντινωνοῦν ᾖ τὸ μέσον \ldots}
\end{quote}
There are three potential readings:
\begin{enumerate}[(a)]
	\item Whenever any three numbers (\emph{arithmōn}), whether \emph{ongkōn} or \emph{dunameōn}, the middle (\emph{to meson}) is such that \ldots
	\item Whenever of any three numbers (\emph{arithmōn}), the middle (\emph{to meson}) between any two that are \emph{ongkōn} or \emph{dunameōn} is such that \ldots
	\item Whenever of three numbers  (\emph{arithmōn}) or \emph{ongkōn} or \emph{dunameōn}, the middle (\emph{to meson}) is such that \ldots
\end{enumerate}
The first reading links the genitives \emph{ongkōn} and \emph{dunameōn} with \emph{arithmōn triōn}. The second reading links the genitives \emph{ongkōn} and \emph{dunameōn} instead with \emph{to meson}. Finally, the third reading takes there to be an implicit \emph{eite} before \emph{arithmōn} and so as an alternative on a par with \emph{ongkōn} and \emph{dunameōn} (\citealt[99]{Taylor:1928qb} provides Platonic parallels for this usage in the \emph{Sophist} 217e1, 224e2). Whereas the first two readings restrict the alternatives to \emph{arithmoi}, the third reading allows the proportionate progression to obtain among \emph{arithmoi} and non-\emph{arithmoi} alike. \citet[97 n12]{Archer-Hind:1888qd} and \citet[59]{Bury:1929jb}, endorse the first reading, \citet[44]{Cornford:1935fk} endorses the second, and \citet[96--99]{Taylor:1928qb} endorses the third, citing Proclus, \emph{In Timaeum}, as a source (see also his translation, \citeyear[28]{Taylor:1929ov}). The third reading is also endorsed by Calcidius (\emph{In Timaeum} 21). \citet{Prtichard:1990aa} argues convincingly for the third reading, and he has been followed by \citet{Zeyl:2000cs} and by Johansen in his revision of Lee's translation \citeyearpar{Lee:2008ca}, who originally adopted the first reading, \citeyearpar{Lee:1965fh}.

The grammar of the opening phrase genuinely allows all three readings. Fortunately, the lexical ambiguities interact with these such that resolving the lexical ambiguities may resolve the syntactic ambiguity. The lexical ambiguities concern \emph{ongkōn} and \emph{dunameōn}. \emph{Dunameōn} is the genitive form of \emph{dunamis} which can mean, and is used by Timaeus to mean, power in a sense that includes sensible qualities. But \emph{dunamis}, in mathematical contexts, may also mean square root or possibly square number (for this latter reading see \citealt[97 n12]{Archer-Hind:1888qd} and \citealt[294 n1]{Heath:1921ys}; \citealt[184]{Prtichard:1990aa} plausibly argues that Heath may have been responsible for the following Liddel \& Scott entry for \emph{dunamis} ``V.I.b square number Pl \emph{Ti} 32a.''). \emph{Ongkōn} is the genitive of \emph{ongkos} and means volume, heap, or mass. But in a mathematical context, paired with \emph{dunamis}, it might mean geometrical solid or even cube, though this is not standardized mathematical vocabulary. As \citet{Prtichard:1990aa} observes, it all comes down to how \emph{dunamis} is read, since there is no reason to give \emph{ongkos} a mathematical reading if \emph{dunamis}, in this passage, does not receive one. 

If \emph{dunamis} has a mathematical reading in this passage, it could not mean the more usual square root, but must mean square number to contrast with \emph{ongkos} read as cube (\citealt[97 n12]{Archer-Hind:1888qd}, \citealt[45--52]{Cornford:1935fk}). However, \citet{Prtichard:1990aa} argues that this cannot be sustained, leaving only the non-mathematical reading of \emph{dunamis}. Aristotle, at least seems happy to describe powers of bodies as standing in ratios, \emph{Physica} 3 204b14--19. And Alcmaeon, a predecessor and in many ways an influence upon Timaeus, does so as well, at least according to Aëtius (\emph{Placita} 5.30.1 = ). Moreover, it is plausibly the opposed powers of fire and earth that cause them to be naturally allocated in the extreme portions of the Cosmos by the winnowing motion of the Receptacle. And since only the third reading does not restrict the members of the progression to \emph{arithmoi}, this is most likely Timaeus' intended reading. Thus resolving the lexical and syntactic ambiguities, we get:
\begin{quote}
	Whenever of three numbers, or volumes, or powers, the middle is such that \dots
\end{quote}

(8--9) Timaeus addresses one final complication before proceeding to derive the remaining Empedoclean ``roots''. The need for proportional intermediaries has been established, but how many, exactly, are required? According to Timaeus, it depends upon what you are compounding. If the Demiurge were compounding a planar figure, only one intermediary would be required. The Cosmos, however, is not a planar figure but is solid. Indeed, the Demiurge makes it solid so as to more perfectly realize its resemblance to the Paradigm, the Living Being. Solids, however, require not one intermediary but two. 

(10) Thus the Demiurge sets air and water between fire and earth and made them as proportionate to one another as was possible to bind them in a three-dimensional whole that constitutes the body of the Cosmos. Specifically, what fire is to air, air is to water, and what air is to water, water is to earth. Explicitly applying Timaeus' characterization of geometrical proportion to the primary bodies we get:
\begin{enumerate}[(a)]
	\item Whenever of numbers, volumes or powers, the middle is such that as the first is to the middle, so the middle is to the last \ldots\ :
	\begin{quote}
		Fire:Air :: Air:Water :: Water:Earth
	\end{quote}
	\item \ldots\ and conversely, as the last is to the middle, so the middle is to the first \dots\ :
	\begin{quote}
		Earth:Water :: Water:Air :: Air:Fire
	\end{quote}
	\item \ldots\ then the middle becomes both first and last, and the last and first also appear in the middle (as there are two middle terms in the present case there are two ways this conclusion is met):
	\begin{quote}
		Water:Earth :: Air:Water :: Fire:Air\\
		Air:Fire :: Water:Air :: Earth:Water
	\end{quote}
\end{enumerate}
Fire, air, water, and earth are bodies. As bodies they encompass volume and have sensible powers. Indeed, these powers precede the formation of the primary bodies---the pre-cosmic chaos involved traces of these powers, with no proportionate organization (for this is a gift of the Demiurge), moving in a disorderly manner. The Demiurge imposes form and number on these furtive powers, assigning them to regular polyhedra that are proportionately organized, thus insuring sufficient intelligibility for them to receive names. Though volume and number are in play in the generation of the primary bodies, strictly speaking, the primary bodies are arrayed by their sensible powers in these geometrical proportions. 

The opposition of fire and earth is thus overcome in the amity of like proportion. (Amity, \emph{philia}, is most likely an allusion to Love in Empedocles's cosmology, though Strife finds no counterpart in Timaeus' speech, see \citealt[99--100]{Taylor:1928qb}, \citealt[44 n4]{Cornford:1935fk}, and \citealt[230 n119]{Broadie:2012vl}, for Plato's own explanation for doing without Strife see \emph{Gorgias} 508a). Having being unified in this manner, Timaeus remarks that this unity cannot be undone by anyone but the one that bound them together. Thus only the Demiurge has the power to rend asunder what He has bound by like proportion. The unity is perfect in the sense that it is as much unity as a generated thing may aspire to. This claim will be echoed in the assurances that the Demiurge gives the Young Gods (41aff), in his discussion of the shock of embodiment (43d), and in his enigmatic remarks about the unknowability by mortal intellect of the proportions involved in color mixture (68b6–8, 68c7–d7). The Young Gods come into being by bonds fitted by the Demiurge. The Demiurge assures the Young Gods that while any bond may be dissolved, only evil would dissolve the bonds of what is well-fitted and good. Implicit in the Demiurge's assurance is that only He has the power to dissolve these bonds and that He is benevolent and ungrudging in a sense inconsistent with dissolving the bonds of what is well-fitted and good. Thus though the Young Gods are dissoluble by Demiurgic power, the benevolent and ungrudging nature of the Demiurge insures their continued unity and being. The chromatic opposed extremes are themselves bound together by intermediary colors that stand to them and to one another in divine proportions. And the amity of like proportions overcomes the opposition of extremes to unify the extremes and the intermediary colors into an intelligible ordering. Just as four primary bodies, when combined in various proportions, suffice to determine the whole range of sensible bodies, four colors, when combined in various proportions, suffice to determine the whole range of color. But the proportions involved in color mixture are unknowable by mortal intellect, since in order to ascertain these one would have to have the power to blend the many into one and dissolve the one into many, and no mortal has this divine power. Again, only the Demiurge has the power to rend asunder what He has bound in amity by like proportion.

What are we to make of this derivation? Timaeus describes his speech as providing a likely account (\emph{eikōs logos}, 30b7) or likely story (\emph{eikōs muthos}, 29d2). If a distinction is to be marked between these expressions, then the derivation is a \emph{eikōs logos} rather than a \emph{eikōs muthos}. Contrast Timaeus description of the generation of the World Soul, which is narrated in terms with no pretense of being a derivation or an argument of any kind. On the supposition that a distinction may be marked, then that narrative is an \emph{eikōs muthos} rather than an \emph{eikōs logos}. The derivation of the four primary bodies, by contrast, presents itself as an account that appeals to the mathematical principle that governs geometrical proportionality. But the account could only ever be likely, and so admit of no conclusive demonstration. It is not a scientific account in the sense of something that can be demonstrated and known. Strictly speaking, then, it is less a derivation than a pseudo-derivation. This prompts \citet{Prtichard:1990aa} to portray it as a parody of a scientific account. While Plato may partially realize the youthful ambitions of Socrates in the guise of Timaeus, perhaps he remains sensitive to the difficulties that drove Socrates to despair of natural philosophy and is not above expressing this with a certain playfulness.

Even if we admit an element of playfulness in the derivation of the elemental composition of the body of the Cosmos, such playfulness is consistent with a seriousness of purpose. Allow me to speculate on but one theme that may be at stake in the presentation of the derivation. Perhaps the playful pseudo-derivation allows Plato, in the guise of Timaeus, to explore a possible reconciliation of Parmenidean and Empedoclean cosmologies, and thus perhaps lay claim to the allegiance of those who separately adhere to them. Begin with two observations that we made previously. First, unlike Empedocles, Timaeus does not merely take it for granted that the corporeal is composed of the four Empedoclean ``roots'' but rather seeks to derive these from their sensible powers. Second, he does by specifying these powers as visibility and tangibility, visibility requiring fire, and tangibility requiring earth. We canvassed some reasons for this specification. Perhaps another is presently relevant. Parmenides presents an account of Becoming, the sensible realm of generation, in the second half of his poem, The Way of Mortal Opinion. In order to account for the sensible realm of generation, Parmenides posits two principles, Fire and Night (DK 28B8 50--9). Surprisingly, Aristotle represents these as fire and earth (\emph{Metaphysica} A 986b31). Perhaps the identification represents an earlier Academic development. Notice that fire and earth are the two primary bodies from which the remaining Empedoclean ``roots'' are derived given the need for proportionate intermediaries to combine these opposed extremes in amity and unity. In so doing, has Timaeus not, among other things, shown how to accommodate the four ``roots'' of Empedocles' cosmology with the two principles of Parmenides? 

We encounter an echo of this in Timaeus' account of color mixture (68b5–c7). There, or so shall I argue (chapter~\ref{sec:the_eyes}), Timaeus attempts to accommodate the Democritean four color scheme within the ancient tradition that takes light and dark as the fundamental chromatic opposition. To be visible is to be colored. To be colored is to possess the power to divide (\emph{diakrisis}) or compact (\emph{sugkrisis}) the visual body (\emph{opsis}) composed of the mild light of day and a fire akin to it emitted from within the eye of the perceiver. Light divides the visual body whereas dark compresses it. Bodies possess this power, and so are colored, by emitting a different kind of fire, flame (\emph{phlox}), since like does not affect like (57a3–5). Vision thus arises from three fires of two kinds (45b–46c, 67c–68d). It is unsurprising, then, that Timaeus understands the fundamental chromatic opposition to be the Parmenidean opposition of Fire and Night, light and dark. Timaeus' contribution consists, instead, in showing how a recent rival scheme, to be found among physicians, painters, and Democritus, that posits four primary colors rather than two, can be understood within the more ancient tradition.

% section the_elemental_composition_of_the_corporeal (end)

\section{The Comprehensiveness of the Cosmos} % (fold)
\label{sec:the_comprehensiveness_of_the_Cosmos}

Not only is the body of the Cosmos composed of the four primary bodies, its corporeal composition is comprehensive. The body of the Cosmos is composed of all of the primary bodies---fire, air, water, and earth. There are no primary bodies, nor part (\emph{meros}) nor power (\emph{dunamis}) of them, outside the Cosmos. An observation is in order about the parts and powers of primary bodies. Primary bodies, Timaeus will tell us, are composed of elemental triangles that are not themselves primary bodies. So not every part of a primary body is a primary body. Moreover, the powers of the primary bodies in a sense pre-exist their formation. In the pre-cosmic chaos, before the creation of the primary bodies, these powers, or traces of them, move in a disorderly fashion. So like their parts, the powers of primary bodies are in a sense distinct from them. There are no primary bodies, nor parts, nor powers of them, outside the Cosmos. Moreover, as Timaeus emphasizes, this was the Demiurge's intent. Timaeus offers three reasons that moved the Demiurge to make the corporeal composition of the Cosmos comprehensive:
\begin{enumerate}[(1)]
	\item That the Cosmos be as whole and complete as possible, made from complete parts
	\item That the Cosmos be single, since if there were parts left over, another Cosmos might be made
	\item That the Cosmos, as a living being, be free from age and sickness
\end{enumerate}
Let us consider these in turn. I shall do three things. First, I shall explain each of the reasons behind the Demiurge's intention. Second, I shall explain how they bear on the corporeal comprehensiveness of the Cosmos, that is, I shall explain why they are reasons for the Demiurge's intention. Finally, I shall consider how these reasons interrelate. It will be convenient to do the first two things and consider their interrelation only once this is done.

(1) The Demiurge endeavors to make the body of the Cosmos as whole and complete as possible. According to Proclus, this is the third gift of the Demiurge (\emph{In Timaeum} 2, \citealt{Diehl:1903re}). Recall that the Demiurge is benevolent and ungrudging and so seeks to make the Cosmos as good as possible. Being partial and being incomplete are imperfections. And if that is right, then, conversely, being whole and being complete are perfections. To see this consider the following. If the Demiurge made only part of the Cosmos when He could have made the whole of it, then it would be less perfect than it could have been. And if the Demiurge made the Cosmos incomplete when He could have completed it, then, again, it would be less perfect than it could have been. Being benevolent and ungrudging, the Demiurge seeks to make the Cosmos as good as possible, and this involves, so Timaeus tells us, the Cosmos being as whole and complete as possible.

Timaeus provides an additional detail on the completeness condition. To be as complete as possible, the Cosmos must be composed of complete parts. I do not think that this additional detail is trivial, there merely for rhetorical emphasis. Suppose that the Cosmos has all of its parts. If some of the parts of which the Cosmos is composed were themselves incomplete, then the Cosmos as a whole would not be as complete as it would have been had it been composed of complete parts. The Cosmos would be complete, in a sense, so long as it had all of its parts. But if some of these parts are themselves incomplete, then the Cosmos itself would not be as complete as possible.

How does being as whole and complete as possible bear on the corporeal comprehensiveness of the Cosmos? Suppose that some of the primary bodies, or their parts, or powers, were outside of the Cosmos. As these are potential parts of the Cosmos, the Cosmos would not not be as whole and as complete as possible. (Compare Socrates' characterization of a whole as that from which nothing is absent, \emph{Theaetetus} 205a4--7, and Parmenides' similar, if distinct, characterization as that from which no part is missing \emph{Parmenides} 137c7--8. For discussion of Platonic mereology see \citealt{Harte:2002tl}.) The wholeness and the completeness of the Cosmos require that its corporeal composition be comprehensive. It is thus a reason for the Demiurge to make a corporeally comprehensive Cosmos.

(2) The Cosmos is unique. Recall, the Demiurge created one Cosmos so that it may better resemble the Paradigm on which it was modeled. Being unique, like being whole and complete, is a perfection, at least by Timaeus' lights. As we have seen, uniqueness is a perfection in itself as well as being a necessary condition on the realization of other perfections such as everlastingness.

How does being unique bear on the corporeal comprehensiveness of the Cosmos? The Cosmos is composed of the primary bodies. Suppose that the Cosmos were not corporeally comprehensive. Suppose, that is, that there were primary bodies, or parts, or powers of them, that existed outside the Cosmos, then there would be material from which another Cosmos, perhaps partial and incomplete, could be made. The uniqueness of the Cosmos thus requires that its corporeal composition be comprehensive.  It is thus a reason for the Demiurge to make a corporeally comprehensive Cosmos.

(3) The Cosmos is a living being created by a benevolent and ungrudging Demiurge. He thus seeks to make the Cosmos as good as possible. Since the Cosmos is a living being, it is better if it did not age and was not subject to sickness as mortal living beings are prone to. Like uniqueness, being free from aging and sickness is a perfection in itself and as well as being a necessary condition on the realization of other perfections. Thus, since the Cosmos is free from aging and sickness, it is everlasting. And being everlasting it thus resembles the eternity of the Paradigm on which it was modeled. Being free from aging and sickness, like being whole, being complete, and being unique, is a perfection.

How does being free from aging and sickness bear on the corporeal comprehensiveness of the Cosmos? The Cosmos, being composed of the four primary bodies, is composite. When the composite body of a mortal living being is surrounded by hot or cold things or any other thing with ``strong powers'', these may weaken or destroy the composite body, causing it to age or to sicken. Since being free from aging and sickness is a perfection, the Demiurge insures that there are no primary bodies, nor parts, nor powers of them, outside of the Cosmos that may surround it and weaken it with strong powers. (There is a linguistic echo of Parmenides here. Specifically, with Parmenides' use of \emph{asulon}---free from violence---in comparing the One Being to a sphere, DK 28B8.48.) That the Cosmos be free from aging and sickness requires that its corporeal composition be comprehensive. It is thus a reason for the Demiurge to make a corporeally comprehensive Cosmos.

Consider now how these reasons interrelate (for discussion see Proclus, \emph{In Timaeum} 2 58.20--59.24, \citealt{Diehl:1903re}). 

First, suppose that the Cosmos is as whole and complete as possible. If it is as whole and complete as possible, then there are no primary bodies, nor parts, nor powers of them, outside of the Cosmos. And if there are no primary bodies, nor parts, nor powers of them, outside of the Cosmos, then there is no material from which another Cosmos may be composed. It follows that the Cosmos would be unique. And since there are no primary bodies, nor parts, nor powers of them, outside of the Cosmos, then there are no strong powers that would act upon the Cosmos causing it to age or sicken. It follows that the Cosmos would be free from aging and sickness (at least on the assumption that the cause of aging and sickness is invariably external).

Second, suppose that the Cosmos is unique. If the Cosmos is unique there is nothing external to it from which another Cosmos may be composed. It must then be composed of all the primary bodies that there are. And if it is, then it is as whole and complete as possible, at least with respect to its corporeal composition. And if there is nothing external to the unique Cosmos, then there are no strong powers that would act upon it causing to age and sicken. It follows that the Cosmos would be free from aging and sickness (again, on the assumption that the cause of aging and sickness is invariably external).

Third, suppose that the living Cosmos is free from aging and sickness. On the supposition that the cause of aging and sickness is invariably external, then there is nothing external to the Cosmos. If there is no primary bodies, nor their parts, nor powers, outside of the Cosmos, then the Cosmos is composed of all the primary bodies that there are. It follows that it would be as whole and complete as possible, at least with respect to its corporeal composition. And if there is nothing external to the living Cosmos free from aging and sickness, then there is no material from which another Cosmos may be composed. It follows that the Cosmos would be unique as well.

So these three reasons---that (1) the Cosmos is as whole and complete as possible, (2) unique, and (3) free from aging and sickness---are mutually supporting and each individually are reasons for the corporeal comprehensiveness of the Cosmos, that there are no primary bodies, nor parts, nor powers of them, external to the Cosmos. Indeed, Timaeus will go on to make an an even stronger claim. Not only are there no primary bodies, nor parts, nor powers of them, external to the Cosmos, neither is there place nor void in which the divided being of the corporeal could extend. The limit of the Cosmos, like the limit of the One Being of Parmenides, is absolute.

That the Cosmos is corporeally comprehensive has an important consequence. If the Cosmos is corporeally comprehensive, then it is not situated in a sensible environment. The sensible is the mark of the corporeal, and there is nothing corporeal external to the Cosmos. Rather than being situated in a sensible environment, the Cosmos, considered as a whole, just is the sensible environment. As we shall see in the next section, like an environment worthy of that name, the Cosmos circumscribes what it contains.

% section the_comprehensiveness_of_the_Cosmos (end)

\section{The Shape of the Cosmos} % (fold)
\label{sec:the_shape_of_the_Cosmos}

The derivation of the elemental composition of the body of the Cosmos proceeded from conditions on the possibility of vision and touch. Perhaps surprisingly, Timaeus' argument for the shape of the Cosmos also has important lessons for the philosophy of perception. As we shall see, Timaeus' argument that the body of the Cosmos is spherical crucially depends upon the idea that sensory powers require an environment in which to operate. Specifically, the objects of perception are essentially environmental (in something like Travis' \citeyear{Travis:2005ys} sense). 

The Timaean Cosmos shares a number of features with Empedocles' Sphere. We have seen that it is composed of the four Empedoclean ``roots'' bound together by \emph{philia}. It is also spherical in shape. And like Empedocles's sphere is has no need for hands and feet.

Timaeus describes three formal features that recommend that the Cosmos be spherical (33b--34a8). These features are later summarized in the \emph{eikos muthos} of the Demiurge interweaving the World Soul with the body of the Cosmos (34a8--34b10). In that summary, Timaeus claims that the shape of the Cosmos must have a smooth and even surface, sides of equal distance from its center, and be whole and complete made up of complete bodies (34b1--3). Timaeus' summary presents these features in reverse order of their original discussion:
\begin{enumerate}[(1)]
	\item a whole and complete body made up of complete bodies (33b1--4)
	\item sides of equal distance from its center (33b4--8)
	\item smooth and even surface (33b8--34a8)
\end{enumerate}
Let us consider them in the order in which Timaeus originally presents them.

(1) \emph{A whole and complete body made up of complete bodies (33b1--4).} Earlier, we saw how the requirement that the body of the Cosmos be as whole and complete as possible, with complete parts, established that its corporeal composition be comprehensive. No primary bodies, nor parts, nor powers of them, are outside of the Cosmos. The Cosmos thus contains all the primary bodies, their parts, and their powers. There is, however, another dimension to the Cosmos being as whole and complete as possible, with complete parts. The Cosmos may be corporeal, but it is also a sensible living being. So the requirement is that a sensible living being be as whole and complete as possible, with complete parts. The added dimension flows from what it means for a sensible living being to be as whole and complete as possible. That the Cosmos is a sensible living being was earlier acknowledged in Timaeus' assurance that it was free from aging and sickness---only sensible living beings are so much as subject to these. (The qualification, sensible, is required, here, since, by contrast, intelligible living beings neither age nor sicken.) However, Timaeus only earlier considered what it would be for the Cosmos to be a whole and complete body. Only now does he consider what it is for the Cosmos to be a whole and complete sensible living being
% The Demiurge gives to the body of the Cosmos the shape appropriate to the kind of thing that it is.
The Cosmos is a sensible living being. For a sensible living being to be as whole and complete as possible, with complete parts, it must contain within itself all other sensible living beings. If there were a sensible living being not a part of the Cosmos and so outside of it, then the Cosmos would not be as whole and complete a sensible living being as possible. But being as whole and complete as possible, with complete parts, is a perfection. And since the Demiurge is benevolent and ungrudging, seeking to make the world as good as possible, He endeavors to endow this perfection upon the Cosmos. And so the Cosmos is a sensible living being that contains within itself all other sensible living beings. The Cosmos is thus comprehensive not only in the sense that it contains all the primary bodies, with all of their parts and powers, but also in the sense that it contains all other sensible living beings. In this way it is like the Paradigm, an intelligible Living Being that contains within itself all other intelligible living beings. In being a comprehensive living being the Cosmos shares with the Paradigm the perfection of being whole and complete, with complete parts.

What is the shape (\emph{schēma}) appropriate to a comprehensive sensible living being? Since the Cosmos is a sensible living being that contains all other sensible living beings, it must have a shape that may encompass the shapes of all the sensible living beings that it contains (see Proclus, \emph{In Timaeum} 2 70.32--72.6, \citealt{Diehl:1903re}). That a sphere contains within itself all other shapes is what makes it the shape appropriate to a sensible living being that contains all other sensible living beings. Notice that what is at stake here is the aptness or appropriateness of a spherical shape for a comprehensive sensible living being. There is no question of the Cosmos being a comprehensive sensible living being entailing that it is spherical. This is part of what makes the present account is an \emph{eikos logos}.

How are we to understand the claim that the sphere contains all other shapes within it? It is plausible to suppose that Timaeus has in mind the more specific claim the five regular polyhedra---the tetrahedron, cube, octahedron, dodecahedron, and icosahedron---may be inscribed within the sphere (\citealt[101--2]{Taylor:1928qb}). Euclid provides proofs of these in the thirteenth book of \emph{Elementa}. Euclid's contribution is the systematic presentation of these proofs. The proofs themselves are of an earlier provenance. Thus the author of the first scholium of book 13 of \emph{Elementa} writes ``the five so-called Platonic figures, which however do not belong to Plato, three of the aforesaid figures being due to the Pythagoreans, namely the cube, the pyramid and the dodecahedron, while the octahedron and the icosahedron are due to Theaetetus'' \citep[438]{Heath:1908th}. Timaeus will go on to assign four of these regular polyhedra to the primary bodies. Fire is a tetrahedron, air is an octahedron, water is a icosahedron, and earth is a cube. The remaining regular polyhedron, the dodecahedron, the Demiurge uses to decorate the Heavens. Given this, it is appropriate that the Cosmos have a shape in which the five regular polyhedra may be inscribed. And a sphere has this formal feature. Again, the appropriateness, here, is not underwritten by entailment. And again, this is part of what makes the present account an \emph{eikōs logos} (\citealt[101]{Taylor:1928qb}).

(2) \emph{Sides of equal distance from its center (33b4--8).} The sides of the Cosmos should be equidistant from the center. The distances in question are finite. Should the distances be infinite, then they would not determine a finite figure, and so would lack shape. In being limited, the Cosmos is determinate. Since the finite distance in every direction from the center is equal, the shape of the Cosmos is complete and uniform. Let us consider these in turn. First, should the Cosmos have a side that is closer to its center than the other sides, then the shape of the Cosmos would be incomplete. Second, should the Cosmos have a side that is closer to its center than the other sides, then not all sides would be equidistant from its center. And if not all sides were equidistant, then the Cosmos would not be uniform. But being complete and uniform is a perfection. (Notice that the sense of uniformity here must differ from the sense in which Timaeus claimed in the \emph{proemium} that what is and never becomes is the object of understanding with an account since it is uniform. Only intellegible objects are uniform in this sense and the Comsos is sensible.) We have already explained why completeness is a perfection. The question remains why uniformity is. The explicit reason that Timaeus offers for why uniformity is a perfection is aesthetic. Timaeus claims that the shape of the body of the Cosmos is fair because of its uniformity. The uniform is self-similar, and the similar is more fair than the dissimilar. Should this aesthetic reason be valid, then uniformity, like completeness, would be a perfection. The Demiurge is benevolent, and ungrudging, and seeks to make the world as good as possible. Since He endeavors to make the Cosmos as good as possible it is appropriate that He gives it a shape that is complete and uniform. And only a body with sides of equal distance from its center would be complete and uniform in the relevant sense.

Timaeus makes two allusions here. 

The first is an allusion to to Anaximander's claim that the Earth is suspended in equipose at the center of the Cosmos (Aristotle, \emph{De caelo} 2.13 295b11–16 = DK 12A26). Plato endorses Anaximander’s claim in the \emph{Phaedo} 108e4–109a7, and Timaeus will do so as well when discussing a popular misconception about the directions, above and below (6210–63a6, chapter~\ref{sec:heavy_and_light}). The Earth may only be suspended in equipose if the sides of the Cosmos are an equal finite distance from its center. 

The second allusion is a deliberate echo of Parmenides. Parmenides compares the One Being to a spherical ball that is equally poised from its center (DK 28B8.41--9, no doubt an influence and inspiration for the Empedoclean Sphere). The sphere in question is not the abstract sphere of stereometry, a discipline not yet developed when Parmenides composed his poem. Rather Parmenides is comparing the One Being to a corporeal sphere with volume. While Timaeus is not a Parmenidean monist, his cosmology is Eleatic to the extent that he envisions the Cosmos to be one, finite, and spherical. Parmenides holds not only that Being is one, but that determinate Being is limited or finite (DK 28B8.42). He thus rules out in advance that Being may be unlimited or infinite. According to Parmenides, since Being is perfect, it is limited. And since its limit is absolute, its perfection is universal. This universal perfection is illustrated by how a sphere is equally poised from its center and is thus self-similar or uniform. What is the relevant dimension of self-similarity or uniformity? A sphere need not be self-similar or uniform in every respect. A corporeal sphere may be multicolored, for example. Perhaps, the dimension of self-similarity or uniformity is constant curvature, for example. The suggestion limits the dimension of uniformity to the geometrical properties of a sphere, allowing corporeal spheres to vary in other ways.  The One Being, like the voluminous corporeal sphere, is itself self-similar and uniform though in this case with respect to its universal perfection. The Cosmos, like the voluminous sphere to which Parmenides compares the One Being, is corporeal. And like the Parmenidean analogy, it is one, finite, determinate, equiposed from its center, and self-similar or uniform, at least with respect to its shape as a whole.

(3) \emph{Smooth and even surface (33b8--34a8).} The Cosmos is corporeal. And being limited, it has a determinate shape. And since the Cosmos is equiposed from its center, its determinate shape has a smooth and even surface. Moreover, the Demiurge has no reason for endowing the Cosmos with parts that would disturb its a smooth and even surface. Being the kind of thing that it is, namely one and corporeally comprehensive, it has no need of such parts. What is directly relevant to the shape of the Cosmos is its spatial parts. However, while Timaeus sometimes lists unneeded parts that would disrupt the smooth and even surface of the Cosmos, sometimes he lists, instead, unneeded powers. The parts and powers are, however, related in a way that gives Timaeus license to move freely between them. The spatial parts in question would be required if the Cosmos had the relevant powers. They are the instruments of these powers. So if the Cosmos lacks the relevant spatial parts, it would lack the relevant power. And if it lacked the relevant power, it would lack the relevant spatial part. This last may be questioned if there is more to having the relevant power than possessing the relevant spatial part. This is not grounds for criticism, however, since, in this context, Timaeus is really only considering potential positive reasons for departures from the smooth and even surface of the Cosmos. And in each case Timaeus argues that they are inapplicable to the Cosmos as it has no need of the relevant parts and powers:
\begin{enumerate}[(a)]
	\item The Cosmos has no need for eyes (33c1--2)
	\item The Cosmos has no need for hearing (33c2--3)
	\item The Cosmos has no need for respiration (33c3--4)
	\item The Cosmos has no need for an organ to receive nutriment nor for an organ to evacuate waste (33c4--33d3)
	\item The Cosmos has no need for hands (33d3--5)
	\item The Cosmos has need for legs nor feet (33d5--34a7)
\end{enumerate}

The Cosmos, like mortal humans, is a sensible living being. However, the Cosmos, unlike mortal humans, has no need of these parts and powers. This passage of the cosmogony (33c--34a) is usefully contrasted with the passage on anthropogony where the Young Gods generate the human body in imitation of Demiurgic activity (42e--46a). The Young Gods do so at the behest of the Demiurge so that they may generate specifically mortal living beings. The Demiurge thus assigns them this task precisely because their imitation of Demiurgic activity is imperfect. The Demiurge Himself may generate only immortals. The irony is that the generation of mortal living beings is required to complete and so perfect the Cosmos as a comprehensive sensible living being. But the mortal living beings that are generated by the Young Gods, in the process of completing the Cosmos, do have need of these parts and powers. The full significance of the Demiurge fashioning the Cosmos with a smooth and even surface thus really only emerges in contrast with Young Gods' fashioning of the human body. 

The powers that the Cosmos has no need of are vision, audition, respiration, nutrition, prehension, repulsion, and locomotion. One thing to observe is the connection with perception. This is explicit in the first two powers that the Cosmos has no need of. But, importantly, it is implicit in the remaining. Specifically, the Cosmos has no need of vision, audition, smell, taste, and touch (Alcinous, \emph{Didaskalikos} 12.3 168 2--4). The Cosmos has need of neither vision nor audition. That much Timaeus tells us explicitly. But notice that without respiration, there is no inhaling the odorous and so no olfaction. And without an organ to receive nutriment, there is no taste. And though touch has no dedicated organ, the Cosmos has no hands to grasp with, a form of haptic touch, and anyway, there is nothing corporeal external to the Cosmos which may be touched. So the Cosmos has no need of the Peripatetic five special senses, vision, audition, smell, taste, and touch. 

This is further reinforced by their ordering. Vision and touch are the extreme oppositions, with the other sensory powers arrayed in between them (section~\ref{sec:the_elemental_composition_of_the_corporeal}). Moreover, as we shall see (chapter~\ref{sec:common_versus_peculiar_pathemata}), they are listed here in descending order of the ethical significance. Further, Timaeus will discuss these senses in reverse order in his discussion of the \emph{pathēmata} (61d--68e). 

The Cosmos has no need of these powers, whether sensory, like vision, or non-sensory, like respiration, since all of these powers are environmental. They are all powers that can only ever be exercised in an environment. But the Cosmos is not situated in an environment. Timaeus' overall lesson, here, for the philosophy of perception, will be that the objects of perception are environmental. A perceiver is embodied and embedded in an environment that circumscribes them, and sensible objects are aspects of that environment. However, there is no environment in which the Cosmos is embedded, and so there are no aspects of its environment to be sensed, and thus the Cosmos lacks the spatial parts required for the relevant sensory powers, instruments the presence of which would disturb the Cosmos' smooth and even surface.

(a) \emph{The Cosmos has no need for eyes (33c1--2)}. The Cosmos has no need for eyes to see with for the simple reason that there is nothing outside of it to see. This is a consequence of the corporeal comprehensiveness of the Cosmos. Recall, for the Cosmos to be corporeally comprehensive, there can be no primary bodies, nor parts, nor powers of them, outside of the Cosmos. The sensible is the mark of the corporeal. As there is nothing corporeal outside of the Cosmos, it follows that there is nothing sensible outside of the Cosmos, not even disordered traces of sensible powers. Indeed, beyond the limits of the Cosmos is neither space nor void. And if there is nothing sensible outside of the Cosmos, then there is nothing for the Cosmos to see outside of itself. It thus has no need for eyes to see with. There is nothing external to be seen. Eyes protruding from the surface of the Cosmos would disturb the smoothness and evenness of that surface. However, there is no reason to endow the Cosmos with sight, and hence no reason disturb the smoothness and evenness of its surface with the instruments of sight.

The overall structure of Timaeus' reasoning is relatively clear. However, it is worth reflecting upon the central claim that the Cosmos has no need for eyes to see with since there is nothing external to see. While itself straightforward enough, it is worth observing that this is an instance of a more general principle. This becomes clear in Timaeus' subsequent discussion of the other parts and powers. What, then, is this more general principle?

Perception is environmental. A perceiver is embodied and embedded in an environment that circumscribes them. And perception is a power of the perceiver that is only ever exercised in an environment that circumscribes that perceiver. The objects of perception are aspects of that environment. The perceiver and the perceptible aspects of their environment are distinct and non-overlapping. If the perceiver is distinct and non-overlapping with the perceptible aspects of the environment, then these are neither the whole of the perceiver nor a proper part of the perceiver, and so must be external to the perceiver. (At least typically. There may be edge cases. Think of Austin's example of feeling a bone caught in one's throat.)

This is the general condition that the Cosmos fails to meet. There is nothing corporeal outside of the Cosmos, neither primary bodies, nor their parts, nor powers. The Cosmos may be embodied but it is not embedded in an environment. Beyond its limits is neither space nor void. Rather, not only is the Cosmos corporeally comprehensive, but it is also a comprehensive sensible living being. That is to say that the Cosmos is a sensible living being that contains within itself all other sensible living beings. The Cosmos is thus an environment that circumscribes the sensible living beings that it contains. These sensible living beings are in the Cosmos that circumscribes them. And should a sensible living being be sentient, other sensible living beings and other aspects of the Cosmos are external to them and perceptible. Unlike the sensible living beings that it contains, the Cosmos is not embedded in an environment, it is the environment.

(b) \emph{The Cosmos has no need for hearing (33c2–3)}. The Cosmos has no need for hearing for the simple reason that there is nothing audible outside of it. This is a consequence of the corporeal comprehensiveness of the Cosmos. There are no primary bodies, nor their parts nor powers, outside of the Cosmos. As the sensible is the corporeal, there is nothing sensible outside of the Cosmos. And if there is nothing sensible outside of the Cosmos, there is nothing specifically audible outside of the Cosmos. It thus has no need for ears to hear with. There is nothing external to be heard. Ears protruding from the surface of the Cosmos would disturb the smoothness and evenness of that surface. Even ears that consist in no more than a cavity would disturb the smoothness and evenness of the surface of the Cosmos. However, there is no reason to endow the Cosmos with audition, and hence no reason to disturb the smoothness and evenness of its surface with the instruments of audition.

A couple of observations. First, Timaeus only explicitly mentions the power, audition, and not the instrument of audition, the ear. (In contrast, he began by explicitly mentioning the instrument, the eye, but not the power, vision.) But only the spatial part, the instrument, is directly relevant to whether or not the Cosmos has a smooth and even surface. Second, this power, audition, like the first, vision, is a sensory power.

The claim that the Cosmos has no need of ears to hear with since there is nothing external to hear is an instance of a more general principle. Perception is environmental. A perceiver is embodied and embedded in an environment that circumscribes them. And perception is a power of the perceiver that is only ever exercised in an environment that circumscribes that perceiver. The objects of perception are aspects of that external environment. But since there is nothing corporeal external to the Cosmos, there is nothing perceptible and, hence, nothing specifically audible external to it. Again, the Cosmos is not embedded in an environment that circumscribes it, it is the environment.

(c) \emph{The Cosmos has no need for respiration (33c3--4)}. The Cosmos has no need for respiration for the simple reason that there is no air for it to breathe. This is a consequence of the corporeal comprehensiveness of the Cosmos. There are no primary bodies, nor their parts nor powers, outside of the Cosmos. Hence, specifically, there is no air outside of the Cosmos. The Cosmos has no need for nostrils to breathe with. There is no external air to breathe. Nostrils to breathe with would disturb the smoothness and evenness of the surface of the Cosmos. However, there is no reason to endow the Cosmos with respiration, and hence no reason to disturb the smoothness and evenness of its surface with the instruments of respiration.

A couple of observations. First, Timaeus explicitly mentions only the power, respiration, and not the part that serves as its instrument, the nostrils. But without some link to the spatial parts of the Cosmos that would affect its shape, the power of respiration would be irrelevant to the discussion. Second, The power explicitly mentioned by Timaeus, respiration, is not a sensory power like vision and audition. As his discussion of the \emph{pathēmata} peculiar to particular parts of the body reveals, however, respiration is linked with a sensory power, olfaction. This is also relevant to our first observation as the instrument by which we breathe is the instrument by which we smell.

Given respiration's link to olfaction, parallel reasoning applies. The Cosmos has no need for olfaction for the simple reason that there is nothing external to smell. This is the joint consequence of the corporeal comprehensiveness of the Cosmos and the environmental nature of perception. There is thus no reason to endow the Cosmos with nostrils to smell with and so disturb the smoothness and evenness of its surface.

(d) \emph{The Cosmos has no need for an organ to receive nutriment nor for an organ to evacuate waste (33c4--33d3)}. The Cosmos has no need for an organ to receive nutriment nor for an organ to evacuate waste for the simple reason that nothing goes into the Cosmos and nothing goes out of it. This is a consequence of the corporeal comprehensiveness of the Cosmos. Nothing goes into the Cosmos since there is nothing corporeal external to it. And nothing goes out of the Cosmos. For suppose it did, then there would be something corporeal external to the cosmos inconsistent with its corporeal comprehensiveness. Organs to receive nutriment and to evacuate waste would disturb the smoothness and the evenness of the surface of the Cosmos. However, there is no need to endow the Cosmos with such organs and so disturb the smoothness and evenness of its surface.

Timaeus uses this discussion to highlight an additional perfection of the Cosmos. In possessing this perfection, the Cosmos, a sensible living being, better resembles its Paradigm, the intelligible Living Being. In lacking organs to receive nutriment and evacuate waste, the Cosmos is distinguished from the sensible living beings that reside within it. These have such organs. Moreover, they have such organs since they are dependent upon their environment to eat and to evacuate waste. The Cosmos, by contrast, feeds of its own unexpelled waste. The parts of the Cosmos that decay are the source of its own nutriment. In not being dependent upon an environment, the Cosmos is, in this regard at least, self-sufficient. Timaeus underscores this by making a stronger, more general claim. The Cosmos is neither acted upon from without nor acts upon anything external. All actions and affections of the Cosmos occur in and by itself. The Cosmos is thus self-sufficient in the this more general sense. Self-sufficiency is a perfection. And the Cosmos, in having the perfection of self-sufficiency, at least to the extent that it does, better resembles the Paradigm of which it is an image, the intelligible Living Being.

A couple of observations. First, Timaeus explicitly mentions the parts, the organs for receiving nutriment and evacuating waste, but these parts are specified with respect to the relevant powers. Second, the powers in question are not sensory powers. However, the first of these, ingestion, is linked with a sensory power, gustation. In receiving external nutriment, mortal animals taste the nutriment that they ingest. 

Given ingestion's link with gustation, parallel reasoning apples. The Cosmos has no need for gustation for the simple reason that there is nothing external to take in and so taste. This is the joint consequence of the corporeal comprehensiveness of the Cosmos and the environmental nature of perception. There is thus no reason to endow the Cosmos with a mouth to receive nutriment with, nor a tongue to taste with, and so disturb the smoothness and evenness of its surface.

(e) \emph{The Cosmos has no need for hands (33d3--5)}. This claim and the next, that the Cosmos has no need for hands and feet, and their connection with the sphericity of the Cosmos is arguably a deliberated echo of Empedocles:
\begin{verse}
	From two branches do not dart from its back\\
	Nor feet nor swift knees nor potent genitals,\\
	But it indeed is equal <to itself> on all sides and totally unbounded,\\
	A rounded sphere rejoicing in its surrounding solitude. (Empedocles DK 31B29\&28; \citealt[233]{Inwood:2001ve})
\end{verse}
This passage mentions only limbs and genitals, but another related passage mentions, in addition, the head:
\begin{verse}
	For [it/he] is not fitted out in [its/his] limbs with a human head,\\
	Nor do two branches dart from [its/his] back\\
	Nor feet, nor swift knees nor shaggy genitals;\\
	But it/he is only a sacred and ineffable thought organ\\
	Darting through the entire cosmos with swift thoughts. (Empedocles DK 31B134; \citealt[263]{Inwood:2001ve})
\end{verse}
Arguably, then, this entire passage (33c--34a), comprising points (a--f), is a Timaean elaboration of an Empedoclean theme.

The Cosmos has no use of hands for the simple reason that there is nothing external to grasp or repel. One graps what is external to the hands. And unless what is grasped is some other part of the body, as when one clutches one's head in one's hands, what is grasped is external to one. The hand as an instrument to repel is even more straightforward. One only needs such an instrument if there is an external threat to repel. But the Cosmos is corporeally comprehensive. There is no external threat to repel, and nothing external to grasp. Hands to grasp and repel with would disturb the smoothness and the evenness of the surface of the Cosmos. However, there is no need to endow the Cosmos with hands and so disturb the smoothness and evenness of its surface.

Grasping, here, is prehensive. Grasping may also be a mode of haptic perception, that \citet{Lederman:1987fr} calls enclosure. As a mode of haptic perception, grasping a body is a way of feeling its overall shape and volume. Prehension may occur independently of enclosure. When grasping a hammer as one uses it, typically on does not attend to its overall shape and volume so much as to the nail that one is aiming to hit. 

A few of observations. First, Timaeus explicitly mentions the parts, the hands, and explicitly mentions the powers that the hands are an instrument for, specifically, the powers to prehend and repel. Second, unlike the organs to receive nutriment and evacuate waste, hands are parts that may be specified independently of the powers they are instruments of. Hands are not specifically needed to prehend or repel. Tails and tentacles are also vital instruments that may prehend and repel. Third, notice, as well, that, unlike the other cases, the hands are multipurpose parts. They are instruments that serve distinct powers. Fourth, the powers in question, to prehend and to repel, are not sensory powers. However, the first of these, prehension, is linked with a sensory power, enclosure. In grasping an external body, one may feel the overall shape and volume of the body enclosed in one's grasp. And even repelling an external threat will involve tactile affection. 

Given prehension's link with enclosure, parallel reasoning applies. The Cosmos has no need for enclosure for the simple reason that there is nothing external to enclose in its grasp. This is the joint consequence of the corporeal comprehensiveness of the Cosmos and the environmental nature of perception. There is thus no reason to endow the Cosmos with hands to grasp with and so disturb the smoothness and evenness of its surface.

(f) \emph{The Cosmos has need for legs and feet (33d5–34a7)}. Legs and feet are instruments of locomotion. The Cosmos has no legs and feet for the simple reason that it has no need for the power of locomotion. Beyond the limits of the Cosmos, there is neither space nor void into which it may move. What is important, here, is the connection between sphericity and the lack of need for the instruments of locomotion. Again this is a deliberate echo of Empedocles (DK 31B29).

Timaeus distinguishes seven motions:
\begin{enumerate}[(i)]
	\item rotation
	\item forward
	\item backward
	\item right
	\item left
	\item upward
	\item downward
\end{enumerate}
The first circular motion, rotation, is rational, while the remaining six rectilinear motions are irrational. Circular motion is rational in being akin to cognitive activity. (How exactly that is so is an important topic addressed in the next chapter.) Rectilinear motion is not circular and so is irrational in not being akin to cognitive activity. Locomotion, a change of place over time, is confined to the six irrational motions. In rotation, the parts of the rotating body may be changing their location over time, but the body as a whole is moving in place and so is not subject to locomotion. 

The Demiurge assigns to the Cosmos only the rational motion of rotation. Being rational, it is the motion proper to the Comos. The Cosmos thus spins uniformly in place thus revolving in a circle. While instruments of locomotion are necessary for the six irrational motions, no such instruments are required for rational motion. Notice only a sphere may rotate in place and remain confined to its own limits. Contrast a rotating cube. Whereas the parts of the rotating sphere only move into places previously occupied by other parts of the sphere, the parts of a rotating cube will move into places previously unoccupied by parts of the cube. So there must at least be space beyond the limits of the rotating cube in the way there is not with a rotating sphere. So the spherical shape of the Cosmos is itself appropriate to is circular motion. Indeed, its required if beyond the limits of the Cosmos there is neither space nor void.


% If the shape of the Cosmos were instead cubic, then there would have to be space beyond its limits since parts of the cube must move into unoccupied space in order to rotate. Being spherical, however, beyond the rotating Cosmos there is neither space nor void.

Locomotion is confined to the six irrational motions. Since only the rational motion is proper to the Cosmos, it has no need of instruments of locomotion. Nor is there even space external to it to move into. Having no need of locomotion, the Demiurge fashioned the Cosmos without legs or feet.

A couple of observations. First, Timaeus explicitly mentions both parts, legs and feet, and power, locomotion. Second, touch does not have a specialized part, tactile affection being common to the body as a whole. Legs and feet, no less than hands, receive tactile affection and thus are linked with the sense of touch. But the Cosmos has no need for touch for the simple reason that there is nothing external to touch. This is the joint consequence of the corporeal comprehensiveness of the Cosmos and the environmental nature of perception.

\vspace{\baselineskip} 

\noindent The surface of the Cosmos is smooth and even since it has no need of powers whose instruments would disturb that surface. While not all of the powers explicitly mentioned by Timaeus (vision, audition, respiration, nutrition, prehension, repulsion, and locomotion) are sensory, those that are not are naturally associated with sensory powers. Thus olfaction is associated with respiration, gustation with nutrition, and touch with prehension, repulsion, and locomotion. All of these powers, sensory and non-sensory, require an environment in which to be exercised, and instruments, specialized spatial parts, to do so. But the Cosmos is not situated in an environment. It is the environment. It thus has no need of these powers whose instruments would disturb its smooth and even surface.

The Cosmos may neither see, nor hear, nor smell, nor taste, nor touch, the sensible within. This raises a question about the cognitive activity of the Cosmos. Can the Cosmos cognize the sensible that resides within, and if so, in what sense? If perception is environmental, its objects external to the percipient, then it would seem that the Cosmos does not perceive the sensible within its limits. According to the epistemology of the \emph{proemium}, the sensible is cognized by perception and opinion. If the Cosmos may not cognize the sensible within by means of perception, perhaps it may do so by means of opinion. One difficulty with this suggestion is that opinion depends upon perception. If opinion depends upon perception, and the sensible aspects within are imperceptible to the Cosmos, then the Cosmos could not so much as opine about them. How then does the Cosmos cognize the sensible? Or is the cognition of the Cosmos confined to the intelligible?

Proclus resolves this puzzle by distinguishing grades of perception (\emph{In Timaeum} 2 83.3--85.31, \citealt{Diehl:1903re}). The first and highest grade of perception belongs to the Cosmos. Mortal living beings residing within the Cosmos have the third grade of perception. The second belongs to immortal living beings within the Cosmos, the visible Gods, such as the fixed stars or the wanderers. And the fourth is a degenerate grade of perception that Timaeus attributes to plants (77b). In the first and highest grade, no distinction is drawn between perceiver and object of perception, and there is no need for instruments of perception. Moreover, the first and highest grade of perception is not divided into distinct sensory modalities like the third is. Perception of the first grade is a mode of self-awareness that Proclus likens to consciousness (\emph{sunaisthēsis}). The sensible within is thus apprehended by a general form of sensory self-awareness. That perception be environmental is thus not a requirement valid for all grades of perception. It is a requirement only on the third grade of perception possessed by mortal living beings contained within the Cosmos. Proclus thus does not deny that human perception is environmental. Indeed, he insists that it is. He claims only that the Cosmos, as befitting a visible God, has a higher mode of perception where no distinction is drawn between perceiver and object of perception, and there is no need for instruments of perception, nor division into a plurality of sensory modalities. Cosmic perception is much more unified than the perception of humans and other mortal living beings. For the Cosmos, cognizing the sensible within involves a general form of sensory self-awareness. And since the Cosmos may perceive the sensible within, it thus may also opine about it. 

Proclus' resolution, regardless of its veracity, draws our attention to important aspects of Timaeus' account. First, it draws our attention to the fact that Tiameus does not explicitly deny that the Cosmos has no perception. He only ever denies that the Cosmos has vision, audition, smell, taste, and touch. So perhaps the Cosmos has a form of perception that is not divided into a plurality of sensory modalities. Second, since the instruments of these sensory powers would disturb the smoothness and evenness of the surface of the Cosmos, the form of perception appropriate to the Cosmos would have no need of these instruments. Third since it is not situated in an environment the way mortal living beings are, but is their environment, no distinction between percipient and object of perception would be marked. I shall not evaluate the Prolcean resolution here. But that it is an intelligible way to develop, if not interpret, Timaeus' position supports key aspects of the present reading. Specifically it supports the claim that the sensory powers associated with mortal living beings that reside within the Cosmos are environmental and require instruments to exercise these powers on aspects of the sensible environment.

% How, if at all, the Cosmos cognizes the sensible within cannot be established on the present basis. Specifically, Timaeus' subsequent discussion of psychogony crucially bears on the cognitive powers of the Cosmos. We shall only be in a position to resolve our puzzle once we discuss the psychogony and its consequences for the cognitive powers of the Cosmos in the next chapter. Only there may we properly evaluate Proclus' resolution.

% section the_shape_of_the_Cosmos (end)

\section{Concluding Observations} % (fold)
\label{sec:concluding_observations_c}

Recall, Timaeus does four things in his cosmogony. He argues that:
\begin{enumerate}[(1)]
	\item the Cosmos is unique in order to better resemble the Paradigm on which it was modeled
	\item the Cosmos is composed of four primary bodies: fire, air, water, and earth
	\item the Cosmos is comprehensive, containing all the primary bodies that there are, and all their parts and powers, leaving nothing corporeal external to it
	\item the Cosmos is spherical in shape
\end{enumerate}
It may be surprising to moderns that Timaeus' cosmogony has so many lessons about the nature of sense and sensibilia. From the fact that the Cosmos is visible and tangible, he derives the fact that it is composed from fire and earth. Fire is thus a necessary precondition for vision and earth a necessary precondition for touch. This is a claim about the primary bodies presupposed by sensory powers. The second claim thus  directly bears upon the nature of sense. Moreover, Timaeus account shows that being sensible is not inconsistent with being composed of primary bodies. The sensible is, after all, the mark of the corporeal. The second claim thus directly bears on the nature of sensibilia as well. We also learned that the sensibles are arrayed between the opposed extremes of the visible and the tangible, and, as a consequence, the corresponding sensory powers are similarly arrayed. The first and third indirectly bear on the nature of sense. The uniqueness of the Cosmos requires that its corporeal composition be comprehensive with the result that there is nothing corporeal, no primary bodies, nor their parts, nor powers, external to the Cosmos. It thus must lack any powers that can only ever be exercised in an environment. Sensory powers, at least those of mortal living beings, are environmental. They are only ever exercised on aspects of the percipient's sensible environment. But the sensible is the mark of the corporeal, and as there is nothing corporeal outside of the Cosmos, there is nothing sensible outside of the Cosmos. There is nothing external to be sensed. It is not just sensory powers that are environmental, other powers of living beings, such as respiration, are environmental, but are not sensory powers. (All the powers explicitly mentioned by Timaeus and all of the sensory powers naturally associated with them are discussed in Aristotle's \emph{De anima} but two: prehension and repulsion.) Moreover, at least with mortal living beings residing within the Cosmos, spatial parts function as the instruments of these environmental powers. This bears on the shape of the Cosmos. Since the Cosmos is not situated in an environment, it has no need of instruments of powers only ever exercised in an environment. And so the Cosmos has no need of spatial parts that would disturb its smooth and even surface. Thus the Cosmos circumscribes the sensible living beings that it contains.

% section concluding_observations (end)


% Chapter cosmogony (end) 