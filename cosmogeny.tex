%!TEX root = /Users/markelikalderon/Documents/Git/timaeus/timaeus.tex

\chapter{Cosmogeny} % (fold)
\label{cha:cosmogeny}

\section{An \emph{Aporia}} % (fold)
\label{sec:an_emph_aporia}

The \emph{aporia} begins with the phenomena of flux. The relevant notion of flux, however, is restricted to the cycle of elemental transformation: the tendency for the four primary bodies---the Empedoclean ``roots'': fire, air, water, and earth---to transform into one another. The relevant example of flux, the cycle of elemental transformation, is of pre-Socratic provenance. It can be found in Anaximenes (Hippolytus of Rome, \emph{Refutatio Omnium Haeresium} 1.7.1--3 = DK 13A7), Heraclitus (DK 22B31, 22B36), Melissus (Simplicius, \emph{In Aristotelis De caelo commentaria} 558.19--559.12 = DK 30B8), and Anaxagoras (Simplicius, \emph{In physica} 179.8 = DK 59B16). The puzzle concerns the possibility of naming or linguistic reference. Specifically, if the Empedoclean ``roots'' are continually transforming into one another, then they fail to provide a sufficiently stable object for deictic reference. Concerning any primary body, it is not posible to refer to it as ``this'' (\emph{touto}) if it is continually transforming into another primary body. At best, we must describe it as ``such like'' (\emph{toiouto}). It is less a substance than a recurring kind (49e5). The primary bodies are thus not enduring objects that can be identified and re-identified but are rather phases in the cycle of qualitative flux.

The puzzle has antecedents in the \emph{Cratylus} (439d--e) and the \emph{Theaetetus} (181c--183b). The \emph{Cratylus} and the \emph{Theaetetus} each has as its target the very coherence of the doctrine of total flux---that everything is changing in every respect at all times. This is a strong claim, and its rejection is consistent with Becoming being reconceived as partial flux, arguably one of the tasks of the \emph{Timaeus}. It is useful to compare, and importantly to contrast, the present puzzle with these earlier treatments. 

In the \emph{Cratylus}, Socrates presents three arguments against the doctrine of total flux. Specifically, that doctrine is said to raise linguistic, ontological, and epistemological problems. Toward the end of that dialogue (439c), Socrates shares a dream that he often has of the Forms---beauty in itself, goodness in itself, and each of the things that are. Whereas a human face does not always retain its beauty, whether through the ravages of time, disease, or some tragic accident, true beauty is invariably beautiful. The Form of Beauty is invariably itself beautiful. If it were not, it would not be possible to say of it that it is a ``this'' or that it is ``such-like'' (439d). Moreover, if it were always changing it could not properly be said to be (439e). Finally, Socrates applies the Eleatic doctrine that knowledge requires stable objects to the Forms. Nothing variable can be known by anyone, for as one approaches it as a matter of inquiry, it changes from what it was (440a-e). A couple of observations are already pertinent. First, in the \emph{Cratylus}, total flux is said to be inconsistent with, not only naming, but predication as well. Second, that doctrine not only raises linguistic difficulties, but ontological and epistemological difficulties as well.

The linguistic difficulty is raised again in the \emph{Theaetetus}. 


% section an_emph_aporia (end)

% Chapter cosmogeny (end) 