%!TEX root = /Users/markelikalderon/Documents/Git/timaeus/timaeus.tex

\chapter{Cosmogeny} % (fold)
\label{cha:cosmogeny}

\section{An \emph{Aporia}} % (fold)
\label{sec:an_emph_aporia}

The \emph{aporia} begins with the phenomena of flux. The relevant notion of flux, however, is restricted to the cycle of elemental transformation: the tendency for the four primary bodies---the Empedoclean ``roots'', fire, air, water, and earth---to transform into one another. The relevant example of flux, the cycle of elemental transformation, is of pre-Socratic provenance. It can be found in Anaximenes (Hippolytus, \emph{Refutatio omnium haeresium} 1.7.1--3 = DK 13A7), Heraclitus (DK 22B31, 22B36), Melissus (Simplicius, \emph{In Aristotelis De caelo commentaria} 558.19--559.12 = DK 30B8), and Anaxagoras (Simplicius, \emph{In physica} 179.8 = DK 59B16). The puzzle concerns the possibility of naming or linguistic reference. Specifically, if the Empedoclean ``roots'' are continually transforming into one another, then they fail to provide a sufficiently stable object for deictic reference. Concerning any primary body, it is not posible to refer to it as ``this'' (\emph{touto}) if it is continually transforming into another primary body. At best, we must describe it as ``such like'' (\emph{toiouto}). The primary bodies are less substances than recurring kinds (49e5). The primary bodies are thus not enduring objects that can be identified and re-identified but are rather recurring phases in the cycle of qualitative flux.

The puzzle has antecedents in the \emph{Cratylus} (439d--e) and the \emph{Theaetetus} (181c--183b). The \emph{Cratylus} and the \emph{Theaetetus} each has as its target the very coherence of the doctrine of total flux---that everything is changing in every respect at all times. This is a strong claim, and its rejection is consistent with Becoming being reconceived as partial flux, arguably one of the tasks of the \emph{Timaeus}. It is useful to compare, and importantly to contrast, the present puzzle with these earlier treatments. 

In the \emph{Cratylus}, Socrates presents three arguments against the doctrine of total flux. Specifically, that doctrine is said to raise linguistic, ontological, and epistemological problems. Toward the end of that dialogue (439c), Socrates shares a dream that he often has of the Forms---Beauty in itself, Goodness in itself, and each of the things that are. Whereas a human face does not always retain its beauty---whether through the ravages of time, disease, or some tragic accident---true beauty is invariably beautiful. The Form of Beauty is itself invariably beautiful. If it were not, it would not be possible to say of it that it is a ``this'' or that it is ``such-like'' (439d). Thus, if the Form of Beauty were not invariably beautiful, then having discovered true beauty, one could not say of it ``This is beautiful''.  Moreover, echoing Parmenides, Plato maintains that that anything always changing could not properly be said to be (439e). Finally, Socrates applies to the Forms the Eleatic doctrine that knowledge requires stable objects. Nothing variable can be known by anyone, for as one approaches it as a matter of inquiry, it changes from what it was (440a-e). Thus in inquiring into true beauty only the Form of Beauty could be known because of all the beautiful things only that Form is invariable and so has the requisite stability to be a potential object of knowledge. A couple of observations are already pertinent. First, in the \emph{Cratylus}, total flux is said to be inconsistent with, not only naming, but predication as well. Second, that doctrine not only raises linguistic difficulties, but ontological and epistemological difficulties as well.

The linguistic difficulty is raised again in the \emph{Theaetetus}. Again the issue is how to coherently describe something in the process of continual change in every respect. Again Plato emphasizes that this process continues even as we speak in attempting to describe it (\emph{Cratylus} 439d10, \emph{Theaetetus} 182d7). There are, however, differences. The ontological problem is not raised in the \emph{Theaetetus}. While, in that dialogue, Eleatic monism is opposed to Ionian Heracliteanism, no Eleatic assumptions about the nature of Being are directly leveraged against the doctrine of total flux. The epistemological concern of the \emph{Cratylus}, that something continually changing in every respect could not be the object of knowledge, is not explicitly raised in the \emph{Theaetetus}. This is reasonably since, in that dialogue, what counts as knowledge is precisely what is at issue. However, one may wonder whether this specific epistemological concern, while not made explicit in the \emph{Theaetetus}, is, nonetheless, implicit in the overall dialectical structure of that dialogue. Notoriously, \emph{Parmenides} raises six \emph{aporiai} concerning the Forms, and the \emph{Theaetetus} makes no mention of the Forms. Perhaps the \emph{Theaetetus} may be read as beginning with the assumption that there are no Forms and drawing out the conclusion that no adequate definition of knowledge is to be had. This, at the very least, approaches Plato's Eleatic conviction that knowledge requires stable objects along with his further insistence that only the Forms have the requisite stability.

In the \emph{Theaetetus}, the doctrine of total flux is expressed as everything being in motion. Socrates begins by clarifying that motion, here, cannot mean narrowly spatial motion, locomotion, but must mean change of any kind including growth and alteration (181d). Socrates goes on to emphasize this by coining a new term \emph{poiotēs}, literally what-sort-ness, for the quality that is changed in alteration (182a). (The English word ``quality'' derives from Cicero's Latin translation of Plato's Greek: ``Qualitates igitur appellavi, quas Graeci {\sbl ποιότητας} appellant, quod ipsum apud Graecos non est vulgi verbum, sed philosophorum; atque id in multis. Dialecticorum vero verba nulla sunt publica, suis, utuntur. Et id quidem commune omnium fere est artium; aut enim nova sunt rerum novarum facienda nomina aut ex aliis transferenda'' \emph{Academicae Quaestiones} 1.25) If everything is in flux, then the white that we see is passing over into another color even as we see it. And this raises the difficulty of how we can call it ``white''. For even as we utter that description the color is changing (182d). The difficulty generalizes. If everything is in flux, so are perceptions such as seeing and hearing. And if language requires a stable object, then nothing may be described as seeing or not seeing, or perceiving or not perceiving more generally (182e). Like in the \emph{Cratylus}, though this is not made explicit, the difficulty seems general, applying equally to naming and predication.

The \emph{Cratylus} and the \emph{Theaetetus} reject the doctrine of total flux as incoherent. But Plato does not restrict Being to what is invariable. For example, in the \emph{Philebus} 27b8, Plato extends the notion of Being to what comes to be. So it would seem that while Plato rejects the doctrine of total flux as incoherent, he is amenable to a world of partial flux (on Plato's persistent attachment to flux see Aristotle \emph{Metaphysics} A 6). However, the \emph{Cratylus} and the \emph{Theaetetus}, while clear about the incoherence of the doctrine of total flux, remain silent about any successor notion. The new metaphysical scheme of the \emph{Timaeus}, involving, the Paradigm, its image, and the Receptacle in which that image appears, is meant, in part, to provide the resources for a coherent successor notion of partial flux.

Another difference concerns the scope of the linguistic difficulty. In the \emph{Timaeus} the cycle of elemental transformations is meant to raise a difficulty about naming the primary bodies, fire, air, water, and earth. No explicit difficulty about predication is raised. This seems \emph{prima facie} odd since the relevant linguistic difficulty seems perfectly general. That is to say, if constant change suffices to make naming impossible parallel reasoning would apply equally as well to the case of predication.

A further difference also concerns the scope of the linguistic difficulty though along another dimension. Unlike in the \emph{Cratylus} and the \emph{Theaetetus}, Timaeus argues not that ``this'' lacks deictic reference, only that it cannot refer to any of the primary bodies. If we say ``This is fiery'', ``this'' may successfully refer to the place in which that primary body appears. Images of the Forms may come and go in the Receptacle, but the spatial matrix of the Receptacle remains invariant, and so it, or at least parts of it, possess the requisite stability to be the object of deictic reference.

% section an_emph_aporia (end)

% Chapter cosmogeny (end) 