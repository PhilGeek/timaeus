%!TEX root = /Users/markelikalderon/Documents/Git/timaeus/timaeus.tex
\chapter{Cosmogeny} % (fold)
\label{cha:cosmogeny}

\section{The Cosmic Significance of Sense and Sensibilia} % (fold)
\label{sec:the_cosmic_significance_of_sensibilia}

Given that the central concern of the present essay is with sense and sensibilia in the \emph{Timaeus}, why devote an entire chapter on the generation of the body of the cosmos? Of what interest is cosmogeny to a philosopher of perception? 

It is hard not to suspect that the anxiety that prompts this question is distinctly modern. Modern cosmology, or those aspects of it that correspond to cosmogeny such as the Big Bang theory, have nothing to teach us about sense and sensibilia. But nothing much follows about ancient cosmogeny in general, and Timaeus' cosmogeny in particular. As we shall see, the pre-cosmic chaos is sensible, and Timaeus derives the elemental composition of the body of the cosmos from the fact that it is visible and tangible. Modern thought about sense and sensibilia is animated by the worry that the scientific image of nature is potentially in conflict with the manifest image of nature that experience affords us. Even those that feel that the conflict is not, in the end, genuine, or may at least be partially overcome, theorize about perception and its objects with an eye to this conflict. One lesson of the present chapter is that, for Timaeus at least, there is no conflict between the scientific and manifest images of nature. This is all the more significant since, though the terminology may be Sellarsian, the roots of the conflict are Parmenidean. So it is not as if Timaeus' speech is delivered in naive ignorance of the potential for such conflict, an ignorance only overcome by scientific modernity. In the face of the Parmenidean explosion, Timaeus denies that the manifest image of nature is upended by its intelligible underpinnings. 

The lack of conflict between the scientific and manifest images of nature concern the nature of sensibilia. Timaeus' cosmogeny has important lessons for the nature of sense as well. Perception is essentially environmental. Perception takes as its object aspects of the environment that circumscribes the perceiver. Without an environment, there is no perception, and no need for sensory apparatus. This important lesson figures in Timaeus' argument that the body of the cosmos is spherical. So not only is cosmogeny relevant to the nature of sensibilia, it is relevant to the nature of sense as well.

Timaeus narrative of the creation of the world is non-linear. He does not, as it were, begin at the beginning and proceed inexorably to the end. Thus, for example, he describes the generation of the body of the cosmos before describing the generation of the World-Soul, even though the World-Soul is prior in dignity and birth. Moreover, Timaeus only describes the pre-cosmic chaos after his cosmogeny and psychogeny. This is no mere happenstance. Timaeus has good reason for the temporal disruption of his narrative. Nor does Timaeus disrupt the temporal order for dramatic reasons the way that Kubrick does in \emph{The Killing}, a device that Tarantino subsequently made much of. Timaeus' reasons are methodological. The first part of his speech concerns the works of Reason, the second concern the works of Necessity, and the third concern the interaction of the Reason and Necessity. The temporal order of the creation narrative does not respect these divisions, and so Timaeus narrates them accordingly.

The present chapter shall not follow Timaeus in this. We shall begin with the pre-cosmic chaos and proceed to the formation of the body of the cosmos. Part of the reason for this departure from the Timaean narrative is to confront a puzzle about naming that informs much of Timaeus' speech. Even a casual reader of the \emph{Timaeus} will notice a singular preoccupation with naming. The \emph{aporia} provides the reason for Timaeus' preoccupation, and it is important to carefully attend to this.

% section the_cosmic_significance_of_sensibilia (end)



% Chapter cosmogeny (end) 