%!TEX root = /Users/markelikalderon/Documents/Git/timaeus/timaeus.tex
\chapter{Cosmogeny} % (fold)
\label{cha:cosmogeny}

\section{The Cosmic Significance of Sense and Sensibilia} % (fold)
\label{sec:the_cosmic_significance_of_sensibilia}

Given that the central concern of the present essay is with sense and sensibilia in the \emph{Timaeus}, why devote an entire chapter on the generation of the body of the cosmos? Of what interest is cosmogeny to a philosopher of perception? 

It is hard not to suspect that the anxiety that prompts this question is distinctly modern. Modern cosmology, or those aspects of it that correspond to cosmogeny such as the Big Bang theory, have nothing to teach us about sense and sensibilia. But nothing much follows about ancient cosmogeny in general, and Timaeus' cosmogeny in particular. As we shall see, the pre-cosmic chaos is sensible, and Timaeus derives the elemental composition of the body of the cosmos from the fact that it is visible and tangible. Modern thought about sense and sensibilia is animated by the worry that the scientific image of nature is potentially in conflict with the manifest image of nature that experience affords us. Even those that feel that the conflict is not, in the end, genuine, or may at least be partially overcome, theorize about perception and its objects with an eye to this conflict. One lesson of the present chapter is that, for Timaeus at least, there is no conflict between the scientific and manifest images of nature. This is all the more significant since, though the terminology may be Sellarsian, the roots of the conflict are Parmenidean. So it is not as if Timaeus' speech is delivered in naive ignorance of the potential for such conflict, an ignorance only overcome by scientific modernity. In the face of the Parmenidean explosion, Timaeus denies that the manifest image of nature is upended by its intelligible underpinnings. 

The lack of conflict between the scientific and manifest images of nature concern the nature of sensibilia. Timaeus' cosmogeny has important lessons for the nature of sense as well. Perception is essentially environmental. Perception takes as its object aspects of the environment that circumscribes the perceiver. Without an environment, there is no perception, and no need for sensory apparatus. This important lesson figures in Timaeus' argument that the body of the cosmos is spherical. So not only is cosmogeny relevant to the nature of sensibilia, it is relevant to the nature of sense as well.

Timaeus narrative of the creation of the world is non-linear. He does not, as it were, begin at the beginning and proceed inexorably to the end. Thus, for example, he describes the generation of the body of the cosmos before describing the generation of the World-Soul, even though the World-Soul is prior in dignity and birth. Moreover, Timaeus only describes the pre-cosmic chaos after his cosmogeny and psychogeny. This is no mere happenstance. Timaeus has good reason for the temporal disruption of his narrative. Nor does Timaeus disrupt the temporal order for dramatic reasons the way that Kubrick does in \emph{The Killing}, a device that Tarantino subsequently made much of. Timaeus' reasons are methodological. The first part of his speech concerns the works of Reason, the second concern the works of Necessity, and the third concern the interaction of the Reason and Necessity. The temporal order of the creation narrative does not respect these divisions, and so Timaeus narrates them accordingly.

% The present chapter shall not follow Timaeus in this. We shall begin with the pre-cosmic chaos and proceed to the formation of the body of the cosmos. Part of the reason for this departure from the Timaean narrative is to confront a puzzle about naming that informs much of Timaeus' speech. Even a casual reader of the \emph{Timaeus} will notice a singular preoccupation with naming. The \emph{aporia} provides the reason for Timaeus' preoccupation, and it is important to carefully attend to this.

% section the_cosmic_significance_of_sensibilia (end)

\section{The Elemental Composition of the Corporeal} % (fold)
\label{sec:the_elemental_composition_of_the_corporeal}

Empedocles took it for granted that the corporeal was composed of four ``roots'' (\emph{rhi\-zō\-ma\-ta}). Initially presented in divinized form (DK 31B6)---Zeus, Hera, Aidoneus (Hades), and Nestis---there is some controversy as to the assignment of roots to deities, apart from Nestis who is explicitly associated with water \citep[165--6]{Wright:1981zr}. \citet[165]{Wright:1981zr} endorses the Theophrastean interpretation on which Zeus is fire, Hera is air, Aidon\-eus is earth, and Nestis, of course, is water. Timaeus will deny that the Empedoclean ``roots'' are elements (\emph{stoicheia}) since they are themselves polyhedra that are further composed of elementary triangles (48b8). It is these triangles that are, in Timaeus' system, the \emph{stoicheia} or \emph{elementa}. Without loosing sight of this important Timaean claim, I will sometimes refer to the Empedoclean ``roots'' as elements. That fire, air, earth, and water are not, strictly speaking, elements is not the only way in which Timaeus departs from Empedocles. Timaeus does not simply take it for granted that these four are the roots of all things. Rather, Timaeus derives the elemental composition of the corporeal from its sensible aspect.

The derivation of the elemental composition of the cosmos is complex. It has a number of distinct steps each of which requires comment. However it is useful to begin with a schematic representation of that derivation before a detailed discussion of it:
\begin{enumerate}
	\item That which comes to be is corporeal and so is visible and tangible
	\item Since the cosmos is visible, it contains fire
	\item Since the cosmos is tangible, it is solid, and since it is solid, it contains earth
	\item Two things may not be joined without a third
	\item There must be an intermediary bond that joins them together
	\item The fairest bond that the most perfectly unites both itself and that which it joins is proportion
	\item If the cosmos were planar, then one middle term would suffice to bind the two
	\item But the cosmos is solid, and so two middle terms are required to bind the two
	\item The cosmos contains fire, air, water, and earth, where air is to water as fire is to air, and water is to earth as air is to water
\end{enumerate}
The derivation consists in two parts or three, depending on how you look at it. The derivation has two parts in the sense that Timaeus first derives fire and earth (1--3), and then goes on to derive the two other ``roots'', air and water (4--9). The derivation of air and water in the second part is advanced on grounds distinct from those from which fire and earth were themselves derived. The cosmos must contain fire and earth in order for it to be visible and tangible. But air and water are not derived from the very possibility of a form of perception. Rather they are derived from the need for proportionate intermediaries in the composition of the corporeal, which is by nature solid. The derivation consists in three parts in the sense that there is a methodological digression (4--6) on the nature of proportional bonds that prefaces the derivation of air and water that may be marked (7--9), thus splitting the second part of the previous scheme in two.

First, that which comes to be is corporeal and so is visible and tangible. Timaeus, here, takes for granted what he earlier argued for, that the cosmos has come to be. Recall, in the \emph{proemium}, Timaeus argued that the cosmos has come to be (28b7--c1):
\begin{enumerate}
	\item The cosmos is visible and tangible
	\item Therefore, the cosmos has a body
	\item Since the cosmos is sensible, it is apprehended by perception and opinion
	\item What is apprehended by perception and opinion has come to be
	\item Therefore, the cosmos has come to be.
\end{enumerate}
The initial claim of Timaeus' derivation of the elemental composition of the cosmos seems to reverse this line of reasoning. Specifically, Timaeus begins with the idea that the cosmos has come to be and concludes first that it is corporeal and then that it is visible and tangible.

Another thing to observe about the initial claim of the derivation is that the cosmos, that which has come to be, is not merely claimed to be sensible but more specifically visible and tangible. Why begin with these two species rather than the genus? And why these two species, rather than the tastable, smellable, or the audible?  Timaeus may have strategic and non-strategic reasons for this specification. The strategic reason is that from this specification, he may immediately conclude that the cosmos is composed of fire and earth, thus beginning the derivation. Notice that while this answers are first question, it does not answer our second. There may, however, be non-strategic reasons as well. While it may be plausible to claim that we can see and feel the cosmos, it is perhaps less plausible to claim that we can taste it, say, even if the cosmos contains parts that we can taste (Proclus, \emph{In Timaeum} 3.1.6 25--6). Nor is the cosmos as a whole smellable and audible, even if it contains parts that are smellable and audible. The cosmos as a whole may be visible and tangible, but the cosmos as a whole is not tastable, smellable, or audible. While the strategic reason addresses our first question, the non-strategic reason addresses our second.

There is another aspect of this specification that is worth mentioning. Proclus suggests that the visible and the tangible are the extreme terms of the sensible (\emph{In Timaeum} 3.1.6--7, see also Calcidius' translation of 31c). So understood, the visible and the tangible are opposed with the other sensibles arrayed as intermediaries. This coheres well with the order in which Timaeus discusses the special sensibles when discussing the affections of the body that are liable to give rise to perception and sensation (61d--68e). Timaeus begins with the tangible and ends with the visible, with the tastable, the smellable, and the audible arrayed as intermediaries. Though Proclus does not mention it, it also coheres well with the natural distribution of the elements in the cosmos. Thanks to the winnowing motion of the Receptacle, like tends to like with the result that while fire naturally tends toward the circumference of the cosmos, earth naturally tends toward the center. As a result, the elements end up being arrayed in terms of their sensible aspect. And according to Timeaus, circumference and center are naturally opposed. And so the fire that naturally tends toward the circumference is opposed to the earth that naturally tends toward the center. On the Proclean interpretation, the specification involves no loss of generality. One may speak of the sensible in general in terms of its opposed extremes. Thus one may speak of temperature in general by speaking of hot and cold since the hot and the cold are the opposed extremes of temperature.

Proclus' professed reason for this claim is non-Timaean, however. Indeed, it is Peripatetic. Proclus claims that the visible is opposed to the tangible since while the former requires a medium, the latter does not (\emph{In Timaeum} 3.1.6 9--11). While Timaeus will claim that the visual body composed of the fire emanating from within the eye and the daylight without mediates the action of chromatic fire, the thought that vision requires a medium is really only made explicit by Aristotle, even it may be found, in embryonic form, in Timaeus' speech. I have characterized Proclus' reason for the opposition between the visible and the tangible as Peripatetic. One might reasonably object that, for Aristotle, the flesh is the medium of touch, and so that the tangible involves a medium just as the visible does. The observation upon which this objection rests is correct, but there is an important shift in vocabulary, if not doctrine, between books 2 and 3 of \emph{De anima}. While in book 2 Aristotle emphasizes that the flesh is the medium by which we feel the tangible, in the opening of book 3, he seems to claim what Proclus' is claiming, that while vision requires a medium, touch does not. The claims of book 3 may be reconciled with the claims of book 2, if we understand the denial that the tangible involves a medium as the denial that it involves an external medium, the flesh being, of course, internal to the percipient (\citealt[chapter 2.1.3]{Kalderon:2015fr}). Just because an argument is bad does not mean that its conclusion is not true. Similarly, just because Proclus' reason for claiming that the visible and the tangible are opposed extremes of the sensible are not Timaeus', does not mean that Timaeus does not accept that the visible and the tangible are the opposed extremes of the sensible. At the very least, I think we should be open to the Proclean suggestion here. As we shall see, the Proclean suggestion can help us understand the general claim about bounds that two things cannot be joined without a third

The initial claim is also the basis of an interesting contrast with Descartes. While Timaeus agrees with Descartes that the corporeal is extended, the corporeal is not identified with extension. Indeed, the defining characteristic of the corporeal seems rather that it is sensible. That the corporeal is sensible, figures not only in the derivation of the elemental composition of the cosmos, but it also figures in the argument, from the \emph{proemium}, that the cosmos has come to be.

Second, from the fact that the cosmos is visible, Timaeus concludes that it must contain fire. As we shall see, vision arises from the interaction of three fires of two kinds (chapter~\ref{sec:the_end_of_sight}). One kind of fire is a mild light that does not burn (58c5–d1). Daylight is this kind of fire (45b4–6), as is presumably the fire that emanates from within the eye and the compound that results from these. Timaeus conceives of colors, the proper objects of vision, as a power of bodies to emit a different kind of fire, a flame (\emph{phlox}, 67c–68d). It is because vision is the result of the interaction of these three fires of two kinds, that Timaeus concludes that the cosmos contains fire from its visibility. Timaeus is not restricting the visible to the fiery. We can see things other than fire, such as earth, air, and water. Rather, Timaeus is claiming that if anything at all is visible, then there must be fire that illuminates the cosmos (Proclus, \emph{In Timaeum} 3.1.7.32--9.8). And if there is, then the cosmos contains fire.

Against this Aristotle will object that what visibility requires is light, and light is not fire. While fire is hot and dry, light is not. It is hard to determine to what extent this is merely a verbal dispute. According to Timaeus, daylight is a mild light that does not burn. Aristotle does not dispute this so much as he disputes Timaeus taking this to be a kind of fire. But even so, the mild light of day that does not burn is, according to Aristotle, contingent upon the presence of a fiery substance. Part of what is at issue is Aristotle's insistence that light is a qualitative state (\emph{hexis}) of a transparent medium actualized by the presence of the fiery substance. Fire, by contrast, is not a state but a body, if a rarefied one.

Third, from the fact that the cosmos is tangible, Timaeus first concludes that it is solid, and then concludes that it must be composed of earth. While the derivation of fire from the visibility of the cosmos is immediate, the derivation of earth from the tangibility of the cosmos is not. What immediately follows from the tangibility of the cosmos is that it is solid. And from its solidity it is further inferred that it must be composed of earth. Let us consider these two inferences in turn. 

Why does solidity of a body follow from its tangibility? Perhaps the thought is that only that which is solid may withstand touch, understood as a mode of perception, since if the object of touch does not withstand being touched it is not so much as touched as it is crushed, say. As crushing a body destroys at least some of its tangible quaities, if the aim in touching it was to discern these, crushing fails to meet this perceptual end. This needs to be understood in a manner that is consistent with what Timaeus has to say about the hard and the soft. The soft is that which yields to touch whereas the hard is that which touch yields to. A surface may be soft and pliant and yield to the percipient's touch all the while withstanding the touch to which it yields. So the requirement on touch is not that the tangible is resistant to touch in the way hard bodies are, for then soft bodies would be intangible. It is rather that the tangible object should withstand its touching.

Why does being composed of earth follow from solidity? Later on, Timaeus will argue that individual bodies of earth are cubic and thus have solid bases. It is the solidity of their bases that determines the solidity of aggregates of earth particles. However, according to Timaeus, this is a sufficient and not a necessary condition on a body's solidity. Bodies may be solid by being composed, in part, of earth, but they may also be solid by by being densely compacted. When the fluid form of water freezes, it becomes solid, not because it has been tainted with earth, but because escaping fire particles have caused the octahedra to fill in the space they have vacated resulting in a densely compacted and so solid body. If being composed of earth is, by Timaeus' lights, a sufficient but not necessary condition for a body to be solid, then there how is this second inference valid?

There is an additional problem here. While we may grant to Timaeus that it is at the very least odd to describe ourselves as touching air or fire, cannot we not, in a straightforward sense, touch water? If punting, for example, may we not idly touch the passing water and feel its coolness and the force with which it streams past? And if we can, then water is tangible. There is no reasons to suppose that this is only so because the water is impure.  And if water, pure water, is tangible, even in its liquid state, then not only is being composed of earth not necessary, neither is being solid, since water in its liquid state is not solid. If so, this threatens the validity of the first inference as well, since solidity would not be a necessary condition on tangibility.

Having derived fire and earth from the visibility and tangibility of the cosmos, Timaeus goes on derive the remaining two Empedoclean ``roots'', air and water. Timaeus does not do so, however, on the basis of their sensible aspects. The presence of air and water in the composition of the body of the cosmos is not derived on the basis of their being a condition on the possibility of a form of perception. Rather they are derived from the way in which fire and earth must be bound together in the composition of the body of the cosmos. Timaeus will first elucidate the nature of this bond before proceeding to the derivation of air and water.

The first claim that Timaeus makes about bonds is that no two things may be joined together without a third. There is a puzzle about this claim that suggests that a suppressed condition may be in play. If that is right, then, as we shall see, the Proclean interpretation that fire and earth, at least in their sensible aspects, are opposed extremes, may do some explanatory work.

Let us consider the puzzle first. The visual body by which we see is composite. It consists, on the one hand, on the fire emanating from within the eye and, on the other hand, the daylight that it encounters. These two fires are akin. Each is a mild light that does not burn. Since they are akin, they grow together or coalesce in the direction that the percipient is looking. What explains this is like's affinity for like. The fire emanating from within the eye and the daylight it encounters compose the visual body without the assistance of any third thing. So it seems that this initial principle of binding is false, at least by Timaeus' lights.

In the present instance what is being bound is fire and earth in the composition of the body of the cosmos. The fire emanating from within the eye and daylight may be alike in being the same kind of fire, but fire and earth are unlike. They are not the same kind of body. Not only are they unlike, they are naturally located in the opposed extremes of the cosmos. Timaeus holds that the cosmos is spherical and that the center of the cosmos is opposed to the points that lie, in extremity, on its circumference. Fire naturally accumulates in the circumference of the cosmos, while earth naturally accumulates in the center. Owing to the winnowing movement of the Receptacle, the Empedoclean ``roots'' are naturally distributed like by like with fire and earth being distributed at the extreme ends of the cosmos. Perhaps this is so since fire and earth are opposed extremes at least with respect to their sensible aspects. Notice that this thought requires the Proclean suggestion that the visible and the tangible are the opposed extremes of the sensible. If we accept all this, then our puzzle may be resolved by claiming that Timaeus' initial claim about bonds is tacitly restricted to opposed extremes: No two things that are opposed extremes may be joined together without a third. If fire and earth are opposed extremes, at least in their sensible aspects, then the principle of like's affinity for like would not suffice for their compounding, and it would make sense that a third thing is somehow needed to compound them and so mediate these extremes.

The second claim about bonds identifies the third thing as the intermediary bond that joins the two opposed extremes. So a bond is a third thing distinct from the two opposed extremes whose function is to mediate between opposed extremes so as to join them. While informative about the nature of bonds, as Timaeus conceives of them, these first two claims taken together still leave open important questions. Most importantly one wants to know what it is to mediate between opposed extremes so as to join these together in a unified compound? 

Timaeus' third claim about bonds is meant to answer that question. The fairest bond that most perfectly unites that which it joins is proportion. Putting all three claims together we get the following. An intermediary third thing between the two opposed extremes may mediate between them so as to join them given the proportions between them. As we shall see, in the present instance, what is required are two intermediaries, air and water, that stand in relevant proportions to themselves and to fire and earth so as to perfectly unite these bodies in the fairest bond.

% section the_elemental_composition_of_the_corporeal (end)


% Chapter cosmogeny (end) 