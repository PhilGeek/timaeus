%!TEX root = /Users/markelikalderon/Documents/Git/timaeus/timaeus.tex
\chapter{Cosmogeny} % (fold)
\label{cha:cosmogeny}

\section{The Cosmic Significance of Sense and Sensibilia} % (fold)
\label{sec:the_cosmic_significance_of_sensibilia}

Given that the central concern of the present essay is with sense and sensibilia in the \emph{Timaeus}, why devote an entire chapter on the generation of the body of the cosmos? Of what interest is cosmogeny to a philosopher of perception? 

It is hard not to suspect that the anxiety that prompts this question is distinctly modern. Modern cosmology, or those aspects of it that correspond to cosmogeny such as the Big Bang theory, have nothing to teach us about sense and sensibilia. But nothing much follows about ancient cosmogeny in general, and Timaeus' cosmogeny in particular. As we shall see, the pre-cosmic chaos is sensible, and Timaeus derives the elemental composition of the body of the cosmos from the fact that it is visible and tangible. Modern thought about sense and sensibilia is animated by the worry that the scientific image of nature is potentially in conflict with the manifest image of nature that experience affords us. Even those that feel that the conflict is not, in the end, genuine, or may at least be partially overcome, theorize about perception and its objects with an eye to this conflict. One lesson of the present chapter is that, for Timaeus at least, there is no conflict between the scientific and manifest images of nature. This is all the more significant since, though the terminology may be Sellarsian, the roots of the conflict are Eleatic (if not in Parmenides, properly understood, at least in a Sophistical take on the Parmenides, or in the work of Melissus). So it is not as if Timaeus' speech is delivered in naive ignorance of the potential for such conflict, an ignorance only overcome by scientific modernity. In the face of the Parmenidean explosion, Timaeus denies that the manifest image of nature is upended by its intelligible underpinnings. 

The lack of conflict between the scientific and manifest images of nature concern the nature of sensibilia. Timaeus' cosmogeny has important lessons for the nature of sense as well. Perception is essentially environmental. Perception takes as its object aspects of the environment that circumscribes the perceiver. Without an environment, there is no perception, and no need for sensory apparatus. This important lesson figures in Timaeus' argument that the body of the cosmos is spherical. So not only is cosmogeny relevant to the nature of sensibilia, it is relevant to the nature of sense as well.

Timaeus narrative of the creation of the world is non-linear. He does not, as it were, begin at the beginning and proceed inexorably to the end. Thus, for example, he describes the generation of the body of the cosmos before describing the generation of the World-Soul, even though the World-Soul is prior in dignity and birth. Moreover, Timaeus only describes the pre-cosmic chaos after his cosmogeny and psychogeny. This is no mere happenstance. Timaeus has good reason for the temporal disruption of his narrative. Nor does Timaeus disrupt the temporal order for dramatic reasons the way that Kubrick does in \emph{The Killing}, a device that Tarantino subsequently made much of. Timaeus' reasons are methodological. The first part of his speech concerns the works of Reason, the second concern the works of Necessity, and the third concern the interaction of the Reason and Necessity. The temporal order of the creation narrative does not respect these divisions, and so Timaeus narrates them accordingly.

In his cosmogeny, Timaeus does four things:
\begin{enumerate}[(1)]
	\item Timaeus argues that the cosmos must be unique in order to best resemble the Paradigm after which it was fashioned
	\item He argues that the cosmos is composed of four primary bodies: fire, air, water, and earth
	\item He argues that the cosmos is comprehensive, containing all that the primary bodies that there are leaving nothing corporeal external to it
	\item Finally, he argues that the cosmos must be spherical in shape
\end{enumerate}
What does the uniqueness and comprehensiveness of the cosmos have to do with philosophy of perception? While the second and fourth arguments are directly relevant to the nature of sense and sensibilia, the first and third are indirectly relevant. As we shall see, the uniqueness of the cosmos is part of Timaeus' reason for its comprehensiveness, and the comprehensiveness of the cosmos is what establishes that there is nothing corporeal external to it. So these two arguments playing a role in establishing the cosmos as environmental, and the objects of perception are environmental if it is perception at all.

As we shall see, there are important parallels between the Demiurge's generation of the body of the cosmos and the young gods' generation of the human body. But more importantly, there will be differences. Mortal human beings are situated within the finite body of the cosmos. The cosmos itself, a visible god, is situated in no other thing. In this way it resembles the self-sufficiency of the Paradigm, the Living Being. Mortal human beings live within an environment, the visible god. This will have consequences for their motion, their shape, their bodily parts, their affection, their cognitive capacities (broadly understood to include their capacity for opinion and perception), and importantly, their salvation.

% section the_cosmic_significance_of_sensibilia (end)

\section{The Uniqueness of the Cosmos} % (fold)
\label{sec:the_uniqueness_of_the_cosmos}



% section the_uniqueness_of_the_cosmos (end)

\section{The Elemental Composition of the Cosmos} % (fold)
\label{sec:the_elemental_composition_of_the_corporeal}

Empedocles took it for granted that the corporeal was composed of four ``roots'' (\emph{rhi\-zō\-ma\-ta}). Initially presented in divinized form (DK 31B6)---Zeus, Hera, Aidoneus (Hades), and Nestis---there is some controversy as to the assignment of roots to deities, apart from Nestis who is explicitly associated with water \citep[165--6]{Wright:1981zr}. \citet[165]{Wright:1981zr} endorses the Theophrastean interpretation on which Zeus is fire, Hera is air, Aidon\-eus is earth, and Nestis, of course, is water. Timaeus will deny that the Empedoclean ``roots'' are elements (\emph{stoicheia}) since they are themselves polyhedra that are further composed of elementary triangles (48b8). It is these triangles that are, in Timaeus' system, the \emph{stoicheia} or \emph{elementa}. Without loosing sight of this important Timaean claim, I will sometimes refer to the Empedoclean ``roots'' as elements. That fire, air, earth, and water are not, strictly speaking, elements is not the only way in which Timaeus departs from Empedocles. Timaeus does not simply take it for granted that these four are the roots of all things. Rather, Timaeus derives the elemental composition of the corporeal from its sensible aspect.

The derivation of the elemental composition of the cosmos is complex. It has a number of distinct steps each of which requires comment. However it is useful to begin with a schematic representation of that derivation before a detailed discussion of it:
\begin{enumerate}[(1)]
	\item That which comes to be is corporeal and so is visible and tangible (31b4--5)
	\item Since the cosmos is visible, it contains fire (31b5--6)
	\item Since the cosmos is tangible, it is solid, and since it is solid, it contains earth (31b6--7)
	\item Two things may not be joined without a third (31bd--31c)
	\item And so there must be an intermediary bond that joins them together (31c)
	\item The fairest bond that the most perfectly unites both itself and that which it joins is proportion (31c1--3)
	\item Whenever there is a middle among three things, such that as the first is to the middle, so the middle is to the last, and conversely, as the last is to the middle, so the middle is to the first, then the middle becomes both first and last, and the last and first also appear in the middle (31c4--32a8)
	\item If the cosmos were planar, then one middle term would suffice to bind the two (32a8--32b1)
	\item But the cosmos is solid, and so two middle terms are required to bind the two (32b1--4)
	\item The cosmos thus contains fire, air, water, and earth, where air is to water as fire is to air, and water is to earth as air is to water (32b4--9)
\end{enumerate}
The derivation consists in two parts or three, depending on how you look at it. The derivation has two parts in the sense that Timaeus first derives fire and earth (1--3), and then goes on to derive the two other ``roots'', air and water (4--10). The derivation of air and water in the second part is advanced on grounds distinct from those from which fire and earth were themselves derived. The cosmos must contain fire and earth in order for it to be visible and tangible. But air and water are not derived as conditions on the possibility of forms of perception. Rather, they are derived from the need for proportionate intermediaries in the composition of the corporeal, which is by nature solid. The derivation consists in three parts in the sense that there is a methodological digression (4--7) on the nature of proportional bonds that prefaces the derivation of air and water that may be marked (8--10), thus splitting the second part of the previous scheme in two.

We shall discuss each of these enumerated steps of Timaeus' argument in turn, but let us begin with some general observations. First, the cosmos that comes to be is corporeal. Timaeus argues that the corporeal is composed of the four Empedoclean ``roots''. Notice that this is a preliminary analysis of the composition of the corporeal. The ``roots'' turn out to be no roots at all but are merely regular polyhedra that are themselves composed of, and so rooted in, elementary triangles. Only the elementary triangles are properly regarded as \emph{stoicheia} or \emph{elementa}. Second, Timaeus' argument has universal scope. It is meant to apply to all bodies. It is not restricted, for example, to sublunary bodies as in Aristotle's cosmology. Finally, and importantly, Timaeus takes it for granted that bodies are sensible. Indeed, being sensible, as opposed to being extended, is the characteristic mark of the corporeal. The analysis of bodies as being composed of the Empedoclean ``roots'' does not undermine their status as sensible. Rather, the assumption that bodies are sensible is a premise in the derivation of their being composed of these roots. Nor does the further analysis of the four primary bodies into regular polyhedra composed of elementary triangles undermine their sensible character. Indeed, the behavior of these polyhedra explain how a sentient animate body is affected so as to give rise to perception and sensation (61d--68e). 

(1) First, that which comes to be is corporeal and so is visible and tangible. Timaeus, here, takes for granted what he earlier argued for, that the cosmos has come to be. Recall, in the \emph{proemium}, Timaeus argued that the cosmos has come to be (28b7--c1):
\begin{enumerate}[(1)]
	\item The cosmos is visible and tangible
	\item Therefore, the cosmos has a body
	\item Since the cosmos is sensible, it is apprehended by perception and opinion
	\item What is apprehended by perception and opinion has come to be
	\item Therefore, the cosmos has come to be
\end{enumerate}
The initial claim of Timaeus' derivation of the elemental composition of the cosmos seems to reverse this line of reasoning. Specifically, Timaeus begins with the idea that the cosmos has come to be and concludes first that it is corporeal and then that it is visible and tangible. Notice that the argument from the \emph{proemium}, like the present argument, involves the claim that if the cosmos is corporeal then it is visible and tangible.

Another thing to observe about the initial claim of the derivation is that the cosmos, that which has come to be, is not merely claimed to be sensible but more specifically visible and tangible. Why begin with these two species, the visible and the tangible, rather than the genus, the sensible? And why these two species, rather than the tastable, smellable, or the audible?  Timaeus may have strategic and non-strategic reasons for this specification. The strategic reason is that from this specification, he may immediately conclude that the cosmos is composed of fire and earth, thus beginning the derivation. These conclusions would not have been warranted merely from the claim that the cosmos was sensible. Notice that while this answers are first question---Why begin with species rather than the genus?---it does not answer our second---Why begin with these two species as opposed to some other species? 

There may, however, be non-strategic reasons as well. While it may be plausible to claim that we can see and feel the cosmos, it is perhaps less plausible to claim that we can taste it, say, even if the cosmos contains parts that we can taste. Nor is the cosmos as a whole smellable and audible, even if it contains parts that are smellable and audible. The cosmos as a whole may be visible and tangible, but the cosmos as a whole is not tastable, smellable, or audible (Proclus, \emph{In Timaeum} 2 25--6, \citealt{Diehl:1903re}). While the strategic reason addresses our first question, the non-strategic reason addresses our second. 

\citet[93]{Taylor:1928qb} offers another reason. By means of sight and touch we may discern the shape and size of bodies. Taylor claims, by contrast, that we cannot taste, smell, or hear the shape and size of bodies. (This last may be questioned, however. Can we not hear the relative size of a room by its characteristic resonance?) Not all bodies, have a taste, smell, or sound, but all bodies have shape and size. So Taylor's idea is that Timaeus begins with sight and touch since it is by means of these senses alone that universal sensible features of the corporeal may be perceived. Perhaps Taylor rightly understands Timaeus here, but the absence of direct textual evidence, and the way Taylor's idea approximates a primary quality conception of the corporeal makes me apprehensive of the potential for anachronism here.

There is another aspect of this specification that is worth mentioning. Proclus suggests that the visible and the tangible are the extreme terms of the sensible (\emph{In Timaeum} 2 6--7, \citealt{Diehl:1903re}, see also Calcidius' translation of 31c). So understood, the visible and the tangible are opposed with the other sensibles arrayed as intermediaries. This coheres well with the order in which Timaeus discusses the special sensibles when discussing the affections of the body that are liable to give rise to perception and sensation (61d--68e). Timaeus begins with the tangible and ends with the visible, with the tastable, the smellable, and the audible arrayed as intermediaries. Though Proclus does not mention it, it also coheres well with the natural distribution of the elements in the cosmos. Thanks to the winnowing motion of the Receptacle, like tends to like, with the result that while fire naturally tends toward the circumference of the cosmos, earth naturally tends toward the center. As a result, the elements end up being arrayed in terms of their sensible aspects. And according to Timeaus, circumference and center are naturally opposed (62c8–d4). And so the fire that naturally tends toward the circumference is opposed to the earth that naturally tends toward the center. On the Proclean interpretation, the specification involves no loss of generality. One may speak of the sensible in general in terms of its opposed extremes. Thus one may speak of temperature in general by speaking of the hot and the cold since the hot and the cold are the opposed extremes of temperature.

Proclus' professed reason for this claim is non-Timaean, however. Indeed, it is Peripatetic. Proclus claims that the visible is opposed to the tangible since while the former requires a medium, the latter does not (\emph{In Timaeum} 2 9--11, \citealt{Diehl:1903re}). While Timaeus will claim that the visual body (\emph{opsis}) composed of the fire emanating from within the eye and the daylight that it meets without mediates the action of chromatic fire (45b--46c, 67c--68d), the thought that vision requires a medium is really only made explicit by Aristotle, even it may be found, in embryonic form, in Timaeus' speech. I have characterized Proclus' reason for the opposition between the visible and the tangible as Peripatetic. One might reasonably object that, for Aristotle, the flesh is the medium of touch, and so that the tangible involves a medium just as much as the visible does. The observation upon which this objection rests is correct, but there is an important shift in vocabulary, if not doctrine, between books two and three of \emph{De anima}. While in \emph{De anima} 2.11 Aristotle emphasizes that the flesh is the medium by which we feel the tangible, in \emph{De anima} 3.1, he seems to claim what Proclus' is claiming, that while vision requires a medium, touch does not. The claims of book three may be reconciled with the claims of book two, if we understand the denial that the tangible involves a medium as the denial that it involves an external medium, the flesh being, of course, internal to the percipient (\citealt[chapter 2.1.3]{Kalderon:2015fr}). 

Just because an argument is bad does not mean that its conclusion is false. Similarly, just because Proclus' reason for claiming that the visible and the tangible are the opposed extremes of the sensible is not Timaeus', does not mean that Timaeus does not accept that the visible and the tangible are the opposed extremes of the sensible. Timaeus, after all, might maintain this claim for some other reason. At the very least, I think we should be open to the Proclean suggestion here. As we shall see, the Proclean suggestion can help us understand the general claim about bonds that two things cannot be joined without a third.

The initial claim is also the basis of an interesting contrast with Descartes. While Timaeus agrees with Descartes that the corporeal is extended, the corporeal is not identified with modes of extension. Indeed, the characteristic mark of the corporeal is not that it is extended but that it is sensible. (Though this contrast is lessoned somewhat if Taylor is right and Timaeus cites the visible and the tangible since only these senses reveal shape and size common to all bodies.) That the corporeal is sensible, figures not only in the derivation of the elemental composition of the cosmos, but it also figures in the argument, from the \emph{proemium}, that the cosmos has come to be.

(2) Second, from the fact that the cosmos is visible, Timaeus concludes that it must contain fire. As we shall see, vision arises from the interaction of three fires of two kinds (chapter~\ref{sec:the_end_of_sight}). One kind of fire is a mild light that does not burn (58c5–d1). Daylight is this kind of fire (45b4–6), as is, presumably, the fire that emanates from within the eye and the compound that results from these. Timaeus conceives of colors, the proper objects of vision, as a power of bodies to emit a different kind of fire, a flame (\emph{phlox}, 67c–68d). It is because vision is the result of the interaction of these three fires of two kinds, that Timaeus concludes that the cosmos contains fire from its visibility. Timaeus is not restricting the visible to the fiery. We can see things other than fire, such as earth, air, and water. Rather, Timaeus is claiming that if anything at all is visible, then there must be fire that illuminates the cosmos (Proclus, \emph{In Timaeum} 2 7.32--9.8, \citealt{Diehl:1903re}). And if there is, then the cosmos contains fire. Fire is a condition on the very possibility of a body's visibility.

Against this Aristotle will object that what visibility requires is light, and light is not fire. While fire is hot and dry, light is not (\emph{De sensu} 2 437b16--19). It is hard to determine to what extent this is merely a verbal dispute. According to Timaeus, daylight is a mild light that does not burn. Aristotle does not dispute this so much as he disputes Timaeus taking this to be a kind of fire. But even so, the mild light of day that does not burn is, according to Aristotle, contingent upon the presence and activity of a fiery substance. Part of what is at issue is Aristotle's insistence that light is a qualitative state (\emph{hexis}) of a transparent medium actualized by the presence and activity of the fiery substance. Fire, by contrast, is not a state but a body, if a rarefied one.

(3) Third, from the fact that the cosmos is tangible, Timaeus first concludes that it is solid, and then concludes that it must be composed of earth. While the derivation of fire from the visibility of the cosmos is immediate, the derivation of earth from the tangibility of the cosmos is not. What immediately follows from the tangibility of the cosmos is that it is solid. And from its solidity it is further inferred that it must be composed of earth. Let us consider these two inferences in turn. 

Why does solidity of a body follow from its tangibility? Why is solidity a condition on the very possibility of a body's tangibility? What, exactly, does Timaeus mean by \emph{stereou} here? Does he mean a geometrical solid or does he mean something hard and resistant to touch? While the nature of geometrical solids will play a role in the derivation, a question arises about its connection with tangibility. If there may be geometrical solids that are incorporeal, these will be intangible, and solidity, so understood, would not be sufficient for tangibility. Perhaps, but it may yet remain a condition on the possibility of touch without being a sufficient condition. Or, perhaps, \emph{stereou} means hard or resistant to touch. So understood, \emph{stereou} would be analogous in sense to the Latin \emph{robustus}. While plausible as a condition on the possibility of touch, this needs to be understood in a manner that is consistent with what Timaeus has to say about the hard and the soft. The soft is that which yields to touch whereas the hard is that which is resistant to touch (62b6–c3). So the requirement on touch is not that the tangible is resistant to touch in the way hard bodies are, for then soft bodies would be intangible. How then is \emph{stereou} to be understood? Perhaps the thought is that only that which is solid may withstand touch, understood as a mode of perception, since if the object of touch does not withstand being touched it is not so much as touched as it is crushed, say. As crushing a body destroys at least some of its tangible qualities, if the aim in touching it were to discern these, crushing would fail to meet this perceptual end. A surface may be soft and pliant and yield to the percipient's touch all the while withstanding the touch to which it yields. It is not that the tangible is resistant to touch the way that hard bodies are but rather that tangible bodies should withstand being touched. Different ways of withstanding being touched correspond to different tangible characters.

Why does being composed of earth follow from solidity? Later on, Timaeus will argue that individual bodies of earth are cubic and thus have solid bases (59d4--7). It is the solidity of their bases that determines the solidity of aggregates of earth particles. However, according to Timaeus, this is a sufficient and not a necessary condition on a body's solidity. Bodies may be solid by being composed, in part, of earth, but they may also be solid by by being densely compacted (59b5). Adamant is hard, not because it contains earth, but because of its density (59b5). And when the fluid form of water freezes, it becomes solid, not because it has been tainted with earth, but because escaping fire particles have caused the octahedra to fill in the space they have vacated resulting in a densely compacted and so solid body (59d4–e5). If being composed of earth is, by Timaeus' lights, a sufficient but not necessary condition for a body to be solid, then how is this second inference valid? Notice that the problem raised here goes beyond the terms of Timaeus' preliminary analysis. The account of solidity in terms of density is only really available once the Empedoclean ``roots'' have been identified with regular polyhedra. Perhaps, then, the derivation is merely probable, adequate only to the present stage of inquiry, and thus, in this sense, an \emph{eikos logos}.

There is, however, an additional problem. While we may grant to Timaeus that it is at the very least odd to describe ourselves as touching air or fire, cannot we not, in a straightforward sense, touch water? If punting, for example, may we not idly touch the passing water and feel its coolness and the force with which it streams past? And if we can, then water is tangible. There is no reasons to suppose that this is only so because the water is impure being tainted with earth (though Londoners' afflicted with limescale may suspect otherwise). And if water untainted with earth is tangible, even in its liquid state, then not only is being composed of earth not necessary, neither is being solid, since water in its liquid state is not solid. Indeed, it is not solid in the geometrical sense or in the sense of being resistant to touch. Water withstands our touching it. I speculatively suggested that \emph{stereou} might be understood as withstanding touch. Should that reading turn out to be unsustainable, then there is a real problem here. Water would not be solid in any of the senses canvassed. If so, this threatens the validity of the first inference as well, since solidity would not be a necessary condition on tangibility. 

(4) Having derived fire and earth from the visibility and tangibility of the cosmos, Timaeus goes on derive the remaining two Empedoclean ``roots'', air and water. Timae\-us does not do so, however, on the basis of their sensible aspects. The presence of air and water in the composition of the body of the cosmos is not derived on the basis of their being conditions on the possibility of forms of perception. Rather, they are derived from the way in which fire and earth must be proportionally bound together in the composition of the body of the cosmos. Timaeus will first elucidate the nature of this bond before proceeding to the derivation of air and water.

The first claim that Timaeus makes in his digression on the nature of bonds (\emph{desmō}) is that no two things may be joined together without a third. First, notice that Timaeus has yet to say anything explicitly about bonds. The connection with bonds is only made explicit in the second claim. Second, an observation should be made about the intended generality of the first claim. Timaeus is not claiming that only a third is required. Rather, Timaeus' claim is that at least a third is required. No two things may be joined together without a third or more. After all, Timaeus goes on to argue that air and water, both, are required to join together fire and earth in a solid body (in the geometrical sense of solidity). Third, there is a puzzle about this claim that suggests that a suppressed condition may be in play. If that is right, then, as we shall see, the Proclean interpretation that fire and earth, at least in their sensible aspects, are opposed extremes, may do some explanatory work.

Let us consider the puzzle first. The visual body (\emph{opsis}) by which we see is composite. It consists, on the one hand, of the fire emanating from within the eye and, on the other hand, the daylight that it encounters without. These two fires are akin. Each is a mild light that does not burn. Since they are akin, they grow together or coalesce in the direction that the percipient is looking. What explains this is the principle like's affinity for like. The fire emanating from within the eye and the daylight that it encounters compose the visual body without the assistance of any third thing since they are alike. So it seems that this initial principle of binding is false, at least by Timaeus' lights.

In the present instance what is being bound is fire and earth in the composition of the body of the cosmos. The fire emanating from within the eye and daylight may be alike in being the same kind of fire, but fire and earth are unlike. They are not the same kind of body. Not only are they unlike, they are naturally located in the opposed extremes of the cosmos. Timaeus holds that the cosmos is spherical and that the center of the cosmos is opposed to the points that lie, in extremity, on its circumference (62c8–d4). Fire naturally accumulates in the circumference of the cosmos, while earth naturally accumulates in the center. Owing to the winnowing movement of the Receptacle, the Empedoclean ``roots'' are naturally distributed, like by like, with fire and earth being distributed at the extreme ends of the cosmos. Perhaps this is so since fire and earth are opposed extremes at least with respect to their sensible aspects. Notice that this thought requires the Proclean suggestion that the visible and the tangible are themselves the opposed extremes of the sensible. If we accept all this, then our puzzle may be resolved by claiming that Timaeus' initial claim about bonds is tacitly restricted to opposed extremes: No two things that are opposed extremes may be joined together without a third (Proclus, \emph{In Timaeum} 2 17.21--18.7, \citealt{Diehl:1903re}). If fire and earth are opposed extremes, at least in their sensible aspects, then the principle of like's affinity for like would not suffice for their compounding, and it would make sense that a third thing is somehow needed to compound them. But more than that, a third thing is required in order to overcome the opposition of the extremes and so mediate between them.

(5) The second claim about bonds identifies the third thing as the intermediary bond (\emph{desmos}) that joins the two opposed extremes. So a bond is a third (or more) thing distinct from the two opposed extremes whose function is to mediate between opposed extremes so as to join them. While informative about the nature of bonds, as Timaeus conceives of them, these first two claims, taken together, still leave open important questions. Most importantly, one wants to know what it is to mediate between opposed extremes so as to join these together in a unified compound? 

(6) Timaeus' third claim about bonds is meant to answer that question. The fairest bond that most perfectly unites that which it joins is proportion (\emph{analogia}). Putting all three claims together we get the following. An intermediary third (or more) thing between the two opposed extremes may mediate between them so as to join them given the proportions that obtain between the opposed extremes and their intermediaries. As we shall see, in the present instance, what is required are two intermediaries, air and water, that stand in relevant proportions to themselves and to fire and earth so as to perfectly unite these bodies in the fairest bond.

Timaeus' third claim requires some unpacking before it provides a full answer to the question how does a bond mediate between opposed extremes so as to join them in a unity. Specifically, what exactly Timaeus means by \emph{analogia}? Nichomachus of Gerasa defines proportion (\emph{analogia}) as follows:
\begin{quote}
	A proportion (\emph{analogia}), then, is in the proper sense, the combination of two or more ratios (\emph{logos}), but by the more general definition the combination of two or more relations (\emph{schesis}), even if they are not brought under the same ratio, but rather a difference, or something else. (Nichomachus of Gerasa, \emph{Arithmetike eisagoge} 2.24.2; \citealt[264--5]{Dooge:1926aa})
\end{quote}
By ratios, Nichomachus seems to have in mind ratios of numbers, nevertheless he is prepared to generalize the notion in speaking of relations. There are two dimensions of generalization. First, proportions may not just be geometrical, but may be arithmetical and harmonic as well \citep[264 n2]{Dooge:1926aa}. Second, in not being restricted to ratios of numbers, the generalized definition is not restricted to proportions found in progressions of numbers specifically. As we shall see, Timaeus, has in mind primarily progressions of geometrical proportions. But he must allow non-arithmetical entities to stand in geometrical proportions. The task, after all, is to establish that there are four primary bodies that compose the body of the cosmos. It is these bodies that are said to stand in geometrical proportions.

(7) Timaeus offers a reason for this third claim about bonds that marks also a transition to the next step of the derivation. Timaeus' claim is convoluted, but the basic idea is simple enough: Whenever there is a middle among three things, such that as the first is to the middle, so the middle is to the last, and conversely, as the last is to the middle, so the middle is to the first, then the middle becomes both first and last, and the last and first also appear in the middle. (``Whenever there is a middle among three things \ldots'' I am being deliberately non-specific here, for, as we shall see, there is controversy attaching to Timaeus' specific wording.) There are three distinguishable components to Timaeus' claim:
\begin{enumerate}[(1)]
	\item Whenever there is a middle among three things, such that as the first is to the middle, so the middle is to the last \ldots\ :
	\begin{quote}
		F:M :: M:L (for example, 2:4 :: 4:8)
	\end{quote}
	\item \ldots\ and conversely, as the last is to the middle, so the middle is to the first \dots\ :
	\begin{quote}
		L:M :: M:F (for example, 8:4 :: 4:2)
	\end{quote}
	\item \ldots\ then the middle becomes both first and last, and the last and first also appear in the middle:
	\begin{quote}
		M:L :: F:M (for example, 4:8 :: 2:4)
	\end{quote}
\end{enumerate}
So understood, Timaeus is describing progressions arrayed in geometrical proportions. (Proclus, due in part to lexical ambiguities that we shall come to, mistakenly takes this passage to describe, not only geometric proportions, but arithmetical and harmonic proportions as well, \emph{In Timaeum}.) The fairness of the bond is undoubtedly due to the symmetry of the proportions. And it is these like proportions that perfectly unite even opposed extremes.  

Unfortunately, the Greek is beset by syntactic and lexical ambiguities that divide interpreters \citep{Prtichard:1990aa}. Let us consider the syntactic ambiguity first. It besets the following opening phrase:
\begin{quote}
{\sbl ὁπόταν γὰρ ἀριθμῶν τριῶν εἴτε ὄγκων εἴτε δυνάμεων ὡντινωνοῦν ᾖ τὸ μέσον \ldots}
% , ὅτιπερ τὸ πρῶτον πρὸς αὐτό, τοῦτο αὐτὸ πρὸς τὸ ἔσχατον, καὶ πάλιν αὖθις, ὅτι τὸ ἔσχατον πρὸς τὸ μέσον, τὸ μέσον πρὸς τὸ πρῶτον, τότε τὸ μέσον μὲν πρῶτον καὶ ἔσχατον γιγνόμενον, τὸ δ᾽ ἔσχατον καὶ τὸ πρῶτον αὖ μέσα ἀμφότερα, πάνθ᾽ οὕτως ἐξ ἀνάγκης τὰ αὐτὰ εἶναι συμβήσεται, τὰ αὐτὰ δὲ γενόμενα ἀλλήλοις ἓν πάντα ἔσται. εἰ μὲν οὖν ἐπίπεδον μέν, βάθος δὲ μηδὲν ἔχον ἔδει γίγνεσθαι τὸ τοῦ παντὸς σῶμα, μία μεσότης ἂν ἐξήρκει τά τε μεθ᾽ αὑτῆς συνδεῖν καὶ ἑαυτήν, νῦν δὲ στερεοειδῆ γὰρ αὐτὸν προσῆκεν εἶναι, τὰ δὲ στερεὰ μία μὲν οὐδέποτε, δύο δὲ ἀεὶ μεσότητες συναρμόττουσιν}
\end{quote}
% Zeyl translates the passage as follows:
% \begin{quote}
% 	For whenever of three numbers (or bulks or powers) the middle term between any two of them is such that when the first term is to it, it is to the last, and, conversely, what the last term is to the middle, it is to the first, then, since the middle term turns out to be both first and last, and the last the fore first likewise turn out to be middle terms, they will all of necessity turn out to have the same relationship to each other, and given this, will all be unified. So if the body of the universe were to have come to be as a two-dimensional plane, a single middle term would have sufficed to bind together its conjoining terms with itself. As it was, however, the universe was to be a solid, and solids are never joined togeether by just one middle term but always by two. (\citealt[17]{Zeyl:2000cs})
% \end{quote}
There are three potential readings:
\begin{enumerate}[(1)]
	\item Whenever any three numbers (\emph{arithmōn}), whether \emph{ongkōn} or \emph{dunameōn}, the middle one (\emph{to meson}) is such that \ldots
	\item Whenever of any three numbers (\emph{arithmōn}), the middle one (\emph{to meson}) between any two that are \emph{ongkōn} or \emph{dunameōn} is such that \ldots
	\item Whenever of three numbers  (\emph{arithmōn}) or \emph{ongkōn} or \emph{dunameōn}, the middle one (\emph{to meson}) is such that \ldots
\end{enumerate}
The first reading links the genitives \emph{ongkōn} and \emph{dunameōn} with \emph{arithmōn triōn}. The second reading links the genitives \emph{ongkōn} and \emph{dunameōn} instead with \emph{to meson}. Finally, the third reading takes there to be an implicit \emph{eite} before \emph{arithmōn} and so as an alternative on a par with \emph{ongkōn} and \emph{dunameōn} (\citealt[99]{Taylor:1928qb} provides Platonic parallels for this usage in the \emph{Sophist} 217e1, 224e2). Whereas the first two readings restrict the alternatives to \emph{arithmoi}, the third reading allows the proportionate progression to obtain among \emph{arithmoi} and non-\emph{arithmoi} alike. Thus it may be proportionate weights or lines or what have you that may be ordered. \citet[97 n12]{Archer-Hind:1888qd} and \citet[59]{Bury:1929jb}, endorse the first reading, \citet[44]{Cornford:1935fk} endorses the second, and \citet[96--99]{Taylor:1928qb} endorses the third, citing Proclus, \emph{In Timaeum}, as a source (see also his translation, \citeyear[28]{Taylor:1929ov}). \citet{Prtichard:1990aa} argues convincingly for the third reading, and he has been followed by \citet{Zeyl:2000cs} and by Johansen in his revision of Lee's translation \citeyearpar{Lee:2008ca}, who originally adopted the first reading, \citeyearpar{Lee:1965fh}.

The grammar of the opening phrase genuinely allows all three readings. Fortunately, the lexical ambiguities interact with these such that resolving the lexical ambiguities may resolve the syntactic ambiguity. The lexical ambiguities concern \emph{ongkōn} and \emph{dunameōn}. \emph{Dunameōn} is the genitive form of \emph{dunamis} which can mean, and is used by Timaeus to mean, power in a sense that includes sensible qualities. But \emph{dunamis}, in mathematical contexts, may also mean square root or possibly square number (for this latter reading see \citealt[97 n12]{Archer-Hind:1888qd} and \citealt[294 n1]{Heath:1921ys}; \citealt[184]{Prtichard:1990aa} plausibly argues that Heath may have been responsible for the following Liddel \& Scott entry for \emph{dunamis} ``V.I.b square number Pl \emph{Ti} 32a.''). \emph{Ongkōn} is the genitive of \emph{ongkos} and means volume, heap, or mass. But in a mathematical context, paired with \emph{dunamis}, it might mean geometrical solid or even cube, though this is not standardized mathematical vocabulary. As \citet{Prtichard:1990aa} observes, it all comes down to how \emph{dunamis} is read, since there is no reason to give \emph{ongkos} a mathematical reading if \emph{dunamis}, in this passage, does not receive one. 

If \emph{dunamis} has a mathematical reading in this passage, it could not mean the more usual square root, but must mean square number to contrast with \emph{ongkos} read as cube (\citealt[97 n12]{Archer-Hind:1888qd}, \citealt[45--52]{Cornford:1935fk}). However, \citet{Prtichard:1990aa} argues that this cannot be sustained, leaving only the non-mathematical reading of \emph{dunamis}. Aristotle, at least seems happy to describe powers of bodies as standing in ratios, \emph{Physica} 3 204b14--19. And Alcmaeon, a predecessor and in many ways an influence upon Timaeus, does so as well, at least according to Aëtius (\emph{Placita} 5.30.1). Moreover, it is plausibly the opposed powers of fire and earth that cause them to be naturally allocated in the extreme portions of the cosmos by the winnowing motion of the Receptacle. And since only the third reading does not restrict the members of the progression to \emph{arithmoi}, this is most likely Timaeus' intended reading. Thus resolving the lexical and syntactic ambiguities, we get:
\begin{quote}
	Whenever of three numbers or volumes or powers, the middle one is such that \dots.
\end{quote}

(8--9) Timaeus addresses one final complication before proceeding to derive the remaining Empedoclean ``roots''. The need for proportional intermediaries has been established, but how many, exactly, are required? According to Timaeus, it depends upon what you are compounding. If the Demiurge were compounding a planar figure, only one intermediary would be required. The cosmos, however, is not a planar figure but is solid. Indeed, the Demiurge makes it solid so as to more perfectly realize its resemblance to the Paradigm, the Living Being. Solids, however, require not one intermediary but two. 

(10) Thus the Demiurge sets air and water between fire and earth and made them as proportionate to one another as was possible to bind them in a three-dimensional whole that constitutes the body of the cosmos. Specifically, what fire is to air, air is to water, and what air is to water, water is to earth. Explicitly applying Timaeus' characterization of geometrical proportion to the primary bodies we get:
\begin{enumerate}[(1)]
	\item Whenever there is a middle among numbers, volumes or powers, such that as the first is to the middle, so the middle is to the last \ldots\ :
	\begin{quote}
		Fire:Air :: Air:Water :: Water:Earth
	\end{quote}
	\item \ldots\ and conversely, as the last is to the middle, so the middle is to the first \dots\ :
	\begin{quote}
		Earth:Water :: Water:Air :: Air:Fire
	\end{quote}
	\item \ldots\ then the middle becomes both first and last, and the last and first also appear in the middle (as there are two middle terms in the present case there are two ways this conclusion is met):
	\begin{quote}
		Water:Earth :: Air:Water :: Fire:Air\\
		Air:Fire :: Water:Air :: Earth:Water
	\end{quote}
\end{enumerate}
Fire, air, water, and earth are bodies. As bodies they encompass volume and have sensible powers. Indeed, these powers precede the formation of the primary bodies---the pre-cosmic chaos involved traces of these powers, with no proportionate organization (for this is a gift of the Demiurge), moving in a disorderly manner. The Demiurge imposes form and number on these furtive powers, assigning them to regular polyhedra that are proportionately organized, thus insuring sufficient intelligibility for them to receive names. Though volume and number are in play in the generation of the primary bodies, strictly speaking, it is the powers of the primary bodies that are arrayed in these geometrical proportions. 

The opposition of fire and earth is thus overcome in the amity of like proportion (amity, \emph{philia}, is most likely an allusion to Love in Empedocles's cosmology, though Strife finds no counterpart in Timaeus' speech). Having being unified in this manner, Timaeus remarks that this unity cannot be undone by anyone but the one that bound them together. Thus only the Demiurge has the power to rend asunder what he has bound by like proportion. The unity is perfect in the sense that it is as much unity as a generated thing may aspire to. This claim will be echoed in the assurances that the Demiurge gives the young gods (41aff) and in his enigmatic remarks about the unknowability by mortal intellect of the proportions involved in color mixture (68b6–8, 68c7–d7). The young gods come into being by the bonds fitted by the Demiurge, The Demiurge assures the young gods that while any bond may be dissolved, only evil would dissolve the bonds of what is well-fitted and good. Implicit in the Demiruge's assurance is that only he has the power to dissolve these bonds and that he is benevolent in a sense inconsistent with dissolving the bonds of what is well-fitted and good. And the proportions involved in color mixture are unknowable by mortal intellect, since in order to ascertain these one would have to have the power to blend the many into one and dissolve the one into many, and no mortal has this divine power.

What are we to make of this derivation? Timaeus describes his speech as providing a likely account (\emph{eikos logos}, 30b7) or likely story (\emph{eikos muthos}, 29d2). If a distinction is to be marked between these expressions, then the derivation is a \emph{eikos logos} rather than a \emph{eikos muthos}. Contrast Timaeus description of the generation of the World-Soul, which is narrated in terms appropriate to the forging of an alloy, with no pretense of being a derivation or an argument of any kind. On the supposition that a distinction may be marked, this is an \emph{eikos muthos} rather than an \emph{eikos logos}. The derivation of the four primary bodies, by contrast, presents itself as an account that appeals to the mathematical principle that governing geometrical proportionality. But it could only ever be likely, and so admit of no conclusive demonstration. It is not a scientific account in the sense of something that can be demonstrated and known. Strictly speaking, then, it is less a derivation than a pseudo-derivation. This prompts \citet{Prtichard:1990aa} to regard it as a parody of a scientific account. While Plato may realize the youthful ambitions of Socrates in the guise of Timaeus, perhaps he remains sensitive to the difficulties that drove Socrates to despair of natural philosophy, and is not above expressing this with a certain playfulness.

Even if we admit an element of playfulness in the derivation of the elemental composition of the body of the cosmos, such playfulness is consistent with a seriousness of purpose. Allow me to speculate on one theme that may be at stake in the presentation of the derivation. Perhaps the playful pseudo-derivation allows Plato, in the guise of Timaeus, to explore a possible reconciliation of Parmenidean and Empedoclean cosmologies, perhaps to lay claim to the allegiance of those who separately adhere to them. Begin with two observations that we made previously. First, unlike Empedocles, Timaeus does not merely take it for granted that the corporeal is composed of the four Empedoclean ``roots'' but rather seeks to derive these from their sensible powers. Second, he does by specifying these powers as visibility and tangibility, visibility requiring fire, and tangibility requiring earth. We canvassed some reasons for this specification. Perhaps another is presently relevant. Parmenides presents an account of Becoming, the sensible realm of generation, in the second half of his poem, The Way of Mortal Opinion. In order to account for the sensible realm of generation, Parmenides posits two principles, Fire and Night (DK 28B8 50--9). Perhaps surprisingly, Aristotle represents these as fire and earth (\emph{Metaphysica} A 986b31). Notice that fire and earth are the two primary bodies from which the remaining Empedoclean ``roots'' are derived given the need for proportionate intermediaries to combine these opposed extremes in amity and unity. In so doing, has Timaeus not, among other things, shown how to accommodate the four roots of Empedocles' cosmology with the two principles of Parmenides'? 

% section the_elemental_composition_of_the_corporeal (end)

\section{The Comprehensiveness of the Cosmos} % (fold)
\label{sec:the_comprehensiveness_of_the_cosmos}



% section the_comprehensiveness_of_the_cosmos (end)

\section{The Shape of the Cosmos} % (fold)
\label{sec:the_shape_of_the_cosmos}

The derivation of the elemental composition of the body of the cosmos crucially depended on conditions on the possibility of vision and touch. Perhaps surprisingly, Timaeus' argument for the shape of the cosmos also has important lessons for the philosophy of perception. As we shall see, Timaeus' argument that the body of the cosmos is spherical crucially depends upon the idea that perceptual capacities require an environment in which to operate in the sense that the objects of perception are essentially environmental (in something like Travis' \citeyear{Travis:2005ys} sense). 

% section the_shape_of_the_cosmos (end)


% Chapter cosmogeny (end) 