%!TEX root = /Users/markelikalderon/Documents/Git/timaeus/timaeus.tex
\chapter{Cosmogeny} % (fold)
\label{cha:cosmogeny}

\section{The Cosmic Significance of Sense and Sensibilia} % (fold)
\label{sec:the_cosmic_significance_of_sensibilia}

Given that the central concern of the present essay is with sense and sensibilia in the \emph{Timaeus}, why devote an entire chapter on the generation of the body of the cosmos? Of what interest is cosmogeny to a philosopher of perception? 

It is hard not to suspect that the anxiety that prompts this question is distinctly modern. Modern cosmology, or those aspects of it that correspond to cosmogeny such as the Big Bang theory, have nothing to teach us about sense and sensibilia. But nothing much follows about ancient cosmogeny in general, and Timaeus' cosmogeny in particular. As we shall see, the pre-cosmic chaos is sensible, and Timaeus derives the elemental composition of the body of the cosmos from the fact that it is visible and tangible. Modern thought about sense and sensibilia is animated by the worry that the scientific image of nature is potentially in conflict with the manifest image of nature that experience affords us. Even those that feel that the conflict is not, in the end, genuine, or may at least be partially overcome, theorize about perception and its objects with an eye to this conflict. One lesson of the present chapter is that, for Timaeus at least, there is no conflict between the scientific and manifest images of nature. This is all the more significant since, though the terminology may be Sellarsian, the roots of the conflict are Parmenidean. So it is not as if Timaeus' speech is delivered in naive ignorance of the potential for such conflict, an ignorance only overcome by scientific modernity. In the face of the Parmenidean explosion, Timaeus denies that the manifest image of nature is upended by its intelligible underpinnings. 

The lack of conflict between the scientific and manifest images of nature concern the nature of sensibilia. Timaeus' cosmogeny has important lessons for the nature of sense as well. Perception is essentially environmental. Perception takes as its object aspects of the environment that circumscribes the perceiver. Without an environment, there is no perception, and no need for sensory apparatus. This important lesson figures in Timaeus' argument that the body of the cosmos is spherical. So not only is cosmogeny relevant to the nature of sensibilia, it is relevant to the nature of sense as well.

Timaeus narrative of the creation of the world is non-linear. He does not, as it were, begin at the beginning and proceed inexorably to the end. Thus, for example, he describes the generation of the body of the cosmos before describing the generation of the World-Soul, even though the World-Soul is prior in dignity and birth. Moreover, Timaeus only describes the pre-cosmic chaos after his cosmogeny and psychogeny. This is no mere happenstance. Timaeus has good reason for the temporal disruption of his narrative. Nor does Timaeus disrupt the temporal order for dramatic reasons the way that Kubrick does in \emph{The Killing}, a device that Tarantino subsequently made much of. Timaeus' reasons are methodological. The first part of his speech concerns the works of Reason, the second concern the works of Necessity, and the third concern the interaction of the Reason and Necessity. The temporal order of the creation narrative does not respect these divisions, and so Timaeus narrates them accordingly.

% The present chapter shall not follow Timaeus in this. We shall begin with the pre-cosmic chaos and proceed to the formation of the body of the cosmos. Part of the reason for this departure from the Timaean narrative is to confront a puzzle about naming that informs much of Timaeus' speech. Even a casual reader of the \emph{Timaeus} will notice a singular preoccupation with naming. The \emph{aporia} provides the reason for Timaeus' preoccupation, and it is important to carefully attend to this.

% section the_cosmic_significance_of_sensibilia (end)

\section{The Elemental Composition of the Corporeal} % (fold)
\label{sec:the_elemental_composition_of_the_corporeal}

Empedocles took it for granted that the corporeal was composed of four ``roots'' (\emph{rhi\-zō\-ma\-ta}). Initially presented in divinized form (DK 31B6)---Zeus, Hera, Aidoneus (Hades), and Nestis---there is some controversy as to the assignment of roots to deities, apart from Nestis who is explicitly associated with water \citep[165--6]{Wright:1981zr}. \citet[165]{Wright:1981zr} endorses the Theophrastean interpretation on which Zeus is fire, Hera is air, Aidon\-eus is earth, and Nestis, of course, is water. Timaeus will deny that the Empedoclean ``roots'' are elements (\emph{stoicheia}) since they are themselves polyhedra that are further composed of elementary triangles (48b8). It is these triangles that are, in Timaeus' system, the \emph{stoicheia} or \emph{elementa}. Without loosing sight of this important Timaean claim, I will sometimes refer to the Empedoclean ``roots'' as elements. That fire, air, earth, and water are not, strictly speaking, elements is not the only way in which Timaeus departs from Empedocles. Timaeus does not simply take it for granted that these four are the roots of all things. Rather, Timaeus derives the elemental composition of the corporeal from its sensible aspect.

The derivation of the elemental composition of the cosmos has a number of distinct steps each of which requires comment:
\begin{enumerate}
	\item That which comes to be is corporeal and so is visible and tangible.
	\item Since the cosmos is visible, it contains fire
	\item Since the cosmos is tangible, it is solid, and since it is solid, it contains earth
	\item Two things may not be joined without a third
	\item There must be an intermediary bond that joins them together
	\item The fairest bond that the most perfectly unites both itself and that which it joins is proportion
	\item If the cosmos were planar, then one middle term would suffice to bind the two
	\item But the cosmos is solid, and so two middle terms are required to bind the two
	\item The cosmos contains fire, air, water, and earth, where air is to water as fire is to air, and water is to earth as air is to water
\end{enumerate}
Let us consider these in turn.

First, that which comes to be is corporeal and so is visible and tangible. Timaeus, here, takes for granted what he earlier argued for, that the cosmos has come to be. Recall, in the \emph{proemium}, Timaeus argued that the cosmos has come to be (28b7--c1):
\begin{enumerate}
	\item The cosmos is visible and tangible
	\item Therefore, the cosmos has a body
	\item Since the cosmos is sensible, it is apprehended by perception and opinion
	\item What is apprehended by perception and opinion has come to be
	\item Therefore, the cosmos has come to be.
\end{enumerate}
The initial claim of Timaeus' derivation of the elemental composition of the cosmos seems to reverse this line of reasoning. Specifically, Timaeus begins with the idea that the cosmos has come to be and concludes first that it is corporeal and then that it is visible and tangible.

Another thing to observe about the initial claim of the derivation is that the cosmos, that which has come to be, is not merely claimed to be sensible but more specifically visible and tangible. Timaeus may have strategic and non-strategic reasons for this specification. The strategic reason is that from this specification, he may immediately conclude that the cosmos is composed of fire and earth, thus beginning the derivation. There may, however, be non-strategic reasons as well. While it may be plausible to claim that we can see and feel the cosmos, it is perhaps less plausible to claim that we can smell it, say, even if the cosmos contains parts that we can smell (Proclus, \emph{In Timaeum} 3.1.6 25--6). Nor is the cosmos as a whole tastable and audible, even if it contains parts that are tastable and audible. The cosmos as a whole may be visible and tangible, but the cosmos as a whole is not tastable, smellable, or audible.

There is another aspect of this specification that is worth mentioning. Proclus suggests that the visible and the tangible are the extreme terms of the sensible (\emph{In Timaeum} 3.1.6--7). So understood, the visible and the tangible are opposed with the other sensibles arrayed as intermediaries. This coheres well with the order in which Timaeus discusses the special sensibles when discussing the affections of the body that are liable to give rise to perception and sensation (61d--68e). Timaeus begins with the tangible and ends with the visible, with the tastable, the smellable, and the audible arrayed as intermediaries. On the Proclean interpretation, the specification involves no loss of generality. One may speak of the sensible in general in terms of its opposed extremes. 

Proclus' professed reason for this claim is non-Timaean, however. Indeed, it is Peripatetic. Proclus claims that the visible is opposed to the tangible since while the former requires a medium, the latter does not (\emph{In Timaeum} 3.1.6 9--11). While Timaeus will claim that the visual body composed of the fire emanating from within the eye and daylight mediates the action of chromatic fire, the thought that vision requires a medium is really only made explicit by Aristotle. I have characterized Proclus' reason for the opposition between the visible and the tangible as Peripatetic. One might reasonably object that, for Aristotle, the flesh is the medium of touch, and so that the tangible involves a medium just as the visible does. The observation upon which this objection rests is correct, but there is an important shift in vocabulary, if not doctrine, between books 2 and 3 of \emph{De anima}. While in book 2 Aristotle emphasizes that the flesh is the medium by which we feel the tangible, in the opening of book 3, he seems to claim what Proclus' is claiming, that while vision requires a medium, touch does not. The claims of book 3 may be reconciled with the claims of book 2, if we understand the denial that the tangible involves a medium as the denial that it involves an external medium, the flesh being, of course, internal to the percipient. Just because an argument is bad does not mean that its conclusion is not true. Similarly, just because Proclus' reason for claiming that the visible and the tangible are opposed extremes of the sensible are not Timaeus', does not mean that Timaeus does not accept that the visible and the tangible are the opposed extremes of the sensible. At the very least, I think we should be open to the Proclean suggestion here.

The initial claim is also the basis of an interesting contrast with Descartes. While Timaeus agrees with Descartes that the corporeal is extended, the corporeal is not identified with extension. Indeed, the defining characteristic of the corporeal seems rather that it is sensible. That the corporeal is sensible, figures not only in the derivation of the elemental composition of the cosmos, but it also figures in the argument, from the \emph{proemium}, that the cosmos has come to be.

From the fact that the cosmos is visible, Timaeus concludes that it must contain fire. As we shall see, vision arises from the interaction of three fires of two kinds. One kind of fire is a mild light that does not burn. Daylight is this kind of fire, as is the fire that emanates from within the eye. Timaeus conceives of colors, the proper objects of vision, as a power of bodies to emit a different kind of fire, a flame (\emph{phlox}). It is because vision is the result of the interaction of these three fires of two kinds, that Timaeus concludes that the cosmos contains fire from its visibility. Timaeus is not restricting the visible to the fiery. We can see things other than fire. Rather, Timaeus is claiming that if anything at all is visible, then there must be fire that illuminates the cosmos. And if there is, then the cosmos contains fire.

While the derivation of fire from the visibility of the cosmos is immediate, the derivation of earth from the tangibility of the cosmos is not. From the fact that the cosmos is tangible, Timaeus first concludes that it is solid, and then concludes that it must be composed of earth.



% section the_elemental_composition_of_the_corporeal (end)


% Chapter cosmogeny (end) 