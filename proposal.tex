%!TEX TS-program = xelatex 

%!TEX encoding = UTF-8 Unicode
%
%  proposal
%
%  Created by Mark Eli Kalderon on 2020-03-04.
%  Copyright (c) 2020. All rights reserved.
%

\documentclass[12pt]{article} 

% Definitions
\newcommand\mykeywords{Aristotle, perception, color}
\newcommand\myauthor{Mark Eli Kalderon}

% Packages
\usepackage{geometry} \geometry{a4paper}

% XeTeX
\usepackage[cm-default]{fontspec}
\usepackage{xltxtra,xunicode}
\defaultfontfeatures{Scale=MatchLowercase,Mapping=tex-text}
\setmainfont{Hoefler Text}

% PDF Stuff
\usepackage[plainpages=false, pdfpagelabels, bookmarksnumbered, backref, pdftitle={Form Without Matter}, pagebackref, pdfauthor={\myauthor}, pdfkeywords={\mykeywords}, xetex, colorlinks=true, citecolor=gray, linkcolor=gray, urlcolor=gray]{hyperref}

%%% BEGIN DOCUMENT
\begin{document}

% % Title Page
\author{\myauthor}
\title{Book Proposal\\
In the Eye of the Cognitive Storm\\
Timaeus on Sense and Sensibilia}
\date{}

\maketitle

% Layout Settings
\setlength{\parindent}{1em}

% Main Content

\section{Overview} % (fold)
\label{sec:overview}

The present essay is on Timaeus' account of senses and sensibilia. Tiameus' views on perception were tremendously influential both in classical antiquity and the Hellenistic period. Allow me to highlight features of the present work that are noteworthy. First, most discussions of perception in the \emph{Timaeus} focus on vision. The present essay discusses all of the senses discussed by Timaeus. Second, most commentators focus on the passages wheree perception is explicitly discussed. However, principles of philosophy of perception are at work in broader aspects of Timaeus's speech on cosmology (for example in his arguments for the elemental composition and morphology of the cosmos). The present essay is the first systematic survey of Timaeus's speech with respect to the philosophy of perception. Third, most commentators discuss the dialogue with an eye to Plato's views of perception. The present essay is by contrast focused on getting straight Timaeus' views (an essential first step in that broader project). Fourth, against the ``literalist'' trend in Anglophone commentary, I argue that the details of the soul's motion are mythical and only by so understanding them can we appreciate the cognitive significance of Timaeus's metaphors

% section overview (end)

\section{The Audience} % (fold)
\label{sec:the_audience}

The present essay has a divided audience. It is addressed both to philosophers of perception as well as historians (as well as graduate students working in these fields). 

% section the_audience (end)

\section{The Competition} % (fold)
\label{sec:the_competition}

While the books on Plato are legion, there are no books on perception in the Timaeus. Among the recent books on the \emph{Timaeus} are

\begin{enumerate}
	\item Thomas K. Johansen. \emph{Plato’s Natural Philosophy, A Study of the Timaeus-Critias}. Cambridge University Press, Cambridge, 2004.
	\item Gabriela Roxana Carone. \emph{Plato’s Cosmology and Its Ethical Dimensions}. Cambridge University Press, Cambridge, 2005.
	\item Sarah Broadie. \emph{Nature and Divinity in Plato’s Timaeus}. Cambridge University Press, Cambridge, 2012.
\end{enumerate}

Perhaps closer in topic is T.M. Robinson's \emph{Plato’s Psychology}, University of Toronto Press, Toronto, 1970. However it is also broader in scope since it is not exclusively about the \emph{Timaeus}.

% section the_competition (end)

\section{Chapters} % (fold)
\label{sec:chapters}

The present essay consists of 10 chapters:

\begin{enumerate}
	\item \emph{Proemium} Discusses the introduction to Timaeus's speech and its metaphysical and epistemological principles.
	\item \emph{Cosmogony}. Discusses Timaeus' arguments concerning the elemental composition, comprehensiveness, and shape of the cosmos and their (surprisingly direct) relevance to his philosophy of perception.
	\item \emph{Psychogony}. Discusses the generation of the World Soul since its substance, proportional divisions, and structure are relevant to its cognitive activities which serve as the model for the cognitive activities of mortal beings.
	\item \emph{Cognitive Revolution}. Discusses the circular motion of the World Soul and its cognitive significance.
	\item \emph{Embodiment}. Discusses the embodiment of mortal beings, and its ethical significance.
	\item \emph{The End of Vision and Audition} Discusses Timaeus' preliminary account of vision and audition and the role they play in our rational salvation.
	\item \emph{Common Pathemata}. Discusses the affections of the body as a whole that give rise to perception and sensation with a emphasis on touch
	\item \emph{Peculiar Pathemata}. Discusses the affections of particular parts of the body that give rise to perception and sensation and so discusses taste, smell, audition, and vision.
	\item \emph{The Anatomy of Tripartition}. Discusses Timaeus' anatomical take on the tripartite psychology.
	\item \emph{The Bonds of Life}. Discusses the soul-body union in mortal beings.
\end{enumerate}

% section chapters (end)

\section{About the Author} % (fold)
\label{sec:about_the_author}

I am a professor of philosophy at UCL and I have written extensively about color and color perception. My previous books include \emph{Moral Fictionalism} OUP 2005, \emph{Form Without Matter, Empedocles and Aristotle on Color Perception} OUP 2015, and \emph{Sympathy in Perception} CUP, 2018.

% section about_the_author (end)

\end{document}
