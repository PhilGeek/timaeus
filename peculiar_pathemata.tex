%!TEX root = /Users/markelikalderon/Documents/Git/timaeus/timaeus.tex

\chapter{Peculiar \emph{Pathēmata}} % (fold)
\label{cha:peculiar_pathemata}

\section{Common versus Peculiar \emph{Pathēmata}} % (fold)
\label{sec:common_versus_peculiar_pathemata}

Whereas the common \emph{pathēmata} are affections common the body as a whole (61d--65b), the peculiar \emph{pathēmata} are affections that are peculiar to particular parts of the body (65b--68e). Just as in the case of the common \emph{pathēmata}, Timaeus will explain the peculiar \emph{pathēmata} in terms of the powers of the agents that cause them and explain how these powers are named in terms of the \emph{pathēmata} they cause. Whereas the common \emph{pathēmata} are involved in the measure of tactile qualities, the peculiar \emph{pathēmata} are involved in the measure of sensory qualities involved in the other four special senses---taste, smell, audition, and vision. Again, though I am anachronistically appealing to Aristotelian vocabulary here, I am bracketing any assumption that the particular parts of the body upon which the peculiar \emph{pathēmata} fall are best conceived by Timaeus as sensory organs. Whether they are shall be discussed only once we have reviewed in detail the four peculiar \emph{pathēmata}. As we shall see, they are not akin to sensory organs, at least as Aristotle conceives of them, but are more like Empedoclean \emph{poroi}. The four particular parts of the body upon which the peculiar \emph{pathēmata} fall are:
\begin{enumerate}
	\item the tongue (65b--66c) (section~\ref{sec:the_tongue})
	\item the nostrils (66c--67a) (section~\ref{sec:the_nostrils})
	\item the ears (or perhaps the brain and the blood) (46c--47e, 67a--c, 80a) (section~\ref{sec:the_ears})
	\item the eyes (45b--46c, 67c--68d) (section~\ref{sec:the_eyes})
\end{enumerate} 

Why does Timaeus discuss the remaining four special senses in the order that he does? Perhaps the relative importance of pleasure and pain provides a clue. These, recall, were claimed to be the most important of the common \emph{pathēmata}. Moreover, I argued that their importance consisted in their ethical significance. Pleasure may incite the greatest of evil, while pain may deter us from the good (68d). Perhaps the present ordering of the four remaining special senses is in ascending order of ethical significance. Taste is, after all, closely connected with appetite, and appetite may lead us astray. Moreover, it was the flood of nutriment that initially disrupts the circles of the Same and the Different in the initial shock of embodiment, a disruption providentially accommodated by our being endowed with audition and vision. That the four remaining special senses are ordered by ascending ethical significance is a speculative hypothesis only, but I see no other alternative explanation. (I emphasize the speculative characer of the hypothesis since inference to an explanation falls well short of inference to the best explanation.)

% section common_versus_peculiar_emph_pathemata (end)

\section{The Tongue} % (fold)
\label{sec:the_tongue}

The first class of \emph{pathēmata} that Timaeus discusses all fall upon the tongue. The \emph{pathēmata} involved in the experience of flavour are all brought about by certain contractions (\emph{sunkrisōs}) or dilations (\emph{diakrisōs}). Timaeus signals this explicitly (65c2--6). Nor is it the first time he has used this vocabulary. It has been explicitly deployed in his discussion of hot and cold (61d5–62a5), and in his discussion of pleasure and pain (64e), and it will reoccur in his discussion of color and vision (67c–68d). And many of the other \emph{pathēmata}, both common and peculiar, while not explicitly so described, may be. Thus, for example, the flesh in yielding to the hardness of a stone in its grasp is compacted. This uniformity is perhaps obscured in English translation given the breadth of the semantic field associated with these terms (thus obscuring as well a potential Empedoclean influence as these plausibly correspond to the operations of Love and Strife). In addition to contractions and dilations, Timaeus also claims that, more than any other sense, the experience of flavour involves the rough and the smooth. 

There are small passages in the tongue leading to the heart. Timaeus describes these as veins (\emph{phlebia}) though there is no explicit indication that these passages are blood vessels. Timaeus only ever mentions them containing particles of earth and air (though since these may be covered with a moist film, presumably they contain particles of water as well.) Indeed, from the perspective of contemporary anatomy they seem to function less like blood vessels and more like nerves. Specifically, these veins function as ``testing instruments'' of the tongue. As instruments, they determine the flavour of the ingested nutriment, depending upon how the nutriment acts upon them. As instruments, they are the measure of the nutriment's flavour. Presumably these instruments are tests since, depending upon the flavour, the nutriment may be either swallowed or spat out if foul. Whether or not the veins in the tongue, with the function that Timaeus ascribes to them, are best understood as blood vessels, cannot be settled with reference to Timaeus' discussion of the tongue alone.

The veins in the tongue thus clearly function as Empedoclean \emph{poroi} (for the role of \emph{poroi} or passages in Empedocles' discussion of \emph{aisthēsis} see Plato \emph{Meno} 76 a--d, Theophrastus \emph{De Sensibus} 7, and \citealt{Beare:1906uq}). There is one notable difference, however. Whereas, for Empedocles, perceived effluences are proportional or commensurate with the \emph{poroi}, the nutriment need not be proportional with the veins in the tongue. However, as we shall see, this aspect of Empedoclean perceptual anatomy is manifest in Timaeus' account of olfaction. Pseudo-Timaeus Locri also emphasizes this Emepdoclean theme: 
\begin{quote}
	What happens with taste is similar to what happens with touch. For things are perceived as either astringent or smooth because of contraction and dilation, as well as because of their movement through the passages and their shapes. (Pseudo-Timaeus Locri, \emph{De natura mundi et animae} 100e; \citealt[59]{Tobin:1984qf})
\end{quote}
Though perhaps this is a Pythagorean approach to taste more broadly, if we accept that Alcmaeon of Croton was a Pythagorean, as Diogenes Laertius (\emph{Vitae Philosophorum} 8.83), Iamblichus (\emph{Vita Pythagorae} 104, 267), and Philoponus (\emph{In De anima} 2 88) attest. For both Theophrastus (\emph{De sensibus} 25) and Plutarch (\emph{Epit} 5 24) attribute a similar view to Alcmaeon of Croton. Even if Alcmaeon was not a Pythagorean, hailing from Croton, a center of Pythagorean activity, he was undoubtedly at least influenced by their thought.

That the veins in the tongue lead to the heart may have influenced Aristotle's conviction that the heart is the primary sensorium. This would be ironic if so, since, in general, Timaeus' anatomy is influenced by, not only Empedocles, but Alcamaeon of Croton as well who maintained that the brain, and not the heart, is centrally involved in perception and sensation:
\begin{quote}
	All the senses are connected in some way with the brain; consequently they are incapable of action if <the brain> is disturbed or shifts its position, for <this organ> stops up the passages through which the senses act. (Theophrastus, \emph{De sensibus} 26; \citealt[89--91]{Stratton:1917vn})
\end{quote}

In addition to containing veins leading to the heart, the tongue is described as moist and soft. Both these features function in taste as well. Thus the moistness of the tongue is said to melt particles of earth prior to their entry into the veins of the tongue (though liquefication will play a larger and more systematic role in Aristotle's account of digestion in \emph{De anima} 2.4). And the softness of the tongue is relevant to its susceptibility to roughening.

Recall the methodological dilemma raised by Timaeus for his account of the \emph{pathēmata} (61c4–d5) (discussed in chapter~\ref{sec:pathemata}). On the one hand, one cannot adequately account for the \emph{pathēmata} without accounting for the origin of the flesh and things pertaining to it and the mortal parts of the soul; on the other hand, one cannot adequately account for the origin of the flesh and things pertaining to it and the mortal parts of the soul without accounting for the \emph{pathēmata}. Perhaps more than with the common \emph{pathēmata}, we can see how Timaeus' discussion of the \emph{pathēmata} presupposes at least the origins of the flesh and things pertaining to it. An account of \emph{pathēmata} peculiar to the tongue would not be possible without making assumptions about the nature and organization of human flesh. The tongue is human flesh. It is moist and soft and contains veins. The way it is affected so as to be liable to give rise to the experience of flavour would be inexplicable without making some such assumptions.

\emph{Pathēmata} peculiar to the tongue are involved in the sensation of:
\begin{enumerate}
	\item astringent and harsh (65c6--d4)
	\item bitter and saline (65d4--65e4)
	\item pungent (65e4--66a2)
	\item sour (66a2--b7)
	\item sweet (66b7--c7)
\end{enumerate}
The flavours stand in a one--many oppositional structure. Sweet is not merely opposed to bitter, but to the whole range of alternative flavours. In this respect it is like the opposition of the center of the cosmos with the points that lie, in extremity, on its circumference, The flavours thus contrast, in different ways, with odours and colors. Odours lack any meaningful oppositional structure admitting only to a loose classification as pleasant and unpleasant. Colors, like flavours and unlike odours, have an oppositional structure, but the opposition is between white and black with the other colors arrayed as intermediaries between them. (This is controversial. Both \citealt{Brisson:1997qr} and \citealt{Ierodiakonou:2005ly} claim that Timaeus is committed to there being four primary colors. We shall see in section~\ref{sec:the_eyes} that there is a way to reconcile these seemingly competing views.) The chromatic oppositional structure is thus one--one as opposed to one--many. This is the more usual pattern of oppositional classification in classical antiquity (see \citealt{Lloyd:1966ly} for discussion). What explains the one--many nature of the opposition here? We shall return to this question after discussing sweetness and its alternatives.

Allow me to make one further preliminary observation before discussing the \emph{pathēmata} involved in tasting the individual flavours. Timaeus begins his discussion of the \emph{pathēmata} peculiar to the tongue by claiming to return to what was omitted in his previous account. This is a backreference to his discussion of the flavours or juices among the \emph{genē}. As we shall see, this is a further confirmation of an important point argued in the previous chapter, namely, that the \emph{pathēmata} are merely the effects on sentient animate bodies, effects that may obtain among inanimate bodies as well.

The \emph{pathēmata} involved in the sensations of astringent and harsh consists in the drying and constriction of the veins of the tongue. Timaeus explains these affections in terms of the character of the primary bodies that compose the nutriment. Specifically, when particles of earth fall upon the moist and soft tongue, these have a tendency to melt with the result that the veins in the tongue constrict and become dryer. Presumably their melting is a consequence of earth's absorption of water. But if the water of the tongue is absorbed, it becomes dryer, and the constriction of the veins of the tongue is a consequence of this drying. When these particles are rougher this process is liable to give rise to a sensation of astringency, and when less rough this process is liable to give rise to a sensation of harshness. (According to Theophrastus, \emph{De sensibus} 66, Timaeus' claim about astringency echoes Democritus'.) Though Timaeus does not explicitly say so, the drying and constriction of the tongue associated with the sensations of astringency and harshness are a departure from the natural state of the tongue. This is what explains their opposition to sweetness. Sweet things are sweet in part by their restoring the tongue to its natural state, hence the pleasure associated with this sensation. 

The \emph{pathēmata} involved in the sensations of bitter and harsh are also affections of the veins in the tongue, though in this case the veins are acted upon in the manner of a cleansing agent or detergent that washes out the veins. Timaeus emphasizes not only that the nutriment acts upon the veins of the tongue as a detergent but does so excessively. I take Timaeus to mean here that this is a departure from the natural state of the tongue. Thus nutriment excessively acting as a detergent is said to dissolve the substance of the tongue, and perhaps such nutriment excessively opens up the tongue's passages as well. Moreover, as we shall see, this departure from the natural state of the tongue explains well these flavours' opposition to sweetness. The difference between the \emph{pathēmata} involved in the sensations of bitter and saline is a matter of degree. When the affection is more excessive, this process is liable to give rise to the sensation of bitterness. When it is less excessive and more agreeable, this process is liable to give rise to the sensation of saline.

The \emph{pathēmata} involved in pungent sensations are more complex and constitute a striking anticipation of the role of retronasal olfaction in taste. In this case the nutriment is both hot and is susceptible to sharing in the heat of the tongue itself. As a result of sharing in the heat of the tongue, the nutriment becomes softened and fiery. This burns the tongue. Moreover, since the nutriment is fiery, particles of the nutriment move upwards (since primary bodies naturally move in the direction of their native environment). These fiery particles cut all that the parts upon which they impinge (because of the acuteness of their angles, the sharpness of their edges, their smallness, and the rapidity of their motion). Think of the way that excessively hot food sensibly affects the nasal cavities. This process is liable to give rise to a pungent sensation. (Timaeus' account of the pungent is an elaboration of the one that Theophrastus, \emph{De sensibus} 67, attributes to Democritus.) Again, though Timaeus does not make this explicit, the \emph{pathēmata} involved in pungent sensations, since these include burning and cutting, are a departure from the natural state of the tongue as well as a departure from the natural state of the parts above it upon which the fiery particles impinge. And if that is right, then the opposition to sweetness is explained.

The \emph{pathēmata} involved in the experience of sourness or acidity is also complex (on this difficult passage see \citealt[chapter 6.4]{OBrien:1984ji}). In this case the nutriment consists in particles whose fineness is the result of decomposition or putrefaction. When these fine particles enter the veins of the tongue, they encounter there particles of earth and air that are proportional to them. Perhaps by proportional Timaeus means that they are like-sized, since these particles differ in shape. (As we shall see, this is the sense in which transparent fire particles are proportional to the fire particles that constitute the visual stream.) The fine particles encountering particles of earth and air proportional to them causes these to circulate. Moreover, the circulation of these particles leads to a process of fermentation. The circular motion involved in fermentation gives rise to bubbles. It does so in the following fashion. This circular motion or churning causes the particles to tumble and to fall into one another, as spaces in the mixture open up, with the result that air particles become encased by a moist film and so create bubbles. There are two kinds of bubbles depending upon the nature of the moist film. If the film is composed of relatively pure water, these are transparent and called ``bubbles''. If the moist film contains earth in addition to water, these are called ``frothing'' or ``fermenting''. Finally, Timaeus claims that the cause of this process is called ``sour''. So sourness is the power of the putrefying nutriment to cause this process of fermentation in the veins of the tongue, and this affection is its measure.

One oddity of Timaeus' account, here, is that the earthy bubbles are said to rise. How could this be consistent with his claims about elemental motion? Shouldn't they sink? Perhaps, though Timaeus does not make this explicit (he is barely explicit about the presence of water), the earthy bubbles involved in frothing or fermentation contain fire as well. Perhaps like pungent nutriment, in the process of fermentation, the earthy bubbles share in the heat of the tongue. There is also a hint of the presence of fire in the transparency of the bubbles whose films are composed of pure water since Timaeus will go on to explain transparency in terms of the size of fire particles (though, as we shall see, strictly speaking, Timaeus provides no explicit account of transparent bodies such as glass, water, ice, and the horns of animals from which lamps were made). The films of the earthy bubbles must contain more fire than the films of bubbles composed of water alone since these do not rise. It is natural to understand that the transparency of the earthy bubbles is obscured by the presence of earth. One final piece of evidence for the unmentioned presence of fire: Earlier, in his discussion of the \emph{genē}, Timaeus describes a frothy substance \emph{opos}, secreted from juices, that dissolves the flesh by burning (60b3--5). \emph{Opos} is clearly sour, and we are being given here in this later passage a more elaborate account of its frothing. Moreover, if we take its description as burning literally, it must involve the presence of fire.

In Timaeus' account of sourness, earth, water, and air are explicitly said to be within the veins. If, in addition, Timaeus is implicitly committed to their be fire, then this may be partial evidence that veins are, after all, vessels of blood, since blood is composed of the four primary bodies.

Timaeus now turns to the \emph{pathēmata} involved in the experience of sweetness. He begins by highlighting sweetness' opposition to all the previous flavours. This is more significant than it might at first seem, for there is no unique affection with which to identify the \emph{pathēmata} involved in the sensation of sweetness. Sweet things restore the tongue to its natural state. That is what makes sweet things sweet. Timaeus mentions three specific affections potentially involved in the sensation of sweetness. Sweet nutriment may:
\begin{enumerate}
	\item smooth the roughened tongue (again an echo of Democritus according to Theophrastus, \emph{De sensibus} 65)
	\item contract dilated veins leading from the tongue
	\item dilate constricted veins leading from the tongue
\end{enumerate}
Here I think that it is important to bear in mind a number of features already in play in Timaeus' account of flavour perception. First, recall, that the experience of flavour involves two pairs of affections: the rough and the smooth and contractions and dilations. Second, all along Timaeus has been emphasizing the natural state of the tongue and how non-sweet flavours all involve, in different ways, a departure from this natural state. Third, the tongue is naturally soft. Putting these together, the three actions of sweet nutriment are readily explicable. If the tongue is naturally smooth, any roughening of it will be a departure from its natural state. This is why only the smoothing of the tongue is mentioned in the process involved in sensing sweetness. The veins in the tongue, depending upon the activity of the nutriment, may be unnaturally contracted or unnaturally dilated. When the veins in the tongue are unnaturally dilated, the sweetness of the nutriment has the tendency to restore these to their natural state by contraction. And when the veins in the tongue are unnaturally contracted, the sweetness of the nutriment has the tendency to restore these to their natural state by dilation. In general, sweet nutriment relaxes and restores the tongue to its natural state. I have been careful to state that sweet nutriment potentially involve these three activities. If the veins of the tongue are not unnaturally dilated, sweet nutriment will not contract them. To do so would be a departure from the natural state of the tongue. And if the veins of the tongue are not unnaturally contracted, sweet nutriment will not dilate them. Again, to do so would be a departure from the natural state of the tongue. So the three \emph{pathēmata} associated with sweet sensations do not invariably occur in our experience of sweetness. What is essential to sweetness is the restoration of the tongue to its natural state. This is why sweetness is by its very nature pleasant.

This is relevant to the one--many nature of the oppositional structure of flavour. Sweetness restores the tongue to its natural state. But the departure from the tongue's natural state is multifarious. There is more than one way in which the tongue may depart from its natural state. This is why Timaeus conceives of the oppositional structure of flavour as one--many. Rather than conceiving of the oppositional structure of flavour on the model of colors, with sweet opposed to bitter, say, and with the others flavours arrayed as intermediaries between them, sweet is opposed to all the different ways in which the flavours involve a departure from the natural state of the tongue.

Recall that Timaeus began his discussion of the \emph{pathēmata} peculiar to the tongue by claiming to return to what was omitted in his previous account. This was a backreference not only to his earlier discussion of \emph{opos} but to \emph{meli} as well. Though \emph{meli} is usually translated as honey it has a broader range of reference than what is produced by bees (see \citealt[254 n6]{Cornford:1935fk}). Of honey, so understood, Timaeus claims that it is an agent that relaxes and restores the passages in the mouth and by this means produces a sweet sensation (60a8--b3). Just as his account of sourness is an elaboration of his remarks on \emph{opos}, his account of sweetness is an elaboration of his remarks on honey.

Theophrastus' objection to Timaeus' account of flavour is a variation of his more general objection that the sensory objects, according ot Timaeus, lack the requisite unity:
\begin{quote}
	Of the sapid substances, he fails to state what severally are their natures \ldots\ But what we seek---since the affections themselves are clear as day---is rather the reality behind them and why they produce this result. (Theophrastus, \emph{De Sensibus} 89; \citealt[149]{Stratton:1917vn})
\end{quote}
Again, I think that Theophrastus has got the emphasis of Timaeus' account wrong (see chapter~\ref{sec:hot_and_cold}). The objects of taste sensation are the powers of the nutriments that produce these affections. Knowledge of these effects suffice for knowing which powers these are. Of course, these powers are themselves explicable in terms of the elemental composition of the nutriment. We have seen examples of this with the flavours opposed to sweetness. But there is no guarantee that the unity which Theophrastus seeks will be found on this level. We have already seen that things may be hard either because they are composed of earth or because they are dense. The unity that Theophrastus seeks is only to be found on the level of the powers of the agents to cause the \emph{pathēmata}. It is the powers of the agents that cause these affections that is the reality behind them and why they result. And since Theophrastus ignores these, he sees only a difficulty for Timaeus' account.

% section the_tongue (end)

\section{The Nostrils} % (fold)
\label{sec:the_nostrils}

Nostrils may be passages leading within, but lining their interior are passages leading further within. Specifically, according to Timaeus, there are veins in the interior of the nostrils. It is these veins that receive odours. Timaeus correctly conceives of odours as massess of particulate matter. However, the particles that we smell are not directly associated with the primary bodies nor even compounds of these. Here, it is important to remember that the primary bodies, the Empedoclean ``roots'', are not, according to Timaeus, strictly speaking elements. If an element is the smallest part that composes a body, then, strictly speaking, the only elements are the elemental triangles that compose the primary bodies. The reason this is important is that none of the primary bodies are proportional (\emph{summetria}, here, curiously, appearing in the Ionian dialect---{\sbl ξυμμετρια} instead of {\sbl συμμετρια}) with the veins in the nostrils with the result that they lack smell. Particles of earth and water are too large to enter the veins in the nostril, and the particles of air and fire are too small. Apparently, they pass through without affecting these veins. What's required to sense an odour is particulate matter that is proportional with these passages and no primary body is. What is smelled is a ``half-formed class''. In the cycle of elemental transformation, specifically as air and water transform into one another, these particles break down into their elemental triangles and reform into polyhedra of the other primary body. Olfaction takes as its objects bodies in these intermediate states. This is the sense in which they are half-formed, and it is only the half-formed that are proportional with the veins in the nostrils.

That the odorous must be proportional or commensurate with passages in the nostrils is due to the overt influence of Empedocles. Thus Theophrastus reports:
\begin{quote}
	For he [Empedocles] attributes our recognition of things to two factors---namely, to likeness and to contact; and so he uses the expression ``to fit''. Accordingly if the smaller touched the larger ones, there would be perception. And likeness also, speaking generally, is out of the question, at least according to him, and commensurateness alone suffices. For he says that substances fail to perceive one another because their passages are not commensurate. But whether the effluence is like or unlike he leaves quite undetermined. (Theophrastus, \emph{De Sensibus} 15; \citealt[79]{Stratton:1917vn})
\end{quote}
Timaeus reduplicates this Empedoclean scheme in his account of smell. We fail to perceive the primary bodies in olfaction since they are not proportional with the passages in the nostrils, the veins leading within. However, whereas for Empedocles this is a general claim, effluences are only perceptible if they fit the relevant passages, Timaeus, here, is only making this claim with respect to the special case of smell.

Recall that unlike his pre-Socratic predecessors, Timaeus denies the full cycle of elemental transformation. Because of the particular kind of triangles that compose earth, no primary body transforms into earth and earth transforms into no other primary body. In this respect at least, the full cycle of elemental transformation is merely apparent. Nevertheless a partial cycle of elemental transformation genuinely remains. Air, water, and fire may transform into one another by these polyhedra decomposing into their elemental triangles and recomposing into polyhedra of the other primary bodies. Only the transformation of air into water and water into air are relevant to olfaction. It is only when bodies are in the intermediary state in the process of these transformations will they be commensurate with the veins in the nostrils. What transforms from air into water is mist, and what transfroms from water into air is vapour. And the particles that compose mist and vapour are smaller than water and bigger than air. Thus only that which is moistened, or putrefied, or melted, or vaporised, may be smelled. For only these give off bodies in the intermediate state that are commensurate with these passages.

The half-formed character of odorous bodies has an important consequence. There are no fixed kinds of smells. While we are capable of smelling a variety of odours, these lack names since they are not associated with the simple forms nor with compounds of these. The simple forms, here, are forms of primary bodies. Recall the primary bodies are only nameable thanks to the Demiurge providentially imposing ``form and number'' upon them (chapter~\ref{sec:an_emph_aporia}). In the intermediate stage of elemental transformation this partially and temporarily breaks down with the result that naming is no longer possible. Thus odours, unlike tastes and colors, lack an oppositional classification scheme. Perhaps Timaeus was being prescient here. Many psychologists and philosophers of mind today deny that there is a quality space associated with olfaction.

While the variety of odours that we smell lack names they nevertheless admit of rough classification as ``pleasant'' and ``painful'' (though perhaps in English it is more natural to describe this latter class as unpleasant). Unpleasant odours violently affect the thorax, the region between the head and the navel, and roughens it, while pleasant odours relax and restore the thorax to its natural condition. There is a contrast here with pleasant tastes. In the case of taste, what is restored are the affections that are primarily involved in flavour perception---the roughening of the tongue, excessive contraction and dilation of its veins. What corresponds to these in the case of olfaction are odorous bodies fitting into the veins in the nostrils. But it is not these affections that involve a departure or restoration of the natural state of the body, but a further effect on the thorax, communicated by the veins themselves via the heart.

Though Aristotle agrees with Timaeus that the four primary bodies lack smell (\emph{De sensu} 5 443a10), against Timaeus he will insist that there are two kinds of smell (\emph{De sensu} 5 443b17ff). Some odours, odours of nutriments, are pleasant incidentally. Think of how good a meal can smell when hungry. Other odours are pleasant in themselves, regardless of the state of the percipient. Echoing a claim made in the \emph{Philebus}, Aristotle cites the pleasant fragrance of flowers as an example. Unfortunately, Aristotle merely asserts, without argument, that it is untrue to say that smell does not admit of species, providing us with no clue as to his criticism of Timaeus. 

Theophrastus, however, explicitly registers a doubt about Timaeus' claim that odours lack fixed kinds (\emph{De sensibus} 90). Just as there are different flavours aren't there different odours? And if there are, aren't there different kinds of odours? Timaeus freely admits there are different odours. And there is a sense in which Theophrastus is surely right about the consequence of this, that there are different kinds of odours. But perhaps the sense in which Theophrastus is right does not align with the sense of Timaeus' denial. After all, Timaeus does not simply deny that there are different kinds of odours, he denies that they are fixed kinds of odours. Consider the case of color. There are fixed kinds of colors since the intermediary colors stand in various proportions to white and black as opposites. The temporary and partial departure from the providential order imposed by the Demiurge in the transformation of air into water and water into air prevents any such analysis. Timaeus is not denying, say, that a rose smells different from honeysuckle. He is merely claiming that due to their half-formed nature these odours are not sufficiently ordered to be named. That is what the qualification, ``fixed'', is alluding to here.

Though Theophrastus' objection to Timaeus' account of olfaction is unsuccessful, it does, at the very least, raise a question  about its completeness. He has explained a body's being odorous with its being commensurate with the veins in the nostrils leading within. But Timaeus has not explained the variety of the affections that are liable to give rise to the experience of different odours. So far this is just a lacuna in Timaeus' account. It might be addressed by postulating a variety of sizes of veins in the nostrils or in some other way. But if his explanation for why the odours may not be named would preclude a more elaborate account of the variety of affections, then Theophrastus' objection would be vindicated in the end. In the same passage, Theophrastus makes a parallel point about the agents of these affections, complaining that Timaeus identifies mist and vapour.

% section the_notrils (end)

\section{The Ears} % (fold)
\label{sec:the_ears}

The ears, for Timaeus, lack significant anatomical structure. They are nothing more than passages or conduits for external motion that results from the blow of the air. And while Timaeus may be following his predecessors in conceiving of the ears as passages through which the movement of the air is conducted so as to impact on the brain or some other internal organ (\citealt[94]{Beare:1906uq}), Timaeus' anatomy remains strikingly schematic. In \emph{De sensibus} (9, 25, 28, 38, 40), Theophrastus attributes this kind of account to Empedocles, Alcmaeon of Croton, Anaxagoras, Clidemus, and Diogenes of Apollonia. While Theophrastus' reports of Anaxagoras, Clidemus, and Diogenes of Apollonia provide no anatomical detail (\emph{De sensibus} 28, 38, 39), both Alcmaeon of Croton and Empedocles are said to discern significant anatomical structure. Thus, Alcmaeon of Croton distinguishes an inner and outer ear. The sound resonates in a cavity of the outer ear which is echoed in the inner ear thus passing on the motion to the brain (\emph{De sensibus} 25). Notice, implicit in Alcmaeon of Croton's description is a boundary between the inner and outer ear which corresponds to the tympanic membrane. And according to Theophrastus (\emph{De sensibus} 9), Empedocles posited a ``fleshy shoot'' as special part of the ear that functions as a ``bell''. Though crude by contemporary anatomical standards, at least Alcmaeon of Croton and Empedocles posit significant anatomical structure whose function contributes to the causal process involved in hearing. No such structure is posited by Timaeus, the ears being reduced to sheer Empedoclean \emph{poroi}.

Timaeus provides accounts of sound and hearing (67a--b) in terms of a movement and countermovement. Sound (\emph{phōne}) is the blow (\emph{plēgē}) of the air upon the brain and the blood through the ears that reaches as far as the soul. Hearing is caused by this impact and is a movement from the head to the region of the liver. These claims are compressed and can be variously interpreted. Let's consider these in turn.

The account of sound has a number of distinct components. There is:
\begin{enumerate}
	\item the blow of the air
	\item the ears
	\item the brain and the blood
	\item the soul
\end{enumerate}

There is a crucial ambiguity that affects how these components are related. (On the ambiguity see \citealt[246 n7]{Archer-Hind:1888qd}, \citealt[99--100]{Cook-Wilson:1889cs}, \citealt[476--7]{Taylor:1928qb}, \citealt[142 n38]{OBrien:1984ji}, \citealt[235, n1]{Lautner:2005aa}.) It is clear that the blow of the air is delivered through (\emph{dia}) the ears. But is the blow delivered as well through the brain and the blood? The grammatical question is whether \emph{dia} governs \emph{egkephalou} and \emph{haimatos} as well. Aëtius (\emph{Placita} 4.16.1 = \emph{Doxographi Graeci} 406) maintains that it does. On this reading the blow of the air passes through, the ears, brain, and the blood and so impacts upon the soul. Theophrastus (\emph{De sensibus} 6, 85) in effect denies that \emph{dia} governs \emph{egkephalou} and \emph{haimatos}. On this reading the blow of the air passes through the ears and impacts upon the brain and the blood before reaching the soul. In the paraphrase of this passage that I gave in previous paragraph, I have followed Theophrastus' reading, a reading endorsed by Archer-Hind, Cook Wilson, Taylor, O'Brien, and Lautner. 

The ambiguity is relevant not only to how the components of Timaeus' account of sound are related, but, what is less well appreciated, it is crucial to identifying the \emph{pathēma} or \emph{pathos} involved in hearing as well. Recall, the account of sound occurs in Timaeus' extended discussion of the \emph{pathēmata} peculiar to particular parts of the body. Moreover, the \emph{pathēmata} are causal intermediaries between the objects of perception and the perceptions or sensations that they are liable to give rise to (chapter~\ref{sec:pathemata}). That means that the \emph{pathēma} is not identified in Timaeus' account of hearing, a form of perception. On the Aëtian reading, it is unclear what the \emph{pathēma} could be. The blow of the air is said to impact upon the soul, but the soul is not a particular part of the body. On the Theophrastian reading, however, the identity of the relevant \emph{pathēma} is clear, it is the impact of the air upon the brain and the blood. The brain and the blood are the primary recipients of the affection in the causal process that gives rise to audition.

The ears, then, are not the primary recipients of the affection in audition but are merely a conduit through which the blow of the air passes before affecting interior parts of the body distinct from the ears. They differ in this way from the veins in the tongue which are affected by the action of the nutriment. The ears, by contrast, are more like the mouth cavity through the nutriment passes before affecting interior parts. The ears also differ from the veins in the nostrils. Odorous particles must be commensurate with these passages, neither too large nor too small, if they are to be sensed. The blow of the air, by contrast, need not be commensurate with the passages in the ears. They are more like the nostrils through which the half-formed particles must pass before entering the veins with which they are commensurate. It is unsurprising, then, that Timaeus discerns no significant anatomical structure in the ears. They are not the particular part of the body that receives the affection involved in audition.

The blow of the air is an event that may occur independently of a percipient. It would exist even should it not impact upon a hearer. Aristotle's disagreement with Timaeus in \emph{De anima} 2.8 419b20--21 highlights a distinctive feature of Timaeus' account. Aristotle agrees with Timaeus that there must be a blow of some kind in the causal process that eventuates in audition, for a blow produces sound (\emph{De anima} 2.8 419b11--13). But he denies that the air is the agent of this action. The blow occurs between two solid objects and the air is merely the medium through which this blow is communicated (\emph{De anima} 2.8 419b9--11). Timaeus, by contrast, makes the air the agent of the blow. 

This blow impacts upon the brain and the blood. Theophrastus reports that the view that the blow impacts specifically upon the brain has antecedents in Alcmaeon of Croton, Anaxagoras, and Diogenes of Apollonia (\emph{De sensibus} 26, 28, 40), though no explicit mention of blood is made in these reports. Here, questions arise. Is the impact upon the brain and the blood simultaneous, or is the blow somehow communicated to the brain through the blood? Or perhaps the reverse is the case, with the brain subsequently communicating the affection to the liver through the blood \citep[477]{Taylor:1928qb}. It ought to be clear, at least in the case of audition, how the relevant \emph{pathēma} depends upon the flesh and things pertaining to it and the mortal part of the soul. Timaeus will discuss the mortal part of the soul at 69c5--72d3, the brain at 73c6--d2, blood at 70a7--c1 and 80d1--81b5, and the liver at 71a3--72c1. A full answer to our questions, or at least as full an answer as the text will allow, will only emerge when we discuss these.

The reference to the soul is also in need of interpretation. Is it the mortal or immortal soul that the blow reaches? The immortal soul, the divine part of the soul, is rational and is embodied in the brain. This might suggest that it is the immortal soul that the blow of the air reaches. But Timaeus' discussion of the mortal soul (69c5ff) at least suggests that perception is a power of the mortal soul. But if it is, then it is located apart from the brain and the head. So at which part of the soul does the blow reach? There is a further interpretative issue as well. How, exactly, are we to understand the blow reaching the soul (\emph{mechri psuchēs})? Does the motion of the blow terminate at the soul? This coheres well with the Aëtian interpretation of Timaeus' account (and would be abetted by a Plutarchian interpretation of the motion of the soul), but it is not forced on one if one accepts, instead, the Theophrastian alternative. Perhaps reaching the soul may be understood as the blow of the air being reported to the \emph{phronimon}. 

\citet[86]{Barker:2000dy} claims that ``nothing will count as sound until it has entered the body in an appropriate way.'' This can seem like a straightforward consequence of Timaeus' account. And those readers of the \emph{Timaeus} who see in it the account of perception that Socrates attributes to Protagoras in the \emph{Theaetetus} (such as \citealt{Cornford:1935fk} and \citealt{Kahn:2013ob}) may see this as further confirmation of their interpretation (compare also \emph{De anima} 3.2 425b29--31). It is clear that Timaeus has provided an account of heard sound. But an account of heard sound is consistent with sounds themselves existing independently of the body of a percipient. And this must be the case if I have correctly identified the \emph{pathēma} involved in the causal process eventuating in audition. The impact upon the brain and the blood is the relevant affection. And as I argued in chapter~\ref{sec:pathemata}, the object of perception is independent of the affection. Indeed, it is the agent that brings about this affection. If that is right, then sound is the blow of the air. And the blow of the air is an event that may occur independently of a percipient.

Consider now Timaeus' account of hearing. In general, hearing is a countermovement to the blow of the air impacting upon the brain and the blood. Specifically, Timaeus claims that it is a movement from the head to the region of the liver caused by the impact upon the brain and the blood. This account has three components:
\begin{enumerate}
	\item the movement
	\item from the head
	\item to the region of the liver
\end{enumerate}
Timaeus does not here make explicit the medium through which this movement is communicated. \citealt[477]{Taylor:1928qb} speculates that it may be carried by the blood. Psuedo-Timaeus Locri makes a similar suggestion, claiming that the motion is communicated by \emph{pneuma} by passages from the ears to the liver (\emph{De natura mundi et animae} 101a). The head is more generic than the brain though of couse the brain is in the head. But we should be cautious about assuming that this movement originates specifically in the brain. As we shall see, the head is the seat of the rational part of the soul and the region of the liver is the seat of the appetitive part. It is plausible, then, that Timaeus thinks that audition has both rational and appetitive effects. Indeed, as we have already seen, the rational effects of audition is the reason that the Demiurge providentially provides us with this perceptual capacity. Notice that hearing, here, seems to be identified with this movement. Hearing just is this movement from the head to the region of the liver. If that is right, then perhaps this movement constitutes the report of the air's blow upon the brain and the blood to the \emph{phronimon}, or perhaps its receipt.

Again, \emph{aisthēsis} may depend upon the \emph{pathēmata} but it depends as well upon the flesh and things pertaining to it and the mortal part of the soul. This latter dependency is clear in the case of hearing. We are not in a position to fully understand Timaeus' account of hearing until we understand more about the relevant anatomy and that which animates it. Not only will understanding more about the head and the liver shed light upon the rational and appetitive effects of audition but it will also provide insight into the Demiurge's providence.

Timaeus completes his account of the \emph{pathēmata} peculiar to the ears with a discussion of pitch, smoothness and harshness, and volume:
\begin{enumerate}
	\item high and low (67b6)
	\item smooth and harsh (67b6--7)
	\item loud and soft (67c1)
\end{enumerate}
These audible qualities, or perhaps their perception, depend upon properties of movement. Pitch depends upon the relative rapidity of the movement. A rapid movement is high-pitched while a slower movement is low-pitched. Smoothness and harshness depend upon the relative uniformity of the movement. A uniform movement is smooth while a non-uniform movement is harsh. Finally, volume depends upon how large the movement is. A large movement is loud and a small movement is soft. In addition, Timaeus announces that a discussion of the concords will be postponed (80a).

Timaeus' account of pitch is not terribly surprising, and the association of relative rapidity of motion with pitch seems to be common to many ancient writers. Thus the Pythagorean Archytas of Tarentum, who sent a ship to rescue Plato from Dionysius II (Diogenes Laertius, \emph{Vitae Philosophorum} 8.79--83; the Platonic \emph{Seventh Letter} 350a), writes:
\begin{quote}
	Now when things strike against our organs of perception, those that come swiftly and powerfully from the impacts appear high-pitched, while those that come slowly and weakly seem to be low-pitched. (Archytas, fragment 1, in Porphyry, \emph{Commentarius in Claudii Ptolemaei Harmonica} 56.5--57.27; \citealt[41]{Barker:1989pi})
\end{quote}
The swift and powerful motion proceeds from an impact. As is clear from an earlier passage in the fragment, the impact, here, is of one body upon another. Aristotle will himself give a similar, if importantly distinct, account (\emph{De anima} 2.8 420a--b, \emph{De Generatione Animalium} 5 786b--787a) as will Pseudo-Timaeus Locri (\emph{De natura mundi et animae} 101b). It is important to recognize that this conception of pitch is distinct from the conception to be found in modern acoustics. In modern acoustics, pitch has to do with the rate of vibration, but high-pitched and low-pitched sounds propagate with the same velocity. At best, this ancient tradition is a partial, if confused, approximation of the truth about pitch. Aristotle's own account is a better approximation than Timaeus' since he says that a high-pitched sound excites to a great extent in a short amount of time, whereas a lower pitched sound excites to a slight extent in a long amount of time (\emph{De anima} 2.8 420a29--32).

How are we to understand Timaeus' contrast between the smooth and the harsh? Pseudo-Timaeus Locri claims that it is the contrast between the musical and the unmusical:
\begin{quote}
	Those arranged according to musical intervals are melodic, but those not arranged in proper intervals are unmelodic and unharmonic. (Pseudo-Timaeus Locri, \emph{De natura mundi et animae}, 101b; \citealt[61]{Tobin:1984qf})
\end{quote}
\citet[477]{Taylor:1928qb} makes a similar suggestion. He understands the contrast as between sounds with pitch as opposed to mere noise. As an interpretation of the \emph{Timaeus}, this cannot be right. As \citet[85--6]{Barker:2000dy} emphasizes, Timaeus' concern with sound is focused on the musical (\emph{mousikē}) though in a more general sense that the narrowly musical. Indeed, the providential role of audition essentially involves the the hearing of melodic and harmonic intervals. Voice is included in the broader class that that Timaeus designates as \emph{mousikē} (47c7--d1). The Peripatetic author of \emph{De Audibilibus} (possibly Strato of Lampsacus, successor of Theophrastus as scholarch of the Lyceum) characterizes the roughness of a voice as follows:
\begin{quote}
	Voices are rough when the impact of all the breath upon the air is not single but divided and dispersed. (\emph{De Audibilibus} 803b2--3; T. Loveday and E. S. Foster in \citealt{Barnes:1984uq})
\end{quote}
If the impact of the air is divided and dispersed, then it is non-uniform. Following the Peripatetic author, perhaps Timaeus' contrast between the smooth and the harsh has to do with the relative evenness of tone, an aspect of timbre. Pseudo-Timaeus Locri and Taylor are, then, right at least to this extent---noises are utterly lacking in evenness of tone. Nevertheless, the contrast is being made within the musical as Timaeus understands it.

Timaeus' account of volume, like his account of pitch, is familiar from ancient sources. Thus, for example, in his discussion of animal voices (\emph{De Generatione Animalium} 5 786b24--787a10), Aristotle claims that loud voices are distinguished from soft voices by the amount of air they move, loud voices moving larger amounts of air and soft voices moving smaller amounts of air. And like the case of pitch, Timaeus' understanding of volume should be contrasted with that which is found in modern acoustics. Archer-Hind's \citeyearpar[246 n7]{Archer-Hind:1888qd} claim that Timaeus has discovered ``that the loudness of a sound is proportionate to the amplitude of the sound-wave'' is clearly anachronistic. Timaeus account may lie at or near the origin of the movement of thought that culminates with this claim of modern acoustics, but they ought not to be confused.

There is a difficulty in understanding Timaeus' account of audible qualities. Audible qualities are naturally understood to qualify sounds. And indeed ancient thinkers who hold similar views to Timaeus take audible qualities to be qualities of sound. However, as \citet[87--88]{Barker:2000dy} observes, the grammar of this passage seems to preclude this natural understanding, ``the feminine accusative forms in which the adjectives naming the attributes appear in agreement with {\sbl άκοή} (\emph{akoē}---`hearing') at the end of the previous clause''. Thus, Barker translates the relevant passage as follows:
\begin{quote}
	\ldots\ [let us state that] the movement generated by [sound], beginning from the head and ending in the region of the liver, is hearing; and as much of it is swift is high-pitched, and as much as is slower is lower-pitched; and that which is alike is even and smooth, while the opposite kind is rough. (Plato, \emph{Timaeus} 67b4--c; \citealt[98 n4]{Barker:2000dy}) 
\end{quote}
If Barker is right about this, and we take this passage, as Barker interprets it, at face value, then Timaeus seems to be saying that pitch, evenness of tone, and volume qualify, not sound, but the experience of sound. But that is an incredible view. Bear in mind that for Timaeus, sounds, even on the Aëtian interpretation, are independent of our hearing them. Hearing only occurs subsequently with the motion from the head to the region of the liver. In this way, Timaeus' view is crucially different from Democritus', DK 68B9. And yet audible qualities are meant to qualify, not these, but our experience of them. There is a puzzle here, for sounds without audible qualities are inaudible. And while we can make sense of a sound being inaudible relative to a percipient because of some deficiency in their capacity for audition, we cannot make sense of sounds being absolutely inaudible. Inaudible sounds are no sounds at all. Can we avoid this puzzle consistent with Barker's grammatical observation?

Allow me to speculate on the basis of principles at work in Timaeus' discussion of the \emph{pathēmata}. The first step is a shift of emphasis that the text may bear even on Barker's reading. We need only attribute to Timaeus a certain compression of expression, a literary proclivity evinced elsewhere in the text. Perhaps Timaeus has in mind the perception of pitch, evenness of tone, and volume. Rather than ascribing audible qualities to auditory experience, the suggestion is that Timaeus is describing our auditory experience of these qualities. Thus, for example, the idea would be that the experience of a high-pitch sound is a swift movement from head to the region of the liver, and not that the swift movement is high-pitched. That suggestion by itself suffices to avoid our puzzle, but it leaves Timaeus without an account of these audible qualities. For on the present suggestion, Timaeus has only told us what the perception of pitch, evenness of tone, and volume are, but he remains silent about pitch, evenness of tone, and volume themselves. If a \emph{pathēma} is the measure of the power of the agent that causes it, then the perception that eventuates from it is a cognizance of what's measured. And if we combine this thought with a plausible assumption, then Timaeus is, implicitly at least, committed to conceptions of these audible qualities in line with the ancient tradition. Specifically it is plausible to suppose that if the blow of the air is swift, so too will be the countermovement from the head to the region of the liver. And if it is slow, so too will be the countermovement. That is to say that the relative rapidity, uniformity, and size of the movement from head to to the region of the liver is inherited from the air's blow to the brain and the blood from which it eventuates. Thus corresponding to the swift countermovement, the hearing of a high-pitched sound, is a swift movement, sound, whose high-pitch consists in its being swift. In this way is audition a measure of its object. Admittedly this is speculative. But it avoids the \emph{aporia} to which Timaeus' account would otherwise be subject and does so consistent with principles at work in his discussion of the \emph{pathēmata}. 

Concerning Timaeus' account of sound Theophrastus objects as follows:
\begin{quote}
	Rather unsatisfactory too, is the definition he gives of sound: for this definition is not applicable to all creatures impartially; and although he tries, he does not state the causes of the sensation. Moreover he seems to be defining, not sound itself, whether inarticulate or vocal, but the sensory processes in us. (Theophrastus, \emph{De sensibus} 91; \citealt[149]{Stratton:1917vn})
\end{quote}
In this passage, Theophrastus presses three objections. 

First, Theophrastus objects that Timaeus' account of sound is not applicable to all animals. Recall that Timaeus' account of sound makes reference to aspects of human anatomy. Sound is the blow of the air upon the brain and the blood through the ears that reaches the soul. It can seem that it is unduly anthropocentric for it would be inapplicable to animals that are capable of audition but that are bloodless or lack a brain. Thus for example, cephalopods were long thought to be incapable of audition, until \citet{Hu:2009ov} discovered that a species of octopus, \emph{octopus vulgaris}, can register sound with the statocyst, a sac-like organ containing a mineralized mass and sensitive hairs. Though they do not hear very well, this primitive form of audition is used to sense nearby predators. While octopuses are sanguineous, they lack a brain. So, it would seem that Timaeus' account of sound is inapplicable to the sound that cephalopods hear. Initially, this seems like a good objection, but recall the consequence of identifying the \emph{pathēma} involved in hearing sound on the Theophrastian interpretation of Timaeus' account. The relevant \emph{pathēma} is the impact upon the brain and the blood by the blow of the air. Since, in general, sense objects are the powers of the agents that cause the \emph{pathēmata}, this has a consequence that Timaeus' account of sound does not, in fact, make essential reference to human anatomy. Sound is simply a blow of the air. Of course, Timaeus, unlike Aristotle (\emph{De anima} 2.8), makes no mention of hearing sound in water rather than air, but there is no in principle obstacle to extending his account appropriately so as to include cephalopods and other animals for whom water is the primary medium of audition.

Second, Theophrastus objects that Timaeus does not state the cause of sensation. This is strikingly odd since Timaeus explicitly characterizes sound and hearing as movement and countermovement. The countermovement which is the hearing of sound is caused by the impact upon the brain and the blood by the blow of the air. What could Theophrastus so much as mean here? A seeming inconsistency with the third objection perhaps provides some insight into Theophrastus' thinking.

Third, Theophrastus objects that Timaeus has accounted for the sensory processes eventuating in audition and not its objects. How is this third objection consistent with the second? If Timaeus has offered an account not of sound but of sensory processes in us, has he not offered, at least in part, an account of the causes of audition? But on the second objection, Timaeus offers no such cause, not even in part. Theophrastus believes that Timaeus has only purported to give an account of sound but in fact has only given us an account of the causal mechanisms in hearing. (Theophrastus might have been subject to interpretative bias here; his primary interest in his doxographical selection lies with the causal mechanisms involved in perception.) If that is right, then on Theophrastus' understanding, since Timaeus' account of sound is really an account of the causes of audition, at least those within us, Timaeus has nothing left to say about such causes given this conflation. In taking Timaeus account of sound instead to be an account of the causal processes eventuating in audition, Theophrastus might be encouraged by the reference to human anatomy in that account. This might have led him to think that Timaeus is offering, in his account of sound, only an account of the causal mechanisms that eventuate in audition. As I have explained, however, this is misleading. On the Theophrastian interpretation, only impacts upon the brain and the blood so much as could be the \emph{pathēmata} involved in audition, and sounds, the causes of these, must be independent of them. Interestingly, Theophrastus might also have been sensitive to Barker's grammatical observation. However, on the speculative interpretation that I have offered, Timaeus' account is not subject to Theophrastus' third objection. The character of the countermovement that constitutes hearing corresponds to the character of the blow to the air and so Timaeus' account is informative about the nature of audibilia.

% section the_ears (end)

\section{The Eyes} % (fold)
\label{sec:the_eyes}



% section the_eyes (end)

\section{Concluding Observations} % (fold)
\label{sec:concluding_observations}



% section concluding_observations (end)

% Chapter peculiar_pathemata (end)
