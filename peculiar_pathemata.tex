%!TEX root = /Users/markelikalderon/Documents/Git/timaeus/timaeus.tex

\chapter{Peculiar \emph{Pathēmata}} % (fold)
\label{cha:peculiar_pathemata}

\section{Common versus Peculiar \emph{Pathēmata}} % (fold)
\label{sec:common_versus_peculiar_pathemata}

Whereas the common \emph{pathēmata} are affections common the body as a whole (61d--65b), the peculiar \emph{pathēmata} are affections that are peculiar to particular parts of the body (65b--68e). Just as in the case of the common \emph{pathēmata}, Timaeus will explain the peculiar \emph{pathēmata} in terms of the powers of the agents that cause them and explain how these powers are named in terms of the \emph{pathēmata} they cause. Whereas the common \emph{pathēmata} are involved in the measure of tactile qualities, the peculiar \emph{pathēmata} are involved in the measure of sensory qualities involved in the other four special senses---taste, smell, audition, and vision. Again, though I am anachronistically appealing to Aristotelian vocabulary here, I am bracketing any assumption that the particular parts of the body upon which the peculiar \emph{pathēmata} fall are best conceived by Timaeus as sensory organs. Whether they are shall be discussed only once we have reviewed in detail the four peculiar \emph{pathēmata}. As we shall see, they are not akin to sensory organs, at least as Aristotle conceives of them, but are more like Empedoclean \emph{poroi}. The four particular parts of the body upon which the peculiar \emph{pathēmata} fall are:
\begin{enumerate}[(1)]
	\item the tongue (65b--66c) (section~\ref{sec:the_tongue})
	\item the nostrils (66c--67a) (section~\ref{sec:the_nostrils})
	\item the ears (or perhaps the brain and the blood) (46c--47e, 67a--c, 80a) (section~\ref{sec:the_ears})
	\item the eyes (45b--46c, 67c--68d) (section~\ref{sec:the_eyes})
\end{enumerate}
Notice that all the particular parts upon which the peculiar \emph{pathēmata} fall are in the head. These at least include what Timaeus earlier described as organs for the forethought of the soul (45b).

Why does Timaeus discuss the remaining four special senses in the order that he does? Perhaps the relative importance of pleasure and pain provides a clue. These, recall, were claimed to be the most important of the common \emph{pathēmata}. Moreover, I argued that their importance consisted in their ethical significance. Pleasure may incite the greatest of evil, while pain may deter us from the good (68d). Perhaps the present ordering of the four remaining special senses is in ascending order of ethical significance. Taste is, after all, closely connected with appetite, as is smell though less so, and appetite may lead us astray. Moreover, it was the flood of nutriment that initially disrupts the circles of the Same and the Different in the initial shock of embodiment, a disruption providentially accommodated by our being endowed with audition and vision. And Timaeus describes vision as the cause of the greatest benefit to us (47a2--6), and so, presumably, a greater benefit than what audition affords. That the four remaining special senses are ordered by ascending ethical significance is a speculative hypothesis only, but I see no other alternative explanation. (I emphasize the speculative characer of the hypothesis since inference to an explanation falls well short of inference to the best explanation despite the apparent practice of some philosophers.) Aristotle, in \emph{De anima}, will reverse this order, as does Theophrastus in his discussion of Plato in \emph{De sensibus}. Curiously the middle Platonist Alcinous in the \emph{Didaskalikos} does so as well, presumably under Peripatetic influence.

% section common_versus_peculiar_emph_pathemata (end)

\section{The Tongue} % (fold)
\label{sec:the_tongue}

The first class of \emph{pathēmata} that Timaeus discusses all fall upon the tongue. The \emph{pathēmata} involved in the experience of flavour are all brought about by certain contractions (\emph{sunkrisōs}) or dilations (\emph{diakrisōs}). Timaeus signals this explicitly (65c2--6). Nor is it the first time he has used this vocabulary. It has been explicitly deployed in his discussion of hot and cold (61d5–62a5), and in his discussion of pleasure and pain (64e), and it will reoccur in his discussion of color and vision (67c–68d). And many of the other \emph{pathēmata}, both common and peculiar, while not explicitly so described, may be. Thus, for example, the flesh in yielding to the hardness of a stone in its grasp is compacted. This uniformity is perhaps obscured in English translation given the breadth of the semantic field associated with these terms (thus obscuring as well a potential Empedoclean influence as these plausibly correspond to the operations of Love and Strife). In addition to contractions and dilations, Timaeus also claims that, more than any other sense, the experience of flavour involves the rough and the smooth. (More than the tactile sensation of the smooth and the rough? Either Timaeus is restricting his claim to the remaining four special senses, or he thinks that the tongue is more sensitive to the smooth and the rough than any other part of the flesh.)

There are small passages in the tongue leading to the heart. Timaeus describes these as veins (\emph{phlebia}) though there is no explicit indication that these passages are blood vessels. Timaeus only ever mentions them containing particles of earth and air (though since these may be covered with a moist film, presumably they contain particles of water as well.) Indeed, from the perspective of contemporary anatomy they seem to function less like blood vessels and more like nerves. Specifically, these veins function as ``testing instruments'' of the tongue. As instruments, they determine the flavour of the ingested nutriment, depending upon how the nutriment acts upon them. As instruments, they are the measure of the nutriment's flavour. Presumably these instruments are tests since, depending upon the flavour, the nutriment may be either swallowed or spat out if foul. Whether or not the veins in the tongue, with the function that Timaeus ascribes to them, are best understood as blood vessels, cannot be settled with reference to Timaeus' discussion of the tongue alone.

The veins in the tongue thus clearly function as Empedoclean \emph{poroi} (for the role of \emph{poroi} or passages in Empedocles' discussion of \emph{aisthēsis} see Plato \emph{Meno} 76 a--d, Theophrastus \emph{De Sensibus} 7, and \citealt{Beare:1906uq}). There is one notable difference, however. Whereas, for Empedocles, perceived effluences are proportional or commensurate with the \emph{poroi}, the nutriment need not be proportional with the veins in the tongue. However, as we shall see, this aspect of Empedoclean perceptual anatomy is manifest in Timaeus' account of olfaction. Pseudo-Timaeus Locri also emphasizes this Emepdoclean theme: 
\begin{quote}
	What happens with taste is similar to what happens with touch. For things are perceived as either astringent or smooth because of contraction and dilation, as well as because of their movement through the passages and their shapes. (Pseudo-Timaeus Locri, \emph{De natura mundi et animae} 100e; \citealt[59]{Tobin:1984qf})
\end{quote}
Though perhaps this is a Pythagorean approach to taste more broadly, if we accept that Alcmaeon of Croton was a Pythagorean, as Diogenes Laertius (\emph{Vitae Philosophorum} 8.83), Iamblichus (\emph{Vita Pythagorae} 104, 267), and Philoponus (\emph{In De anima} 2 88) attest. For both Theophrastus (\emph{De sensibus} 25) and Pseudo-Plutarch (\emph{Epitome} 5 24 = \emph{Doxographi Graeci} 407 a 12) attribute a similar view to Alcmaeon of Croton. Even if Alcmaeon was not a Pythagorean, hailing from Croton, a center of Pythagorean activity, he was undoubtedly at least influenced by their thought.

That the veins in the tongue lead to the heart may have influenced Aristotle's conviction that the heart is the primary sense organ. This would be ironic if so, since, in general, Timaeus' anatomy is influenced by not only Empedocles but also Alcmaeon of Croton who maintained that the brain, and not the heart, is centrally involved in perception and sensation:
\begin{quote}
	All the senses are connected in some way with the brain; consequently they are incapable of action if <the brain> is disturbed or shifts its position, for <this organ> stops up the passages through which the senses act. (Theophrastus, \emph{De sensibus} 26; \citealt[89--91]{Stratton:1917vn})
\end{quote}

In addition to containing veins leading to the heart, the tongue is described as moist and soft (features also mentioned by Alcmaeon of Croton and Pseudo-Timaeus Locri). Both of these features function in taste as well. Thus the moistness of the tongue is said to melt particles of earth prior to their entry into the veins of the tongue (though liquefication will play a larger and more systematic role in Aristotle's account of digestion in \emph{De anima} 2.4). And the softness of the tongue is relevant to its susceptibility to roughening.

Recall the methodological dilemma raised by Timaeus for his account of the \emph{pathēmata} (61c4–d5) (discussed in chapter~\ref{sec:pathemata}). On the one hand, one cannot adequately account for the \emph{pathēmata} without accounting for the origin of the flesh and things pertaining to it and the mortal parts of the soul; on the other hand, one cannot adequately account for the origin of the flesh and things pertaining to it and the mortal parts of the soul without accounting for the \emph{pathēmata}. Perhaps more than with the common \emph{pathēmata}, we can see how Timaeus' discussion of the \emph{pathēmata} presupposes at least the origins of the flesh and things pertaining to it. An account of \emph{pathēmata} peculiar to the tongue would not be possible without making assumptions about the nature and organization of human flesh. The tongue is human flesh. It is moist and soft and contains veins. The way it is affected so as to be liable to give rise to the experience of flavour would be inexplicable without making some such assumptions.

\emph{Pathēmata} peculiar to the tongue are involved in the sensation of:
\begin{enumerate}[(1)]
	\item astringent and harsh (65c6--d4)
	\item bitter and saline (65d4--65e4)
	\item pungent (65e4--66a2)
	\item sour (66a2--b7)
	\item sweet (66b7--c7)
\end{enumerate}
The flavours stand in a one--many oppositional structure. Sweet is not merely opposed to bitter, but to the whole range of alternative flavours. In this respect it is like the opposition of the center of the cosmos with the points that lie, in extremity, on its circumference, The flavours thus contrast, in different ways, with odours and colors. Odours lack any meaningful oppositional structure admitting only to a loose classification as pleasant and unpleasant. Colors, like flavours and unlike odours, have an oppositional structure, but the opposition is between white and black with the other colors arrayed as intermediaries between them. (This is controversial. Though \citealt[480--1]{Taylor:1928qb} accepts the claim, both \citealt{Brisson:1997qr} and \citealt{Ierodiakonou:2005ly} claim that Timaeus is committed to there being four primary colors while \citealt[277]{Cornford:1935fk} claims that there are only three. We shall see in section~\ref{sec:the_eyes} that there is a way to at least partially reconcile these seemingly competing views.) The chromatic oppositional structure is thus one--one as opposed to one--many. Similarly, the thermal opposition of hot and cold is one--one with intermediary temperatures---warm, lukewarm, cool--arrayed between them. This is the more usual pattern of oppositional classification in classical antiquity (see \citealt{Lloyd:1966ly} for discussion). What explains the one--many nature of the opposition here? We shall return to this question after discussing sweetness and its alternatives.

Allow me to make one further preliminary observation before discussing the \emph{pathēmata} involved in tasting the individual flavours. Timaeus begins his discussion of the \emph{pathēmata} peculiar to the tongue by claiming to return to what was omitted in his previous account. This is a backreference to his discussion of the flavours or juices among the \emph{genē}. As we shall see, this is a further confirmation of an important point argued in the previous chapter, namely, that the \emph{pathēmata} are merely the effects on sentient animate bodies, effects that may obtain among inanimate bodies as well.

The \emph{pathēmata} involved in the sensations of astringent and harsh consists in the drying and constriction of the veins of the tongue. Timaeus explains these affections in terms of the character of the primary bodies that compose the nutriment. Specifically, when particles of earth fall upon the moist and soft tongue, these have a tendency to melt with the result that the veins in the tongue constrict and become dryer. Presumably their melting is a consequence of earth's absorption of water. But if the water of the tongue is absorbed, it becomes dryer, and the constriction of the veins of the tongue is a consequence of this drying. When these particles are rougher this process is liable to give rise to a sensation of astringency, and when less rough this process is liable to give rise to a sensation of harshness. (According to Theophrastus, \emph{De sensibus} 66, Timaeus' claim about astringency echoes Democritus'.) Though Timaeus does not explicitly say so, the drying and constriction of the tongue associated with the sensations of astringency and harshness are a departure from the natural state of the tongue. This is what explains their opposition to sweetness. Sweet things are sweet in part by their restoring the tongue to its natural state, hence the pleasure associated with this sensation. 

The \emph{pathēmata} involved in the sensations of bitter and harsh are also affections of the veins in the tongue, though in this case the veins are acted upon in the manner of a cleansing agent or detergent that washes out the veins. Timaeus emphasizes not only that the nutriment acts upon the veins of the tongue as a detergent but does so excessively. I take Timaeus to mean here that this is a departure from the natural state of the tongue. Thus nutriment excessively acting as a detergent is said to dissolve the substance of the tongue, and perhaps such nutriment excessively opens up the tongue's passages as well. Moreover, as we shall see, this departure from the natural state of the tongue explains well these flavours' opposition to sweetness. The difference between the \emph{pathēmata} involved in the sensations of bitter and saline is a matter of degree. When the affection is more excessive, this process is liable to give rise to the sensation of bitterness. When it is less excessive and more agreeable, this process is liable to give rise to the sensation of saline.

The \emph{pathēmata} involved in pungent sensations are more complex and constitute a striking anticipation of the role of retronasal olfaction in taste. In this case the nutriment is both hot and is susceptible to sharing in the heat of the tongue itself. As a result of sharing in the heat of the tongue, the nutriment becomes softened and fiery. This burns the tongue. Moreover, since the nutriment is fiery, particles of the nutriment move upwards (since primary bodies naturally move in the direction of their native environment). These fiery particles cut all that the parts upon which they impinge (because of the acuteness of their angles, the sharpness of their edges, their smallness, and the rapidity of their motion). Think of the way that excessively hot food sensibly affects the nasal cavities. This process is liable to give rise to a pungent sensation. (Timaeus' account of the pungent is an elaboration of the one that Theophrastus, \emph{De sensibus} 67, attributes to Democritus.) Again, though Timaeus does not make this explicit, the \emph{pathēmata} involved in pungent sensations, since these include burning and cutting, are a departure from the natural state of the tongue as well as a departure from the natural state of the parts above it upon which the fiery particles impinge. And if that is right, then the opposition to sweetness is explained.

The \emph{pathēmata} involved in the experience of sourness or acidity is also complex (on this difficult passage see \citealt[chapter 6.4]{OBrien:1984ji}). In this case the nutriment consists in particles whose fineness is the result of decomposition or putrefaction. When these fine particles enter the veins of the tongue, they encounter there particles of earth and air that are proportional to them. Perhaps by proportional Timaeus means that they are like-sized, since these particles differ in shape. (As we shall see, this is the sense in which transparent fire particles are proportional to the fire particles that constitute the visual stream.) The fine particles encountering particles of earth and air proportional to them causes these to circulate. Moreover, the circulation of these particles leads to a process of fermentation. The circular motion involved in fermentation gives rise to bubbles. It does so in the following fashion. This circular motion or churning causes the particles to tumble and to fall into one another, as spaces in the mixture open up, with the result that air particles become encased by a moist film and so create bubbles. There are two kinds of bubbles depending upon the nature of the moist film. If the film is composed of relatively pure water, these are transparent and called ``bubbles''. If the moist film contains earth in addition to water, these are called ``frothing'' or ``fermenting''. Finally, Timaeus claims that the cause of this process is called ``sour''. So sourness is the power of the putrefying nutriment to cause this process of fermentation in the veins of the tongue, and this affection is its measure.

One oddity of Timaeus' account, here, is that the earthy bubbles are said to rise. How could this be consistent with his claims about elemental motion? Shouldn't they sink? Perhaps, though Timaeus does not make this explicit (he is barely explicit about the presence of water), the earthy bubbles involved in frothing or fermentation contain fire as well. Perhaps like pungent nutriment, in the process of fermentation, the earthy bubbles share in the heat of the tongue. Also all water, for Timaeus, in its liquid form contains fire particles. There is also a hint of the presence of fire in the transparency of the bubbles whose films are composed of pure water since Timaeus will go on to explain transparency in terms of the size of fire particles (though, as we shall see, strictly speaking, Timaeus provides no explicit account of transparent bodies). The films of the earthy bubbles must contain more fire than the films of bubbles composed of water alone since these do not rise. It is natural to understand that the transparency of the earthy bubbles is obscured by the presence of earth. One final piece of evidence for the unmentioned presence of fire: Earlier, in his discussion of the \emph{genē}, Timaeus describes a frothy substance \emph{opos}, secreted from juices, that dissolves the flesh by burning (60b3--5). \emph{Opos} is clearly sour, and we are being given here in this later passage a more elaborate account of its frothing. Moreover, if we take its description as burning literally, it must involve the presence of fire.

In Timaeus' account of sourness, earth, water, and air are explicitly said to be within the veins. If, in addition, Timaeus is implicitly committed to their be fire, then this may be partial evidence that veins are, after all, vessels of blood, since blood is composed of the four primary bodies.

Timaeus now turns to the \emph{pathēmata} involved in the experience of sweetness. He begins by highlighting sweetness' opposition to all the previous flavours. This is more significant than it might at first seem, for there is no unique affection with which to identify the \emph{pathēmata} involved in the sensation of sweetness. Sweet things restore the tongue to its natural state. That is what makes sweet things sweet. Timaeus mentions three specific affections potentially involved in the sensation of sweetness. Sweet nutriment may:
\begin{enumerate}[(1)]
	\item smooth the roughened tongue (again an echo of Democritus according to Theophrastus, \emph{De sensibus} 65)
	\item contract dilated veins leading from the tongue
	\item dilate constricted veins leading from the tongue
\end{enumerate}
Here I think that it is important to bear in mind a number of features already in play in Timaeus' account of flavour perception. First, recall, that the experience of flavour involves two pairs of affections: the rough and the smooth and contractions and dilations. Second, all along Timaeus has been emphasizing the natural state of the tongue and how non-sweet flavours all involve, in different ways, a departure from this natural state. Third, the tongue is naturally soft. Putting these together, the three actions of sweet nutriment are readily explicable. If the tongue is naturally smooth, any roughening of it will be a departure from its natural state. This is why only the smoothing of the tongue is mentioned in the process involved in sensing sweetness. The veins in the tongue, depending upon the activity of the nutriment, may be unnaturally contracted or unnaturally dilated. When the veins in the tongue are unnaturally dilated, the sweetness of the nutriment has the tendency to restore these to their natural state by contraction. And when the veins in the tongue are unnaturally contracted, the sweetness of the nutriment has the tendency to restore these to their natural state by dilation. In general, sweet nutriment relaxes and restores the tongue to its natural state. I have been careful to state that sweet nutriment potentially involve these three activities. If the veins of the tongue are not unnaturally dilated, sweet nutriment will not contract them. To do so would be a departure from the natural state of the tongue. And if the veins of the tongue are not unnaturally contracted, sweet nutriment will not dilate them. Again, to do so would be a departure from the natural state of the tongue. So the three \emph{pathēmata} associated with sweet sensations do not invariably occur in our experience of sweetness. What is essential to sweetness is the restoration of the tongue to its natural state. This is why sweetness is by its very nature pleasant.

This is relevant to the one--many nature of the oppositional structure of flavour. Sweetness restores the tongue to its natural state. But the departure from the tongue's natural state is multifarious. There is more than one way in which the tongue may depart from its natural state. This is why Timaeus conceives of the oppositional structure of flavour as one--many. Rather than conceiving of the oppositional structure of flavour on the model of colors, with sweet opposed to bitter, say, and with the others flavours arrayed as intermediaries between them, sweet is opposed to all the different ways in which the flavours involve a departure from the natural state of the tongue.

Recall that Timaeus began his discussion of the \emph{pathēmata} peculiar to the tongue by claiming to return to what was omitted in his previous account. This was a backreference not only to his earlier discussion of \emph{opos} (60b3–5) but to \emph{meli} (60a8--b3) as well. Though \emph{meli} is usually translated as honey it has a broader range of reference than what is produced by bees (see \citealt[254 n6]{Cornford:1935fk}). Of honey, so understood, Timaeus claims that it is an agent that relaxes and restores the passages in the mouth and by this means produces a sweet sensation. Just as his account of sourness is an elaboration of his remarks on \emph{opos}, his account of sweetness is an elaboration of his remarks on honey.

Theophrastus' objection to Timaeus' account of flavour is a variation of his more general objection that the sensory objects, according ot Timaeus, lack the requisite unity:
\begin{quote}
	Of the sapid substances, he fails to state what severally are their natures \ldots\ But what we seek---since the affections themselves are clear as day---is rather the reality behind them and why they produce this result. (Theophrastus, \emph{De Sensibus} 89; \citealt[149]{Stratton:1917vn})
\end{quote}
Again, I think that Theophrastus has got the emphasis of Timaeus' account wrong (see chapter~\ref{sec:hot_and_cold}). The objects of taste sensation are the powers of the nutriments that produce these affections. Knowledge of these effects suffice for knowing which powers these are. Of course, these powers are themselves explicable in terms of the elemental composition of the nutriment. We have seen examples of this with the flavours opposed to sweetness. But there is no guarantee that the unity which Theophrastus seeks will be found on this level. We have already seen that things may be hard either because they are composed of earth or because they are dense. The unity that Theophrastus seeks is only to be found on the level of the powers of the agents to cause the \emph{pathēmata}. It is the powers of the agents that cause these affections that is the reality behind them and why they result. And since Theophrastus ignores these, he sees only a difficulty for Timaeus' account.

% section the_tongue (end)

\section{The Nostrils} % (fold)
\label{sec:the_nostrils}

Nostrils may be passages leading within, but lining their interior are passages leading further within. Specifically, according to Timaeus, there are veins in the interior of the nostrils. It is these veins that receive odours. Timaeus correctly conceives of odours as massess of particulate matter. However, the particles that we smell are not directly associated with the primary bodies nor even compounds of these. Here, it is important to remember that the primary bodies, the Empedoclean ``roots'', are not, according to Timaeus, strictly speaking elements. If an element is the smallest part that composes a body, then, strictly speaking, the only elements are the elemental triangles that compose the primary bodies. The reason this is important is that none of the primary bodies are proportional (\emph{summetria}, here, curiously, appearing in the Ionian dialect---{\sbl ξυμμετρια} instead of {\sbl συμμετρια}) with the veins in the nostrils with the result that they lack smell. Particles of earth and water are too large, and the particles of air and fire are too small. What's required to sense an odour is particulate matter that is proportional with these passages and no primary body is. What is smelled is a ``half-formed class''. In the cycle of elemental transformation, specifically as air and water transform into one another, these particles break down into their elemental triangles and reform into polyhedra of the other primary body. Olfaction takes as its objects bodies in these intermediate states. This is the sense in which they are half-formed, and it is only the half-formed that are proportional with the veins in the nostrils.

That the odorous must be proportional with passages in the nostrils is due to the overt influence of Empedocles. Thus Theophrastus reports:
\begin{quote}
	For he [Empedocles] attributes our recognition of things to two factors---namely, to likeness and to contact; and so he uses the expression ``to fit''. Accordingly if the smaller touched the larger ones, there would be perception. And likeness also, speaking generally, is out of the question, at least according to him, and commensurateness alone suffices. For he says that substances fail to perceive one another because their passages are not commensurate. But whether the effluence is like or unlike he leaves quite undetermined. (Theophrastus, \emph{De Sensibus} 15; \citealt[79]{Stratton:1917vn})
\end{quote}
Timaeus reduplicates this Empedoclean scheme in his account of smell. We fail to perceive the primary bodies in olfaction since they are not proportional with the passages in the nostrils, the veins leading within. However, whereas for Empedocles this is a general claim, effluences are only perceptible if they fit the relevant passages, Timaeus, here, is only making this claim with respect to the special case of smell.

Recall that unlike his pre-Socratic predecessors, Timaeus denies the full cycle of elemental transformation. Because of the particular kind of triangles that compose earth, no other primary body transforms into earth, and earth transforms into no other primary body. In this respect at least, the full cycle of elemental transformation is merely apparent. Nevertheless a partial cycle of elemental transformation genuinely remains. Air, water, and fire may transform into one another by these polyhedra decomposing into their elemental triangles and recomposing into polyhedra of the other primary bodies. Only the transformation of air into water and water into air are relevant to olfaction. It is only when bodies are in the intermediary state in the process of these transformations will they be proportional with the veins in the nostrils. What transforms from air into water is mist, and what transfroms from water into air is vapour. And the particles that compose mist and vapour are smaller than water and bigger than air. Thus only that which is moistened, or putrefied, or melted, or vaporised, may be smelled. For only these give off bodies in the intermediate state that are proportional with these passages.

How exactly are we to understand bodies in a half-formed state? There are two interpretations:
\begin{enumerate}[(1)]
	\item The bodies are half-formed in the sense that they are in a transitional state between octahedra and icosahedra, and so consist of irregular bodies formed from their constituent triangles (Alcinous, \emph{Didaskalikos} 19.2, \citealt[275 n8]{Archer-Hind:1888qd}, \citealt[471]{Taylor:1928qb}, and \citealt[274]{Cornford:1935fk})
	\item The bodies are half-formed in the sense that the particulate matter consists of neither octahedra nor of icosahedra purely but consists, instead, of a combination of these polyhedra (\citealt{Vlastos:1967jw})
\end{enumerate}

Against Archer-Hind, Taylor, and Cornford, Vlastos plausibly suggests that they may have attributed to Timaeus and overly strict understanding of Emepdoclean fit. Perhaps proportional need not be understood as exactly like-sized, here. Rather, according to Vlastos, proportionality with the veins is a matter of the relative viscosity of the particulate matter that varies with the proportion of octahedra to icosahedra. \citet[206-7]{Vlastos:1967jw} postulates an instantaneous transition from octahedra to icosahedra (or icosahedra to octahedra, as the case may be) thus obviating the production of bodies with irregular figures. This is implausible, to my mind, and Timaeus never explicitly claims as much. This is not, however, the main difficulty with Vlastos' account. What \citet[204 n, 209]{Vlastos:1967jw} regards as the chief virtue of his account, the ``victory of persuasion over necessity'' in removing ``the blemish of odd-shaped atoms'' from Timaeus' account, is, ironically, its chief vice. For on Vlastos' interpretation, the lack of fixed kinds of odours is rendered inexplicable. If the odours result from mixtures of icosahedra and octahedra, they could be assigned fixed kinds based on the proportions in which they are mixed. But Timaeus does not claim, as in the case of color, that mortal intellects cannot know, let alone understand, such proportions. Instead, he makes the stronger claim that odours lack fixed kinds that such proportions would give rise to.

If the half-formed character of odorous bodies is understood in terms of their having irregular figures, then the Timaean claim that there are no fixed kinds of smells is readily explicable. While we are capable of smelling a variety of odours, these lack names since they are not associated with the simple forms nor with compounds of these. The simple forms, here, are forms of primary bodies, all of which have regular figures. Recall the primary bodies are only nameable thanks to the Demiurge providentially imposing ``form and number'' upon them (chapter~\ref{sec:an_emph_aporia}). In the intermediate stage of elemental transformation this partially and temporarily breaks down with the production of the irregular figures with the result that naming is no longer possible. Thus odours, unlike tastes and colors, lack an oppositional classification scheme. Perhaps Timaeus was being prescient here. Many psychologists and philosophers of mind today deny that there is a quality space associated with olfaction.

While the variety of odours that we smell lack names, they nevertheless admit of rough classification as ``pleasant'' and ``painful'' (though perhaps in English it is more natural to describe this latter class as unpleasant). In the roughness of his classification, and its hedonic valence, Timaeus is following his predecessors (\citealt{Baltussen:2015aa}). Unpleasant odours violently affect the thorax, the region between the head and the navel, and roughens it, while pleasant odours relax and restore the thorax to its natural condition. There is a contrast here with pleasant tastes. In the case of taste, what is restored are the affections that are primarily involved in flavour perception---the roughening of the tongue, excessive contraction and dilation of its veins. What corresponds to these in the case of olfaction are odorous bodies fitting into the veins in the nostrils. But it is not these affections that involve a departure or restoration of the natural state of the body, but a further effect on the thorax, communicated by the veins themselves perhaps via the heart.

Though Aristotle agrees with Timaeus that the four primary bodies lack smell (\emph{De sensu} 5 443a10), against Timaeus he will insist that there are two kinds of smell (\emph{De sensu} 5 443b17ff). Some odours, odours of nutriments, are pleasant incidentally. Think of how good a meal can smell when hungry. Other odours are pleasant in themselves, regardless of the state of the percipient. Echoing a claim made in the \emph{Philebus} 51b, Aristotle cites the pleasant fragrance of flowers as an example. Unfortunately, Aristotle merely asserts, without argument, that it is untrue to say that smell does not admit of species, providing us with no clue as to his criticism of Timaeus. 

Theophrastus, however, explicitly registers a doubt about Timaeus' claim that odours lack fixed kinds (\emph{De sensibus} 90). Just as there are different flavours are there not different odours? And if there are, are there not different kinds of odours? Timaeus freely admits there are different odours. And there is a sense in which Theophrastus is surely right about the consequence of this, that there are different kinds of odours. But perhaps the sense in which Theophrastus is right does not align with the sense of Timaeus' denial. After all, Timaeus does not simply deny that there are different kinds of odours, he denies that they are fixed kinds of odours. Consider the case of color. There are fixed kinds of colors since the colors stand in various proportions to one another. The temporary and partial departure from the providential order imposed by the Demiurge in the transformation of air into water and water into air prevents any such analysis. Timaeus is not denying, say, that a rose smells different from honeysuckle. He is merely claiming that due to their half-formed nature these odours are not sufficiently ordered to be named. Moreover, there are half-formed since they are in the process of transformation. That is what the qualification, ``fixed'', is alluding to here.

Though Theophrastus' objection to Timaeus' account of olfaction is unsuccessful, it does, at the very least, raise a question  about its completeness. He has explained a body's being odorous with its being proportional with the veins in the nostrils leading within, perhaps on the liberal understanding of this that Vlastos urges. But Timaeus has not explained the variety of the affections that are liable to give rise to the experience of different odours. So far this is just a lacuna in Timaeus' account. It might be addressed by postulating a variety of sizes of veins in the nostrils or in some other way. But if his explanation for why the odours may not be named would preclude an explanation of the variety of affections, then Theophrastus' objection would be vindicated in the end. In the same passage, Theophrastus makes a parallel point about the agents of these affections, complaining that Timaeus identifies mist and vapour. Each is an intermediate state in the transformation of either air into water or water into air. So how may they differ? And if they don't differ, then they won't differ in their effects.

% section the_notrils (end)

\section{The Ears} % (fold)
\label{sec:the_ears}

The ears, for Timaeus, lack significant anatomical structure. They are nothing more than passages or conduits for external motion that results from the blow of the air. And while Timaeus may be following his predecessors in conceiving of the ears as passages through which the movement of the air is conducted so as to impact on the brain or some other internal organ (\citealt[94]{Beare:1906uq}), Timaeus' anatomy remains strikingly schematic. In \emph{De sensibus} (9, 25, 28, 38, 40), Theophrastus attributes this kind of account to Alcmaeon of Croton, Empedocles, Anaxagoras, Clidemus, and Diogenes of Apollonia. While Theophrastus' reports of Anaxagoras, Clidemus, and Diogenes of Apollonia provide no anatomical detail (\emph{De sensibus} 28, 38, 39), both Alcmaeon of Croton and Empedocles are said to discern significant anatomical structure. Thus, Alcmaeon of Croton distinguishes an inner and outer ear. The sound resonates in a cavity of the outer ear which is echoed in the inner ear thus passing on the motion to the brain (\emph{De sensibus} 25). Notice, implicit in Alcmaeon of Croton's description is a boundary between the inner and outer ear which corresponds to the tympanic membrane. And according to Theophrastus (\emph{De sensibus} 9), Empedocles posited a ``fleshy shoot'' as special part of the ear that functions as a ``bell''. Though crude by contemporary anatomical standards, at least Alcmaeon of Croton and Empedocles posit significant anatomical structure whose function contributes to the causal process involved in hearing. No such structure is posited by Timaeus, the ears being reduced to sheer passages.

Timaeus provides accounts of sound and hearing (67a--b) in terms of a movement and countermovement. Sound (\emph{phōne}) is the blow (\emph{plēgē}) of the air upon the brain and the blood through the ears that reaches as far as the soul. Hearing is caused by this impact and is a movement from the head to the region of the liver. These claims are compressed and can be variously interpreted. Let's consider these in turn.

The account of sound has a number of distinct components. There is:
\begin{enumerate}[(1)]
	\item the blow of the air
	\item the ears
	\item the brain and the blood
	\item the soul
\end{enumerate}

There is a crucial ambiguity that affects how these components are related. (On the ambiguity see \citealt[246 n7]{Archer-Hind:1888qd}, \citealt[99--100]{Cook-Wilson:1889cs}, \citealt[476--7]{Taylor:1928qb}, \citealt[142 n38]{OBrien:1984ji}, \citealt[235, n1]{Lautner:2005aa}.) It is clear that the blow of the air is delivered through (\emph{dia}) the ears. But is the blow delivered as well through the brain and the blood? The grammatical question is whether \emph{dia} governs \emph{egkephalou} and \emph{haimatos} as well. Aëtius (\emph{Placita} 4.16.1 = \emph{Doxographi Graeci} 406) maintains that it does, as does Alcinous (\emph{Didaskalikos} 19.1). On this reading the blow of the air passes through, the ears, brain, and the blood and so impacts upon the soul. Theophrastus (\emph{De sensibus} 6, 85) in effect denies that \emph{dia} governs \emph{egkephalou} and \emph{haimatos}, as does Plutarch (\emph{Platonicae quaestiones} 7.9 1006b3--6). On this reading the blow of the air passes through the ears and impacts upon the brain and the blood before reaching the soul. In the paraphrase of this passage that I gave in previous paragraph, I have followed Theophrastus' reading, a reading endorsed by Archer-Hind, Cook Wilson, Taylor, O'Brien, and Lautner. 

The ambiguity is relevant not only to how the components of Timaeus' account of sound are related, but, what is less well appreciated, it is crucial to identifying the \emph{pathēma} or \emph{pathos} involved in hearing as well. Recall, the account of sound occurs in Timaeus' extended discussion of the \emph{pathēmata} peculiar to particular parts of the body. Moreover, the \emph{pathēmata} are causal intermediaries between the objects of perception and the perceptions or sensations that they are liable to give rise to (chapter~\ref{sec:pathemata}). That means that the \emph{pathēma} is not identified in Timaeus' account of hearing, a form of perception. On the Aëtian reading, it is unclear what the \emph{pathēma} could be. The blow of the air is said to impact upon the soul, but the soul is not a particular part of the body. On the Theophrastian reading, however, the identity of the relevant \emph{pathēma} is clear, it is the impact of the air upon the brain and the blood. The brain and the blood are the primary recipients of the affection in the causal process that gives rise to audition.

The ears, then, are not the primary recipients of the affection in audition but are merely a conduit through which the blow of the air passes before affecting interior parts of the body distinct from the ears. They differ in this way from the veins in the tongue which are affected by the action of the nutriment. The ears, by contrast, are more like the mouth cavity through the nutriment passes before affecting interior parts. The ears also differ from the veins in the nostrils. Odorous particles must be proportional with these passages, neither too large nor too small, if they are to be sensed. The blow of the air, by contrast, need not be proportional with the passages in the ears. They are more like the nostrils through which the half-formed particles must pass before entering the veins with which they are proportional. It is unsurprising, then, that Timaeus discerns no significant anatomical structure in the ears. They are not the particular part of the body that receives the affection involved in audition.

The blow of the air is an event that may occur independently of a percipient. It would exist even should it not impact upon a hearer. Aristotle's disagreement with Timaeus in \emph{De anima} (2.8 419b20--21) highlights a distinctive feature of Timaeus' account. Aristotle agrees with Timaeus that there must be a blow of some kind in the causal process that eventuates in audition, for a blow produces sound (\emph{De anima} 2.8 419b11--13). But he denies that the air is the agent of this action. The blow occurs between two solid bodies and the air is merely the medium through which this blow is communicated (\emph{De anima} 2.8 419b9--11). Timaeus, by contrast, makes the air the agent of the blow. 

This blow impacts upon the brain and the blood. Theophrastus reports that the view that the blow impacts specifically upon the brain has antecedents in Alcmaeon of Croton, Anaxagoras, and Diogenes of Apollonia (\emph{De sensibus} 26, 28, 40), though no explicit mention of blood is made in these reports. Here, questions arise. Is the impact upon the brain and the blood simultaneous, or is the blow somehow communicated to the brain through the blood? Or perhaps the reverse is the case, with the brain subsequently communicating the affection to the liver through the blood \citep[477]{Taylor:1928qb}. It ought to be clear, at least in the case of audition, how the relevant \emph{pathēma} depends upon the flesh and things pertaining to it and the mortal part of the soul. Timaeus will discuss the mortal part of the soul at 69c5--72d3, the brain at 73c6--d2, blood at 70a7--c1 and 80d1--81b5, and the liver at 71a3--72c1. A full answer to our questions, or at least as full an answer as the text will allow, will only emerge when we discuss these.

The reference to the soul is also in need of interpretation. Is it the mortal or immortal soul that the blow reaches? The immortal soul, the divine part of the soul, is rational and is embodied in the brain. This might suggest that it is the immortal soul that the blow of the air reaches. But Timaeus' discussion of the mortal soul (69c5ff) at least suggests that perception is a power of the mortal soul. But if it is, then it is located apart from the brain and the head. So at which part of the soul does the blow reach? There is a further interpretative issue as well. How, exactly, are we to understand the blow reaching the soul (\emph{mechri psuchēs})? Does the motion of the blow terminate at the soul? This coheres well with the Aëtian interpretation of Timaeus' account (and would be abetted by a Plutarchian interpretation of the motion of the soul), but it is not forced on one if one accepts, instead, the Theophrastian alternative. Perhaps reaching the soul may be understood as the blow of the air being reported to the \emph{phronimon}. 

\citet[86]{Barker:2000dy} claims that ``nothing will count as sound until it has entered the body in an appropriate way.'' This can seem like a straightforward consequence of Timaeus' account. And those readers of the \emph{Timaeus} who see in it the account of perception that Socrates attributes to Protagoras in the \emph{Theaetetus} (such as \citealt{Cornford:1935fk} and \citealt{Kahn:2013ob}) may see this as further confirmation of their interpretation (compare also \emph{De anima} 3.2 425b29--31). It is clear that Timaeus has provided an account of heard sound. But an account of heard sound is consistent with sounds themselves existing independently of the body of a percipient. And this must be the case if I have correctly identified the \emph{pathēma} involved in the causal process eventuating in audition. The impact upon the brain and the blood is the relevant affection. And as I argued in chapter~\ref{sec:pathemata}, the object of perception is independent of the affection. Indeed, it is the agent that brings about this affection. If that is right, then sound is the blow of the air. And the blow of the air is an event that may occur independently of a percipient.

Consider now Timaeus' account of hearing. In general, hearing is a countermovement to the blow of the air impacting upon the brain and the blood. Specifically, Timaeus claims that it is a movement from the head to the region of the liver caused by the impact upon the brain and the blood. This account has three components:
\begin{enumerate}[(1)]
	\item the movement
	\item from the head
	\item to the region of the liver
\end{enumerate}
Timaeus does not here make explicit the medium through which this movement is communicated. \citealt[477]{Taylor:1928qb} speculates that it may be carried by the blood. Psuedo-Timaeus Locri makes a similar suggestion, claiming that the motion is communicated by \emph{pneuma} by passages from the ears to the liver (\emph{De natura mundi et animae} 101a). Notice that Psuedo-Timaeus Locri makes no mention of the brain here. The head is more generic than the brain, though of course the brain is in the head. But we should be cautious about assuming that this movement originates specifically in the brain. As we shall see, the head is the seat of the rational part of the soul and the region of the liver is the seat of the appetitive part. It is plausible, then, that Timaeus thinks that audition has both rational and appetitive effects. Indeed, as we have already seen, the rational effects of audition is the reason that the Demiurge providentially provides us with this perceptual capacity. Notice that hearing, here, seems to be identified with this movement. Hearing just is this movement from the head to the region of the liver. If that is right, then perhaps this movement constitutes the report of the air's blow upon the brain and the blood to the \emph{phronimon}.

Again, \emph{aisthēsis} may depend upon the \emph{pathēmata} but it depends as well upon the flesh and things pertaining to it and the mortal part of the soul. This latter dependency is clear in the case of hearing. We are not in a position to fully understand Timaeus' account of hearing until we understand more about the relevant anatomy and that which animates it. Not only will understanding more about the head and the liver shed light upon the rational and appetitive effects of audition but it will also provide insight into the Demiurge's providence.

Timaeus completes his account of the \emph{pathēmata} peculiar to the ears with a discussion of pitch, smoothness and harshness, and volume:
\begin{enumerate}[(2)]
	\item high and low (67b6)
	\item smooth and harsh (67b6--7)
	\item loud and soft (67c1)
\end{enumerate}
These audible qualities, or perhaps their perception, depend upon properties of movement. Pitch depends upon the relative rapidity of the movement. A rapid movement is high-pitched while a slower movement is low-pitched. Smoothness and harshness depend upon the relative uniformity of the movement. A uniform movement is smooth while a non-uniform movement is harsh. Finally, volume depends upon how large the movement is. A large movement is loud and a small movement is soft. In addition, Timaeus announces that a discussion of the concords will be postponed (80a).

Timaeus' account of pitch is not terribly surprising, and the association of relative rapidity of motion with pitch seems to be common to many ancient writers. Thus the Pythagorean Archytas of Tarentum, who sent a ship to rescue Plato from Dionysius II (Diogenes Laertius, \emph{Vitae Philosophorum} 8.79--83; the Platonic \emph{Seventh Letter} 350a), writes:
\begin{quote}
	Now when things strike against our organs of perception, those that come swiftly and powerfully from the impacts appear high-pitched, while those that come slowly and weakly seem to be low-pitched. (Archytas, fragment 1, in Porphyry, \emph{Commentarius in Claudii Ptolemaei Harmonica} 56.5--57.27; \citealt[41]{Barker:1989pi})
\end{quote}
The swift and powerful motion proceeds from an impact. As is clear from an earlier passage in the fragment, the impact, here, is of one body upon another. Aristotle will himself give a similar, if importantly distinct, account (\emph{De anima} 2.8 420a--b, \emph{De Generatione Animalium} 5 786b--787a) as will Pseudo-Timaeus Locri (\emph{De natura mundi et animae} 101b). It is important to recognize that this conception of pitch is distinct from the conception to be found in modern acoustics. In modern acoustics, pitch has to do with the rate of vibration, but high-pitched and low-pitched sounds propagate with the same velocity. At best, this ancient tradition is a partial, if confused, approximation of the truth about pitch. Aristotle's own account is a better approximation than Timaeus' since he says that a high-pitched sound excites to a great extent in a short amount of time, whereas a lower pitched sound excites to a slight extent in a long amount of time (\emph{De anima} 2.8 420a29--32).

How are we to understand Timaeus' contrast between the smooth and the harsh? Pseudo-Timaeus Locri claims that it is the contrast between the musical and the unmusical:
\begin{quote}
	Those arranged according to musical intervals are melodic, but those not arranged in proper intervals are unmelodic and unharmonic. (Pseudo-Timaeus Locri, \emph{De natura mundi et animae}, 101b; \citealt[61]{Tobin:1984qf})
\end{quote}
\citet[477]{Taylor:1928qb} makes a similar suggestion. He understands the contrast as between sounds with pitch as opposed to mere noise. As an interpretation of the \emph{Timaeus}, this cannot be right. As \citet[85--6]{Barker:2000dy} emphasizes, Timaeus' concern with sound is focused on the musical (\emph{mousikē}) though in a more general sense that the narrowly musical. Indeed, the providential role of audition essentially involves the the hearing of melodic and harmonic intervals. Voice is included in the broader class that that Timaeus designates as \emph{mousikē} (47c7--d1). The Peripatetic author of \emph{De Audibilibus} (possibly Strato of Lampsacus, successor of Theophrastus as scholarch of the Lyceum) characterizes the roughness of a voice as follows:
\begin{quote}
	Voices are rough when the impact of all the breath upon the air is not single but divided and dispersed. (\emph{De Audibilibus} 803b2--3; T. Loveday and E. S. Foster in \citealt{Barnes:1984uq})
\end{quote}
If the impact of the air is divided and dispersed, then it is non-uniform. Following the Peripatetic author, perhaps Timaeus' contrast between the smooth and the harsh has to do with the relative evenness of tone, an aspect of timbre. Pseudo-Timaeus Locri and Taylor are, then, right at least to this extent---noises are utterly lacking in evenness of tone. Nevertheless, the contrast is being made within the musical as Timaeus understands it.

Timaeus' account of volume, like his account of pitch, is familiar from ancient sources. Thus, for example, in his discussion of animal voices (\emph{De Generatione Animalium} 5 786b24--787a10), Aristotle claims that loud voices are distinguished from soft voices by the amount of air they move, loud voices moving larger amounts of air and soft voices moving smaller amounts of air. And like the case of pitch, Timaeus' understanding of volume should be contrasted with that which is found in modern acoustics. Archer-Hind's \citeyearpar[246 n7]{Archer-Hind:1888qd} claim that Timaeus has discovered ``that the loudness of a sound is proportionate to the amplitude of the sound-wave'' is clearly anachronistic. Timaeus account may lie at or near the origin of the movement of thought that culminates with this claim of modern acoustics, but they ought not to be confused.

There is a difficulty in understanding Timaeus' account of audible qualities. Audible qualities are naturally understood to qualify sounds (see, for example, Aristotle \emph{De anima} 2 11 422b31–2). And indeed ancient thinkers who hold similar views to Timaeus take audible qualities to be qualities of sound. However, as \citet[87--88]{Barker:2000dy} observes, the grammar of this passage seems to preclude this natural understanding, ``the feminine accusative forms in which the adjectives naming the attributes appear in agreement with {\sbl άκοή} (\emph{akoē}---`hearing') at the end of the previous clause''. Thus, Barker translates the relevant passage as follows:
\begin{quote}
	\ldots\ [let us state that] the movement generated by [sound], beginning from the head and ending in the region of the liver, is hearing; and as much of it is swift is high-pitched, and as much as is slower is lower-pitched; and that which is alike is even and smooth, while the opposite kind is rough. (Plato, \emph{Timaeus} 67b4--c; \citealt[98 n4]{Barker:2000dy}) 
\end{quote}
If Barker is right about this, and we take this passage, as Barker interprets it, at face value, then Timaeus seems to be saying that pitch, evenness of tone, and volume qualify, not sound, but the experience of sound. But, as Barker himself acknowledges, that is an incredible view. Bear in mind that for Timaeus, sounds, even on the Aëtian interpretation, are independent of our hearing them. Hearing only occurs subsequently with the motion from the head to the region of the liver. In this way, Timaeus' view is crucially different from Democritus' (DK 68B9) and from the view that Socrates attributes to Protagoras in the \emph{Theaetetus}. And yet audible qualities are meant to qualify, not these, but our experience of them. There is a puzzle here, for sounds without audible qualities are inaudible. And while we can make sense of a sound being inaudible relative to a percipient because of some deficiency in their capacity for audition, we cannot make sense of sounds being absolutely inaudible. Inaudible sounds are no sounds at all. Can we avoid this puzzle consistent with Barker's grammatical observation?

Allow me to speculate on the basis of principles at work in Timaeus' discussion of the \emph{pathēmata}. The first step is a shift of emphasis that the text may bear even on Barker's reading. We need only attribute to Timaeus a certain compression of expression, a literary proclivity evinced elsewhere in the text. Perhaps Timaeus has in mind the perception of pitch, evenness of tone, and volume. Rather than ascribing audible qualities to auditory experience, the suggestion is that Timaeus is describing our auditory experience of these qualities. Thus, for example, the idea would be that the experience of a high-pitch sound is a swift movement from head to the region of the liver, and not that the swift movement is high-pitched. That suggestion by itself suffices to avoid our puzzle, but it leaves Timaeus without an account of these audible qualities. For on the present suggestion, Timaeus has only told us what the perception of pitch, evenness of tone, and volume are, but he remains silent about pitch, evenness of tone, and volume themselves. If a \emph{pathēma} is the measure of the power of the agent that causes it, then the perception that eventuates from it is a cognizance of what's measured. And if we combine this thought with a plausible assumption, then Timaeus is, implicitly at least, committed to conceptions of these audible qualities in line with the ancient tradition. Specifically, it is plausible to suppose that if the blow of the air is swift, so too will be the countermovement from the head to the region of the liver. And if it is slow, so too will be the countermovement. That is to say that the relative rapidity, uniformity, and size of the movement from head to to the region of the liver is inherited from the air's blow to the brain and the blood from which it eventuates. Thus corresponding to the swift countermovement, the hearing of a high-pitched sound, is a swift movement, sound, whose high-pitch consists in its being swift. In this way is audition a measure of its object. Admittedly this is speculative. But it avoids the \emph{aporia} to which Timaeus' account would otherwise be subject and does so consistent with principles at work in his discussion of the \emph{pathēmata}. 

Concerning Timaeus' account of sound, Theophrastus objects as follows:
\begin{quote}
	Rather unsatisfactory too, is the definition he gives of sound: for this definition is not applicable to all creatures impartially; and although he tries, he does not state the causes of the sensation. Moreover he seems to be defining, not sound itself, whether inarticulate or vocal, but the sensory processes in us. (Theophrastus, \emph{De sensibus} 91; \citealt[149]{Stratton:1917vn})
\end{quote}
In this passage, Theophrastus presses three objections. 

First, Theophrastus objects that Timaeus' account of sound is not applicable to all animals. Recall that Timaeus' account of sound makes reference to aspects of human anatomy. Sound is the blow of the air upon the brain and the blood through the ears that reaches the soul. It can seem that it is unduly anthropocentric for it would be inapplicable to animals that are capable of audition but that are bloodless or lack a brain. Thus for example, cephalopods were long thought to be incapable of audition, until \citet{Hu:2009ov} discovered that a species of octopus, \emph{Octopus vulgaris}, and a species of squid, \emph{Sepioteuthis lessoniana}, can register sound with the statocyst, a sac-like organ containing a mineralized mass and sensitive hairs, that is typically used by aquatic animals to maintain balance and orientation. Though they do not hear very well, this primitive form of audition is used to sense nearby predators. While cephalopods are sanguineous, they lack a brain. So, it would seem that Timaeus' account of sound is inapplicable to the sound that cephalopods hear. Initially, this seems like a good objection, but recall the consequence of identifying the \emph{pathēma} involved in hearing sound on the Theophrastian interpretation of Timaeus' account. The relevant \emph{pathēma} is the impact upon the brain and the blood by the blow of the air. Since, in general, sense objects are the powers of the agents that cause the \emph{pathēmata}, this has a consequence that Timaeus' account of sound does not, in fact, make essential reference to human anatomy. Sound is simply a blow of the air. Of course, Timaeus, unlike Aristotle (\emph{De anima} 2.8), makes no mention of hearing sound in water rather than air, but there is no obstacle, in principle, to extending his account appropriately so as to include cephalopods and other animals for whom water is the primary medium of audition. And since cephalopod anatomy differs from human anatomy in that they are brainless, then the effect of the watery blow would be different in cephalopods than in humans. And yet each may hear as the effect of the watery blow.

Second, Theophrastus objects that Timaeus does not state the cause of sensation. This is strikingly odd since Timaeus explicitly characterizes sound and hearing as movement and countermovement. The countermovement which is the hearing of sound is caused by the impact upon the brain and the blood by the blow of the air. What could Theophrastus so much as mean here? A seeming inconsistency with the third objection perhaps provides some insight into Theophrastus' thinking.

Third, Theophrastus objects that Timaeus has accounted for the sensory processes eventuating in audition and not its objects. How is this third objection consistent with the second? If Timaeus has offered an account not of sound but of sensory processes in us, has he not offered, at least in part, an account of the causes of audition? But on the second objection, Timaeus offers no such cause, not even in part. Theophrastus believes that Timaeus has only purported to give an account of sound but in fact has only given us an account of the causal mechanisms in hearing. (Theophrastus might have been subject to interpretative bias here; his primary interest in his doxographical selection lies with the causal mechanisms involved in perception, on Theophrastus' doxography see \citealt{Baltussen:1993fl,Baltussen:2000aa}.) If that is right, then on Theophrastus' understanding, since Timaeus' account of sound is really an account of the causes of audition, at least those within us, Timaeus has nothing left to say about such causes given this conflation. In taking Timaeus account of sound instead to be an account of the causal processes eventuating in audition, Theophrastus might be encouraged by the reference to human anatomy in that account. This might have led him to think that Timaeus is offering, in his account of sound, only an account of the causal mechanisms that eventuate in audition. As I have explained, however, this is misleading. On the Theophrastian interpretation, only impacts upon the brain and the blood so much as could be the \emph{pathēmata} involved in audition, and sounds, the causes of these, must be independent of them. Interestingly, Theophrastus might also have been sensitive to Barker's grammatical observation. However, on the speculative interpretation that I have offered, Timaeus' account is not subject to Theophrastus' third objection. The character of the countermovement that constitutes hearing corresponds to the character of the blow to the air, the sound, and so Timaeus' account is informative about the nature of \emph{audibilia}.

% section the_ears (end)

\section{The Eyes} % (fold)
\label{sec:the_eyes}

The present difficult passage (67c--68d, on its difficulty see \citealt[479]{Taylor:1928qb} and \citealt[276]{Cornford:1935fk}) is not so much focussed on the eye than it is on the colors and their effect on the visual body (\emph{opsis}). Though we do get a further detail about the elemental composition of the eye, the bulk of Timaeus' discussion of the eye is contained in the previous passage 46c--47e. There, Timaeus gave a description of the eye that is a close paraphrase of Empedocles' lantern analogy (Aristotle, \emph{De sensu} 2 437b27--438a3 = DK 31B84). Like Empedocles, Timaeus claims that the eye contains a fire that is emitted through the pupil owing to its fineness. Timaeus claims, in addition, that this fire combines with daylight in the direction that the perceiver is looking to form a body that may be acted upon as a homogenous unity. The further detail that we learn is that the eye contains, not only fire, but water as well. This detail is also of Empedoclean provenance, if Theophrastus is to be believed (\emph{De sensibus} 7), and has an earlier antecedent in Alcmaeon of Croton (\emph{De sensibus} 26). The water of the eye, however, plays a role in Timaeus' account distinct from the role that it plays in Empedocles' (on the eye's water in Empedocles' account see \citealt{Ierodiakonou:2005fk} and \citealt[chapter 1]{Kalderon:2015fr}).

Timaeus tells us about the colors and their effects on the visual body, these latter being, of course, the relevant \emph{pathēmata}. Colors consist in a kind of flame (\emph{phlox}), and since the visual body consists in the fire emitted from the eye combined with the fire that constitutes daylight, each a mild light that does not burn, three fires of two kinds are involved in seeing the colors of things. Colors act upon the visual body in two ways either by \emph{diakrisis} or \emph{sugkrisis}. Timaeus explicitly tells us that the affections at work in our experience of white and black respectively are the same affections at work in our experience of hot and cold and astringent and harsh (though there Timaeus speaks of \emph{sunkrisis} as opposed to \emph{sugkrisis}). Though these affections are identical, they give rise to distinct perceptions and sensations because they differ in their causes (and also, presumably, because these affections are received in different parts of the body). Let us first consider the colors before turning to their effects.

Timaeus makes at least three distinguishable claims about the colors:
\begin{enumerate}[(1)]
	\item colors consist in kind of fire, a flame
	\item colors consist in effluences that emanate from bodies
	\item colors are perceptible if the effluences are proportional with the visual body
\end{enumerate}
Before discussing these, allow me to make a preliminary observation concerning the potential influence of Empedocles. While \citet[248 n3]{Archer-Hind:1888qd} denies that Timaeus' account of color is Empedoclean, \citet[480]{Taylor:1928qb}, especially in light of the third claim, insists that it is (as does Theophrastus, \emph{De sensibus} 91; for criticism of Theophrastus see \citealt[220 n239]{Stratton:1917vn}, on Archer-Hind's denial see \citealt[21 43--6]{Cook-Wilson:1889cs}). While I agree with Taylor that the influence of Empedocles is clear, Timaeus' account is not straightforwardly Empedocles'. As we shall see there are notable differences, making Archer-Hind's denial plausible if misleading.

First, in his discussion of the \emph{genē}, Timaeus claims that there are many kinds of fire but only offers three examples (58c5--d1). First, there is flame (\emph{phlox}). Second, there is the kind that issues from flame but does not burn but supplies light to the eyes. And third, there is the kind that is left behind in embers after the flame is quenched. Timaeus does not explicitly claim that these three kinds exhaust the kinds of fire that there are. The fire emitted by the eye and that constitutes the light of day both are of the second kind. Presumably, their compound, the visual body, itself belongs to the second kind. However, colors are said to consist in flame (\emph{phlox}).

Second, colors consist in effluences (\emph{aporrē}) that emanate from bodies. This is a clear allusion to the account that Socrates attributes to Empedocles in the \emph{Meno} 76a--d, an attribution seconded by Theophrastus in \emph{De sensibus}. Colors may consist in effluences that emanate from bodies, but I do not think that colors are identified with effluences. First, Plato in the \emph{Theaetetus} distinguished alteration from locomotion (181d), going so far as to introduce a neologism for quality, \emph{poiotēs} (182a), that which changes in alteration. Given this, the identification of color, a quality, with effluences, bodies in motion, can seem like a category mistake, and one that Plato is in position to recognize. There are also phenomenological difficulties with the identification of colors with effluences. Typically we take ourselves to see colored particulars. Thus, Timaeus can see the white chiton that Socrates is wearing. But can we really see colored particulars? Or are they occluded from view by the fiery effluences they give off? Moreover, colors seem confined to the remote bounded region in which they appear \citep{Broad:1952kx}, but effluences are not so confined. Besides these difficulties, there is, within the \emph{Timaeus} itself, grounds to resist the identification colors with effluences. The effluences are fire particles, primary bodies, but only secondary bodies are sensible. But colors are sensible if there are colors at all. Moreover, we have observed the following pattern in Timaeus' discussion of the \emph{pathēmata}, that the objects of perception are powers of secondary bodies as determined by the primary bodies that compose them. The chromatic powers would be the powers of secondary bodies to divide or compact the visual body as determined by their emission of fiery effluences. If we adhere to this pattern in interpreting Timaeus' account of color, then it is not susceptible to the three difficulties described above. Colors would not be effluences but powers determined by these and so a kind of \emph{poiotēs} thus avoiding the charge of category mistake. These powers inhere in bounded surfaces of particular bodies, thus respecting the phenomenology of vision. Moreover, they would not be occluded from view, since the effect of the fiery effluences is the measure of these powers whose congizance vision consists in. If that is right, then Timaeus' account is an advance over Empedocles' since Empedocles' account seems susceptible to these difficulties \citep[6]{Kalderon:2015fr}.

Third, colors are perceptible only if the chromatic effluences are proportional with the fire particles that constitute the visual body (here, again, \emph{summetria} appears as {\sbl ξυμ\-μετρια} instead of {\sbl συμμετρια}). Though clearly inspired by Empedocles, Timaeus' position is really the converse of Empedocles' (for discussion of Empedocles on vision see \citealt{Sedley:1992uq}, \citealt{Ierodiakonou:2005fk}, and \citealt[chapter 1]{Kalderon:2015fr}). According to Empedocles, only effluences that fit the \emph{poroi} are perceptible. If an effluence is too large or too small to fit, then it fails to excite perception. However, according to Timaeus, if the fire particles are larger than the fire particles that constitute the visual body, then they compact that body so as to give rise to a perception of black. And if the particles are smaller than the particles that constitute the visual body, then they divide it to give rise to a perception of white. Like-sized particles, on the other hand, do not affect the visual body and so are insensible and are called ``transparent''. So while Empedocles claims that only like-sized particles excite perception, Timaeus' view is that only particles that are not like-sized excite perception. Rather than reduplicating the Empedoclean scheme as Timaeus does in his account of olfaction, Timaeus' account of vision is its inversion. Timaeus retains Empedocles' Eleatic conviction that sense objects are proportional to sense. They differ only in how the relevant proportion is implemented in the sense object's interaction with sense.

There is another notable difference from Empedocles' account of vision. White and black are distinguished by the elemental composition of these effluences. White effluences are composed of fire and black are composed of water. Thus Theophrastus writes: 
\begin{quote}
	The passages <of the eyes> are arranged alternatively of fire and water: by the passages of fire we perceive white objects; by those of water, things black; for in each of these cases <the objects> fit into the given <passage>. colors are brought to our sight by an effluence. (Theophrastus, \emph{De sensibus} 7; \citealt[71--3]{Stratton:1917vn})
\end{quote}
(See also Plutarch's commentary on Empedocles DK 31B94 in \emph{Historia naturalis} 39, \citealt[CTXT-87 137--8]{Inwood:2001ve}.) Timaeus, on the other hand, distinguishes white from black, not by a difference in their elemental composition, but by the size of the fire particles and their different effects on the visual body. Though, as we shall see, the dark waters of the eye do play a role in Timaeus' account of color perception, as well, but not the role that it plays in Empedocles' account.

% That the effluences must be proportional with the visual body for the colors to be perceptible is further evidence for Timaeus' measurement model of perception. The affection of the visual body by proportional effluences is a measure of the power of the secondary bodies that emitted them, and vision consists in a cognizance of what's measured.

Consider now not colors but their effects, the chromatic \emph{pathēmata}. These are not unrelated as colors are powers of the agents that produce these affections and these affections are the measure of these powers whose cognizance vision consists in. The chromatic \emph{pathēmata} are affections of the visual body. There are two fundamental kinds of affections of the visual body, \emph{diakrisis} and \emph{sugkrisis}. These terms, as the occur in the present passage, are usually translated as dilation and contraction, respectively. However, I suspect that this is less than perspicuous if not indeed misleading. As applied to aggregates of natural bodies, \emph{diakrisis} is more naturally understood not as dilation but as a kind of division or dispersal. Timaeus identifies \emph{sugkrisis} with the affection he earlier designated as \emph{sunkrisis}. And as applied to aggregates of natural bodies, \emph{sunkrisis} is naturally understood as compaction. So presumably the present occurence of \emph{sugkrisis} has this sense as well. As we have observed, the size of the fiery effluences determine their effect on the visual body. If the fire particles are smaller than the fire particles that compose the visual body, then they divide the visual body. And if the fire particles are larger than the fire particles that compose the visual body, then they compact the visual body. These affections, division and compaction, are naturally opposed as are the powers to divide and to compact the visual body. Timaeus calls ``white'' (\emph{leukon}) the power to divide the visual body and ``black'' (\emph{melan}) the power to compact the visual body. This is why white and black are the fundamental chromatic opposition. Presumably this is why \citet[480--1]{Taylor:1928qb} claims that, for Timaeus, white and black are the primary colors. 

The transparent (\emph{diaphanē}) is a special case. If the fire particles are like-sized with the particles that compose the visual body, then they neither divide nor compact it but simply have no effect and hence are insensible. This is an instance of the more general principle that like does not affect like (57a3--5). It is worth pausing to think about the lack of effect and its visual consequences. It is implausible to suppose that the visual body extends no further than the like-sized effluences that it confronts. In that case, we could see no further, and these effluences would not properly be understood to be transparent. We would reach a perceptually impenetrable boundary but the transparent is perceptually penetrable (for discussion see \citealt{Crowther:2018yt} and \citealt{Kalderon:2018wn}). Perhaps, then, the like-sized fire particles are assimilated by the visual body in the way that daylight is and so extends its reach, so to speak. Timaeus might be playing with a potential ambiguity in the Greek term \emph{diaphanē}. \emph{Dia} means through, but the occurence of \emph{phanē} in \emph{diaphanē} is potentially ambiguous. It might mean light or sight. The word \emph{phanē} means torch, and \emph{phanos} means light or bright, whereas \emph{phaneros} means open to sight, visible, or manifest. So while it is natural for moderns to understand \emph{diaphanē} as that which may be seen through, perhaps it also means that which may be shone through. Interestingly, in the context of Timaeus' account, these readings are not unrelated. It is only if the visual body shines through does one see through. In the context of Timaeus' account, there might also be a further play on this ambiguity, for, recall, the visual body only shines in the direction in which the perceiver looks.

Timaeus' account of transparency marks a departure from his predecessors. Earlier thinkers both acknowledged the phenomenon of transparency and accounted for it in terms of unobstructed movement from the perceived object. Democritus and Empedocles, in different ways, at the very least have the resources to provide such an account, Democritus by positing a void through which \emph{eidola} may travel (see Aristotle, \emph{De anima} 2.7 418b13–22), Empedocles by circular motion of effluences in a plenum (DK 31B13, 31B100). Timaeus, by contrast, explains transparency in terms of movement from the perceiver, the visual body extending from the perceiver and assimilating like-size particles of fire. Timaeus' account thus emphasizes an aspect of the the phenomenology of the transparent, its perceptual penetrability. 

There is an important limitation of Timaeus' account, however. Timaeus explains only the transparency of emitted fire particles. Strictly speaking, he provides no explicit account of transparent bodies such as glass, water, ice, and the shaved horns of animals from which lamps were made. There are at least two options for extending Timaeus' account. First, just as there are passages in the eye through which interior fire may pass, so perhaps there are passages in transparent bodies through which the visual body may pass. Second, just as the visual body assimilates like-sized fire particles, so perhaps transparent bodies are at least partially composed of like-sized fire particles that the visual body may assimilate and so extend through. Both suggestions are speculative as Timaeus remains silent on this issue.

So depending upon the size of the fire particles, they either divide, or compact, or have no effect on the visual body. Division and compaction are the fundamental affections involved in vision. Moreover, they are naturally opposed. Since white is the power to divide the visual body and black is the power to compact it, these powers are themselves fundamentally and naturally opposed and so were reckoned to be the primary colors. \citet[480--1]{Taylor:1928qb} concurs. This is controversial, however. Thus \citet{Brisson:1997qr} and \citet{Ierodiakonou:2005ly} claim that there are four primary colors, not only white and black but red and what Timaeus calls ``bright'' (\emph{lampron}) or ``brilliant'' (\emph{stilbon}). (Democritus also posits four primary colors: white, black, red, and green, Theophrastus, \emph{De sensibus} 73--5.) \citet[277]{Cornford:1935fk}, by contrast, claims that there are only three. Given the puzzling nature of the brilliant, he discounts it as a color.

I have stated the philosophical reason for counting white and black in the Timaean color scheme as the primary colors. There are, however, additional historical reasons for this claim. This interpretation of the Timaean color scheme is in line with an ancient tradition that includes Homer, Parmenides, and Empedocles before Plato, and Aristotle and Goethe after (on Homer see \citealt{Gladstone:1858fk}, on Parmenides see \citealt[chapter 5.3]{Kalderon:2015fr}, on Empedocles see \citealt{Ierodiakonou:2005fk} and \citealt[chapter 5.4]{Kalderon:2015fr}, and on Aristotle see \citealt{Sorabji:2022qf} and \citealt[chapter 6]{Kalderon:2015fr}). Moreover, there is a general tendency in Greek color vocabulary to emphasize brightness over hue (see \citealt{Gladstone:1858fk}, \citealt{Platnauer:1921bh}, \citealt{Osbourne:1968vn}, \citealt[chapter 1]{Lloyd:2007fk}). As supporting evidence this is pretty weak, admittedly. However, there is a further bit of historical evidence that is puzzling, at the very least. Theophrastus objects that Democritus posits four primary colors but earlier thinkers posit only two, white and black (\emph{De sensibus} 79). What is puzzling is that Theophrastus makes no similar complaint about Plato (though, to be fair, Theophrastus, \emph{De sensibus} 86, omits red and counts brilliant as a species of white in his report of the \emph{Timaeus}). Thus, Theophrastus, like Taylor after him, must hold that there are only two primary colors in the \emph{Timaeus}.

I suspect that these seemingly competing interpretations may be reconciled. Indeed, they are driven by distinct if potentially complementary ideas. Theophrastus and Taylor are moved by the idea that white and black are the fundamental chromatic opposition. Brisson and Ierodiakonou are moved by the idea that white, black, red, and brilliant are mixed to generate the other colors. So perhaps there is no real inconsistency here. Indeed, Timaeus account of color mixture can be read as an attempt to accommodate the Democritean four color scheme within an older tradition that takes white and black as the fundamental chromatic opposition. To get this into view, let us begin by discussing the two other candidate primary colors, brilliant and red. As we shall see, the \emph{pathēmata} that are the exercise of these chromatic powers involve both division and compaction.

Like white bodies, brilliant bodies emit fire particles that divide the visual body. However, the fire particles emitted by brilliant bodies differ in kind and are more rapid than the fire particles emitted by white bodies. Timaeus does not specify what this difference in kind consists in, but presumably it is a difference in the size of the tetrahedra. And since the tetrahedra emitted by brilliant bodies are more rapid than the tetrahedra emitted by white bodies, presumably they are smaller, since the smallness of polyhedra contribute to the speed of their motion. The fire particles emitted by brilliant bodies also differ in their effects. Unlike the fire particles emitted by white bodies, the fire particles emitted by brilliant bodies divide the visual body all the way up to the eye where they produce secondary effects.

Before considering the secondary effects of the fire particles emitted by brilliant bodies, let me pause to make an observation since it signals a further difference between Timaeus' account and Empedocles'. For Empedocles, the effluences must enter passages in the eyes in order to excite perception. But the fire particles emitted by white bodies do not divide the visual body all the way up to the eyes. They thus do not enter the passages in the eyes, the way, as we shall see, that the fire particles emitted by brilliant bodies do. In the case of seeing white things, the \emph{pathēmata} are affections of the visual body that is external to the eyes. Though, of course, the visual body as a whole passes on this affection through the passages of the eyes so that the power of the agent that produced it may be reported to the \emph{phronimon}, but this is not the reception of an effluence but the reception of its effect. 

The fire particles emitted by brilliant bodies enter the eye. As they do, they pass the fire particles that the eye is itself emitting and penetrate and dissolve the very passages from which these are emitted. This would only be possible if there is no exact fit between the fire particles and the passages in the eye, just as there is no exact fit between the odorous particles and the blood vessels in the nose as \citet{Vlastos:1967jw} contends. In the eye, the incoming fire particles encounter water. This has two effects, only the second of which is relevant to visual phenomenology:
\begin{enumerate}[(1)]
	\item the incoming fire particles cause a volume of water and fire to pour from the eye
	\item the incoming fire particles are mixed with the water in the eye causing all kinds of colors to appear
\end{enumerate}
The first secondary effect, the pouring forth of a volume of water and fire, is called ``tears''. All water in its liquid form contains fire. I suspect that Timaeus, here, draws our attention to the fact that the extruded water contains fire as well to emphasize two things, first, the warmth of human tears, and, second, the great amount of fire particles required to displace this volume of water and fire. Though a vital affection, unlike trembling's contribution to our sense of coldness, the production of tears does not contribute to the dazzling visual experience. It is epiphenomenal to the causal process that elicits that experience. In this way it contrasts with the second effect. When the incoming fire is mixed with the eye's moisture, this mixture causes all kinds of colors to appear. Apparently, in a striking anticipation of Newton's discovery (\emph{Opticks} 1), the stream of incoming fire is refracted in the water of the eye (compare Aristotle's claim that a weak light shining through a dense medium will cause all kinds of colors to appear, \emph{Meterologica} 1.5 342b5--8). The resulting visual experience is called ``dazzling'' (\emph{marmarugē}) and the power of the agent that produced it is called ``bright''  (\emph{lampron}) or ``brilliant'' (\emph{stilbon}).

The incoming fire particles dissolve and so destroy, at least in part, the passages in the eyes. Though Timaeus does not make this explicit, this is a departure from the natural state of the eye and is presumably painful (think of staring at the sun). The disruption of the visual body may not be painful since no violent effort is required to do so, but what is dissolved, here, is not the visual body issuing from the eyes but the passages of the eyes from which the visual body issues. These are composed of particles greater than the particles of fire that compose the visual body and so require violent effort to dissolve. Recall, too, that while the fire in the eyes is of a kind that gives light but does not burn, the incoming fire is not of this kind but is, rather, a kind of flame and so does indeed burn. This explains the damage inflicted by seeing brilliant bodies. The dazzling experience of brilliant bodies thus approaches the traumatic experience undergone by the Cyclops as Odysseus blinds him by inserting fire in his eye. Not only are both experiences destructive and painful, but each is a kind of blinding since when dazzled it is difficult to see the details of the scene before one. (For discussion of Euripides' \emph{Cyclops} and vision in the \emph{Timaeus} see \citealt[114]{Johansen:2004dx})

\citet[277--8]{Cornford:1935fk} pronounces brilliance to be puzzling. All kinds of colors are said to appear in a dazzling visual experience and yet in the next breathe Timaeus describes bright or brilliant as a single color. Cornford's puzzle is a chromatic version of the \emph{aporia} about the one and the many (compare the \emph{Philebus} and, of course, the \emph{Parmenides}). Cornford's puzzle is mitigated somewhat if we turn our attention away from flat opaque colors presented against neutral backgrounds such as how they are presented on a Munsell color chip, and consider the metallic green of beetle. Such colors have a metallic sheen that varies with the angle of incidence of the light and yet are a distinctive unitary color. How this is so may remain puzzling, but that this is so is not. (Modern colorimetrists posit extra dimensions of color similarity to explain such phenomena. \citealt{Fairchild:2005vd}, for example, lists five dimensions of color similarity.)

Both white and brilliant are powers to divide the visual body, but what distinguishes brilliance from white is not this affection but a secondary effect, the mixing of fire and water in the eye. In the case of the dazzling, the affection as a whole includes not only division but this mixture as well. Seeing white involves an affection produced by fire alone, a dazzling experience involves an affection produced by fire and water. While the fire that elicits a dazzling experience may be more intensely bright, its mixture with water involves a reduction of chromatic brightness with the effect that all sorts of colors are perceived and not pure white. Mixing with the eye's water darkens. Perhaps it does so by compacting the incoming stream of fire particles that mixes with it, in which case the water of the eye is black. There is an ancient tradition that assigns black as the color of water. Thus, Anaxagoras (DK 59B15), Empedocles (DK 31B94, Theophrastus, \emph{De sensibus} 59, Plutarch, \emph{Historia naturalis} 39), and later Aristotle (\emph{Meteorologica} 3.2 374a2) all held that water is black. If Timaeus subscribes to this tradition, then the \emph{pathēmata} caused by brilliant bodies involves both division and compaction.

Let us now consider Timaeus account of red. Just as brilliant bodies emit a distinctive kind of fire particle, so red things emit a distinctive kind of fire particle. Again, it is plausible to assume that this difference in kind has to do with the size of the tetrahedra  (and not, \emph{pace} \citealt[128]{Ierodiakonou:2009cg}, with their relative brightness). Concerning this, Timaeus says of this kind that it is between ``these'', but there is unclarity about this anaphoric reference. Since Timaeus has just been discussing brilliant, it is plausible that the fire particles emitted by brilliant bodies is one of the kinds of fire to which the fire particles emitted by red bodies is being compared. But what is the other kind of fire? \citet[250 n8]{Archer-Hind:1888qd} claims that it is the fire that is emitted by white bodies while \citet[67 n3]{Bury:1929jb} claims that it is the fire that is emitted by black bodies. Archer-Hind's reading, that \citet[168]{Brisson:1997qr} shares, may be justified in the following manner: Like the fire emitted from brilliant bodies, the fire emitted by red bodies enters they eye and mixes with the water therein to produce a sanguine appearance that we call ``red'' (\emph{eruthron}). It thus must be more like the fire emitted from brilliant bodies than the fire emitted from white bodies. \citet[277 n2]{Cornford:1935fk}, on the other hand, endorses Bury's reading and cites Aristotle's claim that bright light seen through a dark medium appears red (\emph{Meterologica} 3.4 374a3), as when the sun appears through a cloud or smoke. Indeed, Aristotle thinks that red is the intermediary color that is midway between white and black. This, however, can suggest a third interpretation. Perhaps ``these'' does not refer to the kind of fire involved in brilliant and either white or black but simply picks up the earlier opposition between white and black. This would be further reinforced if \citet[10--1]{Levides:2002aa} is right and brilliant is merely a species of white, presumably a lesser species (see also Theophrastus, \emph{De sensibus} 86). \citet[482]{Taylor:1928qb} provides yet another interpretation and sees the contrast between the fire emitted by white bodies and the fire that composes the visual body itself. The matter is unclear. I suspect, however, that Archer-Hind's interpretation is correct. If he is, then the the fire particles emitted by red bodies is smaller than the fire particles emitted by white bodies and larger than the fire particles emitted by brilliant bodies. Nevertheless, they are small enough and rapid enough to divide the visual body all the way up to the eyes and so mix with their waters.

The fire emitted by red bodies, like the fire emitted by brilliant bodies, divides the visual body all the way up to the eye. And like the fire emitted by brilliant bodies, it mixes with the water of the eye to produce a secondary effect. This secondary effect differs from the secondary effect involved in dazzling visual experiences in two ways. First, no tearing is involved. Second, instead of causing all kinds of colors to appear, it causes, instead, a sanguineous appearance that we call ``red''. How are we to understand this? Interestingly, an advocate of Archer-Hind's reading can help themselves to the rationale that Cornford offered for Bury's reading. There is an ancient tradition that assigns black as the color of water. If Timaeus subscribes to this tradition, then the water in the eye is itself black, and so compacts the incoming stream of fire. And the fact that the stream of fire that passes through the dark waters in the eye produces an experience of red is an expression of red being intermediary between white and black. Moreover, if red is not merely intermediate but midway, then Aristotle's claim about red would then turn out to be of Timaean provenance. (It is notable that \citealt[57]{James:1996pb} puts red on the brighter end of the spectrum near brilliant since she focuses exclusively on the fire particles' division of the visual body and ignores their impact on the eye's moisture.)

Like the perception of white, the perception of brilliant and red each involves the division of the visual body. What distinguishes the perception of brilliant and red from the perception of white is not the division of the visual body but the secondary effects, involving the mixture of fire and water, induced by this division. The mixture of fire and water in the eye differ in the case of brilliant and red because of the kind of fire involved, plausibly understood in terms of relative size and a difference in the rapidity of their motion. And in each case the mixture of fire and water involves a reduction of brightness due to water compacting the incoming fire. In the case of brilliant, the mixture results in the appearance of all kinds of colors, many, at least, are not white. And in the case of red, the mixture results in the appearance of an even darker color. And in each case the reduction of brilliance is due to the water being black and so compacting the incoming fire. If the fundamental opposition in the Timaean color scheme derives from the opposition of \emph{diakrisis} and \emph{sugkrisis}, then brilliant and red are intermediary colors between white and black, yielding the series: white, brilliant, red, and black, with white and black being the primary opposition. If, however, one emphasizes the generation of the colors by mixture, then it is natural to understand these four colors as the primary colors as the rest are a mixture of these.

Timaeus prefaces his discussion of color mixture with a warning. It would be foolish to state the exact proportions involved in such mixtures, even if one knew these, since no demonstrative nor even probable reason could be given (68b6--8). And he concludes his discussion of color mixture by charging any experimental inquiry into these proportions with impiety. Only God is wise enough and powerful enough to blend the many into one and dissolve the one into many (68c7--d7). So while we can know which colors need to be mixed to generate another color, it is impossible for mortals to determine by test (\emph{basanos}) the exact proportions involved. \emph{Basanos}, like a lot of Greek epistemological vocabulary, has a judicial origin (see \citealt[chapter 4]{Lloyd:1979lc}). Specifically, it derives from the practice of eliciting evidence by torture. So, in the chromatic domain, only God is wise enough and powerful enough to deploy this judicial procedure. Even if we somehow came to know these proportions, by the testimony of a god or oracular revelation, say, we would still not understand how mixing colors in these proportions generate the colors that they do, for no demonstrative or even probable reason would be available to us.

The full significance of these enigmatic remarks is unclear. \citet[173--4]{Brisson:1997qr} makes the interesting suggestion that implicit in these remarks is a criticism of Empedocles' claim to be equal to a god in knowledge. However they are to be understood, I think that we should resist any cynical interpretation where Timaeus is preemptively silencing his critics, or providing a license to engage in an Athenian parlor game. For one thing, this is an application of an already stated claim. The opposed extremes of fire and earth are united by proportionate intermediaries, and only He that bound them may rend them asunder. Similarly, the opposed extremes of white and black are united by proportionate intermediaries, and only He that bound them may rend them asunder. Rather than manifesting a lack of seriousness, these remarks are the expression of a profound humility. For in making them, Timaeus concedes that the visible may not be fully understood by the intellect of mortals, thus confessing to the limitation of his cosmological project.

Timaeus lists nine colors that are the result of mixing the four colors, white, black, bright or brilliant, and red or the further results of such mixtures (68b5--c7). These are:
\begin{enumerate}[(1)]
	\item golden (\emph{xanthon}) = bright (\emph{lampron}) + red (\emph{eruthron}) + white (\emph{leukon})
	\item purple (\emph{alourgon}) = red (\emph{eruthron}) + black (\emph{melan}) + white (\emph{leukon})
	\item violet (\emph{orphinon}) = black (\emph{melan}) + purple (\emph{alourgon})
	\item tawny (\emph{purron}) = golden (\emph{xanthon}) + grey (\emph{phaion})
	\item gray (\emph{phaion}) = white (\emph{leukon}) + black (\emph{melan})
	\item yellow (\emph{ochron}) = white (\emph{leukon}) + golden (\emph{xanthon})
	\item dark blue (\emph{kuanoun}) = white (\emph{leukon}) + bright (\emph{lampron}) + black (\emph{melas})
	\item light blue (\emph{glaukon}) = dark blue (\emph{kuanoun}) + white (\emph{leukon})
	\item leek green (\emph{prasinon}) = tawny (\emph{purron}) + black (\emph{melan})
\end{enumerate}
The precise sense of Greek color terms is notoriously difficult to capture in translation, and so the translations provided here are only rough equivalents and do not capture precise hue boundaries. (For useful discussion of these terms as they occur in the \emph{Timaeus} see \citet{Platnauer:1921bh}, \citet[483--5]{Taylor:1928qb}, \citet[chapter 10]{Bruno:1977fk}, \citet{Levides:2002aa}, and \citet{Struycken:2003zr}. Timaeus account of color mixture is usefully compared to Democritus' with which there are points of contact, for example, in the mixture that generates golden, Theophrastus, \emph{De sensibus} 76--8, see \citealt{Struycken:2003zr} for discussion.)

If we resolve the combinations with mixed colors into the colors from which they are themselves mixed we get:
\begin{enumerate}[(1)]
	\item golden (\emph{xanthon}) = bright (\emph{lampron}) + red (\emph{eruthron}) + white (\emph{leukon})
	\item purple (\emph{alourgon}) = red (\emph{eruthron}) + black (\emph{melan}) + white (\emph{leukon})
	\item violet (\emph{orphinon}) = black (\emph{melan}) + (red (\emph{eruthron}) + black (\emph{melan}) + white (\emph{leukon}))
	\item tawny (\emph{purron}) = (bright (\emph{lampron}) + red (\emph{eruthron}) + white (\emph{leukon})) + (white (\emph{leukon}) + black (\emph{melan}))
	\item gray (\emph{phaion}) = white (\emph{leukon}) + black (\emph{melan})
	\item yellow (\emph{ochron}) = white (\emph{leukon}) + (bright (\emph{lampron}) + red (\emph{eruthron}) + white (\emph{leukon}))
	\item dark blue (\emph{kuanoun}) = white (\emph{leukon}) + bright (\emph{lampron}) + black (\emph{melan})
	\item light blue (\emph{glaukon}) = (white (\emph{leukon}) + bright (\emph{lampron}) + black (\emph{melan})) + white (\emph{leukon})
	\item leek green (\emph{prasinon}) = (bright (\emph{lampron}) + red (\emph{eruthron}) + white (\emph{leukon})) + (white (\emph{leukon}) + black (\emph{melan})) + black (\emph{melan})
\end{enumerate}
(Compare the tables provided by \citealt[56-7]{James:1996pb} and \citealt[175]{Brisson:1997qr} and \citealt[466-7, 600]{Brisson:1998aa}. I find an aspect of Brisson's notation potentially misleading given Tim\-ae\-us' warning about our ability to know, let alone understand, the proportions involved in color mixture. Consider, for example, Brisson's representation of the color combination involved in the generation of violet. Let A be white, B black, and C red, then the color combination that generates violet is represented as:
\begin{quote}
	(A + B + C) + B = A + 2B + C
\end{quote}
The occurence of 2 is egregious as it can misleadingly suggest that the color mixture involves two parts black. But, as Brisson himself knows well, Timaeus claims no such thing.)

Ordering all thirteen colors from light to dark, we get:
\begin{enumerate}[(1)]
	\item white (\emph{leukon}) 
	\item bright (\emph{lampron}) or brilliant (\emph{stilbon})
	\item yellow (\emph{ochron})
	\item golden (\emph{xanthon})
	\item tawny (\emph{purron})
	\item leek green (\emph{prasinon})
	\item red (\emph{eruthron})
	\item light blue (\emph{glaukon})
	\item dark blue (\emph{kuanoun})
	\item purple (\emph{alourgon})
	\item violet (\emph{orphinon})
	\item grey (\emph{phaion})
	\item black (\emph{melan})
\end{enumerate}
The list was generated given the following assumptions. First, that the four unmixed colors ordered in terms of brightness are white, brilliant, red, and black. Second that brilliant is close to white in brightness. Third, like the Aristotelian color scheme, red is midway between white and black. These assumptions were then applied to Timaeus' color combinations to yield the brightness ordering. As the unknowable proportions can make a difference, the ordering could only be an estimate, but it conforms well to intuitive judgments of relative brightness. Understanding the colors as intermediaries in the opposition between white and black is problematic (\citealt[chapter 6.3]{Kalderon:2015fr}). However, if we bracket these difficulties, this is a reasonably comprehensible ordering and less mysterious than some commentators have made it out to be.

How are we to understand color mixture as Timaeus understands it? This issue potentially ramifies, as it may affect, as well, our understanding of the operative notions of division and compaction. Attempts have been made to understand mixture, here, on the model of the painter's practice of pigment mixture. Bruno's \citeyearpar[chapter 10]{Bruno:1977fk} account is particularly interesting (though see \citealt[]{Gage:1993aa} for criticism; \citealt[278]{Cornford:1935fk} and \citealt{Levides:2002aa} defend this general approach, and \citealt{Struycken:2003zr} criticizes it, though offers, in effect, a mere generalization of this approach). The painter's practice of pigment mixture may have influenced Plato's thinking, at least indirectly, but, in the first instance, we should try to understand Timaeus' account of color mixture in terms of the fiery effluences that colored bodies emit. There are historical, textual, and philosophical reasons for this. 

The historical reason concerns the observed general tendency for the Greeks in antiquity to understand colors in terms of relative brightness. This occurs in both literary and philosophical sources such as Homer, Parmenides, Empedocles, and, later, Aristotle. Plato knew these sources and was in many ways influenced by them, it would be unsurprising should a commitment to this ancient tradition be manifest in Timaeus' account.

The textual reason concerns a consequence of the Timaean account of color. According to Timaeus, colors are powers of secondary bodies to emit fiery effluences. If colors are powers of secondary bodies to emit fiery effluences, then it would be natural to suppose that color combinations are combinations of these powers. Moreover, as we have seen (chapter~\ref{sec:the_elemental_composition_of_the_corporeal}) powers may stand in proportionate ratios. Of course, more needs to be said about what the combination of powers amounts to. But given that colors just are these powers, their combination must be combinations of these powers. A mixture of anything else would not be a color mixture, at least by Timaeus' lights.

The philosophical reason concerns the distinction between color and pigment mixture. The colors are not pigments, and so color mixture is not pigment mixture. It is important to distinguish the mixture of colors that generate different colors from the mixture of bodies that generate differently colored bodies. Pigment mixture is a mixture of bodies that generate differently colored bodies. So we should try to understand color mixture in terms other than pigment mixture, especially in light of the material recalcitrance of the latter. Thus, for example, the color mixing that results from mixing pigments is a subtractive process, and \citet{Helmholtz:1852ab} showed that not every color determined by an additive process can be matched with a color determined by a subtractive process. 

Helmholtz's demonstration was unavailable to the ancients, but it is not anachronistic to suppose that they could mark the distinction between color and pigment mixture. Thus Aristotle criticizes attempts to understand color mixture in terms of pigment mixture in \emph{De sensu} 3. And the Peripatetic author of \emph{De coloribus} writes: 
\begin{quote}
	But we must make our investigation into these things not by mixing colours as painters do, but by comparing the rays which are reflected from those to which we have already referred. For one could especially consider the mixing of rays in nature. (\emph{De coloribus} 2 792b17--21; \citealt[13]{Hett:1936sd})
\end{quote}
Of course, the Peripatetic author is proposing an empirical investigation of the kind that Timaeus would eschew as impious if it includes within its purview the exact proportions involved in color mixture. But the Peripatetic author clearly distinguishes color mixture from pigment mixture. Moreover, he understands the distinction in terms that Timaeus may accept, at least on a certain interpretation of them. The Peripatetic author recommends that color mixture be understood, not in terms of pigment mixture, but in terms of the mixing of reflected rays. That the rays are reflected is a non-Timaean thought. Daylight is necessary for sight, not by providing illumination to be reflected from bodies, but to form the visual body along with the fire emitted from the eyes, to connect, as a continuous unity, to the flame of the colored body. But if we bracket the assumption that the rays are reflected and understand the rays as the streams of fire particles that colored bodies emit, then Timaeus' position, to that extent at least, anticipates the Peripatetic author's.

\citet{Ierodiakonou:2005ly} proposes an account of this kind. She begins by observing that the terms \emph{diakrisis} and \emph{sugkris} appear in Timaeus explanation of the cycle of elemental transformation (58b7). In certain circumstances, smaller primary bodies divide larger ones with the result that, for example, we get two tetrahedra of fire from one octahedron of air. Similarly, in certain circumstances, larger primary bodies may compress smaller ones so as to combine them such that we get one octahedron of air from two tetrahedra of fire. Her suggestion, then, is to apply this model to the division and compaction of the visual body.

If the fire particles emitted from a colored body are smaller than the fire particles that compose the visual body, then these are broken down into smaller tetrahedra. And if the fire particles emitted from a colored body are larger than the fire particles that compose the visual body, then these are compressed and so combined into larger tetrahedra. So Ierodiakonou is advancing a specific interpretation of division and compaction here. The visual body is divided by dividing the tetrahedra that compose it and is compacted by combining smaller tetrahedra into larger tetrahedra.

It is now open to Ierodiakonou to understand color mixture in terms of the mixture different-sized tetrahedra in the visual body as determined by the operations of division and compaction:
\begin{quote}
	Let us take, for example, the simple case of the colour grey which is said to be a mixture of white and black. This, I suggest, is to be understood in the following way: a grey body emits fire-particles of two different sizes; namely they are pyramids which, separated according to size, are of the kind emitted by white and by black bodies, respectively. The pyramids of these two different sizes emitted by the grey body interact with and transform the particles of the visual body into smaller and larger particles so that the visual body ends up containing the same proportion of pyramids of these two sizes as the grey body emits. \citep[228]{Ierodiakonou:2005ly}
\end{quote}

The power of white and black bodies to divide and compact the visual body may produce in it smaller and larger tetrahedra, but is there reason to think that the power of red and brilliant bodies themselves produce tetrahedra of distinctive sizes? After all, their eliciting the visual experiences they do is not due solely to their effect on the visual body but is due as well to their secondary effect on the water of the eye. But this is downstream from the visual body (relative to the motion of the fire particles emitted from the colored body), no matter the sizes of tetrahedra from which the visual body is composed. Moreover, since the secondary effect is downstream, division cannot merely mean the division of constituent tetrahedra, it must mean, as well, the division of the visual body as a whole. Otherwise, how would the small and swift fiery effluences reach the water of the eye? And while it is true that in the cycle of elemental transformation smaller primary bodies may divide larger primary bodies and larger primary bodies may compact and so combine smaller primary bodies, these are all cases of primary bodies of different kinds. Timaeus never explicitly says that that smaller fire particles may divide larger fire particles into their constituent triangles. So while Ierodiakonou's account of Timaean color mixture is both ingenious and of the right kind, I am hesitant to endorse it. 

There are, of course, alternatives. Consider only the most obvious. Associated with white, brilliant, red, and black are distinct affections of the visual body or the eyes from which it issues depending upon the size and rapidity of the fire particles emitted from the colored body. Perhaps what are combined are powers to produce these chromatic \emph{pathēmata}. Recall that there are four of them:
\begin{enumerate}[(1)]
	\item White bodies divide the visual body with small and rapid fire particles
	\item Brilliant bodies divide the visual body with smaller and more rapid fire particles that mix with the waters of the eye that compact them 
	\item Red bodies divide the visual body with larger and less rapid fire particles (though smaller and more rapid than the fire particles emitted by white bodes) that mix with the waters of the eye that compact them
	\item Black bodies compact the visual body with the largest and slowest of the fire particles
\end{enumerate}
Notice that the smaller and more rapid fire particles emitted by brilliant bodies would result in a greater mixture of fire with the eyes' water than would result from the fire particles emitted by red bodies.

The color combinations that Timaeus describes can be understood as combinations of powers to produce these \emph{pathēmata}. So, for example, a gray body will emit small and large fire particles, the smaller and more rapid of which divide the visual body and the larger and slower of which compact it. Thus, gray bodies will act upon the visual body in both the ways that white bodies and black bodies do. Gray bodies thus possess the powers of white and black bodies in a proportion unknowable by mortal intellects. And purple bodies will emit small and large fire particles but where there are two grades of small particles. The large particles, the slowest, will compact the visual body while the small particles will divide it. The smallest and fastest of these, however, divide the visual body all the way up to the eye where they mix with the eye's water and are compacted by it. Thus, purple bodies will act upon the visual body and the eye from which it issues in the ways that white, black, and red bodies do and thus combine these powers in some unknowable proportion. And so on for all the other color combinations that Timaeus describes. Combining the list of the four chromatic \emph{pathēmata}, the exercise of the powers of white, brilliant, red, and black, with the list of Timaeus' color combinations where the combinations with mixed colors are resolved into combinations of colors from which they were mixed, yields a reasonably comprehensible and comprehensive account of color mixture. It is at least as comprehensible as the ordering of the colors from light to dark that it generates, and as we have seen, it does indeed generate a reasonably comprehensible brightness ordering. And with respect to comprehensiveness, all that is missing is what Timaeus claims mortal intellects could never have, knowledge and understanding of the proportions involved in such combinations. Moreover, this account, far from being inconsistent with Ierodiakonou's, may be combined with it, at least to a certain extent. On the hybrid account, an effect of the combined actions on the visual body is that it is composed of differently sized tetrahedra, whether or not brilliant and red are best understood as corresponding to distinctive sizes of tetrahedra.

On the present interpretation, Timaeus can be understood as accommodating Democritus' claim that the mixture of four colors suffices for the generation of the colors within an older chromatic tradition that sees white and black as the fundamental chromatic opposition. Timaeus departs from the four color scheme in substituting brilliant for green  (\emph{chlōron}). But, importantly, the substitution preserves the relative order of brightness. Brilliant, like green, is brighter than red and darker than white. And whereas Democritus only posits seven generated colors, Timaeus posits nine. In this way, it is an elaboration. And while some of the Democritean color combinations are preserved in the Timaean color scheme, there are adjustments given the substitution of brilliant for green, but in a way that preserves the overall structure. Perhaps this accommodation of the Democritean color scheme within the older tradition is an instance of Timaeus' tacit rivalry with Democritean natural philosophy. Democritus had written an entire book on color (\emph{Peri chroōn}, see Diogenes Laertius, \emph{Vitae Philosophorum} 9.46), and his claim that the mixture of four colors suffice to generate the colors was shared with medical and artistic traditions. Thus the medical tradition associated the Democritean four colors with blood, phlegm, and black and yellow bile, respectively \cite[]{Longrigg:1963tt}. And there is literary and archeological evidence that fifth century BCE painters restricted their palette to these four colors \cite[56--7]{Bruno:1977fk}. (On these external influences on Timaeus' account see \citealt[127--8]{Ierodiakonou:2009cg}.) Moreover, accommodating Democritus' four color scheme in a way that preserves the fundamental opposition between white and black would explain why Timaeus is not subject to Theophrastus' criticism of Democritus (\emph{De sensibus} 79). All that was needed was to postulate, in addition to the the power of black bodies, two further powers to reduce brightness that, while distinct, themselves depend upon blackness. Brightness is reduced, not the way black bodies do, by emitting large particles that compact the visual body, but by emitting small particles that are themselves compacted in their mixing with the eyes' waters. These powers differ in the proportion of fire and water in the mixture that they elicit in the eyes. These powers may differ in this way, but they are alike in that neither are wholly independent of the power of black bodies. The water's compaction of the incoming fire particles just is the exercise of its blackness. What are distinguished are degrees of departure from whiteness, a reduction in brightness, where black alone remains in opposition to white and is presupposed by the two other departures from whiteness. That brilliant and red cannot be combined by themselves to generate a color but must be combined with either white or black to do so is further evidence of their derivative nature.

 % The smaller and more rapid fire particles emitted by brilliant bodies would result in a greater mixture of fire with the eyes' water than would result from the larger and less rapid fire particles emitted by red bodies.

The way that Timaeus' account diverges from Empedocles' would also be explained, at least in part. Theophrastus objects to Empedocles' account that while he has explained the perception of white in terms of the reception of fiery effluences and the perception of black in terms of the reception of watery effluences, he has not explained the perception of any other color (\emph{De Sensibus} 17). However, Empedocles may be understood as claiming that the perception of the rest of the colors involve different proportions of fire and water received through the eyes' passages (see, \citealt{Ierodiakonou:2005fk} and \citealt[chapter 5.4]{Kalderon:2015fr}, an idea that may have Parmenidean provenance, see \citealt[5.3]{Kalderon:2015fr}). Timaeus accepts the basic idea that white and black are the fundamental chromatic opposition. Timaeus also accepts the Eleatic idea that the intermediary colors should be understood in terms of mixtures of various proportions (even if knowledge of these proportions would require a god's knowledge that Empedocles' impiously claims for himself). However, accommodating Democritus' four color scheme within the older tradition that Timaeus shares with Empedocles required postulating two further ways of reducing brightness beyond the way that black bodies do, and this, in turn, required corresponding adjustments to the Empedoclean scheme. For example, that only some chromatic effluences enter the eye, those emitted by brilliant and red bodies, is a departure from Empedocles that results from such adjustment, as is the revised role of the eye's water. Thus Taylor was wrong to see Timaeus account of color and its perception as simply an application of Empedocles', just as Archer-Hind was wrong to deny any Empedoclean influence.

% Finally, the persistent allure of interpreting Timaeus' color combinations in terms of the painter's practice of pigment mixture would be explicable, if mistaken. For that which Timaeus seeks to accommodate, Democritus' four color scheme, was shared to an extent with the fifth century four color painters.

% This is speculative, of course. But some account along these lines must be right given the historical, textual, and philosophical reasons to resist understanding Timaeus' account on the model of the painter's practice of pigment mixture. And this would remain so even should it turn out, in the end, that Timaeus' account is overly influenced by this practice. Compare, for example, Aristotle's criticism of understanding color mixture in terms of pigment mixture in \emph{De sensu} 3 (see \citealt[chapter 6]{Kalderon:2015fr}, for discussion).

Theophrastus makes three objections to Timaeus' account of vision and color. Theo\-phrastus understands Timaeus as merely elaborating Empedocles' account of color vision and complains that he does not offer a like account for the other senses (\emph{De sensibus} 91). However, as we have seen, though inspired by Empedocles, Timaeus' account is importantly different in ways dictated, in part, by accommodating Democritus' four color scheme. Theophrastus also doubts whether color could invariably be understood in terms of flame, and he is skeptical about our inability to know let alone understand the proportions involved in color mixture:
\begin{quote}
	It is absurd \ldots\ to say without exception that colour is a flame. For while in some respects the colour white resembles flame, black would seem to be flame's opposite. And in depriving <of all rational necessity> the mixture which produces the other <colors>, he has on the whole made it impossible to assign them to their causes, and has left <his case> in need of argument and warrant. (Theophrastus, \emph{De sensibus} 91; \citealt[149--50]{Stratton:1917vn})
\end{quote}
We can set aside the issue about color mixture, as this is less an objection than an expression of skepticism. The issue about the association of color with flame is more interesting, however. How could the perception of black result from flame? While the worry can seem intuitively compelling, a reply may be made on behalf of Timaeus. Consider a black image on a television or computer screen. Should one turn the screen off, it will be gray and not black. And yet when on, it was emitting light which resulted in a perception of black and not gray. Understanding how this may be so may be puzzling in the way that Theophrastus suggests, but it is demonstratively so nonetheless. What is needed to resolve this puzzlement is the distinction between brightness as amount of light and brightness as a dimension of color similarity. And while the means to empirically establish this requires some method, unavailable in antiquity, of measuring the amount of light, the distinction can be found, in embryonic form, in Timaeus' account. Brilliant bodies, in their rapid emission of small tetrahedra, are very bright, but they elicit in us the appearance of all kinds of colors not all of which are white.

% section the_eyes (end)

\section{Concluding Observations} % (fold)
\label{sec:concluding_observations_pp}

If common \emph{pathēmata} are affections common to the body as a whole, the peculiar \emph{pathēmata} are affections that are peculiar to particular parts of the body. Are the particular parts of the body that receive the peculiar \emph{pathēmata} best understood as sensory organs? On the basis of our detailed review of the peculiar \emph{pathēmata}, we should be doubtful. If as \citet{Lloyd:1978fk} maintains, Aristotle's account of the sensory organs in \emph{Parva Naturalia} is schematic, Timaeus' account of the particular parts of the body that receive the peculiar \emph{pathēmata} are more schematic still. No attempt is made to describe internal structure and no account is given of their function other than to receive the relevant affection. And while anatomical knowledge was relatively impoverished in antiquity, Timaeus' account compares poorly to his predecessors, even those among them from which he draws inspiration. Thus Alcmaeon of Croton and Empedocles each discern internal structure in the ear that contributes to its proper functioning. Timaeus, on the other hand, merely conceives of the ear as a conduit through which the blow of the air may be delivered. While Timaean anatomy does recognize organs properly so-called---think of his elaborate account of the liver---in no sense are the particular parts of the body that receive the peculiar \emph{pathēmata} properly described as sensory organs.

We have seen (chapter~\ref{sec:hot_and_cold} and section~\ref{sec:the_tongue}) Theophrastus raise a general concern about the unity of the sensible objects on Timaeus' account (\emph{De sensibus} 87, 89). Now that we have reviewed all the \emph{pathēmata}, common and peculiar, we are in a position to address this more general concern. It is first introduced in \emph{De sensibus} in the following passage:
\begin{quote}
	Democritus has no uniform account of all <the sensory objects>: some he distinguishes by the size <of their atoms>, others by the shape, and a few by the <atomic> order and position. Plato, on the other hand, refers nearly all of them to effects in us, and to our perceptive faculty. Consequently each of these authors would seem to speak directly counter to his own postulate. For the one of them, who would have sensory objects to be but effects in our perceptive faculty, actually describes a reality resident in the objects themselves; while the other, who attributes the objects' character to their own intrinsic being, ends by ascribing it to the passive change of our perceptive faculty. (Theophrastus, \emph{De sensibus} 60-1; \citealt[121]{Stratton:1917vn})
\end{quote}
An adequate account of sensible objects must provide a uniform account of them, but neither Democritus nor Plato provide such an account, though for different reasons. 

What is the Theophrastian demand for a uniform account exactly? His initial objection to Democritus is that some sensible objects are distinguished by shape while others by order and position. And we have seen (chapter~\ref{sec:hot_and_cold}) Theophrastus object to Timaeus' account of hot and cold on the grounds that while Timaeus explains hot in terms of figure he fails to do so for cold (\emph{De sensibus} 87). Theophrastus seems to subscribe to a monism about the sensible. Consistent with their being distinguishable species of \emph{sensibilia} they must all belong to a uniform class. Is vindicating monism about the sensible a reasonable demand on an adequate account of sensory objects? \citet{Pasnau:1999ss,Pasnau:2000aa} is a contemporary historian and philosopher of perception who explicitly subscribes to monism about the sensible. I have my doubts. Rather than thinking of sensible objects as belonging to a uniform class, I am impressed by the heterogeneity of the sensible. Far from adhering to the monism of the sensible, on Austinian grounds, I am attracted to a pluralism of the sensible. Just consider the variety of \emph{visibilia} alone. We see opaque natural bodies such as Moore's \citeyearpar{Moore:1903uo} blue bead or Price's \citeyearpar{Price:1932fk} red tomato, but we also see translucent volumes, flashes, reflections, mirror images, rainbows, mirages, shadows, holes. Perhaps as \citet{Sorensen:2004jk,Sorensen:2008kx,Sorensen:2009aa} suggests, we can see darkness and hear silence. And all these sensible objects seem to belong to diverse categories and degrees of being. 

Put aside the legitimacy of Theophrastus' demand, and consider now the reasons why neither Democritus nor Plato, in the guise of Timaeus, satisfy it. The irony that Theophrastus discerns is that while Democritus holds that sensible objects are nothing over and above their effects on perceivers, he provides atomic accounts of them. And while Timaeus holds that sensible objects are independent of the effects on perceivers, he provides accounts of them in terms of these effects.

Democritus is said to fail to vindicate the monism of the sensible since some sensible objects are explained in terms of the shape of atoms while others are explained in terms of order and position. And this despite the fact that Democritus' official position is that sensible objects are nothing over and above their effects on perceivers. Here, Theophrastus is alluding to the position that Sextus Empiricus reports:
\begin{quote}
	Democritus sometimes does away with what appears to the senses, and says that none of these appears according to truth but only according to opinion: the truth in real things is that there are atoms and void. ``By convention sweet'', he says, ``by convention bitter, by convention hot, by convention cold, by convention color: but in reality atoms and void.'' (Sextus Empiricus, \emph{Adversus mathematicos} 8 136 = DK 68B9; \citealt[410]{Kirk:1983ly})
\end{quote}
However, if we take Democritus at his word, and hold fast to the idea that sensible objects are nothing over and above the effects on perceivers, diversity among the atomic causes of these effects is consistent with monism about the sensible.

According to Theophrastus, Timaeus holds that sensible objects are independent of their effects on perceivers and so subscribes to the position that Democritus denies, at least officially. We have seen evidence for this as well. The \emph{pathēmata} are the measure of these sensible objects, and so distinct from them, and perception is our cognizance of what is measured thereby. But despite this being Timaeus' official position, Theophrastus maintains that Timaeus only accounts for sensible objects, in a Democritean fashion, in terms of their effects on perceivers. Theophrastus, like Taylor and Cornford after him, obviously identifies Timaean sensible objects with the \emph{pathēmata}. However, as I argued in chapter~\ref{sec:pathemata}, this is a mistake.

Theophrastus develops his claim that Timaeus fails to provide a uniform account of sensible objects in his discussion of the special senses and their objects. Again, we have seen an example of this when Theophrastus objects that while Timaeus explains hot in terms of figure he fails to do so for cold (\emph{De sensibus} 87; chapter~\ref{sec:hot_and_cold}). However, there is more unity to be found in Timaeus' account of sensible objects than Theophrastus allows. Think of the central role of \emph{diakris} and \emph{sunkrisis}. These are \emph{pathēmata} to be sure, and so affections of sentient animate bodies. But Timaeus conceives of sensible objects as powers of secondary bodies to produce such affections, many of which may obtain in inanimate bodies as well. Sensible objects are, by and large, powers to divide and compact, and are further distinguished by the different parts of the body they affect and by their secondary effects such as trembling in the case of feeling cold, and the mixture of fire and water in the case of seeing brilliant or red. If we focus on the powers of secondary bodies that are determined by the primary bodies that compose them, a much more unified picture of sensible objects emerges from Timaeus' speech. These powers are generally conceived in terms of pairs of opposites (though, as we have seen in section~\ref{sec:the_tongue}, the oppositional structure of flavor is not one--one but one--many, as is the opposition between the center of a sphere and the points on its circumference, chapter~\ref{sec:heavy_and_light}). What unites these opposing powers are the proportions in which there intermediaries stand (compare 31c4).

% section concluding_observations_pp (end)

% Chapter peculiar_pathemata (end)
