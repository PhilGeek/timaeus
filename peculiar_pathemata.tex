%!TEX root = /Users/markelikalderon/Documents/Git/timaeus/timaeus.tex

\chapter{Peculiar \emph{Pathēmata}} % (fold)
\label{cha:peculiar_pathemata}

\section{Common versus Peculiar \emph{Pathēmata}} % (fold)
\label{sec:common_versus_peculiar_pathemata}

Whereas the common \emph{pathēmata} are affections common the body as a whole (61d--65b), the peculiar \emph{pathēmata} are affections that are peculiar to particular parts of the body (65b--68e). Just as in the case of the common \emph{pathēmata}, Timaeus will explain the peculiar \emph{pathēmata} in terms of the powers of the agents that cause them and explain how these powers are named in terms of the \emph{pathēmata} they cause. Whereas the common \emph{pathēmata} are involved in the measure of tactile qualities, the peculiar \emph{pathēmata} are involved in the measure of sensory qualities involved in the other four special senses---taste, smell, audition, and vision. Again, though I am anachronistically appealing to Aristotelian vocabulary here, I am bracketing any assumption that the particular parts of the body upon which the peculiar \emph{pathēmata} fall are best conceived by Timaeus as sensory organs. Whether they are shall be discussed only once we have reviewed in detail the four peculiar \emph{pathēmata}. As we shall see, they are not akin to sensory organs, at least as Aristotle conceives of them, but are more like Empedoclean \emph{poroi} (though this latter claim is subject to important qualification). The four particular parts of the body upon which the peculiar \emph{pathēmata} fall are:
\begin{enumerate}
	\item the tongue (65b--66c) (section~\ref{sec:the_tongue})
	\item the nostrils (66c--67a) (section~\ref{sec:the_nostrils})
	\item the ears (or perhaps the brain and the blood) (46c--47e, 67a--c, 80a) (section~\ref{sec:the_ears})
	\item the eyes (45b--46c, 67c--68d) (section~\ref{sec:the_eyes})
\end{enumerate} 

Why does Timaeus discuss the remaining four special senses in the order that he does? Perhaps the relative importance of pleasure and pain provides a clue. These, recall, were claimed to be the most important of the common \emph{pathēmata}. Moreover, I argued that their importance consisted in their ethical significance. Pleasure may incite the greatest of evil, while pain may deter us from the good (68d). Perhaps the present ordering of the four remaining special senses is in ascending order of ethical significance. Taste is, after all, closely connected with appetite. Moreover, it was the flood of nutriment that initially disrupts the circles of the Same and the Different in the initial shock of embodiment, a disruption providentially accommodated by our being endowed with audition and vision. That the four remaining special senses are ordered by ascending ethical significance is a speculative hypothesis only, but I see no other alternative explanation. (I emphasize the speculative characer of the hypothesis since inference to an explanation falls well short of inference to the best explanation.)

% section common_versus_peculiar_emph_pathemata (end)

\section{The Tongue} % (fold)
\label{sec:the_tongue}

The first class of \emph{pathēmata} that Timaeus discusses all fall upon the tongue. The \emph{pathēmata} involved in the experience of flavour are all brought about by certain contractions (\emph{sunkrisos}) or dilations (\emph{diakrisos}). Timaeus signals this explicitly (65c2--6). Nor is it the first time he has used this vocabulary. It has been explicitly deployed in his discussion of hot and cold (61d5–62a5), and in his discussion of pleasure and pain (64e), and it will reoccur in his discussion of color and vision (67c–68d). This uniformity is perhaps obscured in English translation given the breadth of the semantic field associated with these terms (thus obscuring as well a potential Empedoclean influence as these plausibly correspond to the operations of Love and Strife). In addition to contractions and dilations, Timaeus also claims that, more than any other sense, the experience of flavour involves the rough and the smooth. 

There are small passages in the tongue leading to the heart. Timaeus describes these as veins (\emph{phlebia}) though there is no indication that these passages are blood vessels. Timaeus only ever mentions them containing particles of earth and air (though since these may be covered with a moist film, presumably they contain particles of water as well.) Indeed, from the perspective of contemporary anatomy they seem to function less like blood vessels and more like nerves. Specifically, these veins function as ``testing instruments'' of the tongue. As instruments, they determine the flavour of the ingested nutriment, depending upon how the nutriment acts upon them. As instruments, they are the measure of the nutriment's flavour. Presumably these instruments are tests since, depending upon the flavour, the nutriment may be either swallowed or spat out if foul. The veins in the tongue thus clearly function as Empedoclean \emph{poroi} (for the role of \emph{poroi} or passages in Empedocles's discussion of \emph{aisthēsis} see Plato \emph{Meno} 76 a--d, Theophrastus \emph{De Sensibus} 7, and \citealt{Beare:1906uq}.) Whether or not the veins in the tongue, with the function that Timaeus ascribes to them, are best understood as blood vessels, cannot be settled with reference to Timaeus' discussion of the tongue alone.

In addition to containing veins leading to the heart, the tongue is described as moist and soft. Both these features function in taste. Thus the moistness of the tongue is said to melt particles of earth prior to their entry into the veins of the tongue (though liquefication will play a larger and more systematic role in Aristotle's account of digestion, \emph{De anima} 2.4). And the softness of the tongue is relevant to its susceptibility to roughening. 

Recall the methodological dilemma raised by Timaeus for his account of the \emph{pathēmata} (61c4–d5) (discussed in chapter~\ref{sec:pathemata}). On the one hand, one cannot adequately account for the \emph{pathēmata} without accounting for the origin of the flesh and things pertaining to it and the mortal parts of the soul; on the other hand, one cannot adequately account for the origin of the flesh and things pertaining to it and the mortal parts of the soul without accounting for the \emph{pathēmata}. Perhaps more than with the common \emph{pathēmata}, we can see how Timaeus' discussion of the \emph{pathēmata} presupposes at least the origins of the flesh and things pertaining to it. An account of \emph{pathēmata} peculiar to the tongue would not be possible without making assumptions about the nature and organization of human flesh. The tongue is human flesh. It is moist and soft and contains veins. The way it is affected so as to be liable to give rise to the experience of flavour would be inexplicable without making some such assumptions.

\emph{Pathēmata} peculiar to the tongue are involved in the sensation of:
\begin{enumerate}
	\item astringent and harsh (65c6--d4)
	\item bitter and saline (65d4--65e4)
	\item pungent (65e4--66a2)
	\item sour (66a2--b7)
	\item sweet (66b7--c7)
\end{enumerate}
The flavours stand in a one--many oppositional structure. Sweet is not merely opposed to bitter, but to the whole range of alternative flavours. In this respect it is like the opposition of the center of the cosmos with the points that lie, in extremity, on its circumference, The flavours thus contrast, in different ways, with odours and colors. Odours lack any meaningful oppositional structure admitting only to a loose classification as pleasant and unpleasant. Colors, like flavours and unlike odours, have an oppositional structure, but the opposition is between white and black with the other colors arrayed as intermediaries between them. The chromatic oppositional structure is thus one--one as opposed to one--many. This is the more usual pattern of oppositional classification in classical antiquity (see \citealt{Lloyd:1966ly} for discussion).

Allow me to make one further preliminary observation before discussing the \emph{pathēmata} involved in tasting the individual flavours. Timaeus begins his discussion of the \emph{pathēmata} peculiar to the tongue by claiming to return to what was omitted in his previous account. This is a backreference to the his discussion of the flavours or juices among the \emph{genē}. As we shall see, this is a further confirmation of an important point argued in the previous chapter, namely, that the \emph{pathēmata} are merely the effects on sentient animate bodies, effects that may obtain among inanimate bodies as well.

The \emph{pathēmata} involved in the sensations of astringent and harsh consists in the drying and constriction of the veins of the tongue. Timaeus explains these affections in terms of the character of the primary bodies that compose the nutriment. Specifically, when particles of earth fall upon the moist and soft tongue, these have a tendency to melt with the result that the veins in the tongue constrict and become dryer. When these particles are rougher this process is liable to give rise to a sensation of astringency, and when less rough theis process is liable to give rise to a sensation of harshness. Though Timaeus does not explicitly say so, the constriction and drying of the tongue associated with the sensations of astringency and harshness are a departure from the natural state of the tongue. This is what explains their opposition to sweetness. Sweet things are sweet in part by their restoring the tongue to its natural state, hence the pleasure associated with this sensation. 

The \emph{pathēmata} involved in the sensations of bitter and harsh are also affections of the veins in the tongue, though in this case the veins are acted upon in the manner of a detergent washing out the veins. Timaeus emphasizes not only that the nutriment acts upon the veins of the tongue as a detergent but does so excessively. I take Timaeus to mean here that this is a departure from the natural state of the tongue. Thus nutriment excessively acting as a detergent is said to dissolve the substance of the tongue. Moreover, as we shall see, this departure from the natural state of the tongue explains well these flavours opposition to sweetness. The difference between the \emph{pathēmata} involved in the sensations of bitter and saline is a matter of degree. When the affection is more excessive, this process is liable to give rise to the sensation of bitterness. When it is less excessive and more agreeable, this process is liable to give rise to the sensation of saline.

The \emph{pathēmata} involved in pungent sensations are more complex and constitute a striking anticipation of the role of retronasal olfaction in taste. In this case the nutriment is both hot and is susceptible to sharing in the heat of the tongue itself. As a result of sharing in the heat of the tongue, the nutriment becomes softened and fiery. This burns the tongue. Moreover, since the nutriment is fiery, particles of the nutriment move upwards (since primary bodies naturally move in the direction of their native environment). These fiery particles cut all that the parts upon which they impinge. Think of the way that excessively hot food sensibly affects the nasal cavities. This process is liable to give rise to a pungent sensation. Again, though Timaeus does not make this explicit, the \emph{pathēmata} involved in pungent sensations, since these include burning and cutting, are a departure from the natural state of the tongue. And if that is right, then the opposition to sweetness is explained.

% section the_tongue (end)

\section{The Nostrils} % (fold)
\label{sec:the_nostrils}



% section the_notrils (end)

\section{The Ears} % (fold)
\label{sec:the_ears}



% section the_ears (end)

\section{The Eyes} % (fold)
\label{sec:the_eyes}



% section the_eyes (end)

% Chapter peculiar_pathemata (end)
