%!TEX root = /Users/markelikalderon/Documents/Git/timaeus/timaeus.tex

\chapter{Peculiar \emph{Pathēmata}} % (fold)
\label{cha:peculiar_pathemata}

\section{Common versus Peculiar \emph{Pathēmata}} % (fold)
\label{sec:common_versus_peculiar_pathemata}

Whereas the common \emph{pathēmata} are \emph{pathēmata} common the body as a whole (61d--65b), the peculiar \emph{pathēmata} are \emph{pathēmata} that are peculiar to particular parts of the body (65b--68e). Just as in the case of the common \emph{pathēmata}, Timaeus will explain the \emph{pathēmata} in terms of the powers of the agents that cause them and explain how these powers are named in terms of the \emph{pathēmata} they cause. Whereas the common \emph{pathēmata} are involved in the measure of tactile qualities, the peculiar \emph{pathēmata} are involved in the measure of sensory qualities involved in the other four special senses---taste, smell, audition, and vision. Again, though I am anachronistically appealing to Aristotelian vocabulary here, I am bracketing any assumption that the particular parts of the body upon which the peculiar \emph{pathēmata} fall are best conceived by Timaeus as sensory organs. Whether they are shall be discussed only once we have reviewed in detail the four peculiar \emph{pathēmata}. As we shall see, they are not akin to sensory organs, at least as Aristotle conceives of them, but are more like Empedoclean \emph{poroi} (though this latter claim is subject to important qualification). The four particular parts of the body upon which the peculiar \emph{pathēmata} fall are:
\begin{enumerate}
	\item the tongue (65b--66c) (section~\ref{sec:the_tongue})
	\item the nostrils (66c--67a) (section~\ref{sec:the_nostrils})
	\item the ears (or perhaps the brain and the blood) (46c--47e, 67a--c, 80a) (section~\ref{sec:the_ears})
	\item the eyes (45b--46c, 67c--68d) (section~\ref{sec:the_eyes})
\end{enumerate} 

Why does Timaeus discuss the remaining four special senses in the order that he does? Perhaps the relative importance of pleasure and pain provides a clue. These, recall, were claimed to be the most important of the common \emph{pathēmata}. Moreover, I argued that their importance consisted in their ethical significance. Pleasure may incite the greatest of evil, while pain may deter us from the good (68d). Perhaps the present ordering of the four remaining special senses is in ascending order of ethical significance. Taste is, after all, closely connected with appetite. Moreover, it was the flood of nutriment that initially disrupts the circles of the Same and the Different in the initial shock of embodiment, a disruption providentially accommodated by our being endowed with audition and vision. That the four remaining special senses are ordered by ascending ethical significance is a speculative hypothesis only, but I see no other alternative explanation. (I emphasize the speculative characer of the hypothesis since inference to an explanation falls well short of inference to the best explanation.)

% section common_versus_peculiar_emph_pathemata (end)

\section{The Tongue} % (fold)
\label{sec:the_tongue}

The first class of \emph{pathēmata} that Timaeus discusses all fall upon the tongue. The \emph{pathēmata} involved in the experience of flavour are all brought about by certain contractions (\emph{sunkrisos}) or dilations (\emph{diakrisos}). Timaeus signals this explicitly (65c2--6). Nor is it the first time he has used this vocabulary. It has been explicitly deployed in his discussion of heat and cold, and in his discussion of pleasure and pain (64e). This uniformity is perhaps obscured in English translation given the breadth of the semantic field associated with these terms (thus obscuring as well a potential Empedoclean influence as these plausibly correspond to the operations of Love and Strife). In addition to contractions and dilations, Timaeus also claims that, more than any other sense, the experience of flavour involves the rough and the smooth. Since the tongue is what receives these affections, it is the mouth cavity that corresponds to Empedoclean \emph{poroi}, for it is a passage through which nutriment is channelled into the interior where it may be tasted by the tongue. (For the role of \emph{poroi} or passages in Empedocles's discussion of \emph{aisthēsis} see Plato \emph{Meno} 76 a--d, Theophrastus \emph{De Sensibus} 7, and \citealt{Beare:1906uq}.)

\emph{Pathēmata} peculiar to the tongue are involved in the perception or sensation of:
\begin{enumerate}
	\item astringent and harsh (65c6--d4)
	\item bitter and saline (65d4--65e4)
	\item pungent (65e4--66a2)
	\item acid (66a2--b7)
	\item sweet (66b7--c7)
\end{enumerate}

% section the_tongue (end)

\section{The Nostrils} % (fold)
\label{sec:the_nostrils}



% section the_notrils (end)

\section{The Ears} % (fold)
\label{sec:the_ears}



% section the_ears (end)

\section{The Eyes} % (fold)
\label{sec:the_eyes}



% section the_eyes (end)

% Chapter peculiar_pathemata (end)
