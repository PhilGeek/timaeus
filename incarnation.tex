%!TEX root = /Users/markelikalderon/Documents/Git/timaeus/timaeus.tex
\chapter{Incarnation} % (fold)
\label{cha:incarnation}

\section{Cosmic and Encosmic Incarnation} % (fold)
\label{sec:cosmic_and_encosmic_incarnation}

There are similarities between cosmic and encosmic incarnation. The generation of the Cosmos and its ensoulment and the generation of the human body and its ensoulment are in many ways similar. But, importantly, there are differences. 

First, there is a difference in agency. Whereas, the Demiurge generates the body of the Cosmos and the soul that animates it, the young gods, generated by the Demiurge, act on His behest in generating the human body. At least the immortal part of the soul is generated by the Demiurge with similar materials, if less pure, than the materials from which the World-Soul was generated. The difference in agency is explained by the limitations of the young gods' power of generation. In generating mortal living beings, the young gods imitate Demiurgic activity. Their imitation is imperfect, however. Whereas the Demiurge generates immortals, the young gods generate only mortals. This is precisely why they are assigned this task. In order for a sensible living being to be truly comprehensive, it must contain within itself all manner of mortal living beings. 

Second, there is a difference in body. The bodies of mortal living beings are importantly different from the body of the Cosmos that contains them. Mortal living beings contained within the Cosmos are situated in an environment with strong powers. The Cosmos, by contrast, is not situated in an environment. There is nothing beyond the limits of the Cosmos, neither strong powers, nor space, nor void. Mortal living beings depend upon and must defend against the environment and consequently have a number of powers that are only ever exercised in an environment. That mortal living beings are equipped with the instruments of these powers explains the way their bodies differ from the body of the Cosmos.

Third, there is a difference in soul. The human soul has mortal and immortal parts. While the young gods generate the mortal part of the soul, the Demiurge generates the immortal part from the same material, if less pure, that He used to generate the World-Soul. Like the World-Soul, it is composed of the Circles of the Same and the Different, and though Timaeus does not say so explicitly, presumably the immortal part of the human soul is structured in the same way with arithmetic, geometric, and harmonic intervals. However, whereas the motions of the World-Soul are unwaverable, the motions of the immortal part of the human soul are waverable, they may be affected by strong powers in the sensible environment. Moreover, unlike the Cosmos, human beings have, in addition, mortal parts of the soul. That mortal living beings are situated in an environment with strong powers explains why their souls have mortal parts and why the motions of their soul may waver.

Fourth, soul and body are differently related. The World-Soul encompasses the body of the Cosmos. Though some commentators may try to resist it, Timaeus spatial language is clear. The body of the Cosmos is contained within the World-Soul. The immortal part of the human soul, by contrast, is encompassed by the human body. The immortal part of the soul is contained within the human body. It is bound to the marrow that composes the brain which is encased in the skull which is itself covered by flesh and hair. Just as the flesh protects the skull that it contains, providing padding that will soften any blow that may befall it, the skull protects the brain that it contains and so preserves its bond with the immortal part of the soul. In this way, the body is organized so as to protect its bond with the immortal soul that it contains and that animates it. Mortal living beings are situated within the immortal Cosmos in the way that the Cosmos is not. That mortal living beings are situated in an environment with strong powers explains how their soul and body are differently related.

Fifth, there is a difference in narrative order. Timaeus narrates the generation of the body of the Cosmos before narrating how it became ensouled. This narrative is reversed when it comes to the generation of human beings. Timaeus narrates the incarnation of the human soul before narrating the generation of the human body. This narrative reversal is accompanied by a perspectival shift. Timaeus narrates the incarnation of the human soul from the perspective of the human soul, the dramatic highpoint of which is Timaeus vivid description of the shock of embodiment. The ensoulment of the body of the Cosmos, by contrast, is impersonally presented. This perspectival shift is ethically significant. In dramatizing the shock of embodiment undergone by a newly incarnate human soul, Timaeus makes vivid the need for salvation, thus laying the groundwork for his soteriology (chapter~\ref{cha:the_end_of_vision_and_audition}).

The present chapter will follow the order in which Timaeus narrates the incarnation of the human soul and the generation of the human body. The differences highlighted above shall be further explained and their significance assessed (all but one---the mortal soul shall be discussed in chapter~\ref{cha:the_flesh_and_the_mortal_soul}). Just as the Cosmos lacks certain features since it lacks an environment, mortal living beings have certain features since they are situated in an environment with strong powers that they depend upon and must defend against. Again, the principle lesson for the philosophy of perception shall be the environmental nature of perception. Sensory powers are only ever exercised on aspects of the sensible environment in which the percipient is situated.

% section cosmic_and_encosmic_incarnation (end)

\section{The Demiurge's Address} % (fold)
\label{sec:the_demiurge_addressing_the_gods}

The Demiurge addresses two classes of gods. Specifically, the Demiurge addresses, on the one hand, the gods that manifestly revolve (celestial bodies such as the fixed stars) and, on the other hand, the gods that manifest themselves at will (the Titanic and Olympian gods familiar from the poets and state religion). The young gods are the instruments of Demiurgic activity, and the Demiurge, in his address, reveals both his intentions, and the reason for the involvement of the young gods in carrying out His intentions.

The address itself divides into three parts:
\begin{enumerate}[(1)]
	\item A discussion on the nature of bonds forms the basis of the assurance that the Demiurge gives the young gods. Though not wholly immortal, they need not fear death, since only what is evil would undo the bonds of what is well-fitted and good.
	\item The Demiurge instructs the young gods to generate the three remaining kinds of mortal beings. This they must do, since the Demiurge can only generate immortal beings, and mortal beings are required to complete the Cosmos conceived as a comprehensive sensible living being.
	\item The Demiurge explains that his role in the generation of mortal beings will be limited to generating the immortal part and instructs the young gods to weave what is mortal into what is immortal.
\end{enumerate}


(1) The Demiurge's discussion of bonds (\emph{desmō}) has antecedents in Timaeus' speech, specifically in the proportionate bonds that unite the opposed extremes of fire and earth in amity in the body of the Cosmos. The young gods, like the body of the Cosmos, are compounded. And what unites the constituents of the compound into a whole are bonds. Though the Demiurge does not say, presumably these are proportionate bonds. After all, the fairest bond that most perfectly unites what it joins is proportion (\emph{analogia}, 31c2--4). 

The first thing that the Demiurge does is explicitly state a principle that may be cause for concern, at least for those young gods that may fear death. All that may be joined together by bonds may be undone. This is the reason that the young gods are not wholly immortal and indissoluble. The qualifier ``wholly'' already hints at Demiurgic assurance. His assurance consists in the claim that only what is evil would undo the bonds of what is well-fitted and good. That principle only constitutes an assurance against the background of two further assumptions. First, that the young gods are well-fitted and good. For if they are not, the principle does not say that it would be evil to undo that which is ill-fitted and bad, and so no assurance would be given. Second, that the Demiurge, as the maker of these bonds, is not evil. The Demiurge is not at all evil, but benevolent and ungrudging. 

There is a further implicit commitment here, namely, that only the Demiurge has the power to undo the bonds by which the young gods were compounded. For suppose this power was not the Demiurge's alone. Then some other agent could be the cause of their undoing. Even if the Demiurge, being benevolent and ungrudging, shall not undo the bonds that join them, some other agent may yet exercise this power should they be suitably evil. This last implicit commitment was earlier made explicit in Timaeus discussion of the elemental composition of the body of the Cosmos (32c3--5) and will be made explicit again in Timaeus' warning that it is impious to empirically investigate color mixture. Only God is wise enough and powerful enough to blend the many into one and dissolve the one into many (68c7–d7).

The Demiurge ends this portion of his address with hyperbolic claim that should be understood as the expression of His overflowing goodness. The Demiurge is benevolent and ungrudging and would not will the undoing what is well-fitted and good. The Demiurge now represents his will has a greater bond than what bound them in geenration. By Timaeus' lights at least, this Demiurgic claim is hyperbolic. Again, he has explicitly stated that the fairest bond that most perfectly unites what it joins is proportion. If the Demiruge's will is greater still, does that mean that the bond that it constitutes is fairer and more perfectly unites what it joins? The hyperbole is not a boast, nor meant to mislead, but rather dramatizes the benevolent and ungrudging nature of the Demiurge whose overflowing goodness is the real guarantee of the immortality of the young gods. 

(2) In the second part of His address, the Demiurge instructs the young gods to generate the three remaining kinds of mortal beings. So according to Timaeus, the Cosmos contains four kinds of living beings:
\begin{enumerate}[(a)]
	\item the young gods
	\item the beings that live in the air
	\item the beings that live in the water
	\item the beings that live on the earth
\end{enumerate}

% section the_demiurge_addressing_the_gods (end)

\section{The Laws of Destiny} % (fold)
\label{sec:the_laws_of_destiny}



% section the_laws_of_destiny (end)

\section{The Shock of Embodiment} % (fold)
\label{sec:the_shock_of_embodiment}



% section the_shock_of_embodiment (end)

\section{The Generation of the Human Body} % (fold)
\label{sec:structuring_the_human_body}



% section structuring_the_human_body (end)


% chapter incarnation (end)