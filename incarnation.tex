%!TEX root = /Users/markelikalderon/Documents/Git/timaeus/timaeus.tex
\chapter{Incarnation} % (fold)
\label{cha:incarnation}

\section{The Demiurge Addressing the Gods} % (fold)
\label{sec:the_demiurge_addressing_the_gods}



% section the_demiurge_addressing_the_gods (end)

\section{The Laws of Destiny} % (fold)
\label{sec:the_laws_of_destiny}



% section the_laws_of_destiny (end)

\section{The Shock of Embodiment} % (fold)
\label{sec:the_shock_of_embodiment}



% section the_shock_of_embodiment (end)

\section{Structuring the Human Body} % (fold)
\label{sec:structuring_the_human_body}



% section structuring_the_human_body (end)

\section{Vision} % (fold)
\label{sec:vision}

The disruption incurred by the shock of embodiment is accommodated by the Demiurge providentially providing human beings with the capacities for vision and audition. In exercising these capacities in the right way, the circles of the Same and the Different return to their proper revolutions. 

Timaeus' discussion of vision is structured around the contrast between \emph{aitia} and \emph{sunaitia}.


The young gods present task is the construction of the face, the forward part of the vessel of the head. The face is arrayed with organs for the forethought of the soul, and the eyes are the first of the organs that the young gods construct. How are the eyes organs for the forethought of the soul? And why are they a priority for the young gods? Forethought, here, carries suggestions of distance and time. If one can see something in the distance one can plan how best to act in the time before encountering it. This may be of vital concern should the thing that one sees in the distance be predator or prey. Timaeus is hinting right at the beginning that vision is a great benefit to humanity. Moreover, as we shall see, the benefit is not narrowly confined to its survival value. The temporal implications of forethought are especially important. By means of vision we have a view of the celestial clockwork and so can distinguish days and nights, months, and years. In observing these we acquire number and even greater feats of forethought are made possible both in agriculture and seafaring. In exercising our reason in acquiring and deploying number we become capable of engaging even in philosophy. Moreover, in so doing we align the circles of the Same and the Different in the immortal part of our soul with the circles the Same and the Different in the World-Soul whose movement is visibly manifest in the celestial clockwork. We see divine forethought and thus may offset the disturbing effects of the shock of embodiment. In this way is vision the greatest benefit that the gods bestow upon us. Notice that in answering our first question, concerning the eyes as organs for the forethought of the soul, we have also answered our second question, concerning the priority of the eyes. The eyes are a priority for the young gods precisely because they are the greatest benefit to humanity. This issue of priority will be echoed in Timaeus' later discussion of the affections peculiar to particular parts of the body. For there, he will discuss the parts of the body whose affection is liable to give rise to perception or sensation in ascending order of ethical significance culminating with the eyes (chapter~\ref{sec:common_versus_peculiar_pathemata}).

The Timaean anatomy of the eye is broadly Empedoclean (DK 31B84--88). Each represents the eye as the product of divine craftsmanship, with the young gods in Timaeus' account playing Aphrodite's role in Empedocles'. Each maintains that there is a fire in the eye, likened to light, that flows through the center owing to its fineness. Thus the organs that the young gods affix to the face are said to be light-bearing eyes (\emph{phōsphora ommata}). Though Empedocles does not use the adjective \emph{phōsphora} in DK 31B84, the lantern analogy that he there develops for the eye clearly implies that they are, in fact, light-bearing (in some sense or another, depending upon how that analogy is to be understood). A variant form of that adjective, \emph{phaesphorōi}, is used and applied to the eye in Euripides' \emph{Cyclops} 462 (see \citealt[489-90]{Seaford:1984vb} and \citealt[114]{Johansen:2004dx}). In a cruel irony Odysseus, in conveying his plan to brand the Cyclops' eye describes it as truly light-bearing. Notice that the irony only gets a grip against the background of some view, like Empedocles', that the eyes anyway bear light, though perhaps not in the way that the Cyclops' will. This is a view that Timaeus shares not only with Empedocles, but also Alcmaeon of Croton (see \citealt[11--13]{Beare:1906uq}), and it was widely enough held to be the basis for Euripides' irony his satyr play.

In his discussion of the \emph{gēne}, Timaeus claims that there are many kinds of fire and gives three examples though he does not claim that these exhaust the kinds of fire that there are (58c5–d1). First, there is flame (\emph{phlox}). Second, there is the kind that issues from flame but does not burn but supplies light to the eyes. And third, there is the kind that is left behind in embers after the flame is quenched. The young gods contrive daylight, a body proper to each day, out of the second kind of fire, that gives a mild light but does not burn (45b4--6). There is a play on words here that hints at an etymology. The Greek word for day \emph{hemeras} clearly echoes the adjective \emph{hemeron} that means mild. In the \emph{Cratylus} 418 c--d, Socrates spurns the etymology that takes \emph{hermon} as the root of \emph{hemeras} preferring the older etymology that takes \emph{himera} as the root. So understood, it is not the mildness of its light by which the day is called but our longing for the light after the dark. Both resonances may be in play here. Timaeus may be emphasizing both the mildness of the day's light and our longing for it. Timaeus will argue that the mild light that does not burn proper to the body of the day is necessary for vision. It is among the auxiliary causes of vision. But the longing for light naturally motivates our looking toward the celestial clockwork and so contributes to the end of sight as well. And just as Socrates suggests that the older etymology provides us with a deeper understanding of the significance of \emph{hemeras}, the longing for light, given its connection with providential end of sight, provides us with a deeper understanding of vision. Indeed, this is an instance of a methodological point that Timaeus himself makes, looking to the \emph{aitia} one gains a better understanding of the \emph{sunaitia}.

Sunlight, then, is a body proper to the day composed of a mild light that does not burn. The fire within the eye is said to be akin, \emph{adelphon}, literally brother, to the mild light of the day. Being sons of the same mother the fire within the eye and the sunlight without belong to the same family. Perhaps Timaeus is suggesting in this way that each is the same kind of fire, that gives a mild light but does not burn. I hedge only because Timaeus stops just short of making this claim explicit. It is nonetheless plausible that the fire within the eye is mild and does not burn since, if it did, the eye would be damaged, as the fire of Odysseus' brand damages the Cyclops' eye. Timaeus himself will give an example of this kind. Brilliant (\emph{stilbon}) bodies emit a flame whose particles enter the passages in the eye and dissolves them. Beyond its kinship with sunlight, it is independently plausible, then, that the fire within the eye is of the kind that does not burn.

The young gods cause this fire within, brother to the sunlight, to flow through the eyes (45b6--c2, for the grammar of this passage see \citealt[63 123--127]{Cook-Wilson:1889cs} and \citealt[277]{Taylor:1928qb}). With this in mind, they compacted the whole of the eye and made its center, presumably the pupil or perhaps the cornea, especially smooth and dense so as to only allow the internal fire to escape, owing to its fineness, while keeping within all that is coarser. The center of the eye is an efficient filter. Only the finest and purest of fire is said to escape. The fire emanating from within is said to have two consequences depending upon the occasion. When it is day, the fire emanating from within forms a compound with the sunlight, and when it is night and the body proper to the day withdraws, it is quenched and so extends no further.

% section vision (end)

\section{Audition} % (fold)
\label{sec:audition}



% section audition (end)

% chapter incarnation (end)