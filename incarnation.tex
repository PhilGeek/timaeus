%!TEX root = /Users/markelikalderon/Documents/Git/timaeus/timaeus.tex
\chapter{Incarnation} % (fold)
\label{cha:incarnation}

\section{The Demiurge Addressing the Gods} % (fold)
\label{sec:the_demiurge_addressing_the_gods}



% section the_demiurge_addressing_the_gods (end)

\section{The Laws of Destiny} % (fold)
\label{sec:the_laws_of_destiny}



% section the_laws_of_destiny (end)

\section{The Shock of Embodiment} % (fold)
\label{sec:the_shock_of_embodiment}



% section the_shock_of_embodiment (end)

\section{Structuring the Human Body} % (fold)
\label{sec:structuring_the_human_body}



% section structuring_the_human_body (end)

\section{Vision} % (fold)
\label{sec:vision}

The disruption incurred by the shock of embodiment is accommodated by the Demiurge providentially providing human beings with the capacities for vision and audition. In exercising these capacities in the right way, the circles of the Same and the Different return to their proper revolutions. 

Timaeus' discussion of vision is structured around the contrast between \emph{aitia} and \emph{sunaitia}.


The young gods present task is the construction of the face, the forward part of the vessel of the head. The face is arrayed with organs for the forethought of the soul, and the eyes are the first of the organs that the young gods construct. How are the eyes organs for the forethought of the soul? And why are they a priority for the young gods? Forethought, here, carries suggestions of distance and time. If one can see something in the distance one can reason how best to act in the time before encountering it. This may be of vital concern should the thing that one see be predator or prey. Timaeus is hinting right at the beginning that vision is a great benefit to humanity. Moreover, as we shall see, the benefit is not narrowly confined to its survival value. The temporal implications of forethought are especially important. By means of vision we have a view of the celestial clockwork and so can distinguish days and nights, months, and years. In observing these we acquire number and even greater feats of forethought are made possible both in agriculture and seafaring. In exercising our reason in acquiring and deploying number we become capable of engaging even in philosophy. Moreover, in so doing we align the circles of the Same and the Different in the immortal part of our soul with the circles the Same and the Different in the World-Soul whose movement is visibly manifest in the celestial clockwork. We see divine forethought and thus may offset the disturbing effects of the shock of embodiment. In this way is vision the greatest benefit that the gods bestow upon us. Notice that in answering our first question, concerning the eyes as organs for the forethought of the soul, we have also answered our second question, concerning the priority of the eyes. The eyes are a priority for the young gods precisely because they are the greatest benefit to humanity. This issue of priority will be echoed in Timaeus' later discussion of the affections peculiar to particular parts of the body. For there, he will discuss the parts of the body whose affection is liable to give rise to perception or sensation in ascending order of ethical significance culminating with the eyes (chapter~\ref{sec:common_versus_peculiar_pathemata}).

The Timaean anatomy of the eye is broadly Empedoclean (DK 31B84--88). Each represents the eye as the product of divine craftsmanship, with the young gods in Timaeus' account playing Aphrodite's role in Empedocles'. Each maintains that there is a fire in the eye, likened to light, that flows through the center owing to its fineness. Thus the organs that the young gods affix to the face are said to be light-bearing eyes (\emph{phōsphora ommata}). Though Empedocles does not use the adjective, \emph{phōsphora}, in DK 31B84, the lantern analogy that he there develops for the eye clearly implies that they are, in fact, light-bearing (in some sense or another, depending upon how that analogy is to be understood). A variant form of that adjective, \emph{phaesphorōi}, is also applied to the eye in Euripides' \emph{Cyclops} 462 (see \citealt[489-90]{Seaford:1984vb} and \citealt[114]{Johansen:2004dx}). In a cruel irony, Odysseus, in describing his plan to brand with fire the Cyclops' eye says that it shall be truly light-bearing. Notice that the irony only gets a grip against the background of some view, like Empedocles', that the eyes anyway bear light, though perhaps not in the way that the Cyclops' will. This is a view that Timaeus shares not only with Empedocles, but also Alcmaeon of Croton (see \citealt[11--13]{Beare:1906uq}), and it was widely enough held to be the basis for Euripides' irony in his satyr play.

In his discussion of the \emph{gēne}, Timaeus claims that there are many kinds of fire and gives three examples though he does not claim that these exhaust the kinds of fire that there are (58c5–d1). First, there is flame (\emph{phlox}). Second, there is the kind that issues from flame but does not burn but supplies light to the eyes. And third, there is the kind that is left behind in embers after the flame is quenched. The young gods contrive daylight, a body proper to each day, out of the second kind of fire, that gives a mild light but does not burn (45b4--6). There is a play on words here that hints at an etymology. The Greek word for day, \emph{hemeras}, clearly echoes the adjective, \emph{hemeron}, that means mild. In the \emph{Cratylus} 418 c--d, Socrates spurns the etymology that takes \emph{hemeron} as the root of \emph{hemeras} preferring the older etymology that takes \emph{himera} as the root. So understood, it is not the mildness of its light by which the day is called but our longing for its light after the dark of night. Both resonances may be in play here. Timaeus may be emphasizing both the mildness of the day's light and our longing for it. Timaeus will argue that the mild light that does not burn proper to the body of the day is necessary for vision. It is among the auxiliary causes of vision. But the longing for light naturally motivates our looking toward the celestial clockwork and so contributes to the end of sight as well. And just as Socrates suggests that the older etymology provides us with a deeper understanding of the significance of \emph{hemeras}, the longing for light, given its connection with providential end of sight, provides us with a deeper understanding of vision. Indeed, this is an instance of a methodological point that Timaeus himself makes, looking to the \emph{aitia} one gains a better understanding of the \emph{sunaitia}.

Daylight, then, is a body proper to the day composed of a mild light that does not burn. The fire within the eye is said to be akin, \emph{adelphon}, literally brother, to the mild light of the day (so like the English ``akin'', ``\emph{adelphon}'' may convey either similarity or familial relationship). Being sons of the same mother, the fire within the eye and the daylight without belong to the same family. Perhaps Timaeus is suggesting in this way that each is the same kind of fire, that gives a mild light but does not burn. I hedge only because Timaeus stops just short of making this claim explicit. It is nonetheless plausible that the fire within the eye is mild and does not burn since, if it did, the eye would be damaged, as the fire of Odysseus' brand damages the Cyclops' eye. Timaeus himself will give an example of this kind. Brilliant (\emph{stilbon}) bodies emit a flame whose particles enter the passages in the eye and dissolves them (chapter~\ref{sec:the_eyes}). Beyond its kinship with daylight, it is independently plausible, then, in the framework of the Timaeus, that the fire within the eye is of the kind that does not burn.

The young gods cause this fire within, brother to daylight, to flow through the eyes (45b6--c2, for the grammar of this passage see \citealt[63 123--127]{Cook-Wilson:1889cs} and \citealt[277]{Taylor:1928qb}). With this in mind, they compacted the whole of the eye and made its center, presumably the pupil or perhaps the cornea, especially smooth and dense so as to only allow the internal fire to escape, owing to its fineness, while keeping within all that is coarser. This closely corresponds in description and function to the screen of shaved horn, or perhaps of flaxen linen, in Empedocles' lantern analogy (DK 31B84). The center of the eye is an efficient filter. Only the finest and purest of fire is said to escape. The fire emanating from within is said to have two consequences depending upon the occasion. When it is day, the fire emanating from within meets what is like it and forms a compound with the daylight, and when it is night and the body proper to the day withdraws, it meets what is unlike and is quenched and so extends no further. Let us consider these in turn.

First, when it is day, the fire within goes forth like to like (\emph{homoion pros homoion}). The fire within is brother to the daylight and is like it in giving a mild light that does not burn. So when it is day, the fire emanating from within meets what is like it. This encounter of like with like has an important consequence. The fire emanating from within is compacted with the daylight (\emph{sumpages}, appearing in the text as {\sbl ξυμπαγές} as opposed to {\sbl συμπαγές}). This happens in whatever direction the perceiver is looking when the internal fire comes into contact with an external body. As it happens, the body proper to the day is like the internal fire in being a mild light that does not burn. In compacting the two fires thus form a unified homogenous body (\emph{hen sōma oikeiōthen}) and so the sons of the same mother come to share a common household. Later, not only are the two fires said to be compacted but also grown together (\emph{sumphues} appearing as {\sbl ξυμφυὲς}) (64d8), further emphasizing that the compound is a unified body. This unified homogenous body is aligned in the direction in which the eyes are looking. So Timaeus reifies lines of sight, conceiving of them as a unified homogenous bodies compounded from the fire within and brother daylight without that it encounters in the direction that the perceiver is looking. In Greek, ``\emph{opsis}'' means vision or sight and yet Timaeus will use this term to refer to the body compounded of internal and external fire. Following \citet[221]{Ierodiakonou:2005ly}, I shall refer to the compound as the visual body.

The principle governing the compounding of the visual body is like's affinity for like. We should not confuse this with another application of this principle in the vicinity. It is the compounding of the visual body, an auxiliary cause of vision, and not vision itself that is governed by the principle of like's affinity for like. In \emph{De anima} (1.2 404b7--15), Aristotle claims that Empedocles explains perception by like's affinity for like, citing the following passage as evidence:
\begin{verse}
	By earth we see earth; by water, water;\\
	by aither, shining aither; but by fire, blazing fire;\\
	love by love and strife by baneful strife (Empedocles DK 31B109; \citealt[221]{Inwood:2001ve})
\end{verse} 
(This is controversial, however, see \citealt{Kamtekar:2009fk}.) Aristotle goes on to claim that Timaeus, like Empedocles, similarly explains perception in terms of the principle of like's affinity for like (\emph{De anima} 1.2 404b16--9). But, in the present passage at least, Timaeus has in mind a more limited application of this principle. The principle is only invoked to explain the compounding of the visual body. Sight only occurs when the visual body is affected as a whole and this affection is conveyed to the body where it reaches the soul. So there are further phases in the causal process eventuating in perception subsequent to the compounding of the visual body. Thus it is only the compounding of the visual body, a necessary if insufficient condition for perception, and not the perception that may eventuate, that is here being explained.

Since internal and external fire are compacted and grown together to form a unified homogenous body, this body, the visual body, is affected as a whole. Moreover it is the principle of its compounding, like's affinity for like, that explains this. It is because the visual body is uniformly formed that it is uniformly affected (\emph{homoiopathes dē di homoiotēta}). (The unity and homogeneity of the visual body and its consequences for how it may be affected is usefully compared to the role of sympathy in uniting \emph{eidola} emanating from sensible objects in Epicurus' \emph{Letter to Herodotus}, see \citealt{Lee:1978yz}.) In the discussion of pleasure (64a2--c7), Timaeus explains that the communication of the affection depends upon the mobility of the particles which compose the relevant part of the body. Fire particles are particularly mobile, and so it might be thought that the affection is communicated through the visual body owing to the mobility of its constituents. But the present discussion of vision suggests an alternative model. The visual body is a unified homogenous body that is uniformly affected. It is this unified body as a whole that receives the affection which it passes on to the body before reaching the soul (45c--d; see \citealt[72, 92 n46]{Hahm:1978ny} for discussion). So the affection is not communicated part by part within the visual body as one might be tempted to think. The mobile nature of the recipient would seem to apply to the way the visual body as a whole passes around the affection to the rest of the body before it reaches the soul.

This feature of Timaeus' account was widely accepted even among those who propounded alternatives to it \cite[chapter 1]{Lindberg:1977aa}. Aristotle explicitly denies that vision has any extramissive element (\emph{De sensu} 437b10--14). There is no need for daylight to compound with fire emanated from within since daylight already, by itself, constitutes a homogenous unity that may be acted upon as a whole. And this formal feature of Timaeus' account is preserved in the stick analogy that Alexander of Aphrodisias attributes to the Stoics (\emph{De anima} 130 14; the Stoic analogy was criticized by Galen in \emph{De placitis Hippocratis et Plotonis} 2.5, 2.7, and by Tideus in \emph{De speculis}, and echoed with a distinctively modernist interpretation in Descartes' \emph{La Dioptrique}). In poking something with a stick, it is not as if the poke propagates through the stick. The stick, being a continuous unity, pokes as a whole. And should the stick rebound from striking a hard surface, the rebound does not propagate through the stick, the stick rebounds as whole. So Peripatetic and Stoic accounts of vision, self-consciously propounded as alternatives to Timaeus' account of vision, nevertheless preserve this formal feature, namely, that which mediates the distal object of vision be a homogenous unity that is acted upon as a whole. Moderns tend to anachronistically elide this feature since it is alien to our present conception in which we are reasonably confident.

As ``\emph{homoiopathes}'' may suggest, we might expect the uniform affection to be described as \emph{pathē} or what will emerge as his preferred term \emph{pathēma} (61d--68e). But Timaeus instead speaks of motion (\emph{kinēsis}). It is a motion that the visual body passes on to the body before it reaches the soul. I suspect that the fact that Timaeus here speakes of motion rather than affection merely reflects the active nature of vision as Timaeus conceives of it, since it involves the emission of fiery effluences, rather than signalling a shift in doctrine (between 45b--d and 61d--68e). When the visual body touches an object, or an object touches it, it communicates this motion through its body as a whole until it reaches the soul. And this is what we call ``seeing''. Timaeus remains silent about the objects of vision, though it will emerge that they are colors conceived as a kind of flame (\emph{phlox}) (67c–68d), and so a fire of a different kind from the fire that composes the visual body. It is presumably because colors are a different kind of fire that they may act upon the visual body rather than being assimilated to it by being compacted and grown together (with, as we shall see, the possible exception of the transparent, chapter~\ref{sec:the_eyes}).

So far we have been discussing what happens when the fire emanating from within meets what is like it. We have yet to discuss what happens when the fire emanating within meets what is unlike it. Before we do, however, allow me to bring up another point of comparison between Timaeus account of vision and Empedocles'. In \emph{De sensu} (2 437b27--438a3), Aristotle notes that Timaeus, like Empedocles, seeks to explain vision in terms of fire emanating from the eyes and cites the lantern analogy to support this. But he complains that while he sometimes explains vision in terms of fire emanating from the eyes, sometimes Empedocles explains vision in terms of emanations from visible objects (\emph{De sensu} 438a3--4). Aristotle has in mind the kind of account that Socrates attributes to Empedocles in the \emph{Meno} (76a--d) where chromatic effluences emanating from colored bodies are received through passages in the eye. Is Empedocles being inconsistent as Aristotle charges? It is hard to say given the fragmentary state of Empedocles' poem as it has come down to us, but I am inclined to try to reconcile these seemingly contradictory elements of his thought and see the extramissive elements, the emanation of fire within, as somehow making possible the intromissive elements, the reception of chromatic effluences (for an admittedly speculative interpretation of this kind see \citealt[chapter 1.3]{Kalderon:2015fr}). However exactly things stand with Empedocles', Timaeus is offering an account of just this kind. Perhaps, he better succeeds at this task. The emanation of fire from within and the formation of the visual body along the line of sight is what makes possible the reception of an affection or motion that is communicated through the body to the soul. The second case where the fire emanating from within meets what is unlike it, will further substantiate this point.

When it is night, and the body proper to the day has withdrawn, the fire emanating from within meets with what is unlike it. Since the air at night is dark and the mild light of day no longer pervades it, there is nothing in the dark air of night that is very much like the fire emanating from the eye. It is isolated and cut off from its kin. Rather than assimilating the darkness to form a compound body, the dark air, being unlike (\emph{anomoion}) and unakin alters the fire emanating from within and quenches it. Two observations are relevant here. First, this is an instance of the more general principle that like does not affect like (57a3--5). Since the darkness of the night air is unlike the fire emanating from the eyes it may act upon this fire and so alter it. Second, the darkness of the night air alters the fire emanating from within by quenching or extinguishing it. The language here, perhaps archaic at the time of Plato's writing, if not for the speech of Timaeus' generation, suggests that darkness is a moist substance which is why it extinguishes the eye's fire. Timaeus' expression, at least, is picking up on an older tradition that declined to understand darkness as the privation of light but instead conceived of it as a moist substance with a positive nature of its own. This prompts Aristotle's criticism, in \emph{De sensu} (437b 15--23), that Timaeus cannot explain why we can see in the rain.

I do not, in fact, think that we are meant to understand Timaeus as subscribing to this tradition. And if he does not so subscribe, then he is improperly lampooned by Aristotle. Parmenides had already described the moon as shining with a foreign light (DK 28B14), and Timaeus describes the sun as illuminating the wanderers, the planets, so that they may be seen (39b3--c1). Empedocles explicitly claims that the night is the Earth's shadow (DK 31B48, but remains willing to describing the night as blind-eyed, DK 31B48, and so a Cyclops roughly handled by Odysseus). And subsequently Aristotle will correctly describe the lunar eclipse as being the result of the Earth's shadow (\emph{Analytica Posteriora} 2.8 93a29—93b3). It is very likely, then, that Timaeus understood the withdrawal of the body proper to the day as the casting of the Earth's shadow. Timaeus accedes to this older usage even as he gives no credence to the conviction that gave rise to it, just as we accede to talk of sunrises even while accepting a heliocentric astronomy. 

Aristotle's objection, even if inapplicable to the position that Timaeus actually held, nonetheless raises a question about Timaeus' acceding to this older usage. Why accede to this older usage when it is liable to mislead? This is a good question. I am unsure whether I know the whole answer. However, Timaeus is keen to preserve a connection between water and darkness. It is no so much that Timaeus thinks that the dark is wet, but, rather, that he thinks that water is dark. Thus, Anaxagoras (DK 59B15), Empedocles (DK 31B94, Theophrastus, \emph{De sensibus} 59, Plutarch, \emph{Historia naturalis} 39), and later Aristotle (\emph{Meteorologica} 3.2 374a2) all held that water is dark. Timaeus will use similar language when describing how fire particles from brilliant and red bodies are quenched in the waters of the eye, in the way that the fire particles from white bodies are not, with a consequent reduction in the brightness of the color appearances that these give rise to. As Timaeus conceives of them, aranging these colors in descending order of brightness yields the series: white, brilliant, and red. And what explains the reduction of brightness in the case of brilliant and red bodies is the mixing of the fire particles that they emit in the dark waters of the eye (chapter~\ref{sec:the_eyes}). 

% section vision (end)

\section{Audition} % (fold)
\label{sec:audition}



% section audition (end)

% chapter incarnation (end)