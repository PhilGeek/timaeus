%!TEX root = /Users/markelikalderon/Documents/Git/timaeus/timaeus.tex
\chapter{Incarnation} % (fold)
\label{cha:incarnation}

\section{Cosmic and Encosmic Incarnation} % (fold)
\label{sec:cosmic_and_encosmic_incarnation}

There are similarities between cosmic and encosmic incarnation. The generation of the human body and its ensoulment are in many ways similar to the generation of the body of the Cosmos and its ensoulment. But, importantly, there are differences. 

First, there is a difference in agency. Whereas, the Demiurge generates the body of the Cosmos and the soul that animates it, the young gods, generated by the Demiurge, act on His behest in generating the human body. At least the immortal part of the soul is generated by the Demiurge with similar materials, if less pure, than the materials from which the World-Soul was generated. The difference in agency is explained by the limitations of the young gods' power of generation. In generating mortal living beings, the young gods imitate Demiurgic activity. Their imitation is imperfect, however. Whereas the Demiurge generates immortals, the young gods generate only mortals. This is precisely why they are assigned this task. In order for a sensible living being to be truly comprehensive, it must contain within itself all manner of mortal living beings. 

Second, there is a difference in body. The bodies of mortal living beings are importantly different from the body of the Cosmos that contains them. Mortal living beings contained within the Cosmos are situated in an environment with strong powers. The Cosmos, by contrast, is not situated in an environment. There is nothing beyond the limits of the Cosmos, neither strong powers, nor space, nor void. Mortal living beings depend upon and must defend against the environment and consequently have a number of powers that are only ever exercised in an environment. That mortal living beings are equipped with the instruments of these powers explains the way their bodies differ from the body of the Cosmos.

Third, there is a difference in soul. The human soul has mortal and immortal parts. While the young gods generate the mortal part of the soul, the Demiurge generates the immortal part from the same material, if less pure, that He used to generate the World-Soul. Like the World-Soul, it is composed of the Circles of the Same and the Different, and though Timaeus does not say so explicitly, presumably the immortal part of the human soul is structured in the same way with arithmetic, geometric, and harmonic intervals. However, whereas the motions of the World-Soul are unwaverable, the motions of the immortal part of the human soul are waverable, they may be affected by strong powers in the sensible environment. Moreover, unlike the Cosmos, human beings have, in addition, mortal parts of the soul. That mortal living beings are situated in an environment with strong powers explains why their souls have mortal parts and why the motions of their soul may waver.

Fourth, soul and body are differently related. The World-Soul encompasses the body of the Cosmos. Though some commentators may try to resist it, Timaeus spatial language is clear. The body of the Cosmos is contained within the World-Soul. The immortal part of the human soul, by contrast, is encompassed by the human body. The immortal part of the soul is contained within the human body. It is bound to the marrow that composes the brain which is encased in the skull which is itself covered by flesh and hair. Just as the flesh protects the skull that it contains, providing padding that will soften any blow that it may befall it, the skull protects the brain that it contains and so preserves its bond with the immortal part of the soul. In this way, the body is providentially provided with the means to protect its bond with the immortal soul that it contains. Mortal living beings are situated within the immortal Cosmos in the way that the Cosmos is not. That mortal living beings are situated in an environment with strong powers explains how their soul and body are differently related.

Fifth, there is a difference in narrative order. Timaeus narrates the generation of the body of the Cosmos before narrating how it became ensouled. This narrative is reversed when it comes to the generation of human beings. Timaeus narrates the ensoulment of the human body before narrating the generation of the human body. This narrative reversal is accompanied by a perspectival shift. Timaeus narrates the ensoulment of the human body from the perspective of the human soul, the dramatic highpoint of which is Timaeus vivid description of the shock of embodiment. The ensoulment of the Cosmos, by contrast, is impersonally presented. This perspectival shift is ethically significant. In dramatizing the shock of embodiment undergone by a newly incarnate human soul, Timaeus makes vivid the need for salvation, thus laying the groundwork for his soteriology (chapter~\ref{cha:the_end_of_vision_and_audition}).

The present chapter will follow the order in which Timaeus narrates the incarnation of the human soul and the generation of the human body. The differences highlighted above shall be further explained and their significance assessed (all but one---the mortal soul shall be discussed in chapter~\ref{cha:the_flesh_and_the_mortal_soul}). Just as the Cosmos lacks certain features since it lacks an environment, mortal living beings have certain features since they are situated in an environment with strong powers that they depend upon and must defend agains. Again, the principle lesson for the philosophy of perception shall be the environmental nature of perception. Sensory powers are only ever exercised on aspects of the sensible environment in which the percipient is situated.

% section cosmic_and_encosmic_incarnation (end)

\section{The Demiurge's Address} % (fold)
\label{sec:the_demiurge_addressing_the_gods}



% section the_demiurge_addressing_the_gods (end)

\section{The Laws of Destiny} % (fold)
\label{sec:the_laws_of_destiny}



% section the_laws_of_destiny (end)

\section{The Shock of Embodiment} % (fold)
\label{sec:the_shock_of_embodiment}



% section the_shock_of_embodiment (end)

\section{The Generation of the Human Body} % (fold)
\label{sec:structuring_the_human_body}



% section structuring_the_human_body (end)


% chapter incarnation (end)