%!TEX root = /Users/markelikalderon/Documents/Git/timaeus/timaeus.tex

\chapter{\emph{Pathēmata}} % (fold)
\label{cha:pathemata}

\section{\emph{Pathēmata}} % (fold)
\label{sec:pathemata}

\emph{Pathēmata} or \emph{pathē} (Plato, in the \emph{Timaeus} at least, uses these terms interchangeably) play an important role in the process eventuating in \emph{aisthēsis}. ``\emph{Pathēmata}'' is perhaps most naturally translated in English as affections. In a not quite archaic usage, affections are conditions or characteristics of a thing brought about by its being affected in some way. (Thus Theophrastus claims that Plato, in contrast to Democritus, emphasizes the effects of sensory objects, \emph{De Sensibus} 60--61.) But the \emph{pathēmata} that play a role in the causal process eventuating in perception or sensation in Timaeus' account are a specification of the more generic conception provided by the natural English translation. It is surprisingly controversial, however, what, exactly, ``\emph{pathēmata}'' means in this context. Thus \citet[429-31]{Taylor:1928qb} understands the \emph{pathēmata}, in this context, as sensible qualities that a thing may have independently of the sensibility of a percipient being affected in any way. They are ``characters of the various bodies themselves'' as opposed to ``the effects produced by the bodies on a percipient''. \citet[258-9]{Cornford:1935fk}, by contrast, wants to assimilate Timaeus' account of perception to the account given in the \emph{Theaetetus}. As a result, he contrasts the \emph{pathēmata} with those ``properties which bodies are supposed to possess in the absence of any sentient being, such as the shapes of the microscopic particles, which are never perceived'' \citep[259]{Cornford:1935fk}, for, according to the Secret Doctrine that Socrates attributes to Protagoras in the \emph{Theaetetus} (156a-b), a sensible quality and a perceiver's perception of it are ``twin births'' with the consequence that nothing has a sensible quality without a perception of it. Since it is controversial how to understand \emph{pathēmata} in this context, it is worth discussing this explicitly before discussing Timaeus' specific examples of \emph{pathēmata} that eventuate in perception or sensation.

Timaeus distinguishes two classes of \emph{pathēmata}. There are:
\begin{enumerate}
	\item \emph{pathēmata} that are common to the body as a whole (61d--65b)
	\item \emph{pathēmata} that are peculiar to particular parts of the body (65b--68e)
\end{enumerate}
The body, here, is the human body or, perhaps, the body of a sentient being more generally. Thus \citet[431]{Taylor:1928qb} is wrong to deny that the \emph{pathēmata} are ``the effects produced by the bodies on a percipient''. In describing the common \emph{pathēmata}, Timaeus' explicit task is to describe the affections of the body as a whole and the names of the agents that produce these affections (65b). Timaeus undertakes a similar task with the peculiar \emph{pathēmata}, that is, he is meant to describe the affections of particular parts of the body and the names of the agents that produce these affections (65b6-c1). Since the agents that produce these affections, whether common or peculiar, are the objects of perception, this is inconsistent with Taylor and Cornford's common insistence that the \emph{pathēmata} are sensible qualities \citep[see][225, n8]{Archer-Hind:1888qd}. And this remains so whether or not sensible qualities are best conceived, in Timaeus' account, as characteristics of bodies had independently of their effects on perceivers or as twinned with the appropriate perception. 

Common \emph{pathēmata}, affections of the body as a whole (discussed in section \ref{sec:common_emph_pathemata}), are involved in the perception or sensation of:
\begin{enumerate}
 	\item hot and cold (61d5--62a5) (section \ref{sub:hot_and_cold})
 	\item hard and soft (62b6--c3) (section \ref{sub:hard_and_soft})
 	\item heavy and light (62c3--63e8) (section \ref{sub:heavy_and_light})
 	\item smooth and rough (63e8--64a1) (section \ref{sub:smooth_and_rough})
 	\item pleasure and pain (64a2--65b3) (section \ref{sub:pleasure_and_pain})
\end{enumerate}

Peculiar \emph{pathēmata} are not affection of the body as a whole, but are rather affections of particular parts of the body that are liable to give rise to perception or sensation (discussed in section \ref{sec:peculiar_emph_pathemata}). Timaeus describes four such parts:
\begin{enumerate}
	\item the tongue (65b--66c) (section \ref{sub:the_tongue})
	\item the nostrils (66c--67a) (section \ref{sub:the_nostrils})
	\item the ears (or perhaps the brain and the blood) (46c--47e, 67a--c, 80a) (section \ref{sub:the_ears})
	\item the eyes (45b--46c, 67c--68d) (section \ref{sub:the_eyes})
\end{enumerate}

\emph{Pathēmata} peculiar to the tongue are involved in the perception or sensation of:
\begin{enumerate}
	\item astringent and harsh (65c6--d4)
	\item acrid and agreeable (65d4--65e4)
	\item pungent (65e4--66a2)
	\item acid (66a2--b7)
	\item sweet (66b7--c7)
\end{enumerate}

\emph{Pathēmata} peculiar to the nostrils are involved in the perception or sensation of:
\begin{enumerate}
	\item odours of water transforming into air or air into water (66c-67a)
	\item pleasant and unpleasant odours (66d1--67a6)
\end{enumerate}

\emph{Pathēmata} peculiar to the ears (or perhaps the brain and the blood) are involved in the perception of:
\begin{enumerate}
	\item high and low (67b6)
	\item smooth and harsh (67b6-7)
	\item loud and soft (67c1)
\end{enumerate}

Finally, \emph{pathēmata} peculiar to the eyes are involved in the perception of:
\begin{enumerate}
	\item colors (45b--46c, 67c--68d)
\end{enumerate}

If common \emph{pathēmata} are the \emph{pathēmata} involved in tactile perception and sensation (compare Ps-Timaeus Locri who claims that these are named in relation to the sense of touch, \emph{De natura mundi et animae} 100d), then the peculiar \emph{pathēmata} are the \emph{pathēmata} involved in the remaining four special senses: taste, smell, hearing, and vision. Here I am helping myself, in full knowledge of its anachronism, to the vocabulary of Aristotle in \emph{De anima} and \emph{De sensu}. Undoubtedly the \emph{Timaeus} account of perception was an influence on Aristotle's own account in these works. However, for now at least, I want to bracket any suggestion that talk of the special senses may carry that the particular parts of the body affected are sense organs. Whether or not the particular parts of the body affected, as Timaeus conceives of them, are properly deemed sense organs shall be assessed as we proceed.

\emph{Pathēmata} are not the objects of perception as Taylor and Cornford maintain but, rather, they are a causal intermediary between the objects of perception and the perceptions or sensations that they are liable to give rise to (\citealt[138]{OBrien:1984ji}; \citealt{Brisson:1997qr}). The named agents of these affections, explicitly contrasted with the \emph{pathēmata} themselves, are the objects of perception. Moreover, perceptions are themselves explicitly contrasted with the \emph{pathēmata}. \emph{Pathēmata}, understood as affections of a sentient body, while necessary for the production of perception or sensation, are not sufficient. And if \emph{pathēmata} may obtain without perception, then perceptions are distinct from \emph{pathēmata}. Let's consider these claims in turn.

First, Timaeus contrasts the agents that cause the \emph{pathēmata} with the \emph{pathēmata} themselves. Moreover, these agents are the objects of perception or sensation (as we shall see in section~\ref{sub:pleasure_and_pain}, pleasure and pain are an exceptional case). Touch, for example, affords us with thermal perception. We can feel that a body is hot or cold by touching it. When our flesh is in contact with a hot or cold body it is affected in a certain way, and this affection is liable to give rise to thermal perception. When we touch something hot, for example, we don't merely feel the consequent warmth of our flesh, we feel the warmth of the body that we are in contact with. And both the affection, the \emph{pathēma}, and the agent that causes it are called by the same name, ``hot'' or ``cold'' as the case may be (61e--62b6). This is true not only for the common \emph{pathēmata} (with the possible exception of pleasure and pain), but it is true, as well, of the peculiar \emph{pathēmata}. Timaeus makes this general claim explicit. \emph{Aisthēsis} occurs when the power of the agent that caused the \emph{pathos} is reported to the \emph{phronimon}, the seat of intelligence or consciousness (65b4). Note well, what is reported is not the affection of the body but the power of the agent that caused that affection. And if that report determines the content of the perception then what is perceived is not the \emph{pathos}, but its external cause. But if the \emph{pathēmata} are contrasted with the agents that cause them and these latter are, on the whole at least, the objects of perception and sensation, then the \emph{pathēmata} are not sensible qualities as Taylor and Cornford maintain.

Second, \emph{pathēmata} while necessary for the production of \emph{aisthēsis} are not sufficient. \emph{Pathēmata} may obtain without eliciting perception or sensation. When an affection is violent, intense, and contrary to the natural state of the body, it is painful. However, when an affection is mild and gradual, it is insensible (64d4). An insensible affection is not the object of any perception or sensation. And earlier (64a--d) Timaeus tells us that when the \emph{pathos} falls upon the body, whether or not it produces \emph{aisthēsis} depends upon the nature of the body receiving that \emph{pathos}. Parts of the body may be more or less mobile, depending upon the character of the particles that compose them. If the affected part of the body is mobile it passes the affection around to other mobile parts until it reaches the \emph{phronimon}, the seat of intelligence or consciousness, where it reports the power of the agent that caused the affection. Whereas if the part of the body upon which the \emph{pathos} falls is immobile, as in the case of hair and nails, though it is affected, it does not pass this affection around with the consequence that the power of the agent that caused that affection is not reported to the \emph{phronimon}. So \emph{pathēmata} may obtain without eventuating in \emph{aisthēsis} (see also \emph{Philebus} 33d). Thus, \emph{pathēmata} while necessary for the production of \emph{aisthēsis} are not sufficient. If that is right, then the \emph{pathēmata} are not twinned with a perception in the way that Cornford maintains.

Timaeus inaugurates the discussion of the \emph{pathēmata} with a methodological dilemma (61c4--d5). Timaeus distinguishes between the \emph{pathēmata}, on the one hand, and the origin of the flesh and things pertaining to it and the mortal parts of the soul, on the other. The problem is twofold. First, one cannot adequately account for the \emph{pathēmata} having to do with \emph{aisthēsis} without accounting for the origin of the flesh and things pertaining to it and the mortal part of the soul. But second, one cannot account for the origin of the flesh and things pertaining to it and the mortal parts of the soul without accounting for the \emph{pathēmata}. The problem is not one of mutual entailment or that these subject matters somehow presuppose one another directly. The problem is rather that an account of either subject matter each presupposes \emph{aisthēsis} with the consequence that an account of each must make reference to the other. Hence the methodological dilemma: Timaeus cannot account for either without reference to the other, but it is impossible to give an account of both subjects at once. Timaeus solution is to first account for the \emph{pathēmata} (61d--68d) and then subsequently account for the origin of the flesh and things pertaining to it (73b1ff) and the mortal parts of the soul (69c5ff). In part, Timaeus' decision, here, is dictated by the structure of his speech. Discussion of the \emph{pathēmata} more naturally follows on from his discussion of the \emph{genē} or \emph{eidē} than would discussion of the origin of the flesh and things pertaining to the flesh or discussion of the mortal parts of the soul. If the \emph{pathēmata} presuppose \emph{aisthēsis}, and if an account of \emph{aisthēsis} makes essential reference to the flesh and the mortal soul, then Timaeus in accounting for the \emph{pathēmata} will have to make assumptions about the flesh and mortal soul, assumptions that will only later receive a full account.

Two questions arise, only the first of which we are in position to answer. First how exactly and in what sense does an account of the \emph{pathēmata} presuppose \emph{aisthēsis}? Second how exactly and in what sense does an account of the flesh and the mortal soul presuppose \emph{aisthēsis}? Consider, then, the first question. \emph{Pathēmata} are affections to be sure. And some at least of the common \emph{pathēmata} may obtain not only in animate bodies capable of sentience but inanimate bodies as well. Thus in Timaeus' discussion of hard and soft (62b6--c3) we are told that we call ``hard'' that which the flesh yields to and ``soft'' what yields to the flesh and then immediately claims that these terms also apply more generally to bodies in relation to one another (see \citealt[228, n6]{Archer-Hind:1888qd} and \citealt[110]{OBrien:1984ji}). Thus, for example, a body composed of earth, such as a stone, is harder than water since water yields to earth. So yielding is the \emph{pathēma} and not only does the flesh yield to a stone, say, but do does water. Still, the \emph{pathēmata}, whether common or peculiar, that Timaeus discusses are \emph{pathēmata} of sentient animate bodies. The relevant \emph{pathēmata} are \emph{pathēmata} that when they fall upon a sentient body are liable to produce \emph{aisthēsis} (should they be intense enough, and fall upon a part of the body composed of sufficiently mobile particles, and so on). The connection with \emph{aisthēsis}, then, is this: The \emph{pathēmata}, whether common or peculiar, are such that when they fall upon a sentient body, should the circumstances be propitious, they are liable to produce \emph{aisthēsis}.

Before we begin the detailed discussion of the \emph{pathēmata}, three preliminary observations are in order concerning naming the agents that cause the \emph{pathēmata}, the prevalent use of \emph{summetria}, and a potential connection with the Empodoclean principles of Love and Strife (though these last two are perhaps less observations than preliminary hypotheses to be verified by the subsequent detailed discussion of the \emph{pathēmata}).

Timaeus, in accounting for the \emph{pathēmata}, seeks to describe the agents that cause them and to explain how these are named. This linguistic interest is picking up from an earlier discussion of the impossibility of naming in the pre-cosmic chaos. Timaeus' intent is to show that the powers of secondary bodies, as determined by their being composed by primary bodies, can be named thanks to the Demiurge providentially imposing order upon the pre-cosmic chaos.

Throughout the discussion of the \emph{pathēmata}, Timaeus' use of the word \emph{summetria} is prevalent. \emph{Summetria} means something like proportion. Thus medieval translations of Euclid's \emph{Elementa} used the Latin \emph{commensurabile} to translate \emph{summetria}. Given the \emph{pathēmata}'s role in the process eventuating in \emph{aisthēsis}, Timaeus evidently conceives of perception on the model of measurement \citep[155--6]{Brisson:1997qr}. (Perhaps Aristotle picks up on this in \emph{Metaphysica} I 1053a 31 where he claims that knowledge and perception are a measure of things.) Given the emphasis on proportion in perception, Plato may have been influenced by Empedocles (see \citealt{Ierodiakonou:2005fk,Ierodiakonou:2005ly}; the Pythagorean influence on Empedocles in connection with proportion is well attested by the ancient commentators, see Porphyry \emph{Vita Pythagorae} 30, Simplicius \emph{In de anima} 68.5--8, Philoponus \emph{In de anima} 176.32--177, Sophon \emph{De anima paraphrasis} 68.5--8). There is an evident connection with naming here. If Timaeus is indeed modelling perception on measurement, then his explanation of the names of the powers of the agents that cause the \emph{pathēmata} is both natural and intelligible. Specifically, if the \emph{pathēmata} are the measure of the powers of the agents that cause them, it is natural that the names of the \emph{pathēmata} should apply to the powers they measure as well. 

Many of the \emph{pathēmata} discussed by Timaeus involve either a kind of compression or compaction (\emph{sunkrisis}), on the one hand, or a kind of  division or dispersal (\emph{diakrisis}), on the other. Indeed, Timaeus explicitly claims as much (65c3--7). Once this generalization comes to light, it is hard not to suspect a further Empedoclean influence as these operations correspond to the principles Love and Strife respectively.

% section pathemata (end)

\section{Common \emph{Pathēmata}} % (fold)
\label{sec:common_emph_pathemata}

\subsection{Hot and Cold} % (fold)
\label{sub:hot_and_cold}

Timaeus provides an account of how it is that we call bodies ``hot'' and ``cold'' (61d5--62a5). Hot things have a power to divide and cold things have the power to compact. Timaeus, here, is drawing on an earlier tradition. Thus, for example, if Plutarch is to believed (\emph{De primo frigido} 7 947f--948a = DK 13B1), Anaximenes maintained that hot was rare and cold was dense. These ideas are related if distinct---dividing makes bodies rarified, compacting makes them dense.

Timaeus begins with the way in which fire affects our body, by dividing (\emph{diakrisis}) or cutting. This much is evident from our experience. A hot flame, should we be in contact with it, gives rise to a sharp sensation. (In \emph{De generatione et corruptione} 2.2 329b26, Aristotle will dispute this, albeit without argument. For, there, he claims that heat's apparent power to divide (\emph{diakrisis}) is really a power to associate (\emph{sunkrisis}) things of the same kind. The power to divide attributed to fire is really the power to associate things of the same kind since it eliminates what is foreign or alien. Aristotle plainly has in mind the purificatory power of fire.) Notice that it is our body, the body of a sentient human being, whose affection is presently being discussed, and not an inanimate body, though fire will divide and cut wood as well, say. So the \emph{pathēma} relevant to what we call ``hot'' is the division of the sentient body and the sharp sensation associated with extreme heat is the \emph{aisthēsis} that the \emph{pathēma} is liable to give rise to. 

Timaeus explains the \emph{pathēma} of fire in terms of its geometrical and kinematic properties. When the tetrahedron or pyramid is assigned to fire Timaeus proclaims it to be the sharpest of the regular solids that constitute the \emph{genē} (56a5). A body is consumed when burnt, since when a body is surrounded by a large mass of tetrahedra, it is cut up or divided owing to the acuteness of the angles and sharpness of the edges of the particles of fire (56e8--57a2). Notice that the \emph{pathēma} associated with the heat of fire, division, obtains not only when it falls upon an animate sentient body, but when it falls upon inanimate bodies altogether lacking in sentience. However, when explaining how it is that we call fire ``hot'', Timeaus mentions two further factors, the relative smallness of the tetrahedra and their rapid motion. Thus four factors are relevant to fire dividing or cutting the body in a manner that is liable to give rise to sensation. Specifically, tetrahedra of fire have the power to cut or divide the body owing to:
\begin{enumerate}
	\item the acuteness of its angles,
	\item the sharpness of its edges,
	\item the smallness of the tetrahedra, and
	\item the rapidity of their motion.
\end{enumerate}
These geometrical and kinematic properties of the tetrahedra that compose sensible fire explain its power to divide or cut. And, in cases of thermal perception, the feeling of heat is the measure of the power to divide as determined by these geometrical and kinematic properties of fiery tetrahedra. Moreover, we call both this power of fire and its affect on us ``hot''. So ``hot'' names not just a common \emph{pathēma}, but a power of the agent that produced this affection. 

Notice that these geometrical properties were providentially imposed upon the pre-cosmic chaos by the benevolent Demiurge. What about the relevant kinematic property, the swiftness of the fire particles? While the disorderly motion in the pre-cosmic chaos was due to the pre-Socratic principle that like attracts like, this motion is channelled, so to speak, by the Demiurge's providential imposition of ``shape and number''. Thus, the swiftness of the tetrahedra of fire, for example, is due, as well, to the Demiurge making them small. Moreover, making the tetrahedra of fire small is for the best and hence what Reason requires. In this way is the swiftness of fire particles due to Reason's persuasion of Necessity. Since the geometrical and kinematic properties explain the power of the agent to produce the relevant \emph{pathēma}, then we are only able to name this \emph{pathēma} and power thanks to the Demiurge providentially imposing ``shape and number'' upon the disorderly powers in the pre-cosmic chaos. Implicit in these remarks is a double rebuke to pre-Socratic cosmologists. Not only is the principle that like attracts like insufficient for the production of a cosmos, but if it were the sole principle of motion, the sensible realm of Becoming would lack sufficient stability for its inconstant and inharmonious powers to be named.

It is phenomenologically important that ``hot'' names not just a common \emph{pathēma} but a power of the agent that produced this affection. Specifically, it provides Timaeus with the resources to distinguish thermal perception from sensation (a fact that is perhaps obscured by \emph{aisthēsis} applying to both). When we feel the heat of a fire, when its power to divide or cut our flesh is reported to the \emph{phronimon}, its the fire whose heat we feel. This is what makes it a case of thermal perception. Its object is exogenous. Contrast feeling hot, not because of the action of an external agent, but because we are suffering a fever, say. In this case, it is our own bodies' heat that we feel. This is less a case of thermal perception than bodily sensation (see \citealt{Yrjonsuuri:2008aa}). In this case, the heat that we feel is not exogenous but endogenous. While the object of bodily sensation is the affection of the body, the object of thermal perception is not the affection of the body but the power of the agent that produced it.  It is becase ``hot'' names not only the \emph{pathēma} of division but the power of the agent that produced this affection, and because this power is reported to the \emph{phronimon}, that we can suffer the ravaging heat of the fire in our experience of it.

If fire is called ``hot'' due to its power to divide, then things are called ``cold'' not due to division but compaction. Like the case of heat, the relevant \emph{pathēma} obtains not only when it falls upon an animate sentient body but when it falls upon inanimate bodies altogether lacking in sentience as well. For example, in his discussion of the \emph{genē}, Timaeus describes the cooling and solidification of melted metal (59a1--8) and the freezing of the fluid kind of water (59d4--e5) in terms of the operation of compaction. 

Metal belongs to the fusible as opposed to the liquid kind of water (59bff). A natural fact central to the art of metallurgy is that metal melts when sufficiently heated. Casting, for example, would not be possible if metal did not melt. Another natural fact central to metallurgy is that melted metal solidifies as it cools. Only in this way could casting melted metal produce hard solid objects. Consider Timaeus' explanation of this latter fact, that melted metal sets as it cools. According to the first natural fact, metal melts when sufficiently heated. A body in a liquid state is mobile, in the sense of being easily set in motion. Timaeus connects the mobility of a body with (1) the non-uniformity of its material composition and (2) the shape and figure of the particles that compose it. (With respect to the second condition, Timaeus may have in mind the relative stability of earth, being cubic, as compared to the icosahedra of water.) Conversely, a body with a uniform material composition of particles with the appropriate shape and figure will be immobile. Thus metal, in its melted state, contains large quantities of fire particles. Melted metal thus has a non-uniform material composition and the icosahedra that compose it lack a stable base and is consequently mobile and easily set in motion. The melted metal cools as these fire particles escape. However, the fire particles, when escaping from the melted metal, do not move into a void but immediately meet the air. As a result the air is compacted. And the compacted air presses down on the metal forcing its particles to fill in the spaces vacated by the fire particles. As a result of this process, the metal becomes uniform and immobile. Melted metal sets when cool because cooling, involving the escape of fire particles, results in compaction.

% \citet[248--50]{Cornford:1935fk} speculates that another force is at work here, namely the reconstruction of uniformly sized octahedra from smaller octahedra. If he is right then this is an additional force at work in what is essentially a process of compaction.

Compaction is also at work in the production of hail, ice, snow, and frost. The difference between these is explained by whether they occur above or upon the earth and how frozen they are. Thus hail and ice are completely frozen, hail above the earth and ice upon it. While snow and frost are half frozen, snow above the earth and frost upon it. The fluid kind of water is subject to freezing when sufficiently cooled. In this case, fire and air particles escape making the water more uniform, presumably in the sense of being more consistently composed of water particles alone. Due to the force of the fire and air particles escaping the water particles close in on one another filling in the space vacated by the escaped particles. As a result of this process the mass of water particles become compacted. Water freezes when sufficiently cooled because, once again, cooling and solidification are involved in a common process of compaction.

% Pick fight with Cornford?

Like the setting of metal, and the freezing of water, the cooling of the body involves a process of compaction. Like the setting of metal and freezing of water, the cooling of the human body involves the escaping of particles. And though Timaeus does not explicitly say that the particles that escape in the cooling of the human body are fire particles, he does describe them as small, and fire particles are the smallest of the primary bodies. Like setting metal and freezing water, the human body becomes more uniform as it cools. And like setting metal and freezing, the more uniform mass, as spaces vacated by the escaping particles are filled in, becomes more compact and less mobile. Timaeus mentions, in addition, a related affection. The trembling or shivering that results when the human body is cold. Since the compaction, at a certain point, is contrary to the natural state of the human body, the body fights against it, and thus begins to tremble or shiver. This affection as a whole (presumably, the compaction and consequent trembling) is called ``cold'', a name that is extended to the power of the agent that caused this affection. In the case of thermal perception, then, feeling cold measures the escape of fire particles from the human body.

A small observation: Earlier, I claimed that some at least of the common \emph{pathēmata} obtain among sentient animate bodies as well as inanimate bodies. Timaeus' discussion of hard and soft (62b6–c3), discussed in the next section, makes this claim explicit. Here, however, we have a partial exception as the relevant \emph{pathēma} has an essentially animate aspect. The affection as a whole is not something potentially shared with inanimate bodies. Inanimate bodies do not fight against a departure from their normal state when compacted and so tremble in the cold. Compaction due to escaping fire particles is an affection shared with inanimate bodies, but the affection as a whole includes not only this but shivering, a vital response. Nevertheless, what the affection as a whole measures, the power of the agent that caused the affection, is something objective. Coldness, compaction do to escaping fire particles, is independent of the vital response that figures in its measure.

In \emph{De Sensibus}, Theophrastus objects to Timaeus' account of thermal \emph{aisthēsis} as follows:
\begin{quote}
	First of all, he gives no uniform account of all <our sensory objects>, not even those that belong to the same class. For he describes heat in terms of figures, but he has not given a like account of cold. (Theophrastus, \emph{De Sensibus} 87; \citealt[147]{Stratton:1917vn})
\end{quote}
There is a more general and a more specific objection here. We shall address Theophrastus' more general objection, that Timaeus provides no uniform account of sensory objects, once we have discussed all the \emph{pathēmata}. For now, consider only the more specific objection, that while Timaeus accounts for heat in terms of figures, he fails to do so for coldness. I suspect that Theophrastus misses the emphasis of Timaeus' account. While fire's power to divide is explained in terms of the geometrical and kinematic properties of tetrahedra (and so, not just figure, but figure and motion), the sensation of heat is due to the sentient body's division, the exercise of the power of the agent that caused this affection. What's more the power to divide and the power to compact are naturally opposed. And if Timaeus subscribes to something like the Heraclitean doctrine of the unity of opposites, then heat and cold have the requisite unity \emph{pace} Theophrastus. 

Perhaps Theophrastus' objection can be interpreted as turning on the previous observation in the following manner: While the affection involved in the perception or sensation of cold essentially involves a vital response---shivering---the affection involved in the perception or sensation of heat does not, or at least insofar as Timaeus has described it. While such an interpretation captures Theophrastus' charge of asymmetry, that charge was meant to apply to the objects of thermal perception and sensation and not the affections of sentient animate bodies. Shivering may be a way of feeling cold, or at least a part thereof, but it is not the cold that we feel.

% subsection hot_and_cold (end)

\subsection{Hard and Soft} % (fold)
\label{sub:hard_and_soft}

Timaeus provides an account of what we call ``hard'' and ``soft'' (62b6--c3). Specifically, what we call ``hard'' is that which the flesh yields to and what we call ``soft'' is that which yields to the flesh. (Notice that when our flesh yields to a stone in our grasp it becomes compacted, whereas something soft to the touch, such as a sponge, is compacted as it yields to our flesh.) So hard is that which is resistant and soft that which is yielding. Our flesh yields to the resistant and resists that which is yielding. So the affection of the flesh of a sentient animate body explains what we call ``hard'' and ``soft'' and these terms describe the power of the agent that caused these affection.

So far, so straightforward. Timaeus, however, immediately qualifies this claim. Timaeus claims that the same is true of things called ``hard'' and ``soft'' in relation to one another (\emph{pros allēla te houtos}). That is to say that these terms apply more generally to bodies in relation to one another (see \citealt[228, n6]{Archer-Hind:1888qd} and \citealt[110]{OBrien:1984ji}) and not merely to that which is resistant or yielding to the flesh.

Here Timaeus is drawing our attention to a claim that he has already made in his discussion of the \emph{gēne}. Specifically, in discussing the liquid form of water in its fluid state, Timaeus claims that water is ``soft'' in relation to earth (54d4--7). By earth, here, Timaeus presumably has in mind stone as opposed to soil and so need not be read as denying water's role in soil erosion. If water yields to stone, the relevant \emph{pathēmata} obtain not only in sentient animate bodies but among inanimate bodies as well. So the affections of the flesh that explains the use of these terms obtains among bodies generally whether or not they are sentient and animate. Timaeus' position thus contrasts, \emph{pace} \citet[185, n2]{Beare:1906uq}, with Locke's who writes ``And, indeed, hard and soft are names that we give to things \emph{only} [my emphasis] in relation to the constitution of our own bodies'' (\emph{Essay Concerning Human Understanding} 2.4.4). This reading is seconded by Theophrastus when he complains that it is an implication of Timaeus' account of hard and soft that water, air, and fire are soft (\emph{De Sensibus} 87).

The powers of the agents that cause these affections are explained in terms of the broadly geometrical properties (the reason for the hedge will become immediately clear). Specifically, whatever stands on a small base yields and is this called ``soft'', whereas whatever stands on a square base is more resistant and is thus called ``hard''. Timaeus mentions, however, an additional source of hardness. Whatever is dense is especially resistant. Density is reckoned as a broadly geometrical property since density, here, is a matter of how tightly packed in the primary bodies that compose the thing are, and so is a matter of the internal configuration of a body's parts. If I may help myself to a useful anachronism, hardness is, as it were, multiply realized.

Thus in touching a stone, the hardness of the stone is measured by the flesh yielding to it, and the softness of water is measured by water yielding to our flesh. What is measured is an objective feature of things, the powers of resistance and yielding, as determined by the broadly geometrical properties of bodies, while what does the measuring is the affection of a sentient animate body, in the present case, the yielding or resistance of the flesh.

Earlier I alluded to Theophrastus complaint that water is not ordinarily described as soft. Given that Timaeus, especially in his discussion of time, is ready to convict ordinary speech of confusion, this objection can seem question begging. However, in the same passage, Theophrastus makes a related and, to my mind at least, stronger objection to Timaeus' account:
\begin{quote}
	nor in general is it held that a thing is soft that moves freely around and behind <the entering body>; but only what yields in ``depth'', without <free> change of place. (Theophrastus, \emph{De Sensibus} 87; \citealt[147]{Stratton:1917vn})
\end{quote}
Theophrastus' ideas is that soft things yield in the direction in which they are touched and form a depression rather than being scattered so as to flow around the flesh of the body (see \citealt[213, n230]{Stratton:1917vn}). Plausibly, Theophrastus derives this objection from Aristotle's claim in the \emph{Meterologia} 3.1 299b11 that the soft withdraws into itself. Relatedly, Ps-Timaeus Locri distinguishes hard and soft from yielding to touch and resistance to touch in \emph{De natura mundi et animae} 100d. If Aristotle, Theophrastus, and Ps-Timaeus Locri are right, then Timaeus' account would have to be modified. But in so far as I can tell, any such modification would be consistent with Timaeus' more general principles at work in his discussion of the \emph{pathēmata}.

% subsection hard_and_soft (end)

\subsection{Heavy and Light} % (fold)
\label{sub:heavy_and_light}

Timaeus prefaces his account of heavy and light by an account of direction, specifically the directions above and below. He does so since he will go on to identify the affection involved in the sense of weight in terms of resistance to the direction of movement. In his account of direction Timaeus does four things:
\begin{enumerate}
	\item Timaeus identifies a popular misconception of direction (62c5--8)
	\item He distinguishes the circumference from the center of a sphere (62c8--d4).
	\item Since the cosmos is spherical, Timaeus offers the distinction between circumference and center as a correction to the popular misconception (62d4--10).
	\item He argues that the popular misconception is incoherent (6210--63a6).
\end{enumerate}
Let's consider these in turn.

First, Timaeus identifies a popular misconception about these directions, involving the idea that the universe can be divided into two opposed and mutually exclusive regions. Bodies tend to sink in the direction of one region, the region denominated ``below''. And bodies resist moving in the direction of the other region, the region denominated ``above''. Though Timaeus does not attribute this misconception to any particular thinker, Simplicius gives as an example the atomists' conviction that the only natural movement is downward, \emph{In Aristotelis De Caelo commenteria} 269 4--14.

Second, Timaeus considers some elementary geometrical facts about spheres. Thus the points on the circumference of the sphere are all equidistant from the center. They are thus at the extremity of this figure and in this respect all alike. This has the consequence that all the points on the extremity of the sphere are opposed, not to each other, but to the center and the center alone. Moreover, the center, being equidistant to every point on the circumference of the sphere is itself opposed to all of these. Whereas the opposition in the popular misconception is one--one, the present opposition is one--many (or many--one, depending on whether we privilege the center or the points on the circumference).

Third, since the cosmos is, according to Timaeus, spherical, the only applicable opposition is not between a top half of the cosmos denominated ``above'' and a bottom half denominated ``below''. Rather there is only the opposition between the center of the cosmos and the points that lie, in extremity, on the circumference of its outer boundary. No point on the circumference could justly be called ``above'' or ``below'' the center since the point on the opposite side of the sphere would have equal claim. This is self-consciously put forward as a correction of the popular misconception.

Fourth, such a correction is needed since the popular misconception is incoherent. Timaeus begins by asking us to imagine that there is a spherical body at the center of the cosmos. Though he does not say, this is plainly meant to be the Earth. Timaeus declines to make this explicit, restricting himself to a purely geometrical description of his thought experiment, since elemental considerations will come to the fore when he turns from his account of direction to his account of weight. (This aspect of Timaeus' thought experiment is reproduced in Aristotle's paraphrase of it in \emph{De caelo} 4.1 308a17.) In a clear allusion to Anaximander, Timaeus also suggests that we imagine that this body rests in equipose at the center of the cosmos. (Plato endorses Anaximander's claim in the \emph{Phaedo} 108e4--109a7.) Presumably the central body does not move in the direction of any of the points on the extremity since, in an application of the principle of sufficient reason, there is no reason to. Why move in the direction of a given point when every other point on the extremity is just alike in being equidistant from the center? We are to further imagine a traveler circumnavigating this body. Timaeus tells us that at different points in the circumnavigation the same part of the central body will be called ``above'' and ``below''. This is, at the very least, inconsistent with the popular misconception's contention that the designated regions are mutually exclusive. But Timaeus draws a stronger conclusion, that no part of the central body may be described as ``above'' or ``below'' \citep[23--24]{OBrien:1984ji}. Timaeus' assumption seems to be that if something can be given opposite descriptions, even relative to different contexts, then it cannot, by its very nature, possess the designated features (a similar assumption is at work in \emph{Phaedo} 102c). Contrast Aristotle's position in \emph{De caelo} 4.1 where he maintains, not that neither the center nor any part of it are, by its nature, above or below, but rather that the center is below and the points on the circumference of the cosmos are above. Ps-Timaeus Locri also endorses the Aristotelian position, \emph{De natura mundi et animae} 100d-e, further evidence for the pseudepigraphal nature of this work.

Timaeus regards the habits of speech that lead to the popular misconception as explicable despite the incoherence of that misconception. To make this plain, Timaeus again has us consider a thought experiment. Specifically we are invited to compare the activity of two observers, located on the circumference of the cosmos where fire accumulates and the center of the cosmos where earth accumulates, respectively. While the previous thought experiment was framed in purely geometrical terms, the present thought experiment appeals to the elemental composition of bodies. The introduction of elemental considerations is crucial since the thought experiment will rely on the idea that a primary body has the tendency to move towards its native element.

The first observer is standing on the inner circumference of the cosmos. There is nothing beyond the body of the cosmos, neither space nor void, so the observer could not be standing on its outer circumference. The observer weighs a larger and smaller quantity of fire with a pair of scales by lifting it into the region of the air. The larger quantity of fire will weigh more than the smaller quantity. Timaeus' explanation of this crucially relies on the elemental nature of fire, for it is more difficult to move a larger quantity of fire from the region where fire naturally accumulates. In the background is the claim that, owing to the movement of the Receptacle, the primary bodies tend to accumulate in different regions of the cosmos. What explains the greater difficulty of moving the larger quantity of fire into the region of the air is like's affinity for like, which is itself explained by the winnowing motion of the Receptacle. Timaeus concludes that the larger quantity of fire is moving downwards or below and is heavy, while the smaller quantity of fire is moving upwards or above and is light.

The second observer is standing on the outer circumference of the Earth. The observer weighs a larger and smaller quantity of earth with a pair of scales by, again, lifting it into the region of the air. The larger quantity of earth will weigh more than the smaller quantity. Again the explanation is that it is more difficult to move a larger quantity of earth from the region in which earth naturally accumulates. Again, in the background is the derivative principle of like's affinity for like. Timaeus concludes that the larger quantity of earth is moving downwards or below and is heavy, while the smaller quantity of earth is moving upwards or above and is light.

Notice that each of the terms, ``above'' and ``below'', ``heavy'' and ``light'' have the opposite significance when deployed on the outer circumference of the Earth than they would were they deployed on the inner circumference of the cosmos, in the region of fire. Thus a large quantity of fire, while heavy and moving downwards when weighed on the inner circumference of the cosmos, will be light and moving upwards when weighed on the surface of the Earth. (Again contrast Aristotle's and Ps-Timaeus Locri's position that the center of the cosmos is below and the points on the circumference of the cosmos are above. On that position, fire is light and moving upwards wherever it should be weighed.) According to Timaeus, then, the habits of speech that lead to the popular misconception are due to our Earth-bound perspective and a failure to consider alternative perspectives.

Finally, Timaeus concludes that for any body the movement towards its native element makes that body heavy and makes the direction of this movement below. Correspondingly, for any body the movement away from its native element makes that body light and makes the direction of this movement above. Timaeus' explanation, here, is not purely geometrical but crucially depends on the elemental composition of bodies.

The affection involved in our sense of the weight of a body is its resistance to our moving it upwards. The resistance to movement is, in this way, a measure of a body's weight. A fact emphasized by Timaeus' use of scales in his second thought experiment. Notice that this affection is not limited to sentient animate bodies but may obtain among inanimate bodies as well.

Weight is not here being defined in terms of the resistance to movement, only its measure is so defined. What then does resistance to movement measure? What is the power of the agent that causes this affection? Weight, obviously, but what, according to Timaeus, is weight?

This is a notoriously vexed question. In an earlier passage, Timaeus seems to characterize the weight of a primary body in terms of the number of elemental triangles that compose it (56b1--2). If one took resistance to movement to define, not the affection involved in the sense of weight, but as itself a definition of weight, then it would seem that Timaeus has defined weight twice over. This prompts \citet[136--d]{Cherniss:1944ma} to dismiss this earlier claim about weight as a ``passing remark''. While some commentators reject the earlier characterization of weight, others have tried to reconcile it with the present passage, albeit in different ways. Thus \citet[chapter 8]{OBrien:1984ji} offers a ``conciliationist'' interpretation on which weight is defined in terms of the number of elemental triangles. \citet{Code:2010bz}, on the other hand, offers a different reconciliation. He observes that the earlier characterization applies only to primary bodies. And since only secondary bodies are sensible, he concludes that weight, understood as a sensible quality, is not in fact defined at 56b1--2. I take resistance to movement to be the affection of perceivers caused by the weight of a body and not as a definition of weight. So the question of choosing between two inconsistent definitions does not arise. I am thus sympathetic to commentators, such as O'Brien and Code, who see no inconsistency between the earlier and later claims about weight. However, determining what exactly, according to Timaeus, weight is would take us too far afield, if it can be done decisively at all.

% subsection heavy_and_light (end)

\subsection{Smooth and Rough} % (fold)
\label{sub:smooth_and_rough}

Timaeus regards the causes of smoothness and roughness to be obvious. Anyone should be able to not only discern these but to explain them to others. The causes of smoothness and roughness thus stand in sharp contrast to the Maker and Father of the universe who is difficult to discover and impossible to explain to all (28c3--5). Specifically, then, the roughness of a thing consists in its hardness and irregularity, while the smoothness of a thing consists in its regularity and density. The role of regularity and irregularity is straightforward. However, the inattentive might initially be surprised by the asymmetry in the second conditions. Why is the hardness of the rough contrasted with density of the smooth as opposed to softness? But of course smooth things may be hard. Think of a marble table top. And insufficient density would itself be a source of discernable irregularity.

Timaeus's discussion of the smooth and the rough is brief owing to the obviousness of its subject matter. Perhaps it is all too brief. Notice that, here, Timaeus, in contrast with the other pairs of common \emph{pathēmata}, discusses only the broadly geometrical explanation of the power of the agent to cause the relevant \emph{pathēmata}. The \emph{pathēmata} themselves go unmentioned. Perhaps just as the sharpness of the sensation of heat reveals the power to divide of the tetrahedra of fire, so in offering a broadly geometrical explanation of the power to cause the relevant \emph{pathēmata}, the \emph{pathēmata}, the relevant affections of the flesh, are themselves obvious. Owing to this brevity, it is difficult to discern exactly how the relevant \emph{pathēmata} are proportional to the power of the agents that cause them and so a measure of these powers.

% subsection smooth_and_rough (end)

\subsection{Pleasure and Pain} % (fold)
\label{sub:pleasure_and_pain}

Finally, Timaeus turns to pleasure and pain (64a2–65b3). These are described as the most important of the common \emph{pathēmata}, but Timaeus does not immediately elaborate. However, their importance is not far to seek given their ethical significance: pleasure may incite the greatest of evil while pain may deter us from the good (68d)

Pleasure and pain, according to Timaeus, are not a fifth pair of \emph{pathēmata} common to the body as a whole. Rather they are potential elements in the four pair of \emph{pathēmata} that Timaeus has already discussed. Nor are they confined to the common \emph{pathēmata}. They may figure in the peculiar \emph{pathēmata} themselves. Thus odours, for example, may be pleasant or unpleasant. Timaeus will explain that \emph{pathēmata} may or may not determine perception or sensation. And of those that do, they may or may not be attended with pleasure or pain.

Timaeus prefaces his discussion of pleasure and pain with an explanation of why some \emph{pathēmata} yield perception and sensation while others do not (64a2--c7). When a \emph{pathēma} falls upon a body, whether or not it produces \emph{aisthēsis} depends upon that nature of the body receiving it. Timaeus' explanation of the nature of the recipient makes crucial use of the notion of mobility that he discussed earlier (54bff, 57d7ff). 

Consider a body composed of mobile particles, such as fire or air. If an affected part of that body is mobile, it passes that affection around. The affected part affects an adjacent part which in turn affects an adjacent part thus transmitting the affection in a circle. In the case of sentient living beings, this process continues until the affection reaches the \emph{phronimon}, the seat of intelligence or consciousness. Importantly, what is reported to the \emph{phronimon} is not the affection that has been passed around, but the power of the agent that produced it. The object of \emph{aisthēsis}, then, is not the \emph{pathēma} but the power of the agent that caused it, namely the power of the external agent that the \emph{pathēma} measures.

Matters are different, however, if the particles that compose the body are not mobile, but stable. It is presumably for this reason that the pseudepigraphical \emph{De natura mundi et animae} 100b--c speaks of earth here, since of the four primary bodies, earth is most stable. If the particles of the body are stable, it lacks circular motion. While the initial part may be affected, this affection is not passed around to the other parts. In the case of a living being, if the affected part of the animate body is composed of stable parts, then the affection is not passed on and so does not reach the \emph{phronimon}. And since the affection does not reach the \emph{phronimon}, the power of the agent that caused that affection is not reported with the result that \emph{aisthēsis} does not arise. Bones and nails are composed of stable parts and so lack circular motion. In contrast, in the case of vision and hearing, circular motion is at work since these involve parts of the body in which fire and air are the greater part and fire an air particles are mobile.

So whether or not a \emph{pathēma} gives rise to \emph{aisthēsis} depends upon the nature of the recipient, namely, whether or not it is composed of mobile or stable particles. (The pseudepigraphical \emph{De natura mundi et animae} 100c reports an additional factor: how strong or weak the affecting motion is.) Timaeus' explanation raises some important questions. Does the \emph{phronimon} belong to the mortal or immortal part of the soul? Where in the living body is it located? Is \emph{aisthēsis} the report, or is it, so to speak, the \emph{phronimon}'s receipt and understanding of that report. We shall return to these questions as we are not yet in a position to answer them.

Timaeus' explanation of pleasure and pain picks up on a feature earlier introduced in his discussion of cold. If the human body is sufficiently cold, it becomes compacted in a way that exceeds its natural state. An animate body will fight against this departure from its natural state with the consequence that we shiver or tremble in the cold. Timaeus explanation of pleasure and pain crucially relies on the idea that an animate body has a natural state. Pain involves a departure from the natural state of the body whereas pleasure involves a return. (\citealt[448--9]{Taylor:1928qb}, regards Timaeus' explanation as a generalization of an ancient medical tradition that traces back to Alcmaeon of Crotona.) However, not any departure or return to the natural state of the body results in pain or pleasure. Specifically, pain results when the natural state of the body is suddenly and violently disturbed whereas pleasure results from a sudden return to that natural state (64d1--3). Timaeus' explanation seems restricted to bodily pleasures and pains as opposed to psychic, such as the pain experienced in the loss of a loved one \citep[447--8]{Taylor:1928qb}.

Pain and pleasure each involve the suddenness of the departure or return to the natural state of the body. If the return to the natural state of the body is sufficiently gradual it will be insensible with the result that no pleasure is experienced. And if the departure from the natural state is gradual but the return sudden, no pain will be experienced though the sudden return may occasion intense pleasure, as when we experience pleasant odours. (In the \emph{Republic} 9 584b--c, Plato describes this class of pleasures as pure. And in the \emph{Philebus} 51b Plato understands the pleasure of smell as resulting from a want, though one that is not experienced as a want or painful.) Conversely, it the departure from the natural state is sudden but the return gradual, pain is experienced but no pleasure in the return to the natural state, as in the healing of cuts and burns.

Pain involves in addition to the suddenness of the departure from the natural state of the body the condition that the departure be violent. If the sudden departure from the natural state is not violent no pain will be experienced. Moreover, Timaeus explains this in terms of the size of the particles that compose the affected parts of the body. Specifically, if the particles are small, they will yield easily. However, if the particles are large, they will not yield easily and thus require violent effort to dislodge and so pass around the relevant affection. Moreover, it is only in cases of this latter kind that we are liable to experience pain and pleasure. Fire particles are the smallest of the primary bodies and thus yield easily. Consider then the visual stream compounded out of the fire emitted from the eye and daylight. Given the smalleness of the fire particles that compose it, the stream is easily disrupted and easily returns to its natural state, but no pain or pleasure attends this.

\citet[269]{Cornford:1935fk} understands the disruption of the visual stream as violent and so as an exception to the general claim that a sudden and violent departure from the natural state results in pain. There are three problems with this. First, Timaeus does not explicitly describe the disruption of the visual stream as violent. Second, on Cornford's reading, Timaeus' explanation of pain is insufficient, indeed, no explanation at all. And third, Timaeus contrasts violent affections with facile and explicitly links the size of the particles that compose the affected part of the body with the facility or ease of their affection. \citet[447]{Taylor:1928qb}, by contrast, makes no such error.

Pleasure and pain are manifestly not the measure of anything exogenous. (Compare Wittgenstein's enigmatic thought experiment concerning pain-patches, \emph{Philosophical Investigations} 312.) If Timaeus really understands \emph{aisthēsis} on the model of measurement, what then does pleasure and pain measure? While nothing exogenous, perhaps pleasure and pain are understood as measuring something endogenous, the sudden departure from or return to the natural state of the body. This would cohere well with pleasure and pain being bodily sensations.\\

Not all affected parts of the body are subject to pleasure and pain, only sufficiently mobile parts of the body are. The affection of bones and hair (and presumably any other parts, such as hair, that are composed of sufficient quantities of earth) do not give rise to pleasure or pain. We have also observed that not only common \emph{pathēmata} but peculiar \emph{pathēmata} may occasion pleasure and pain. Why then are pleasure and pain discussed along with the affections of the body as a whole? 

Perhaps Timaeus is moved by a phenomenological insight here. Should I injure my hand, there is a sense in which the pain that I feel is localized to that part of my body, but there is another sense in which it is the living being as a whole that feels the pain of the part. Perhaps it is this feature of the phenomenology of pleasure and pain that leads Timaeus to discuss it along with affections common to the body as a whole. To get a better sense of this consider Augustine who, like Timaeus, emphasizes the ethical significance of pleasure and pain and shares his conception of pain as a departure from the natural state of the animal (\emph{De musica} 11):
\begin{quote}
	it is the entire soul that feels the pain of a part of the body, yet it does not feel it in the entire body. When, for instance, there is an ache in the foot, the eye looks at it, the mouth speaks of it, and the hand reaches for it. (Augustine, \emph{De immortalitate animae} 16.25; \citealt[46]{Schopp:1947df})
\end{quote}

\noindent Let me conclude the discussion of the common \emph{pathēmata} with some general observations. 

\emph{Pathēmata} are not sensible qualities as Taylor or Cornford maintain. They are affections of a sentient animate body that are liable to give rise to perception or sensation. Indeed, Timaeus tends to conceive of them as a measure of the sensible qualities that cause them.

Timaeus' discussion of the common \emph{pathēmata} has revealed an important fact about the causal process involved in perception and sensation quite generally. Perception and sensation depends upon two factors:
\begin{enumerate}
	\item the affection of the sentient animate body
	\item the nature of the recipient of the affection
\end{enumerate}
Not only must the body of the living being be affected in some suitable manner but the part of the body that receives this affection must have a suitable nature, it must be composed of mobile particles so that the affection may be passed around.

The mobile nature of the recipient is a necessary part of the causal process leading to \emph{aisthēsis} since only if the affection is passed around may it be reported to the \emph{phronimon}, the seat of intelligence or consciousness. Perception or sensation only occurs with this report. 

Earlier (section~\ref{sec:pathemata}) I claimed that the common \emph{pathēmata} are affections of the sentient animate body involved in tactile perception and sensation. If that is right, then arguably Timaeus' account is incomplete. For in addition to the five pair of \emph{pathēmata}, there are other sensory contrasts manifest in touch. Thus, in \emph{De anima} 2.11 Aristotle mentions, in addition, dry and wet and in \emph{De generatione et corruptione} 2.2 viscous and brittle and coarse and fine. This is not necessarily a problem for Timaeus since he only needs to say enough to render his account more likely than any other. What would be a problem is if a sensory contrast manifest in touch could not be accommodated in terms of the principles governing Timaeus' account. Indeed this would threaten Timaeus' more general cosmological project. But that requires more than the observation that the five pair of \emph{pathēmata} do not exhaust the sensory contrasts manifest in touch. One would need to argue, in addition, that missing tactile contrasts could not in principle be accounted for by Timaeus.

Though some may be tempted to think otherwise, common \emph{pathēmata} are not common sensibles. According to Aristotle, common sensibles are perceptible in themselves and perceptible to more than one sense (\emph{De anima} 2.6, for discussion see \citealt[chapter 4.2]{Kalderon:2015fr}). The common \emph{pathēmata}, however, satisfy neither condition. The common \emph{pathēmata}, with the possible exception of pleasure and pain, are not sensory objects but their effects. The powers of the agents that cause them are what are perceptible in themselves. Contrary to Taylor and Cornford, these powers, and not the \emph{pathēmata}, are what are reported to the \emph{phronimon} when \emph{aisthēsis} occurs. Moreover, excepting pleasure and pain, the powers of the agents that cause the four pair of common \emph{pathēmata}---hot and cold, hard and soft, heavy and light, smooth and rough---are proper sensibles. One may see that the fire is hot, but not by seeing its heat. Heat may be a sensible quality, but it is tactile and not visible. Nor is it audible, nor available to any other sensory modality.

% subsection pleasure_and_pain (end)

% section common_emph_pathemata (end)


\section{Peculiar \emph{Pathēmata}} % (fold)
\label{sec:peculiar_emph_pathemata}

\subsection{The Tongue} % (fold)
\label{sub:the_tongue}



% subsection the_tongue (end)

\subsection{The Nostrils} % (fold)
\label{sub:the_nostrils}



% subsection the_notrils (end)

\subsection{The Ears} % (fold)
\label{sub:the_ears}



% subsection the_ears (end)

\subsection{The Eyes} % (fold)
\label{sub:the_eyes}



% subsection the_eyes (end)

% section peculiar_emph_pathemata (end)

% Chapter pathemata (end) 