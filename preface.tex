%!TEX root = /Users/markelikalderon/Documents/Git/timaeus/timaeus.tex
\chapter*{Preface} % (fold)
\markboth{\MakeUppercase{Preface}}{}
\addcontentsline{toc}{chapter}{Preface}
\label{cha:preface}

The books on Plato are legion. It is with some reluctance that I add to the heap. I thus feel obliged to provide some reason for doing so. Whether, in the end, I have sufficient reason is for the reader to decide.

I have long been intersted in extramission theories of visual perception for which Empedocles' lantern analogy ((Aristotle, \emph{De sensu} 2 437b27–438a3 = DK 31B84)) and the Timaean account of vision (45b–46c, 67c–68d) that it inspired are important sources. In the current naturalistic climate, this is an unfashionable interest, since extramission theories provide false causal models of visual perception. However, my interest in extramission theories is less with the causal model they provide for vision, than what phenomenological insights they may retain despite providing such models. For they make vivid the active outward directionality of vision, and I felt that it was important to recover whatever insights they may afford into this aspect of visual phenomenology \cite[chapter 5]{Kalderon:2018oe}. A proper understanding of Timaeus' account of vision is an essential step for such a project.

Partly for ideological reasons, I have a broader interest in the premodern history of the philosophy of perception. The Timaeus account of visual perception plays a crucial role in thinking about visual perception in both classical antiquity and the Hellenistic period (see \citealt[chapter 1]{Lindberg:1977aa}). A proper understanding of Timaeus' account of vision is an essential step, as well, for this broader project.

My interest in the Timaeus was not limited to vision, however. Timaeus provides accounts of the five special senses, if I may help myself here to this Peripatetic anachronism. And even if one were exclusively interested in Timaean vision, no adequate understand of vision as Timaeus conceives of it is possible by exclusively focussing on the two passages where Timaeus discusses vision (45b–46c, 67c–68d). Thus, for example, an important general claim about the aetiology of perception only emerges in Timaeus account of pleasure (64a2–65b3), and as \citet{Barker:2000dy} and \citet{Lautner:2005aa} argue, Timaeus account of audition has implications for his general understanding of perception. Moreover, I have been moved by the emerging general consensus in the philosophy of perception that it is a mistake to focus on vision to the exclusion of the other senses and so welcomed the opportunity to closely examine Timaeus' general survey of these.

Not only is my discussion of the \emph{Timaeus} not limited to the account of vision, neither is it limited to the passages that explicitly discuss perception more generally. As I continued to work on the \emph{Timaeus}, it struck me as being akin to a hologram, where each part is an image of the whole. Even a narrow focus on perception will reveal that the providence of the benevolent Demiurge is not completely understood without detailed knowledge of the workings of the liver. Indeed no major part of the Timaeus' speech is without direct relevance to the nature of perception as Timaeus conceives of it. The proemium, the cosmogeny, the psychogeny, the discussion of the mortal soul and the flesh that contains it, no less than Timaeus' discussion of auxiliary and true causes of vision and audition and the affections common to the body as a whole and peculiar to particular parts of the body in giving rise to perception and sensation, are all directly relevant. At first this occasioned despair. Short of writing a linear commentary on the entirety of Timaeus speech, which threatened to be of comparable length to Proclus' own monstrously long commentary, how could one manage this material? Nevertheless, I persisted and somehow muddled through, hopefully not producing a muddle in the process.

In thinking about the Timaeus, I have relied on a variety of different sources. While \citet{Skemp:1942oc} is surely right that an understanding of Timaeus' pre-Socratic sources is relevant to understanding the positions he develops, it seems odd to restrict oneself to these. Skemp himself relies on the works of the great British commentators, Archer-Hind, Taylor, and Cornford. But if one can help oneself to the insights of these, why not also help oneself to whatever insights might be afforded by Plutarch, Calcidius, and Proclus? Or Aristotle and Theophrastus, not to mention Pseudo-Timaeus Locrus? As an initiate, it struck me as impardonable and audacious to ignore these important contributions to our collective understanding. And throughout, I have relied not only on these but on the work of art historians, medical historians, classicists, as well as historians of ancient philosophy. As I had no idea what I was doing when I began, I couldn't very well ignore any help provided, no matter its source.

% Chapter preface (end)