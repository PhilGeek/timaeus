%!TEX root = /Users/markelikalderon/Documents/Git/timaeus/timaeus.tex
\chapter*{Preface} % (fold)
\markboth{\MakeUppercase{Preface}}{}
\addcontentsline{toc}{chapter}{Preface}
\label{cha:preface}

The books on Plato are legion. It is with some reluctance that I add to the heap. I thus feel obliged to provide some reason for doing so. Whether, in the end, I had sufficient reason is for the reader to decide.

I have long been interested in extramission theories of visual perception for which Empedocles' lantern analogy (Aristotle, \emph{De sensu} 2 437b27–438a3 = DK 31B84) and the Timaean account of vision (45b–46c, 67c–68d) that it inspired are important sources. In the current naturalistic climate, this is an unfashionable interest, since extramission theories provide manifestly false causal models of visual perception. However, my interest in extramission theories is less with the causal model they provide for vision, than what phenomenological insights they may retain despite providing such models. For they make vivid the active outward directionality of vision, and I felt that it was important to recover whatever insights they may afford into this aspect of visual phenomenology \cite[see][chapter 5, for an initial attempt]{Kalderon:2018oe}. A proper understanding of Timaeus' account of vision is an essential step for such a project.

Partly for ideological reasons, I have a broader interest in the premodern history of the philosophy of perception. I believe that contemporary philosophy of perception is hampered by assumptions inherited from early modern philosophy. Much of modern philosophy of perception still works within the modern paradigm, even the work of philosophers who self-consciously seek to go beyond it. Thinking hard about premodern philosophy of perception is a much needed propaedeutic in the search for alternatives. Timaeus' account of visual perception plays a crucial role in thinking about visual perception in both classical antiquity and the Hellenistic period (see \citealt[chapter 1]{Lindberg:1977aa}). A proper understanding of Timaeus' account of vision is an essential step, as well, for this broader project.

My interest in the Timaeus was not limited to vision, however. Timaeus provides accounts of the five special senses, if I may help myself here to this Peripatetic anachronism. And even if one were exclusively interested in Timaean vision, no adequate understand of vision as Timaeus conceives of it is possible by exclusively focussing on the two passages where Timaeus discusses vision (45b–46c, 67c–68d). Thus, for example, an important general claim about the aetiology of perception only emerges in Timaeus account of pleasure (64a2–65b3), and as \citet{Barker:2000dy} and \citet{Lautner:2005aa} argue, Timaeus account of audition has implications for his general understanding of perception. Moreover, I have been moved by the emerging general consensus in the philosophy of perception that it is a mistake to focus on vision to the exclusion of the other senses and so welcomed the opportunity to closely examine Timaeus' general survey of these.

Not only is my discussion of the \emph{Timaeus} not limited to the account of vision, neither is it limited to the passages that explicitly discuss perception more generally. As I continued to work on the \emph{Timaeus}, it struck me as being akin to a hologram, where each part is an image of the whole. Even a narrow focus on perception will reveal that the providence of the benevolent Demiurge is not completely understood without detailed knowledge of the workings of the liver. Indeed no major part of the Timaeus' speech is without direct relevance to the nature of perception as Timaeus conceives of it. The proemium, the cosmogeny, the psychogeny, the discussion of the mortal soul and the flesh that contains it, no less than Timaeus' discussion of auxiliary and true causes of vision and audition and the affections common to the body as a whole and peculiar to particular parts of the body in giving rise to perception and sensation, are all directly relevant. Conversely, perception, a specialist interest, assumes a cosmic significance in Timaeus' system. At first this occasioned despair. Short of writing a linear commentary on the entirety of Timaeus speech, which threatened to be of comparable length to Proclus' own monstrously long commentary, how could one manage this material? Nevertheless, I persisted and somehow muddled through, hopefully not producing a muddle in the process.

In thinking about the Timaeus, I have relied on a variety of different sources. While \citet{Skemp:1942oc} is surely right that an understanding of Timaeus' pre-Socratic sources is relevant to understanding the positions he develops, it seems odd to restrict oneself to these. Skemp himself relies on the works of the great British commentators, Archer-Hind, Taylor, and Cornford. But if one can help oneself to the insights of these, why not also help oneself to whatever insights might be afforded by Plutarch, Calcidius, and Proclus? Or Aristotle and Theophrastus, not to mention Pseudo-Timaeus Locri? As an initiate, it seemed audacious of me to ignore these important contributions to our collective understanding. And throughout, I have relied not only on these but on the work of art historians, medical historians, classicists, as well as historians of ancient philosophy. As I had no idea what I was doing when I began, I couldn't very well ignore any help provided, no matter its source.

As I wrote the present essay, I experimented with different ways of organizing the material. My initial thought was to organize the material by following the causal process that eventuated in perception. I experimented as well with other schemes. I found that I could follow none of them to the end. With each departure from Timaeus' presentation I eventually found my thoughts curbed or hampered in some way. I came to have a deep appreciation for the care and insight with which Timaeus organizes and presents his cosmology in a way that I would not had I not fought so hard against it. In the end, then, I have largely followed Timaeus' lead. This too has its drawbacks. Promissory notes must be issued and made good later on in the narrative. The discussion thus oftentimes loops back and picks up on earlier threads. Perhaps this is a manifestation of Plato's intent to provoke his audience to actively engage with the text. I confess that I often entertained doubts about whether I was up to this task. I am not sure that I have completely allayed these doubts. My greatest hope is that, at least in some small way, I have facilitated Plato's intent, that the present essay may itself provoke the reader to actively engage with the \emph{Timaeus}. It is not for nothing that Raphael depicts Plato as holding that book in his fresco \emph{Scoula di Atene} in the \emph{Stanze di Raffaello} in the Apostolic Palace.

Theophrastus in reporting the accounts of sense and sensibilia found in the \emph{Timaeus} straightforwardly attributes these to Plato (\emph{De sensibus}). \citet{Taylor:1928qb}, by contrast, denies any such an attribution. Rather, he took Plato to be reporting the distintively Pythagorean views of Timaeus. Taylor's interpretation has been criticized by \citet{Cornford:1935fk}. Despite the merits of Cornford's case, I do not believe that we are thereby licenesed to follow Theophrastus in straightforwardly attributing Timaeus' views to Plato. Throughout, I have focused on Timaeus' views quite apart from the question of whether or not, or to what extent, they may be attributed to Plato. Thus I make limited use of the broader Platonic corpus. If the views of Timaeus are compared to views that emerge in other Platonic dialogues, this is not done for the sake of establishing what Plato really thought. Such comparisons may reveal, for example, what is distinctive about Timaeus' development of a view. So for example, Timaeus has a distinctive take on the tripartite psychology, emphasizing its anatomical aspect.

To understand perception's cognizance of its object, on must understand Timaeus' general account of cognition. Timaeus associates cognitive acitvity with circular motion. Ever since Burnyeat's Cambridge seminar on the \emph{Timaeus}, there has been a growing trend to interpret such motion literally. This is in sharp contrast with the great British commentators who, influenced by the neo-Platonists, were inclined to offer non-literal interpretations of various aspects of Timaeus' speech. 

There are many salutary aspects of the literalist tend. Formost among these, to my mind at least, is the way that a literalist interpretation forces us to focus on the oftentimes unruly details of Timaeus' speech. However, this is sometimes conflated with the more general precept that to take an account seriously one must understand it literally \citep[see][7]{Broadie:2012vl}. Something like that precept is accepted both by the literalists and their opponents. Indeed, the latter refuse to take seriously certain aspects of Timaeus' account and so offer non-literal interpretations of it.

The general precept is false. Metaphor is not, or not merely, a device of non-commit\-ment. Nor is it merely ornamentation. To think otherwise is to ignore the important role metaphors play in our cognitive lives. Adherents of the general precept, be they literalist or not, are blind to the cognitive content of metaphorical thought. Ironically it was following the lead of the literalists and focusing on the details of Timaeus' speech that convinced me that a literal interpretation of the soul's motion is impossible. Literalists, such as \citet{Sedley:1997kr}, can take seriously, and so literally, certain aspects of Timaeus' speech only by ignoring others. Specifically, Timaeus offers two spatial descriptions of the World-Soul that conflict. Attending to one description and not the other masks that conflict. Worse still, not only do these spatial descriptions conflict, but each are self-contradictory or otherwise internally incoherent (chapter~\ref{cha:psychogeny}). If that is right, then there is no question of interpreting the spatial attributes of the soul literally. And, hence, no question of interpreting the motion of the soul literally. Taking the association of cognitive activity and circular motion seriously requires that we understand circular motion as a metaphor for cognitive activity (chapter~\ref{cha:cognitive_revolution}). 

This approach was fruitful with respect to other aspects of Timaeus speech as well. Thus, for example, Timaeus deploys a weaving metaphor for the union of soul and body. Attending to textile archealogy and the religous significance of weaving in the festivities of the Panathenea allow us to understand that metaphor and so better understand the soul-body union as Timaues conceives of it. Careful attention to metaphor also sheds light on the moral significance of Timaean anatomoy, too often overlooked by modern commentators (\citealt{Steel:2001ay} is a notable exception).

My staunch opposition to literalism is not an expression of ingratitude. It is only because I have learned so much from scholars defending literalism that I have been emboldened to criticize their views.



% Chapter preface (end)