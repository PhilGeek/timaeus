%!TEX root = /Users/markelikalderon/Documents/Git/timaeus/timaeus.tex

\chapter{Psychogeny} % (fold)
\label{cha:psychogeny}

\section{Psychogeny and the Philosophy of Perception} % (fold)
\label{sec:psychogeny_and_the_philosophy_of_perception}

What is there to learn about the philosophy of perception from Timaeus' psychogeny? The Demiurge's procedure in the generation of the World-Soul falls into three parts. First, He generates the substance of the soul by mixing indivisible and divisible Being, Sameness, and Difference. Second, He establishes proportional divisions in the substance of the soul. And finally, from this substance, He fabricates the Circles of the Same and the Different. The structure of the soul and the materials from which it is fabricated determine its capacities. 

But what capacities are these? Ancient opinion divides. According to one view, associated with Xenocrates, the composition of the World-Soul explains the soul as the principle of motion. According to another, associated with his student Crantor, the composition of the World-Soul explains the soul as the principle of cognition. (For these views see Plutarch, \emph{De animae procreatione in Timaeo} 1012e--1013a.) While the details of Xenocrates' and Crantor's accounts are inconsistent with one another, the general claims that the soul is the principle of motion and that it is the principle of cognition are not, as Aristotle explicitly recognizes:
\begin{quote}
	For having constituted the soul out of the elements and having divided it accordance with the harmonic numbers so that it might have an innate perception of harmony and so that the whole universe might be borne in harmonious orbits, he bent the straight into a circle. (Aristotle, \emph{De anima} 1 406b28--31; \citealt[10]{Shields:2016ix})
\end{quote}
According to Aristotle, then, the composition of the World-Soul both explains why it cognizes harmony and moves the Cosmos in harmonious orbits. The composition of the World-Soul, on Aristotle's view, thus explains the soul as the principle of motion and the principle of cognition. Our focus shall be on the latter explanation.

Since the soul of mortal beings is constructed from the same materials, even if the mixture is less pure, with the same proportional divisions and structure, the soul of mortal beings at the very least have similar cognitive capacities. So an understanding of the psychogney promises to shed light on the cognitive capacities of mortal beings.

This is not yet to establish psychogeny's relevance to perception. Psychogeny may be relevant to the cognitive capacities of the immortal soul, but perception is linked, instead, to the mortal soul. But, as we shall see (chapter~\ref{cha:common_pathemata}), perception involves reporting its object to the \emph{phronimon}, the seat of cognizance. Timaeus conceives of perception as a kind of measurement. In perception, the body is affected and this affection is the measure of the powers of the agents that cause this affection. When these powers are reported to the \emph{phronimon}, perception supervenes. Though elusive, the \emph{phronimon} is naturally understood as among the cognitive capacities of the immortal soul. The young gods may endow mortal beings with a mortal soul, in part, so that they may perceive, but the power to do so depends upon the cognitive capacities they have thanks to their immortal soul.


% section psychogeny_and_the_philosophy_of_perception (end)

\section{Incarnation and Psychogeny} % (fold)
\label{sec:embodiment_and_psychogeny}

If cosmogeny describes the generation of the body of the Cosmos, psychogeny describes the generation of the World-Soul. Timaeus goes out of his way to point out that he treats these subjects out of order. We are not to suppose that the body of the Cosmos was created first and only then is the soul that animates it created. The World-Soul rules over the body of the Cosmos and it is only fitting that the elder should rule over the younger. If Timaeus' narrative departs from the natural order of things, this is solely due to human limitations in understanding such matters.

The psychogeny is not only preceded by the cosmogeny in Timaeus' speech, but it is also preceded by an account of the soul--body union. Timaeus concludes his account of the generation of the body of the Cosmos with an account of its ensoulment. Moreover, this account is repeated at the conclusion of the psychogeny. Thus the Timaean psychogeny is bookended with accounts of the union of the body of the Cosmos with the World-Soul. Here, however, Timaeus does not remark about the oddity of his procedure. If one thinks about the all too human demands of exposition, would it not be more natural to give an account of the soul--body union only after giving accounts of the soul and the body? And why repeat the account of soul--body without apparent embellishment? 

Other puzzles, however, will be our central concern. The accounts of the ensoulment of the body of the Cosmos and the incarnation of the World-Soul ascribe spatial properties to the World-Soul, as does the psychogeny proper. However, these straightforwardly conflict. On the first, the World-Soul is stretched from the center throughout the body of the Cosmos so as to entirely encompass it. The World-Soul is thus depicted as a voluminous sphere. On the second, it consists in circles, the Circles of the Same and the Different, with the Circle of the Same containing the Circle of the Different, which is itself divided into smaller circles. On the second account, unlike the first, the World-Soul is not a voluminous sphere. What is more, each is internally incoherent. On the first account, the World-Soul is said to encompass the spherical body of the Cosmos. But beyond the limits of the Cosmos is neither space nor void. If beyond the limits of the body of the Cosmos there is neither space nor void, there is no space in which the World-Soul may encompass it. On the second account, the Circle of the Different is said to be contained within the Circle of the Same and so smaller than it, and yet each are produced from two strips of material of equal length. 

These puzzles do not admit of a proper resolution. There is no way to read the account of soul--body union and the psychogeny both literally and coherently. Moreover, these puzzles are so blatant and on the surface of the text that is not possible to read Timaeus as being unaware of them. On this basis, I shall argue that the ascriptions of spatial properties to the World-Soul cannot be interpreted literally. This, in turn, has an important consequence. Not only does Timaeus ascribe spatial properties to the World-Soul, but he also ascribes motion to it. But if we cannot interpret the ascription of spatial properties to the World-Soul literally, neither can we interpret the ascription of motion to it literally. 

% section embodiment_and_psychogeny (end)

\section{The Soul-Body Union} % (fold)
\label{sec:the_embodiment_of_the_world_soul}

Timaeus accounts for the union of the World-Soul and the body of the Cosmos twice over. And though there are minor differences in detail, there is no substantive embellishment in the second retelling. What, then, could justify the retelling? One thought might be that Timaeus does so for emphasis. After all, the soul--body union is an important topic. However, his account of the the union of the World-Soul and the body of the Cosmos is brief, much briefer and less involved that his account of the psychogeny proper. If emphasis was really Timaeus' concern, would not a more elaborate account be a better way to accomplish it? 

I suspect that the same Demiurgic activity is being described from two different perspectives. The Cosmogeny ends with a description of the ensoulment of the body of the Cosmos. Similarly, the psychogeny ends with a description of the incarnation of the World-Soul. Each describes the same event brought about by Demiurgic activity, though from the perspective of different participants of that event.

Timaeus describes the ensoulment of the body of the Cosmos at 34a8--34b10. Timae\-us begins by summing up, in reverse order, his discussion of cosmic morphology: The Demiurge has created the body of the cosmos with a smooth and even surface (33b8--34a8), with sides of equal distance from its center (33b4--8), a complete body made up of complete bodies (33b1--4). Now he describes what happens to it, the body of the Cosmos, as a result of Demiurgic activity. The Demiurge begins by placing soul in the center of the Cosmos. Note well the lack of a definite article. Timaeus does not say that that he placed the soul in the midst of the Cosmos, but soul (a detail keenly observed by Proclus, \emph{In Timaeum} 2 107, though perhaps he owes this observation to Syrianus). The occurrence of \emph{psuchē}, here, is thus better interpreted as a mass noun rather than a count noun. This contrasts with the ensoulment of mortal beings. The soul of mortal beings is placed in the midst of their body, bound to marrow and encased by bone and flesh. Not so the World-Soul. Having placed soul in the middle of the body of the Cosmos, the Demiurge stretches (\emph{eteinen}) the soul throughout the whole of it. The World-Soul thus permeates the body of the Cosmos. There is a potential echo of Empedocles here: ``but it/he is only a sacred and ineffable thought organ darting through the entire cosmos with swift thoughts'' (DK 31B134, \citealt[263]{Inwood:2001ve}). Not only does the the World-Soul permeate the body of the Cosmos but it encompasses it from without, covering it all around (\emph{perikalupsen}). 

Proclus understands the imagery, here, as illuminationist. Soul having been placed in the center of the Cosmos illuminates by its powers the whole of the Cosmos and beyond (\emph{In Timaeum} 2 108). If the aptness of the illuminationist imagery is unclear, compare Robert Grosseteste's vivid description of the propagation of light:
\begin{quote}
	For light of its very nature diffuses itself in every direction in such a way that a point of light will produce instantaneously a sphere of light of any size whatsoever, unless some opaque object stand in the way. (Robert Grosseteste, \emph{De luce}, \citealt[10]{Riedl:1942it})
\end{quote}
The retelling of this event, from the perspective of the World-Soul, invokes, instead, a textile metaphor. Soul is said to be interwoven into the body of the Cosmos. The imagery in both passages, though distinct, need not conflict and may be complementary. Indeed, as we shall see, the weaving metaphor in the retelling potentially lends significance to the occurrences of \emph{eteinen} and \emph{periekalupsen} in the initial telling. 

We are not to envision the World-Soul's encompassment of the body of the Cosmos as their merely being coincident. Rather, the World-Soul does so from without (\emph{exōthen}). \citet[58]{Cornford:1935fk} tries to resist this interpretation, claiming, instead, that the World-Soul is coincident with the body of the Cosmos (see also \citealt[105]{Taylor:1928qb}, \citealt[85]{Skemp:1942oc}, \citealt[70]{Robinson:1970lq}). He cites Alcinous as a precedent (\emph{Didaska\-likos} 14.4). There are two problems with this bit of evidence. First, there is no consensus among ancient commentators on this interpretation as Proclus reports (\emph{In Timaeum} 2 104--8) And, second, Alcinous, there, seems to conflate the exterior portion of the World-Soul with the Circle of the Same in a way not borne out by the text (though see \citealt[105]{Taylor:1928qb}). Cornford, in addition, cites other occurrences of \emph{exōthen} such that, if the present occurrence is read in light of these, it would not carry the implication that the World-Soul extends beyond the body of the Cosmos. Cornford cites occurrences of \emph{exōthen} in Aristotle (``the parts of animals on the outside'', \emph{ta exōthen moria tōn zōōn}, \emph{Historia Animālium} 494a22) and in Plotinus (``the circumference on the outside'' of a circle, \emph{hē exōthen periphereia}, \emph{Ennead} 2.2.1). Neither occurrence serves Cornford's purpose, however. Animals and corporeal circles exist in an environment which their outer surface or circumference faces. But the Cosmos does not itself exist in an environment. It is the environment. While it may have an inner circumference, it lacks an outer circumference. It is not outer facing since there is nothing outer to face. The only Platonic occurrence Cornford cites is \emph{Sophist} 253d where the specific Forms are embraced from the outside by the generic Form, remarking that ``the genus does not extend farther than the species it contains''. But the Forms are by nature a-spatial. There is no question of reading \emph{exōthen} as anything other than a metaphor here. This contrasts with the present case where it is very much an open question whether the spatial language that Timaeus uses to describe the World-Soul is best understood literally. And even if it is not, to discern its significance, we first must take it seriously. Thus, for example, Proclus, following his teacher Syrianus, takes the World-Soul to be non-coincident with the body of the Cosmos and understands its extending beyond the Cosmos as indicating a hypercosmic aspect (\emph{In Timaeum} 2 105--6).

If I am right, and we have to take seriously Timaeus' claim that the World-Soul encompasses the body of the Cosmos from without, then there is a puzzle here. (Indeed, I suspect that Cornford's motivation for his alternative interpretation consists, in part, in the avoidance of this puzzle.) Beyond the limits of the Cosmos is neither space nor void. There is no space, then, for the World-Soul to extend beyond the limits of the Cosmos. If we take the spatial language of \emph{exōthen} seriously, then there is no coherent way to understand it literally. What then is its significance? 

\citet[144--5]{OBrien:1969ty} sees here a Platonic criticism of Empedocles:
\begin{quote}
	The position of Plato's soul is the same as the position we have attributed to Love, stretching out from the centre in all directions in the form of a sphere. But Plato makes a point of wrapping soul round the \emph{outside} of the elements as well. This is very likely in order to make Love occupy the position of Strife. In this way Plato shows agains his denial of the existence of Empedocles' evil god, precisely as he had done in the \emph{Politicus}, and by slight implication in the Sophist. \citep[145]{OBrien:1969ty}
\end{quote}
O'Brien may well be right about this. Despite many points of contact between Timaeus' cosmogeny and Empedocles, Strife is conspicuous in its absence. I doubt that the significance of the World-Soul encompassing the body of the Cosmos is exhausted by this criticism of Empedocles, however. I suspect that Timaeus is making a positive as well as a negative point. Soul's encompassment of the corporeal is meant, in addition, to mark a contrast between the ensoulment of the body of the Cosmos and the ensoulment of the bodies of mortal beings. Whereas the World-Soul encompasses the body of the Cosmos, the body of a mortal being encompasses its soul. We shall discuss this contrast further in chapter~\ref{cha:incarnation}.

The incarnation of the World-Soul is taken up after the completion of the psychogeny. The ensoulment of the body of the Cosmos is now narrated from the perspective of the World-Soul, the other participant in the soul--body union. When the Demiurge completes the construction of the World-Soul to his satisfaction, He fabricates within it all that is corporeal. This marks the first difference with the ensoulment of the body of the Cosmos. According to that earlier narrative, soul is placed in the center of the Cosmos and is stretched throughout so as to encompass it from without. According to the present narrative, the encompassment of the body of the Cosmos from without is clear from the very beginning. The corporeal is generated within the World-Soul. Timaeus also adds a detail not present in the narrative of the ensoulment of the body of the Cosmos. There we were told that soul was placed in center of the Cosmos. When narrating the incarnation of the World-Soul, Timaeus claims, in addition that they are placed center to center. The new detail, then, is that the World-Soul has a center and that this is coincident with the center of the body of the Cosmos. Timaeus goes on to claim that the World-Soul is interwoven everywhere (\emph{pantē daiaplakeisa}) throughout the Heaven (\emph{ouranos}, here, used as equivalent to \emph{kosmos}) and encompassing it from without (\emph{exōthen perikalupsasa}). Proclus notes a further potential difference between our two passages:
\begin{quote}
	These things differ not at all from the words that are presently before us, for the phrases `extending it from the middle in every way' and `interwoven from the middle on out in every direction to the outermost edge of the heaven' come to the same thing except that in the one case [the retelling] the soul itself from itself has illuminated by its own powers the centre of the universe and the whole sphere, but in the present case [the initial telling] the Demiurge is the cause of the ensoulment when he transplants the soul into the universe. (Proclus, \emph{In Timaeum} 2 108, \citealt[64]{Baltzly:2009bc})
\end{quote}
If Proclus is right, it is doubtful that Timaeus is denying Demiurgic agency in the retelling of the event. Rather, he is emphasizing that soul is the cause of intelligent and unceasing life of the Cosmos and that its body is a passive recipient of these.

The weaving metaphor is not incidental. It is invoked again and at greater length in Timaeus' discussion of the soul--body union of mortal beings (41d, discussed in chapter~\ref{sec:the_demiurge_addressing_the_gods}). Indeed, weaving imagery is not infrequent in the Platonic corpus (\emph{Theaetetus} 202b3, \emph{Sophist} 259e6, 262d4, \emph{Statesman} 279a--283a, and \emph{Cratylus} 387e--390b). The earlier occurrences of \emph{eteinen} and \emph{perikalupsen} are, perhaps, explicable in terms of the weaving metaphor now invoked. First consider \emph{eteinen}. In a loom the warp threads are stretched tight and used as support for the weft which is drawn through. The soul of the Cosmos, that within which its body is generated, seems analogous to the warp in the support in lends to the corporeal. This is why it is stretched tight throughout. Second, \emph{exōthen perikalupsasa} itself has textile connotations. Homer uses similar language to describe a cloak that surrounds a body as an image for sleep enveloping the sleeper (\emph{Illiad} 14.359). And a cloak is itself a product of weaving. The weaving imagery, more prominent in the retelling of the event than in its initial telling, need not be inconsistent will the illuminationist imagery that Proclus discerns (should it indeed be of genuine Timaean provenance).

% section the_embodiment_of_the_world_soul (end)

\section{Soul Mixture} % (fold)
\label{sec:soul_mixture}

The first step in the psychogney is the generation of the substance of the World-Soul from the mixture of indivisible and divisible Being, Sameness, and Difference. The dominant modern interpretation of this mixture (\citealt{Grube:1932qr}, \citealt[60--1]{Cornford:1935fk} \citealt[70--1]{Robinson:1970lq}) is due to Proclus, \emph{In Timaeum} 2 (\citealt{Diehl:1903re}, for discussion of alternative interpretations see \citealt[106--36]{Taylor:1928qb}, \citealt[51--54]{Shorey:1889sx}, and \citealt[352]{Shorey:1928ar}). According to the Proclean interpretation, the mixture occurs in four stages. Specifically, the Demiurge mixes:
\begin{enumerate}[(1)]
	\item indivisible and divisible Being
	\item indivisible and divisible Sameness
	\item indivisible and divisible Difference
	\item the three resulting mixtures (1--3)
\end{enumerate} 
We shall examine these in turn. But first let us consider the contrast between the indivisible and the divisible.

Timaeus marks a contrast between what is indivisible and what is divisible. With respect to indivisible (\emph{ameristou}) Being it is said to remain always the same (\emph{aei kata tauta}), where as divisible Being is said specifically to be divided around bodies (\emph{peri ta somata} \ldots\ \emph{meristēs}) and generated (\emph{gignomenēs}). Since \emph{gignomenēs} contrasts with \emph{aei kata tauta} presumably it means something like becoming or subject to change rather than being generated in the narrow sense of having come to be as the product of some agency. The associations of being changeless with the indivisible and being subject to change with the divisible presumably carry over to the application of this distinction to Sameness and Difference as well. Presumably so too does the specification of the divisible as being divided around bodies. The specification also provides a hint about the intended contrast. Recall that, for Timaeus, the mark of the corporeal is that it is sensible rather than being extended as it is for Descartes. This carries the strong suggestion that the indivisible is, by contrast, intelligible.

(1) \emph{Indivisible and divisible Being.} Being that is divided around bodies and subject to change is the kind of being suffered by the sensible and the corporeal. That much is clear. But what about indivisible Being? If we follow the suggestion that it is intelligible, then it is natural to suppose that it is the kind of Being enjoyed by the Forms. Individual Forms, it might be thought, are incomposite and hence indivisible (\citealt[71]{Robinson:1970lq}). The corporeal objects of sense, by contrast, are composite. We learn that they are composed of elemental triangles themselves too small to be seen. One difficulty with this suggestion, perhaps not insurmountable, is that a prominent example of a changeless intelligible being in the \emph{Timaeus} is the Paradigm. But the Paradigm is in a sense composite. It is a comprehensive living being that contains within itself all other intelligible living beings. If the Paradigm is indivisible it must be so in a sense consistent with being the comprehensive intelligible living being.

If the substance of soul is generated, in part, by mixing indivisible and divisible Being, the point of this Demiurgic procedure must surely be to endow soul with affinities to each. While the body of the Cosmos is visible the World-Soul is invisible (36e8). In this respect it is like indivisible Being and unlike divisible Being. Moreover, the World-Soul permeates the body of the Cosmos. If this does not come to the same thing as being extended at the very least it is present to every part of the Cosmic body. (I hedge only because being extended comes dangerously close to just being divided around bodies.) In this respect it is like divisible Being and unlike divisible being. The World-Soul, though generated, at least according to the \emph{eikos muthos}, is imperishable like indivisible Being (since perishing is a kind of change and indivisible Being is changeless.) Nevertheless, its activities are variable. Among its activities is forming true opinions about what transpires in the sensible realm. These will vary as the sensible realm varies. In this respect, it is like divisible Being.

This last observations suggests a different if complementary take on the point of Demiurgic mixture. If Timaeus subscribes to the principle that like is known by like, then part of the point of mixing indivisible and divisible Being is so that the World-Soul may cognize each. Indeed it is capable of both contemplating the intelligible, like its maker and Father, and opining about the sensible and the corporeal. As Shorey observes, the point generalizes:
\begin{quote}
	It is necessary that the Soul should recognize everywhere \ldots\ the \emph{same}, the \emph{other} and \emph{essence}, those three {\sbl μέγιστα γένη} of the \ldots\ \emph{Sophist}. Hence, on the Greek principle that like is known by like, Plato makes real substances out of these three abstractions and puts them as plastic materials into the hands of the Demiurgus for the formation of the Soul.
\end{quote}

(2) \emph{Indivisible and divisible Sameness.} If Forms are indivisible, in some suitable sense, and every From is the same as itself, then the Forms enjoy a kind of indivisible Sameness. The sensible and the corporeal are divisible. And yet they too may, in some sense, remain the same. Though the pages of a book may yellow with age, it remains the same copy despite this variability. The sensible and corporeal thus suffer a divisible form of Sameness.

(3) \emph{Indivisible and divisible Difference.}

(4) \emph{The resulting mixtures mixed.}

% section soul_mixture (end)

\section{Proportional Division} % (fold)
\label{sec:proportional_division}



% section proportional_division (end)

\section{The Circles of the Same and the Different} % (fold)
\label{sec:the_circles_of_the_same_and_the_different}



% section the_circles_of_the_same_and_the_different (end)

\section{The Cognitive Significance of Circular Motion} % (fold)
\label{sec:the_cognitive_significance_of_circular_motion}



% section the_cognitive_significance_of_circular_motion (end)

\section{Concluding Observations} % (fold)
\label{sec:concluding_observations_p}



% section concluding_observations_p (end)

% Chapter psychogeny (end) 