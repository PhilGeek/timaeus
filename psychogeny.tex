%!TEX root = /Users/markelikalderon/Documents/Git/timaeus/timaeus.tex

\chapter{Psychogeny} % (fold)
\label{cha:psychogeny}

\section{Incarnation and Psychogeny} % (fold)
\label{sec:embodiment_and_psychogeny}

If cosmogeny describes the generation of the body of the Cosmos, psychogeny describes the generation of the World-Soul. Timaeus goes out of his way to point out that he treats these subjects out of order. We are not to suppose that the body of the Cosmos was created first and only then is the soul that animates it created. The World-Soul rules over the body of the Cosmos and it is only fitting that the elder should rule over the younger. If Timaeus' narrative departs from the natural order of things, this is solely due to human limitations in understanding such matters.

The psychogeny is not only preceded by the cosmogeny in Timaeus' speech, but it is also preceded by an account of the soul--body union. Timaeus concludes his account of the generation of the body of the Cosmos with an account of its ensoulment. Moreover, this account is repeated at the conclusion of the psychogeny. Thus the Timaean psychogeny is bookended with accounts of the ensoulment of the body of the Cosmos and the incarnation of the World-Soul. Here, however, Timaeus does not remark about the oddity of his procedure. If one thinks about the all too human demands of exposition, would it not be more natural to give an account of the soul--body union only after giving accounts of the soul and the body? And why repeat the account of soul--body without apparent embellishment? 

Other puzzles, however, will be our central concern. The accounts of the ensoulment of the body of the Cosmos and the incarnation of the World-Soul ascribe spatial properties to the World-Soul, as does the psychogeny proper. However, these straightforwardly conflict. On the first, the World-Soul is stretched from the center throughout the body of the Cosmos so as to entirely encompass it. The World-Soul is thus depicted as a voluminous sphere. On the second, it consists in circles, the Circles of the Same and the Different, with the Circle of the Same containing the Circle of the Different, which is itself divided into smaller circles. On the second account, unlike the first, the World-Soul is not a voluminous sphere. What is more, each is internally incoherent. On the first account, the World-Soul is said to encompass the spherical body of the Cosmos. But beyond the limits of the Cosmos is neither space nor void. If beyond the limits of the body of the Cosmos there is neither space nor void, there is no space in which the World-Soul may encompass it. On the second account, the Circle of the Different is said to be contained within the Circle of the Same and so smaller than it, and yet each are produced from two strips of material of equal length. 

These puzzles do not admit of a proper resolution. There is no way to read the account of soul--body union and the psychogeny both literally and coherently. Moreover, these puzzles are so blatant and on the surface of the text that is not possible to read Timaeus as being unaware of them. On this basis, I shall argue that the ascriptions of spatial properties to the World-Soul cannot be interpreted literally. This, in turn, has an important consequence. Not only does Timaeus ascribe spatial properties to the World-Soul, but he also ascribes motion to it. But if we cannot interpret the ascription of spatial properties to the World-Soul literally, neither can we interpret the ascription of motion to it literally. 

% section embodiment_and_psychogeny (end)

\section{The Soul-Body Union} % (fold)
\label{sec:the_embodiment_of_the_world_soul}

Timaeus accounts for the union of the World-Soul and the body of the Cosmos twice over. And though there are minor differences in detail, there is no substantive embellishment in the second retelling. What, then, could justify the retelling? One thought might be that Timaeus does so for emphasis. After all, the soul--body union is an important topic. However, his account of the the union of the World-Soul and the body of the Cosmos is brief, much briefer and less involved that his account of psychogeny proper. If emphasis was really Timaeus' concern, would not a more elaborate account be a better way to accomplish it? 

I suspect that the same Demiurgic activity is being described from two perspectives. The Cosmogeny ends with a description of the ensoulment of the body of the Cosmos. Similarly, the psychogeny ends with a description of the incarnation of the World-Soul. Each describes the same event brought about by Demiurgic agency, though from the perspective of different participants of that event.

Timaeus describes the ensoulment of the body of the Cosmos at 34a8--34b11. Timaeus begins by summing up, in reverse order, his discussion of cosmic morphology: The Demiurge has created the body of the cosmos with a smooth and even surface (33b8--34a8), with sides of equal distance from its center (33b4--8), a complete body made up of complete bodies (33b1--4). Now he describes what happens to it, the body of the Cosmos, as a result of Demiurgic activity. The Demiurge begins by placing soul in the middle of the body of the Cosmos. Note well the lack of a definite article. Timaeus does not say that that he placed the soul in the midst of the Cosmos, but soul. The occurrence of \emph{psuchē}, here, is thus better interpreted as a mass noun rather than a count noun. This contrasts with the ensoulment of mortal beings. The soul of mortal beings is placed in the midst of their body, bound to marrow and encased by bone and flesh. Not so the World-Soul. Having placed soul in the middle of the body of the Cosmos, the Demiurge stretches (\emph{eteinen}) the soul throughout the whole of it. The World-Soul thus permeates the body of the Cosmos. Not only does the the World-Soul permeate the body of the Cosmos but it encompasses it from without, covering it all around (\emph{perikalupsen}). (The potential significance of \emph{eteinen} and \emph{periekalupsien} shall emerge in the retelling of this event, from the perspective of the World-Soul.) We are not to envision the World-Soul's encompassment of the body of the Cosmos as their merely being coincident. Rather, the World-Soul does so from without (\emph{exōthen}).

\citet[58]{Cornford:1935fk} tries to resist this interpretation, claiming instead that the World-Soul is coincident with the body of the Cosmos. He cites Alcinous as precedent (\emph{Didaska\-likos} 14.4). There are two problems with this bit of evidence. First, there is no consensus among ancient commentators on this interpretation as Proclus reports (\emph{In Timaeum} ) And, second, Alcinous, there, seems to conflate the exterior portion of the World-Soul with the Circle of the Same in a way not borne out by the text (though see \citealt[105]{Taylor:1928qb}). Cornford, in addition, cites other occurrences of \emph{exōthen} such that, if the present occurrence is read in light of these, it would not carry the implication that the World-Soul extends beyond the body of the Cosmos. The only Platonic occurrence he cites is \emph{Sophist} 253d where the specific Forms are embraced from the outside by the generic Form, remarking that ``the genus does not extend farther than the species it contains''. But the Forms are by nature a-spatial. There is no question of reading \emph{exōthen} as anything other than a metaphor. This contrasts with the present case where it is very much an open question whether the spatial language that Timaeus uses to describe the World-Soul is best understood literally. And even if it is not, to discern its significance, we first must take it seriously. Thus, for example, Proclus, following his teacher Syrianus, takes the World-Soul to be non-coincident with the body of the Cosmos and understands its extending beyond the Cosmos as indicating a hypercosmic aspect.

If I am right, and we have to take seriously Timaeus' claim that the World-Soul encompasses the body of the Cosmos from without, then there is a puzzle here. Beyond the limits of the Cosmos is neither space nor void. There is no space, then, for the World-Soul to extend beyond the limits of the Cosmos. If we take the spatial language of \emph{exōthen} seriously, then there is no coherent way to understand it literally. What then is its significance? While I am uncertain as to the full significance of \emph{exōthen}, I am certain as to its partial significance. It is intended, at least in part, to mark a contrast between the ensoulment of the body of the Cosmos and the ensoulment of the bodies of mortal beings. Whereas the World-Soul encompasses the body of the Cosmos, the body of a mortal being encompasses its soul. The significance of this contrast is discussed in chapter~\ref{cha:incarnation}.

The incarnation of the World-Soul is taken up after the completion of the psychogeny. The ensoulment of the body of the Cosmos is now narrated from the perspective of the World-Soul, the other participant in the soul--body union. When the Demiurge completes the construction of the World-Soul to his satisfaction, He fabricates within it all that is corporeal. This marks the first difference with the ensoulment of the body of the Cosmos. According to that earlier narrative, soul is placed in the middle of the body of the Cosmos and is stretched throughout so as to encompass it from without. According to the present narrative, the encompassment of the body of the Cosmos from without is clear from the very beginning. The corporeal is generated within the World-Soul. Timaeus also adds a detail not present in the narrative of the ensoulment of the body of the Cosmos. There we were told that soul was placed in middle of the Cosmos. When narrating the incarnation of the World-Soul, Timaeus claims, in addition that they are placed center to center. The World-Soul is interwoven everywhere (\emph{pantē daiaplakeisa}) throughout the Heaven (\emph{ouranos}, here, used as an equivalent to \emph{kosmos}) and encompassing it from without (\emph{exōthen perikalupsasa}). 

The weaving metaphor is not incidental. It is invoked again and at greater length in Timaeus' discussion of the soul--body union of mortal beings (discussed in chapter~\ref{sec:the_demiurge_addressing_the_gods}). The earlier occurrences of \emph{eteinen} and \emph{perikalupsen} are, perhaps, explicable in terms of the weaving metaphor now invoked. In a loom the warp threads are stretched tight and used as support for the weft which is drawn through. The soul of the Cosmos, that within which its body is generated, seems analogous to the warp in the support in lends to the corporeal. This is why it is stretched tight throughout. \emph{Exōthen perikalupsasa} itself has textile connotations. Homer uses similar language to describe a cloak that surrounds a body as an image for sleep (\emph{Illiad} 14.359).

% When the construction of the World-Soul is completed, the Demiurge constructs within it all that is corporeal. The soul of the Cosmos and its body are united center to center. The Demiurge then fits them together by interweaving them. The soul of the Cosmos is interwoven everywhere (\emph{pantē daiaplakeisa}) into its body.

% section the_embodiment_of_the_world_soul (end)

% Chapter psychogeny (end) 