%!TEX root = /Users/markelikalderon/Documents/Git/timaeus/timaeus.tex

\chapter{Cognitive Revolution} % (fold)
\label{cha:cognitive_revolution}

\section{Psychogeny and Cognitive Power} % (fold)
\label{sec:psychogeny_and_psychic_power}

The psychogeny, understood strictly as Timaeus' account of the generation of the World-Soul, was the topic of the last chapter. Timaeus' account of the generation of the World-Soul is important since the nature of its substance, its proportional division, and its being formed into the Circles of the Same and the Different determine its kinetic and cognitive powers. Our understanding of these powers is incomplete unless they are understood to be determined, at least in part, by the substance, proportional division, and structure that the World-Soul has thanks to Demiurgic activity. The details of the Demiurge's manufacture and construction of the World-Soul thus go some way to explaining the powers that He invests it with. Timaeus makes this connection explicit with respect to its cognitive powers. The World-Soul has its cognitive powers inasmuch as it is a mixture of Being, Sameness, and Difference, proportionally divided, and revolves around itself (37a). The explanation of the exercise of the World-Soul's cognitive powers would thus involve its substance, proportional division, and structure.

The World-Soul is elder sister to the souls of mortal beings. The Demiurge is the maker and Father of the World-Soul as well as the souls of mortal beings. The souls of mortal beings are generated after the generation of the World-Soul. The souls of mortal beings are mixed from the same material, though impurely, that was used to mix the substance of the World-Soul. Moreover, the Demiurge invests the souls of mortal beings with the same proportional divisions and structure as the World-Soul. The World-Soul is prior not only in birth but in dignity. It is more perfect than the souls of mortal beings as signaled by the impurity of their substance. Being perfect it is an exemplar for the souls of mortal beings. Thus like an elder sister, the souls of mortal beings look to her for guidance. Literally. The cognitive powers of the World-Soul are visibly manifest in the harmony of celestial motion. And by rationally attending to this motion in the right sort of way the circles in the souls of mortal beings are aligned to the circles in the World-Soul making them more rational and virtuous. In this way should we strive to be like our elder sister.

The way that the psychogeny explains the exercise of the cognitive powers of the World-Soul is relevant to the psychology of mortal beings. Given the relationship between the World-Soul and the souls of mortal beings, the explanation of the exercise of the cognitive powers of the souls of mortal beings should parallel the explanation of the exercise of the cognitive powers of the World-Soul. 

This chapter will discuss the cognitive powers of the World-Soul as Timaeus presents them to be at 37a--c. The passage largely consists in two long and syntactically complex sentences that have been variously interpreted. There Timaeus describes:
\begin{enumerate}[(1)]
	\item The objects of these cognitive powers, divisible and indivisible beings
	\item The World-Soul's ``contact'' (\emph{ephaptētai}) with its objects
	\item The thought that ``contact'' elicits
	\item The two kinds of cognitive powers that the World-Soul enjoys, the power to know the indivisible and the power to have true opinion about the divisible.
\end{enumerate}
Let us consider these in turn.

% section psychogeny_and_psychic_power (end)

\section{The Objects of Cognition} % (fold)
\label{sec:the_objects_of_cognition}

The objects of cognition have either divisible or indivisible Being. Divisible Being is associated with Becoming (\emph{gignomenēs}) and is divided around bodies (\emph{peri ta somata} \ldots\ \emph{meristēs}), whereas indivisible Being always remains the same (\emph{aei kata tauta}). Being divided around bodies, divisible Being is corporeal. And since the sensible is the mark of the corporeal, divisible beings are sensible. Indivisible beings, by contrast, are not divided around bodies, and so are not sensible and corporeal, but are, rather, intelligible.

How is this related to the psychogeny? Timaeus specifies the different kinds of objects of cognition in terms a difference in their Being. One kind enjoys indivisible Being, and the other suffers divisible Being. Being is mixed into the substance of the World-Soul. And though Timaeus does not make this explicit in his summary statement of the soul mixture at 37a, this involved mixing indivisible and divisible Being. That the substance of the World-Soul consists in a mixture of divisible and indivisible Being perhaps explains, at least in part, that the objects of its cognition are either divisible or indivisible beings. If so, perhaps Timaeus subscribes, as Aristotle (\emph{De anima} 1 2 404b) suggest, to some version of the principle that like is known by like. 

It will emerge that different cognitive powers are exercised on the different objects of cognition. The World-Soul knows indivisible beings and has true opinions about divisible beings. The stability of indivisible beings, that they always remain the same, make them appropriate objects of knowledge. Whereas the instability of divisible beings, belonging to the realm of Becoming, makes them at best the objects of true opinion.

% section the_objects_of_cognition (end)

\section{\emph{Ephatētai}} % (fold)
\label{sec:_emph_ephatetai}

The World-Soul is in ``contact'' (\emph{ephaptētai}) with the objects of cognition, be they divisible or indivisible, sensible or intelligible. As a result of this contact, the World-Soul is moved throughout its being and so announces what it is in contact with. How one understands this contact naturally constrains how one understands the motion of the World-Soul through its whole being by which it announces the object of its cognition. 

So far I have followed the bloodless convention of translating the verb \emph{ephaptētai} as a kind of contact. That verb, however, has a range of uses. The different senses conveyed by these may prove relevant. Let's begin by reviewing at least some of these.

In a range of uses the verb is used to convey \emph{binding}, somehow \emph{fixing} or \emph{holding fast} (Homer, \emph{Odyssey} 22 41). Observe that the emphasis is on the activity of binding rather than on the state of being bound. In another range of uses the verb has more tactile associations. It may be used to convey \emph{reaching} (Euripides, \emph{Helen} 556) or \emph{laying hands on} (Homer, \emph{Odyssey} 5 348). This second range of uses also emphasizes the activity of the subject of the verb. Reaching and laying hands on are activities. They are something done by embodied mortal beings embedded in an environment. They \emph{apply themselves} to that environment (Pindar, \emph{Olympian} 1 186). Moreover, reaching is an exploratory activity in a way that laying hands on may at least sometimes be. The senses associated with these different uses need not conflict. Indeed, they may combine. Thus, for example, \emph{grasping} involves reaching and is a way to lay hands on something, but it also binds what is in its grasp. The verb has gustatory uses that emphasize not binding so much as assimilation, as when one \emph{partakes} of food (Iamblichus, \emph{Vita Pythagorae} 3 17). The verb also has purely cognitive uses that designate a kind of cognitive \emph{apprehension} (Plato, \emph{Symposium} 212a). (For discussion of a similar semantic field associated with perceptual apprehension see \citealt[chapters 1--2]{Kalderon:2018oe})

\citet[]{Betegh:2019fq} has argued that the present occurrence of \emph{ephaptētai} must itself be understood in an active sense. The usual bloodless translation where the World-Soul is in contact with divisible and indivisible beings obscures the activity conveyed by the Greek. The World-Soul instead applies itself to divisible and indivisible beings. Reaching out and grasping is an apt metaphor for the World-Soul's cognitive apprehension of divisible and indivisible being. Cognitive apprehension is the World-Soul's activity, the way grasping is an activity. And, insofar as the object is apprehended, it is apt to think of this as a kind of binding, as when something is bound in one's grasp. And this remains true even when the binding could not be corporeal the way a grasping must be. 

Perhaps there are echoes of the other senses as well. The soul, in thinking, applies itself to the object of thought. Perhaps the soul in applying itself to divisible or indivisible being assimilates to these. In applying itself to the object of thought the soul partakes of what it thinks and becomes like it, in some suitable sense. This might be manifest in the ethical challenges that Timaeus understands the sensible to pose. But more compellingly, Timaeus' sensory soteriology presupposes something like the mortal soul' ability to assimilate to the object of its contemplation. The young gods, acting on the Demiurge's behest, providentially provide mortal beings with eyes to see with. In rationally attending to the harmony of celestial revolutions, the mortal being may align the circles in their soul with the circles in the World-Soul. Providentially provided sight is a means of salvation only insofar as it is a means by which the souls of mortal beings may assimilate to the soul of an immortal being and so become like their elder sister, insofar as that is possible.

To begin to understand what \emph{ephaptētai} means in this context, let us begin by considering an obvious hypothesis about a special case. Begin by considering the special case where the World-Soul applies itself to a divisible being and, hence, to something sensible and corporeal. Since the object of cognition is sensible and corporeal, it is natural to consider whether \emph{ephaptētai} might involve a kind of corporeal contact. The tactile imagery conveyed by some uses of the verb---reaching, laying hands on---may suggest this. Just as a body may set into motion another body by coming into contact with it, perhaps the sensible and corporeal object of cognition sets the World-Soul into motion in just the same way. When the World-Soul applies itself to its corporeal object, this causes it to move throughout its whole being. The World-Soul is said not merely to move, but to move throughout its whole being. How are we to understand this? The World-Soul, though incorporeal, is conceived as a sphere. Perhaps circular motion such as axial rotation would be a way for a sphere to move throughout its whole being.  




There are problems with understanding \emph{ephaptētai} as corporeal contact. Doing so requires the corporeal object to be solid and offer resistance, and the incorporeal soul to itself be solid in order to be moved by contact with the corporeal object. One problem is that solidity, on some understanding of that notion, is a condition on the possibility of touch (see chapter~\ref{sec:the_elemental_composition_of_the_corporeal}). If the solidity of the extended if incorporeal soul is suitably understood, then while the World-Soul may be invisible, it is tangible. When Timaeus claims that the World-Soul is invisible, does he allow for this? Or is invisible () a metonym for insensible, in which case the World-Soul is intangible. That Timaeus thinks visibility and tangibility to be opposed extremes of the sensible provides some support for the more general reading.

% But is it really plausible that the World-Soul may be felt if unseen? Or is the imputation of invisibility really the claim that it is insensible more generally? If the latter, then soul is intangible and so not solid in the way required to be the recipient of corporeal contact.


% To illustrate this connection, let us consider a literalist interpretation of contact. I do so not to endorse it, but because it is simple an vivid, and thus a good illustration of the connection between the conception of contact and the conception of the resulting motion. Indeed far from endorsing literalist interpretation, I believe that there are compelling grounds for its rejection. So the literalist interpretation will serve not only as an illustration of the connection between the conception of contact and the conception of the resulting motion but will also serve as a stalking horse for a better alternative

Beginning with our assumption that \emph{ephaptētai} involves corporeal contact, we were led to conceive of the resulting motion as the axial rotation of a sphere. This could only be so if the World-Soul were spatial. Only spatial things are spherical. Importantly, the World-Soul is also set into motion as a result of its contact with the divisible being, making the axial rotation subject to mechanical explanation. Doing so requires the sensible body to be solid and offer resistance, and the incorporeal soul to itself be solid in order to be moved by contact with the sensible body. One problem is that solidity, on some understanding of that notion, is a condition on the possibility of touch (see chapter~\ref{sec:the_elemental_composition_of_the_corporeal}). If the solidity of the extended if incorporeal soul is suitably understood, then while the World-Soul may be invisible, it is tangible. But is it really plausible that the World-Soul may be felt if unseen? Or is the imputation of invisibility really the claim that it is insensible more generally? If the latter, then soul is intangible and so not solid in the way required by the mechanical explanation of cognition.

Johansen is sensitive to the problem and is led thereby to deny that the World-Soul is solid. Johansen observes that the sensible is the mark of the corporeal and that soul is insensible. Since the World-Soul is insensible, it is intangible. And since it is intangible, the World-Soul lacks the features that would make it tangible. Consider now not static touch, passively resting one's hand on an independently supported corporeal surface, but reaching out and grasping a body. What would a body have to be like to be tangible in this way. It would have to have depth and solidity. Anything that lacks depth would elude our grasp, and in grasping something with depth, it offers resistance to our grasp. And by experiencing this resistance we can have a sense of the overall shape and volume of the body in our grasp. Marking the sensible--insensible contrast thus involves making different spatial attributions to bodies and souls. Only bodies have depth and solidity as required for their tangibility.

Part of the charm of the corporeal understanding \emph{ephaptētai} in the special case where the World-Soul applies itself to a divisible being is that it provides the resources for a mechanical explanation of the World-Soul's opinion about the sensible. In applying itself to divisible being, the World-Soul is in corporeal contact with its object and this causes it to engage in axial rotation. The mechanical explanation exploits the spatial character of the World-Soul. The World-Soul does occupy the space of the Cosmos---it is stretched throughout that space. But perhaps it does not occupy space in the same way as the body of the Cosmos. Something like the following thought might have motivated Johansen: If the World-Soul had depth an solidity, it would exclude the body of the Cosmos from occupying the same space. However, if the World-Soul lacks depth and solidity, it may spatially overlap the body of the Cosmos. Johansen is right to insist that whereas the corporeal is tangible, the World-Soul is intangible. What Johansen failed to see was that the World-Soul must meet the requirements on tangibility if the mechanical explanation of the soul's movement succeeds on its own terms. In order to move the soul, the divisible being must resist the soul's application to it. But such resistance is only possible if the World-Soul is itself solid. It is difficult to understand how the mechanical explanation of the World-Soul's opining could be consistent the sensible being the mark of the corporeal, the conditions on tangibility, and the World-Soul being insensible.

% Understanding \emph{ephaptētai} as the World-Soul applying itself to a divisible being so as to be in corporeal contact with it, not only led us to conceive of the World-Soul as engaged in axial rotation, but also that the World-Soul was set into motion as a result of the contact that it sought. When one body moves another, both are solid and source of mutual resistance. But the World-Soul is not solid. Being insensible, it is intangible. And being intangible, it lacks depth and solidity. But if the World-Soul lacks solidity then in applying itself to a divisible being, how can it encounter sufficient resistance to be moved throughout the whole of its being?

 % That turned out to be inconsistent with the mechanical explanation of the motion of the World-Soul. But perhaps there are other ways to understand this general idea. Perhaps the World-Soul is wholly present to each part of space in a way that contrasts with the way a body occupies space. A body occupies space by being partly here and partly there. It is not wholly present to every region of space that it occupies. Perhaps the World-Soul, while being spatial, is not extended, if that implies that it is divided around bodies.

There are other difficulties with the mechanical explanation. Following \citet{Betegh:2019fq} we have retained an active sense for \emph{ephaptētai}. The present interpretation may have the World-Soul in corporeal contact with divisible being, but only by the World-Soul seeking out such contact. Still, the World-Soul's applying itself to the divisible being merely makes it a passive recipient of imparted motion. Even if the World-Soul only has the power to receive such motion if it actively applies itself to the sensible and the corporeal, there are elements of the passage that suggests that the World-Soul is the initiator of that movement and not merely its recipient. On the present interpretation, the opinion, the soul's announcement concerning the divisible being that it applies itself to, is the motion throughout the whole being of the World-Soul that results from that contact. But later, Timaeus will speak of the announcement as being born through the self-moved. This suggests that the World-Soul, in applying itself to a divisible being, moves itself and is not itself moved in reaction to the encounter. If the World-Soul moves itself and is not moved by its encounter with a divisible being, then its movement is not subject to mechanical explanation. 

Even if this interpretation, or some variant of it, succeeds perfectly on its own terms, its ambitions are limited. It only considers a special case, the World-Soul's applying itself to a divisible being. The World-Soul also applies itself to indivisible beings, intelligible beings such as the Forms. There is no question of conceiving the World-Soul's applying itself to the Forms as corporeal contact. Indivisible beings are not only incorporeal but inextended as well. And since only extended things may be solid, indivisible beings are not solid and so offer no resistance in corporeal contact. Either contact means different things when the World-Soul is said to be in contact with divisible Being and when it is said to be in contact with indivisible Being, or contact means something sufficiently general to cover both cases. Put another way, does \emph{ephaptētai} admit of homonymous or non-homonymous reading?

Consider a homonymous reading of \emph{ephaptētai}. When applied to divisible being it denotes a kind of corporeal contact. When applied to indivisible being it denotes a kind of non-corporeal contact. One problem with the homonymous interpretations of \emph{ephaptētai} is the grammatical context of its occurrence. It occurs in a context where we are being asked to consider the World-Soul applying itself to either a divisible or indivisible being. The homonymous reading should rule out such constructions. Either reading of \emph{ephaptētai}, as involving corporeal contact or not, would rule out the intelligible occurrence of one of its objects  (compare the standard linguistic tests for lexical ambiguity, \citealt{Zwicky:1975hl}). If \emph{ephaptētai} is interpreted as involving corporeal contact, while taking a divisible being as its objects is intelligible, taking an indivisible being as its object is not. And if \emph{ephaptētai} is interpreted as involving non-corporeal contact, while taking an indivisible being as its object is intelligible, taking a divisible being as its object is not. Such a construction could only be intelligible on a non-homonymous reading of \emph{ephaptētai}. The verb must mean something sufficiently general to intelligibly apply to both divisible and indivisible beings.

The homonymous reading of \emph{ephaptētai} is, if not incoherent, then inconsistent with the text. As a consequence, one should not understand the World-Soul's applying itself to divisible being as corporeal contact. Whatever the verb means, it must mean something sufficiently general to apply to sensible and intelligible objects, and it does not on the corporeal interpretation. Of its range of uses perhaps only its cognitive uses are sufficiently general (for example, Plato, \emph{Symposium} 212a). One obstacle to this suggestion is that the most straightforward understanding of it as false. So consider those uses where \emph{ephaptētai} means something like cognitive apprehension. The activity conveyed by \emph{ephaptētai} may be a part of the apprehension of a sensible object in opinion but it is not the whole of it. Opining involves not only the World-Soul applying itself to a sensible object, but moving itself throughout the whole of its being in response. So if \emph{ephaptētai} receives a cognitive reading, it must be less than the whole cognitive act effected, opining as applied to sensible objects, and knowing as applied to intelligible objects. Perhaps \emph{ephaptētai} might be understood as analogous to attentive contact. Attending to a sensible object is not to opine about it. Nevertheless, in attending to a sensible object, the World-Soul moves itself throughout its whole being thus announcing the character of the sensible object attended to. Similar remarks apply to the intelligible case. Attending to intelligible objects is not yet to know them. Perhaps they must be brought into relation with other intelligible objects over the course of dialectic before they are truly known. And yet one only comes to know the intelligible by attending to it. In attending to an intelligible object, the World-Soul moves itself throughout its whole being thus announcing the character of the intelligible object attended to. 


% section _emph_ephatetai (end)

\section{Cognition} % (fold)
\label{sec:cognition}

The World-Soul applies itself to divisible or indivisible beings. On the cognitive reading of \emph{ephaptētai}, this is a kind of pre-cognitive attentive contact. In attending to the divisible or indivisible beings the World-Soul moves throughout the whole of its being and announces the character of the object attended to. A question immediately arises. Is the movement throughout the whole of its being the same as the announcement? Or is the movement merely a phase in the causal process eventuating in cognition?

Identifying the movement with the announcement is in one way parsimonious. Timaeus does not have to provide a further account of what this announcement is. Though some explanation is owed as to how this motion is an announcement. Indeed this puts an important constraint on any such identification. The movement must be understood in such a way that it is intelligibly an announcement. If the movement could not be understood in such a way as to intelligibly be an announcement, that would be good reason to regard the movement as merely a phase in the causal process eventuating in cognition.

I have spoken blandly of the character of the attended object being announced. Talk of character is misleading in its generality given that Timaeus has provided us something like a proto-categories. What is announced may be the character of the object attended to, but that character consists in what it is the same as or different from, in what relation, where, when, and how each thing exists and is acted upon


% section cognition (end)

\section{Knowledge and Opinion} % (fold)
\label{sec:knowledge_and_opinion}



% section knowledge_and_opinion (end)

\section{Concluding Observations} % (fold)
\label{sec:concluding_observations_cr}



% section concluding_observations (end)

% chapter cognitive_revolution (end)