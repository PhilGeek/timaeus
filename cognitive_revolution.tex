%!TEX root = /Users/markelikalderon/Documents/Git/timaeus/timaeus.tex

\chapter{Cognitive Revolution} % (fold)
\label{cha:cognitive_revolution}

\section{Psychogeny and Cognitive Power} % (fold)
\label{sec:psychogeny_and_psychic_power}

The psychogeny, understood strictly as Timaeus' account of the generation of the World-Soul, was the topic of the last chapter. Timaeus' account of the generation of the World-Soul is important since the nature of its substance, its proportional division, and its being formed into the Circles of the Same and the Different determine its kinetic and cognitive powers. Our understanding of these powers is incomplete unless they are understood to be determined, at least in part, by the substance, proportional division, and structure that the World-Soul has thanks to Demiurgic activity. The details of the Demiurge's manufacture and construction of the World-Soul thus go some way to explaining the powers that He invests it with. Timaeus makes this connection explicit with respect to its cognitive powers. The World-Soul has its cognitive powers inasmuch as it is a mixture of Being, Sameness, and Difference, proportionally divided, and revolves around itself (37a). The explanation of the exercise of the World-Soul's cognitive powers would thus involve its substance, proportional division, and structure.

The World-Soul is elder sister to the souls of mortal beings. The Demiurge is the maker and Father of the World-Soul as well as the souls of mortal beings. The souls of mortal beings are generated after the generation of the World-Soul. The souls of mortal beings are mixed from the same material, though impurely, that was used to mix the substance of the World-Soul. Moreover, the Demiurge invests the souls of mortal beings with the same proportional divisions and structure as the World-Soul. The World-Soul is prior not only in birth but in dignity. It is more perfect than the souls of mortal beings as signaled by the impurity of their substance. Being perfect it is an exemplar for the souls of mortal beings. Thus like an elder sister, the souls of mortal beings look to her for guidance. Literally. The cognitive powers of the World-Soul are visibly manifest in the harmony of celestial motion. And by rationally attending to this motion in the right sort of way the circles in the souls of mortal beings are aligned to the circles in the World-Soul making them more rational and virtuous. In this way should we strive to be like our elder sister.

The way that the psychogeny explains the exercise of the cognitive powers of the World-Soul is relevant to the psychology of mortal beings. Given the relationship between the World-Soul and the souls of mortal beings, the explanation of the exercise of the cognitive powers of the souls of mortal beings should parallel the explanation of the exercise of the cognitive powers of the World-Soul. 

This chapter will discuss the cognitive powers of the World-Soul as Timaeus presents them to be at 37a--c. The passage largely consists in two long and syntactically complex sentences that have been variously interpreted. There Timaeus describes:
\begin{enumerate}[(1)]
	\item The objects of these cognitive powers, divisible and indivisible beings
	\item The World-Soul's ``contact'' (\emph{ephaptētai}) with its objects
	\item The thought that ``contact'' elicits
	\item The two kinds of cognitive powers that the World-Soul enjoys, the power to know the indivisible and the power to have true opinion about the divisible.
\end{enumerate}
Let us consider these in turn.

% section psychogeny_and_psychic_power (end)

\section{The Objects of Cognition} % (fold)
\label{sec:the_objects_of_cognition}

The objects of cognition have either divisible or indivisible Being. Divisible Being is associated with Becoming (\emph{gignomenēs}) and is divided around bodies (\emph{peri ta somata} \ldots\ \emph{meristēs}), whereas indivisible Being always remains the same (\emph{aei kata tauta}). Being divided around bodies, divisible Being is corporeal. And since the sensible is the mark of the corporeal, divisible beings are sensible. Indivisible beings, by contrast, are not divided around bodies, and so are not sensible and corporeal, but are, rather, intelligible.

How is this related to the psychogeny? Timaeus specifies the different kinds of objects of cognition in terms a difference in their Being. One kind enjoys indivisible Being, and the other suffers divisible Being. Being is mixed into the substance of the World-Soul. And though Timaeus does not make this explicit in his summary statement of the soul mixture at 37a, this involved mixing indivisible and divisible Being. That the substance of the World-Soul consists in a mixture of divisible and indivisible Being perhaps explains, at least in part, that the objects of its cognition are either divisible or indivisible beings. If so, perhaps Timaeus subscribes, as Aristotle (\emph{De anima} 1 2 404b) suggest, to some version of the principle that like is known by like. 

It will emerge that different cognitive powers are exercised on the different objects of cognition. The World-Soul knows indivisible beings and has true opinions about divisible beings. The stability of indivisible beings, that they always remain the same, make them appropriate objects of knowledge. Whereas the instability of divisible beings, belonging to the realm of Becoming, makes them at best the objects of true opinion.

% section the_objects_of_cognition (end)

\section{\emph{Ephatētai}} % (fold)
\label{sec:_emph_ephatetai}

The World-Soul is in ``contact'' (\emph{ephaptētai}) with the objects of cognition, be they divisible or indivisible, sensible or intelligible. As a result of this contact, the World-Soul is moved throughout its being and so announces what it is in contact with. How one understands this contact naturally constrains how one understands the motion of the World-Soul through its whole being by which it announces the object of its cognition. 

So far I have followed the bloodless convention of translating the verb \emph{ephaptētai} as a kind of contact. That verb, however, has a range of uses. The different senses conveyed by these may prove relevant. Let's begin by reviewing these.

In a range of uses the verb is used to convey \emph{binding}, somehow \emph{fixing} or \emph{holding fast} (Homer, \emph{Odyssey} 22 41). Observe that the emphasis is on the activity of binding rather than on the state of being bound. In another range of uses the verb has more tactile associations. It may be used to convey \emph{reaching} (Euripides, \emph{Helen} 556) or \emph{laying hands on} (Homer, \emph{Odyssey} 5 348). This second range of uses also emphasizes the activity of the subject of the verb. Reaching and laying hands on are activities. They are something done by embodied mortal beings embedded in an environment. They \emph{apply themselves} to that environment (Pindar, \emph{Olympian} 1 186). Moreover, reaching is an exploratory activity in a way that laying hands on may at least sometimes be. The senses associated with these different uses need not conflict. Indeed, they may combine. Thus, for example, \emph{grasping} involves reaching and is a way to lay hands on something, but it also binds what is in its grasp. The verb has gustatory uses that emphasize not binding so much as assimilation, as when one \emph{partakes} of food (Iamblichus, \emph{Vita Pythagorae} 3 17). The verb also has purely cognitive uses that designate a kind of cognitive \emph{apprehension} (Plato, \emph{Symposium} 212a).

\citet[]{Betegh:2019fq} has argued that the present occurrence of \emph{ephaptētai} should itself be understood in an active sense. The usual bloodless translation where the World-Soul is in contact with divisible and indivisible beings obscures the activity conveyed by the Greek. The World-Soul instead applies itself to divisible and indivisible beings. Reaching out and grasping is an apt metaphor for the World-Soul's cognitive apprehension of divisible and indivisible being. Cognitive apprehension is the World-Soul's activity, and insofar as the object is apprehended it is apt to think of this as a kind of binding, as when something is bound in one's grasp. And this remains true when the binding could not be corporeal the way a grasping must. Perhaps there are echoes of the other senses as well. The soul, in thinking, applies itself to the object of thought. Perhaps the soul in applying itself to divisible or indivisible being assimilates to these. In applying itself to the object of thought the soul partakes of what it thinks and becomes like it, in some sense. This might be manifest in the ethical challenges that Timaeus understands the sensible to pose. But more compellingly, Timaeus' sensory soteriology presupposes something like the mortal soul' ability to assimilate to the object of its contemplation. The young gods, acting on the Demiurge's behest, providentially provide mortal beings with eyes to see with. In rationally attending to the harmony of celestial revolutions, the mortal being may align the circles in their soul with the circles in the World-Soul. Providentially provided sight is a means of salvation only insofar as it is a means by which the souls of mortal beings may assimilate to the soul of an immortal being and so become like their elder sister, insofar as that is possible.

To begin to understand what \emph{ephaptētai} means in this context, let us begin by considering an obvious hypothesis of a special case. Begin by considering the World-Soul applying itself to a divisible being and, hence, to something sensible and corporeal. Since the object of cognition is sensible and corporeal, it is natural to consider whether \emph{ephaptētai} might involve a kind of corporeal contact. Just as a body may set into motion another body by coming into contact with it, perhaps the sensible and corporeal object of cognition sets the World-Soul into motion in just the same way. When the World-Soul applies itself to its corporeal object, it meets resistance which causes it to move throughout its whole being. 

% To illustrate this connection, let us consider a literalist interpretation of contact. I do so not to endorse it, but because it is simple an vivid, and thus a good illustration of the connection between the conception of contact and the conception of the resulting motion. Indeed far from endorsing literalist interpretation, I believe that there are compelling grounds for its rejection. So the literalist interpretation will serve not only as an illustration of the connection between the conception of contact and the conception of the resulting motion but will also serve as a stalking horse for a better alternative


In order for a body to be the agent of another body's motion, it must be solid and so offer resistance to the body that it comes into contact with. If when encountering such resistance the World-Soul is set into motion, must not the Wold-Soul itself be not only extended but solid as well? If we model the World-Soul's cognitive contact on corporeal contact there is reason to conceive of the resulting motion of the World-Soul as the motion of an extended, solid incorporeal being. The World-Soul is said not merely to move, but to move throughout its whole being. How are we to understand this? The World-Soul, though incorporeal, is conceived as a solid voluminous sphere. Perhaps circular motion such as axial rotation would be a way for a solid voluminous sphere to move throughout its whole being.  

How one conceives of cognitive contact has consequences for how one conceives of the World-Soul being moved throughout its whole being. Thus by conceiving of cognitive contact as corporeal contact, we were led to conceive of the resulting motion as the axial rotation of an incorporeal solid voluminous sphere.

Even if this interpretation, or some variant of it, succeeds perfectly on its own terms, its ambitions were limited. It only considered one case of cognitive contact, the World-Soul's contact with a divisible being. The World-Soul also comes into contact with indivisible beings, intelligible beings such as the Forms. There is no question of conceiving the World-Soul's cognitive contact with the Forms as corporeal contact. Indivisible beings are not only incorporeal but inextended as well. And since only extended things may be solid, indivisible beings are not solid and so offer no resistance in corporeal contact. Either contact means different things when the World-Soul is said to be in contact with divisible Being and when it is said to be in contact with indivisible Being. Or contact means something sufficiently general to cover both cases. Put another way, does \emph{ephaptētai} admit of homonymous or non-homonymous reading?

\citet{Betegh:2019fq} offers a homonymous reading of \emph{ephaptētai}. When applied to divisible being it denotes a kind of corporeal contact. When applied to indivisible being it denotes a kind of non-corporeal contact. One problem with the homonymous interpretations of \emph{ephaptētai} is the grammatical context of its occurrence. It occurs in a context where we are being asked to consider the World-Soul being in contact with a divisible or indivisible being. On the homonymous reading, a single occurrence of \emph{ephaptētai} would receive different meanings to coordinate with its divisible and indivisible objects. But how could this be? Surely the homonymous reading should rule out such constructions. That is to say that surely a single occurrence of \emph{ephaptētai} should receive a single reading which would rule out the intelligibility of one of its objects (compare the standard linguistic tests for lexical ambiguity, \citealt{Zwicky:1975hl}).

If a homonymous reading of \emph{ephaptētai} is, if not incoherent, then inconsistent with the text, then one should not understand the World-Soul's contact with the divisible Being as corporeal contact. There are, at any rate, independent problems with understanding \emph{ephaptētai} as corporeal contact. Doing so requires the corporeal agent to be solid and offer resistance, and the incorporeal soul to itself be solid in order to be moved by contact with the corporeal agent. One problem is that solidity, on some understanding of that notion, is a condition on the possibility of touch (see chapter~\ref{sec:the_elemental_composition_of_the_corporeal}). If the solidity of the extended if incorporeal soul is suitably understood, then while the World-Soul may be invisible, it is tangible. But is it really plausible that the World-Soul may be felt if unseen? Or is the imputation of invisibility really the claim that it is insensible more generally? If the latter, then soul is intangible and so not solid in the way required to be the recipient of corporeal contact.

% section _emph_ephatetai (end)

\section{Cognition} % (fold)
\label{sec:cognition}



% section cognition (end)

\section{Knowledge and Opinion} % (fold)
\label{sec:knowledge_and_opinion}



% section knowledge_and_opinion (end)

\section{Concluding Observations} % (fold)
\label{sec:concluding_observations_cr}



% section concluding_observations (end)

% chapter cognitive_revolution (end)