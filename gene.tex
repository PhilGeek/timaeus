%!TEX root = /Users/markelikalderon/Documents/Git/timaeus/timaeus.tex

\chapter{\emph{Genē}} % (fold)
\label{cha:gene}

\section{A New Start} % (fold)
\label{sec:a_new_start}

If the first part of Timaeus' speech concerned the operation of Reason (27d--47e), the second part of Timaeus' speech concerns, instead, the operation of Necessity (47e--69a). Timaeus' insight is that Cosmos is generated as the joint product of Reason and Necessity. In his New Start, Timaeus articulates the workings of Necessity. Timaeus distinguishes two phases in the workings of Necessity. In the pre-cosmic chaos, prior to the Demiurge's imposition of form and number, Necessity was a wandering cause. In imposing form and number, however, the Demiurge persuades Necessity to act for the best.

% [Why persuasion? Hoackforth suggests that persuasion marks a contrast with violence. On Hackforth's understadning, Reason is not forcing Necessity to act for the best, but nevertheless acheives this reult through discursive persuasion. I wonder, however, if this gets the emphasis of Timaeus' talk of persuasion right. Later in the same section, Timaeus will contrast understanding and opinion. Opinion is based upon perception and so concerns the sensible realm of Becoming. Opinion is subject to persuasion and this marks a contrast with understanding. Understanding may be achieved through a reasoned account but it is not subject to persuasion.]

In describing the works of Reason in the first part, Timaeus distinguished the intelligible model, the Paradigm, and its sensible copy, the Cosmos. Now, in describing the works of Necessity, Timaeus introduces a third kind, the Receptacle. The Receptacle is baffling and obscure, perhaps more obscure than the difficult to discover deity, impossible to explain to all. The Receptacle is that in which the Paradigm is imaged. Reflection on elemental transformation reveals the existence of the Receptacle, as the invariant place in which the sensible fleetingly appears and so as the only invariant thing in the process of continual change that so much as could be the object of demonstration. We also learn that the Receptacle is receptive to all and that this requires that it be without form. And while the Receptacle is the place in which the sensible appears, it is not itself sensible and so is invisible like the intelligible Paradigm.

In persuading the wandering cause to act for the best, the Demiurge imposes form and number upon the chaotic contents of the Receptacle. These consist in sensible powers that are traces marvellously stamped by the Forms of fire, air, water, and earth. Timaeus' discussion of the Receptacle thus has an important lesson about the nature of the sensible powers. There are no perceivers in the pre-cosmic chaos, and yet it consists in a disorderly mass of trace sensible powers. So sensible powers can exit, at least after a fashion, in the absence of perceivers. 

The discussion of the Receptacle harbors another important lesson about the sensible. We have seen how, for Timaeus, the sensible is the mark of the corporeal. The discussion of the generation of the primary bodies from the chaotic contents of the Receptacle deepens and elaborates this theme. The trace sensible powers in the pre-cosmic chaos are unlike and unbalanced. They thus move in a disorderly fashion. Order is established only by the Demiurgic imposition of form and number upon these trace sensible powers. Specifically, the Demiurge imposes regular polyhedra of a kind that both captures the character of these trace sensible powers and extends that character sufficiently so that they are enough like the Forms to be called by their names. Thus fire is formed from the fiery, the trace sensible power marvellously stamped in the Receptacle by the Form of fire, by imposing the form of a tetrahedron upon it, given how it captures the phenomenal aspects of the fiery and extends them so that it may be worthy to be named after the Form of fire. There is no potential, here, for the construction of the corporeal from regular polyhedra formed from elemental triangles to undermine the sensible character of the corporeal.

% section a_new_start (end)

\section{The Cosmological Project} % (fold)
\label{sec:the_cosmological_project}

Cosmology, at least in antiquity, is comprehensive, taking as its object the All. Cosmology aims to account for the whole of it, at least in principle. Given the sensible diversity of the All, to account for the whole of it would be it to account for the entire range of sensible phenomena. There is a challenge, here, since cosmologies tend to operate with a limited number of principles. So to establish the adequacy of a cosmology would be to show how our experience of the entire range of sensible phenomena can be explained in terms of the limited number of principles postulated by that cosmology.

It is easy to see how this challenge can degenerate into conflict as a result of impotence. If the principles cannot explain our experience of the entire range of sensible phenomena, and are to that extent impotent, then a conflict arises between the Scientific Image of Nature, encapsulated by the principles of that cosmology, and the Manifest Image of Nature, the sensible phenomena as we experience them to be but the principles fail to explain. A cosmologist facing such a conflict, should they be more confident in their principles than the alleged phenomena that they fail to explain, may attempt to defend their claim to comprehensiveness by dismissing the recalcitrant sensible phenomena as illusory or otherwise misleading.

Timaeus aims to meet the challenge, at least on the whole. If successful, on his own terms, then the challenge, being met, does not degenerate into conflict. It is true that Timaeus is willing to criticize common ways of talking. For example, he thinks that are temporal language is confused. Moreover, as we shall see, Timaeus is willing to criticize common conceptions of sensible phenomena. Thus, for example, while the Empedoclean ``roots'' appear to transform into one another, this is misleading. While three of the ``roots'' may transform into one another, a fourth does not. Nevertheless, while the cycle of elemental transformation is a misleading approximation of the truth, Timaeus' account explains why it should appear as if it were so. So, on the whole, Timaeus' attitude toward sensible phenomena is broadly ecumenical and conciliationist. It is a qualified endorsement of the Manifest Image of Nature.

Timaeus thus aims to explain the sensible diversity of the All. This is strikingly exemplified in his account of the \emph{genē}. There are four primary \emph{genē}---fire, air, water, and earth. And there is variability in kind within the primary \emph{genē}. Thus, as we have seen, there are different kinds of fire---flame, a mild light that does not burn, and that which is retained in embers. Every sensible phenomena that we experience is an aggregate of these variable kinds of \emph{genē}. Timaeus posits two kinds of elemental triangles from which he derives five regular polyhedra. The simplest four of these are shapes imposed upon powers in the Receptacle by the Demiurge to constitute the primary bodies. The variable kinds within the primary \emph{genē} are explained in terms of variable sizes of triangles that compose them. So given two kinds of elemental triangles, the Demiurge imposes form and number upon the pre-cosmic chaos, to give rise to primary and secondary bodies. Timaeus' account will only be comprehensive if the the sensible diversity of the secondary bodies (the primary bodies being too small to be visible) matches the sensible diversity of the world as we experience it to be.

In explaining the sensible diversity that we experience in terms of elemental triangles that construct regular polyhedra and the way that these aggregate, Timaeus' cosmology invites comparison to the atomism of Leucippus and Democritus. Upon reflection, the similarities are superficial and the differences striking. That there should be differences is perhaps unsurprising. Diogenes Laertius reports Aristoexenus' claim that Plato wished to burn all of the writings of Democritus (\emph{Vitae Philosophorum} 9.40). Even if apocryphal, the competitive attitude expressed by the anecdote may have been genuine. Though Plato never directly discusses Democritus, his student Aristotle does extensively, and we can be sure that Plato was familiar with the writings that Aristoexenus purports he wished to burn. One difference is that the atomists do not understand the sensible Cosmos to be governed by \emph{nous}, and so for the best, the way Timaeus does and Socrates despaired of (\emph{Phaedo} 96a6–99d2). Thus the Demiurge, associated with \emph{nous}, makes no appearance, even under a distinct avatar, in an atomist cosmology. Atoms exist within the void, but Timaues maintains that there is neither void within the Cosmos nor without. The Cosmos, is, instead, a qualified plenum. Moreover, the atomists posited nothing like the Receptacle, that in which the Paradigm is imaged, a truly Timaean innovation. Importantly, for our purposes, there is also a difference in their attitudes toward the sensible. While Timaeus' attitude toward the sensible is broadly ecumenical and conciliationist, the atomists' attitude is not (for a comparison of Platonic and atomist epistemologies see \citealt{Lee:2005qr}). Thus consider Sextus Empiricus' dramatic report of Democritus' attitude toward the sensible:
\begin{quote}
	And Democritus in some places abolishes things that appear to the senses and asserts that none of them appears in truth but only in opinion, the true fact in things existent being the existence of atoms in the void; for ``By convention,'' he says, ``is sweet, by convention bitter, by convention hot, by convention cold, by convention color; but by verity atoms and void.'' (Sextus Empiricus, \emph{Adversus Mathematicos} 7; \citealt{Bury:1997uq})
\end{quote}
Clearly, abolishing things that appear to the senses is not an ecumenical and conciliationist attitude toward the sensible, even broadly construed.

The apparent cycle of elemental transformation raises a puzzle or \emph{aporia} whose resolution requires the postulation of the Receptacle. We shall begin by discussing this \emph{aporia} and then discuss the construction of the regular polyhedra from elemental triangles and their assignment to the four primary \emph{genē}, and the sensibly diverse secondary bodies that these give rise to. 

% section the_cosmological_project (end)

\section{An \emph{Aporia}} % (fold)
\label{sec:an_emph_aporia}

The \emph{aporia} begins with the phenomena of flux. The relevant notion of flux, however, is restricted to the cycle of elemental transformation, the tendency for the four primary bodies---the Empedoclean ``roots'', fire, air, water, and earth---to transform into one another. The relevant example of flux, the cycle of elemental transformation, is of pre-Socratic provenance. It can be found in Anaximenes (Hippolytus, \emph{Refutatio omnium haeresium} 1.7.1--3 = DK 13A7), Heraclitus (DK 22B31, 22B36), Melissus (Simplicius, \emph{In Aristotelis De caelo commentaria} 558.19--559.12 = DK 30B8), and Anaxagoras (Simplicius, \emph{In physica} 179.8 = DK 59B16). The puzzle concerns the possibility of naming or being the object of demonstration. Specifically, if the Empedoclean ``roots'' are continually transforming into one another, then they fail to provide a sufficiently stable object for deictic reference. Concerning any primary body, it is not possible to refer to it as ``this'' (\emph{touto}) if it is continually transforming into another primary body. At best, we must describe it as ``such-like'' (\emph{toiouto}). The primary bodies are less substances than recurring kinds (49e5). The primary bodies are thus not enduring objects that can be identified and re-identified but are rather recurring phases in the cycle of qualitative flux.

Within the Platonic corpus, the puzzle has antecedents in the \emph{Cratylus} (439d--e) and the \emph{Theaetetus} (181c--183b). The \emph{Cratylus} and the \emph{Theaetetus} each has as its target the very coherence of the doctrine of total flux---that everything is changing in every respect at all times. This is a strong claim, and its rejection is consistent with Becoming being reconceived as partial flux, arguably one of the tasks of the \emph{Timaeus}. It is useful to compare, and importantly to contrast, the present puzzle with these earlier treatments. 

In the \emph{Cratylus}, Socrates presents three arguments against the doctrine of total flux. Specifically, that doctrine is said to raise linguistic, ontological, and epistemological problems. Toward the end of that dialogue (439c), Socrates shares a dream that he often has of the Forms---Beauty in itself, Goodness in itself, and each of the things that are. Whereas a human face does not always retain its beauty---whether through the ravages of time, disease, or some tragic accident---true beauty is invariably beautiful. The Form of Beauty is itself invariably beautiful. If it were not, it would not be possible to say of it that it is a ``this'' or that it is ``such-like'' (439d). Thus, if the Form of Beauty were not invariably beautiful, then having discovered true beauty, one could not say of it ``This is beautiful''.  Moreover, echoing Parmenides, Socrates maintains that that anything always changing could not properly be said to be (439e). Finally, Socrates applies to the Forms the Eleatic doctrine that knowledge requires stable objects. Nothing variable can be known by anyone, for as one approaches it as a matter of inquiry, it changes from what it was (440a-e). Thus in inquiring into true beauty only the Form of Beauty could be known because of all the beautiful things only that Form is invariable and so has the requisite stability to be a potential object of knowledge. Three observations are already pertinent. First, in the \emph{Cratylus}, total flux, and not partial flux, is the object of criticism. Second, total flux is said to be inconsistent with, not only naming, but predication as well. Third, that doctrine not only raises linguistic difficulties, but ontological and epistemological difficulties as well.

The linguistic difficulty is raised again in the \emph{Theaetetus}. Again, the issue is how to coherently describe something in the process of continual change in every respect. And again, Plato emphasizes that this process continues even as we speak in attempting to describe it (\emph{Cratylus} 439d10, \emph{Theaetetus} 182d7). There are, however, differences. The ontological problem is not raised in the \emph{Theaetetus}. While, in that dialogue, Eleatic monism is opposed to Ionian Heracliteanism, no Eleatic assumptions about the nature of Being are directly leveraged against the doctrine of total flux. The epistemological concern of the \emph{Cratylus}, that something continually changing in every respect could not be the object of knowledge, is not explicitly raised in the \emph{Theaetetus}. This is reasonable, however, since, in that dialogue, what counts as knowledge is precisely what is at issue. 

% However, one may wonder whether this specific epistemological concern, while not made explicit in the \emph{Theaetetus}, is, nonetheless, implicit in the overall dialectical structure of that dialogue. Notoriously, \emph{Parmenides} raises six \emph{aporiai} concerning the Forms, and the \emph{Theaetetus} makes no mention of the Forms. Perhaps the \emph{Theaetetus} may be read as beginning with the assumption that there are no Forms and drawing out the conclusion that no adequate definition of knowledge is to be had. This, at the very least, approaches Plato's Eleatic conviction that knowledge requires stable objects along with his further insistence that only the Forms have the requisite stability. \citet{Cornford:1951ei} and \citet{Kahn:2013ob} offer interpretations of this kind. This issue, however, is beyond the scope of the present essay.

In the \emph{Theaetetus}, the doctrine of total flux is expressed as everything being in motion. Socrates begins by clarifying that motion, here, cannot mean narrowly spatial motion, locomotion, but must mean change of any kind including growth and alteration (181d). Socrates goes on to emphasize this by coining a new term \emph{poiotēs}, literally what-sort-ness, for the quality that is changed in alteration (182a). (The English word ``quality'' derives from Cicero's Latin translation of Plato's Greek: ``\emph{Qualitates igitur appellavi, quas Graeci} {\sbl ποιότητας} \emph{appellant, quod ipsum apud Graecos non est vulgi verbum, sed philosophorum; atque id in multis. Dialecticorum vero verba nulla sunt publica, suis, utuntur. Et id quidem commune omnium fere est artium; aut enim nova sunt rerum novarum facienda nomina aut ex aliis transferenda}'' \emph{Academicae Quaestiones} 1.25.) If everything is in flux, then the white that we see is passing over into another color even as we see it. And this raises the difficulty of how we can call it ``white''. For even as we utter that description the color is changing (182d). The difficulty generalizes. If everything is in flux, so are perceptions such as seeing and hearing. And if language requires a stable object, then nothing may be described as seeing or not seeing, or perceiving or not perceiving more generally (182e). Like in the \emph{Cratylus}, the difficulty seems general, applying equally to naming and predication.

The \emph{Cratylus} and the \emph{Theaetetus} reject the doctrine of total flux as incoherent. But Plato does not, in general, restrict Being to what is invariable. For example, in the \emph{Philebus} 27b8, Plato seems to extend the notion of Being to what comes to be. So it would seem that while Plato rejects the doctrine of total flux as incoherent, he is amenable to a world of partial flux (on Plato's persistent attachment to flux see Aristotle \emph{Metaphysica} A 6). However, the \emph{Cratylus} and the \emph{Theaetetus}, while clear about the incoherence of the doctrine of total flux, remain silent about any successor notion. The new metaphysical scheme of the \emph{Timaeus}, involving, the Paradigm, its image, and the Receptacle in which that image appears, is meant, in part, to provide the resources for a coherent successor notion of partial flux.

Another difference concerns the scope of the linguistic difficulty. In the \emph{Timaeus} the cycle of elemental transformations is meant to raise a difficulty about naming the primary bodies, fire, air, water, and earth. No explicit difficulty about predication is raised. This seems \emph{prima facie} odd since the relevant linguistic difficulty seems perfectly general. That is to say, if constant change suffices to make naming impossible parallel reasoning would apply equally as well to the case of predication. However, it is worth noting that Timaeus, in his discussion of the gold (50a--c), strikes a cautious note even about describing what comes to be as ``such-like''. We should be content that, here, we can even call such things ``such-like'' with any safety. A question naturally arises: Why is there sufficient safety, here, even to call such things ``such-like''?

% To protest against Timaeus' linguistic recommendations, then, would be insenstive to the danger of the difficulty generalizing as it does in the \emph{Cratylus} and the \emph{Theaetetus}.

Timaeus does not offer an answer to our question. However, his account has the resources to provide an answer, though it is difficult to determine whether it would be the answer that Timaeus would give were he to explicitly consider our question. Begin with the observation that the cycle of elemental transformation concerns the primary bodies. Fire as apposed to fleeting fiery traces in the pre-cosmic chaos. Later we shall learn that the primary bodies are the result of the Demiurge imposing form and number on chaotic contents of the Receptacle. Perhaps the sensible traces in the pre-cosmic chaos are too fleeting even for them to be called ``such-like'' with any safety. It is only once the Demiurge imposes form and number that the primary bodies are generated and have sufficient stability to be called ``such-like'', at least with due caution. Within the Cosmos, primary bodies, despite their variable nature, may be called ``such-like'' with any safety thanks to the Demiurgic imposition of form and number.

A further difference also concerns the scope of the linguistic difficulty, though along another dimension. Unlike in the \emph{Cratylus} and the \emph{Theaetetus}, Timaeus argues not that ``this'' lacks deictic reference, only that it cannot refer to any of the primary bodies. If ``this'' cannot refer to any of the primary bodies, and it successfully refers, then it must refer to something other than the primary bodies. What then does ``this'' refer to? According to Timaeus, if we say ``This is fiery'', ``this'' may successfully refer to the place in which that primary body appears. Images of the Forms may come and go in the Receptacle, but the spatial matrix of the Receptacle remains invariant, and so it, or at least parts of it, possess the requisite stability to be the object of deictic reference. If ``This is fiery'' is true, then ``this'' picks out the place wherein the fiery appeared, a place wherefrom it will depart when it perishes, by being quenched or otherwise extinguished. ``This'' may refer to the place wherein the fiery appeared, but we should not suppose that the fiery appeared from some other place (like the guest of a talk-show making their appearance from the greenroom). The fiery, when quenched, departs from the place wherein it appeared. And again we should not suppose the fiery has departed to some other place (like the guest subsequently returning to the greenroom after their appearance).

Timaeus is engaged in linguistic revisionism here. Ordinarily, we would take ``this fire'' to refer to the demonstrated fire and not the place in which the fiery body appeared. This is not the first time in his speech that Timaeus has engaged in linguistic revision. Thus, for example, Timeaus thinks that our temporal language is confused. Observing that Timaeus is engaged in linguistic revisionism is not a coy way of pressing a worry or objection. Philosophers have been known to recommend a number linguistic revisions for a number of reasons. Timaeus departs from ordinary usage for a principled reason. Timaeus sees our ordinary usage as aporetic since it insists that we refer to something that lacks the requisite stability to be the object of demonstration. That Timaeus is prescribing what is by his lights the only intelligible usage rather than describing a pre-existing usage is relevant to assessing his proposal. 
% As we shall see, Aristotle retains the ordinary usage, and some, at least, of his criticisms of Timaeues in \emph{De generatione et corruptione} seem to presuppose this.

Timaean linguistic revisionism, however, raises a question about the nature of his argument. We are meant to be given an argument for the existence of the Receptacle. But appealing to what a hypothetical usage could so much as refer to seems weak as existence proofs go, even by the standards of a ``likely'' account. 

Though Timaeus' linguistic recommendations depart from ordinary usage, not only is that usage subject to \emph{aporia}, but Timaeus believes that his recommended usage better conforms with a deeply held commitment. Moreover the force of his argument for the existence of the Receptacle depends upon this commitment. (In this regard, Timaeus perhaps deserves the Orwellian epithet, ``Conservative Revisionist''.) Without endorsing a corporealist metaphysics, Timaeus accepts a fundamental corporealist commitment. In the \emph{Sophistes}, the Eleatic Stranger transforms the Gigantomachy, the battle for political supremacy over the Cosmos between the Giants, the sons of Gaia and Uranus, and the Olympian Gods, into a metaphysical dispute between corporealists and the Friends of the Forms. The Giants press their case as follows:
\begin{quote}
	One party is trying to drag everything down to earth out of heaven and the unseen, literally grasping rocks and trees in their hands, for they lay hold upon every stock and stone and strenuously affirm that real existence belongs only to that which can be handled and offers resistance to the touch. (Plato, \emph{Sophistes} 246a; Cornford in \citealt{Hamilton:1989fk})
\end{quote}
Timaeus, of course, denies that existence belongs only to that which can be handled and offers resistance to touch. Nevertheless, according to Timaeus, the Giants are right at least to this extent: Only that which appears in some place, like every stock and stone, enjoys an encosmic existence. And so while Timaeus' linguistic recommendation departs from ordinary usage, it conforms to a deeply held commitment that captures part of what was right about a primordial corporeal metaphysics. It is really this commitment---that only that which appears in some place may suffer encosmic corporeal existence---that is being leveraged in Timaeus' case for the Receptacle.

The sensible that appears in our experience within the living Cosmos is, at best, ``such-like''. If ``this'' does in fact refer, it must refer to something insensible. And the Receptacle, like the Living Being and intelligible things more generally, is invisible and insensible. Moreover, it is insensible in the way that intelligible things are as opposed to the way that elemental triangles are. Elemental triangles may be too small to be perceived, but aggregates of elemental triangles are sensible. The intelligible, by contrast, remain insensible even in aggregate. That the Receptacle is insensible has further epistemological consequences. Elemental triangles are the object of \emph{doxa} or empirical judgment. But the Receptacle is not the object of \emph{doxa} or \emph{pistis}. \emph{Doxa} is based upon perception or sensation. The Receptacle is insensible. So the Receptacle could not be the object of \emph{doxa}. Rather, it is known, if at all, through a ``bastard'' reasoning based upon non-perception, hardly an object of belief.

% section an_emph_aporia (end)

\section{Corporeal Media} % (fold)
\label{sec:corporeal_media}

Timaeus introduces at this point a corporeal image. The corporeal image involves a craftsman, specifically a goldsmith, continually fashioning gold into a range of distinct figures. The image is meant to do three things. First, it is meant to lend credence to Timaeus' linguistic recommendations. Second, in so far as it does, it is meant to bolster Timaeus' case for the existence of the Receptacle. Third, the image is meant to disclose a substantive claim about the nature of the Receptacle.

We are meant to imagine a goldsmith constantly working and reworking a piece of gold into a range of different possible figures. The goldsmith's work is unceasing. As soon as they have fashioned a figure, they begin fashioning another. If during this continuing process, the goldsmith were asked ``What is this?'' the safest possible answer would be to say that this is gold. ``This'' could not refer to any of the figures that the goldsmith worked since these are always changing. The only invariant aspect of this process is the gold in which the figures appear.

First, consider how the corporeal image is meant to lend credence to Timaeus' linguistic recommendations. If the appearance of the sensible and the corporeal in the spatial matrix of the Receptacle is really analogous to the appearance of figures in the constantly reworked piece of gold, and we accept that the objects of demonstration must have the requisite stability, then we are naturally encouraged to accept the negative and positive claims behind Timaeus' recommendation. The negative claim consists in the thought that variation undermines being the object of demonstration. There is nothing in the sensible realm of Becoming with the requisite stability to be the object of demonstration. So nothing sensible and corporeal so much as could be the object of demonstration. The positive claim consists in the thought that since the place wherein the sensible and the corporeal appear and wherefrom they depart, like the gold, is an invariant aspect of a process of continual change with sufficient stability to be the object of demonstration. Just as ``this'' could only refer to the gold in the case of the unceasing goldsmith, ``this'' could only refer to the place in which the sensible and the corporeal appear.

There is an apparent puzzle that potentially sheds light on how the corporeal image is meant to work. First, the puzzle. If we accept that the corporeal image supports Timaeus' linguistic recommendations, then ``this'' in the case of the goldsmith could not refer even to the gold. The gold may be soft and pliable and thus receptive to the various figures that the goldsmith imposes upon it, but it remains itself sensible and corporeal. And the sensible and the corporeal are meant to lack the requisite stability to be the objects of demonstration. If that is right, then not only could ``this'' not refer to the figures fashioned in the gold, but it could not even refer to the gold. We seem to be in the paradoxical position where if we accept the moral of the corporeal image then the case is not, after all, analogous.

I am uncertain whether and to what extent this is a problem for Timaeus, even a pragmatic one. The narrative of the unceasing work of the goldsmith might still succeed in reinforcing the idea that instability undermines being the object of demonstration even should it not be strictly speaking analogous. Notice that it is disanalogous in the sense that, by strict Timaean standards, there is no invariant element in the process involving the gold other than the place in which it occurs. So any lessons not based upon this invariant element remain secure, such as the way in which variation undermines eligibility for demonstration. Moreover, even if in this way disanalogous, the narrative might still help frame our experience of life within the living Cosmos so as to facilitate recognizing the only invariant thing that we so much as could demonstrate, the place wherein the sensible and the corporeal appear and wherefrom they depart. So these negative and positive claims behind Timaeus' linguistic recommendations arguably survive the failure of the corporeal analogy. Perhaps, then, the gold is less an analogy, strictly speaking, than a framing metaphor. Metaphors are not analogies. Juliet is not a celestial body and any analogies that might be drawn between Juliet and the Sun are themselves metaphorical. Juliet may be analogus to the Sun in being a source of light for Romeo, but that too is metaphorical. 

Second, suppose that Timaeus' corporeal image, does, in fact, in one way or another, lend credence to his linguistic recommendations. How does this bolster his case for the existence of this third kind, distinct from Paradigm and image, but that in which the image of the paradigm is formed, the Receptacle? If we accept that nothing sensible and corporeal has sufficient stability to be the object of demonstration, then ``this'' could refer to nothing sensible and corporeal. And while the sensible and the corporeal come to be and pass away, the place wherein they appear and wherefrom they depart remains. As this is the only thing to which ``this'' could refer, if we accept that it does, we must accept the existence of the object of that demonstration, the Receptacle, the Nurse and Mother of all Becoming. While the reasoning here is structured around what it takes to be the object of demonstration, there is a crucial additional element as well. Timaeus in his narrative gets us to independently attend to the place wherein the sensible and the corporeal appear. This indicates an independent positive commitment to something with the requisite stability to be the object of demonstration. Again, Timaeus is leveraging the primordial commitment that animates corporeal metaphysics---that there must be a place in which the sensible and the corporeal appear.

The Receptacle is a uniquely Timaean innovation of his cosmology, perhaps even more so than the Demiurge. Timaeus' cosmological account is ``likely''. There is no question, then, of demonstrating the existence of the Receptacle in the sense required for the Receptacle to the apprehended in understanding with an account. Part of what prevents Timaeus' reasoning from being a demonstration of the existence of the Receptacle concerns the status of the commitment to there being a place in which the sensible and the corporeal appear. What is the source of this commitment? 

It appears to us in dreams (52b), and in myth (the Gigantomachy, at least as retold by the Eleatic Stranger in the \emph{Sophistes}). While mundane dreams (45d--46, chapter~\ref{sec:sleep_and_dreams}) may mislead, dreams may also divinatory (71e--72b, chapter~\ref{sec:appetite}), though it may take wisdom to distinguish mundane from the divine, the false from the true. Similarly, myths may mislead (think of the banishment of the mimetic poets in \emph{Res Publica} 10), but they may also succeed in meeting the positive epistemic standard of being ``likely'' in the sense of fitting, fair, natural, or reasonable, even if this standard falls short of what is required for understanding with an account. Some myths may even be divinely inspired. Like divinatory dreams, they are the divine disclosure of some truth. And the source of this truth is exogenous and divine. Like divinatory dreams, myth requires interpretation. And interpretation requires wisdom, not only to distinguish the mundane from the divinatory, but to understand the divinely disclosed truth as well. So appearing in dream and myth may be consistent with being true or at least being a reasonable representation of the truth. That the place in which the sensible and the corporeal appear is made explicit as the object of dream and myth introduces an element of ambiguity. Are the dreams in which the Receptacle presents itself mundane or divinatory? Do the myths present divine truths or dangerous falsehoods? While the ambiguity may prompt the mind to wander, genuine wisdom knows the path and can discern the true from the false. Timaeus tells us that the Receptacle is only revealed in a ``bastard'' reasoning. The source of the conviction that the sensible and the corporeal must appear in some place is ambiguous. If bastards are of ambiguous heritage, then perhaps the reasoning from this conviction with an ambiguous source is at least part of what makes it a ``bastard'' reasoning. We shall return to the epistemic standing of the Receptacle.

Third, not only is the corporeal image meant to lend credence to Timaeus' linguistic recommendations and bolster his case for the existence of the Receptacle, but in terms of it, Timaeus, in addition, makes some substantive claims about the nature of the Receptacle. 

The gold is always called gold, even as the craftsman continually works new figures from it. Similarly, the Receptacle has its own name because it has a constant character. So we learn two things. First, that the Receptacle, like the gold, has its own name. It is, at the very least, the object of demonstration. Second, we learn why the Receptacle has its own name. The Receptacle can have a name proper to it because it retains its character through the coming and going of corporeal appearances. That the Receptacle has a constant character and the nature of that character are the substantive claims about the Receptacle introduced by the corporeal image. If that image is apt, then the gold too has its name because of its constant character. Presumably the constant character of the gold consists in its softness and pliability since these explain its receptivity to the figures imposed upon it by the goldsmith. The constant character of the Receptacle is similar if crucially, and puzzlingly, different.

This character, from which it never departs, is, Timaeus tells us, the catholic power to receive all things. This is similar to the constant character of the gold since each is a power to receive. In the case of the gold, the gold has the power to receive the figures that the goldsmith could impose upon it. And this because of its corporeal characteristics, its softness and pliability. Throughout being worked upon, the gold retains its softness and pliability, and so retains its ability to receive potential figures imposed upon it by the goldsmith. The Receptacle, we are told, has its proper name because of its character, from which it never departs, the great catholic power to receive all things.

The gold's power to receive figures imposed upon it by the goldsmith is quotidian, familiar from life within the living Cosmos (at least to the urbanites that were the most likely readers of Plato's dialogue). And it is also, importantly limited, as the contrast with the Receptacle makes clear. The gold may have the power to receive a range of potential figures, but it manifestly lacks the great catholic power to receive all things. A corporeal medium's power to receive is inexorably limited by the material recalcitrance of the corporeal medium itself. 

This specific difference with the great catholic power of the Receptacle reveals something general about how Timaeus conceives of powers to receive. Clearly Timaeus conceives of such powers as arrayed from Great to Small, in order of the range of things that they may receive. So understood, the catholic power to receive all things is Great and the power to receive a limited range of potential figures is Small. Nevertheless, Timaeus thinks that by reflecting upon the way that some bodies have the power to receive a limited range of sensible forms will help explicate the Receptacle's great catholic power to receive all things.

 % The Receptacle is meant to be analogous to the gold. But to what extent is the analogy apt?

Timaeus begins to elaborate upon the Receptacle's power to receive all things, but his remarks are modeled on what a craftsman would look for in a corporeal medium which is to receive a sensible form. The gold that the craftsman works is capable of receiving all the figures that the craftsman imposed. Of course, the Receptacle's capacity to receive is greater still. The gold may be capable of receiving a range of figures, but it is not capable of receiving all things the way that the Receptacle can. There may, for example, be forms too small to be molded in gold. (Think of the form of an elemental triangle.) This is a manifestation of the material recalcitrance of gold. Timaeus, however, draws our attention to a different potential limitation. In order for the gold to receive a range of figures, it must not retain any of the figures that were imposed upon it. Were the gold to retain the figure, no new figure could be imposed, and even if it could, it would be distorted by the presence of the retained figure. Not only are the powers to receive arrayed from Great to Small, in proportion with the range of things that they may receive, but this is constrained by the range of sensible forms retained by the body in which that power inheres. Any retained sensible form is an obstacle to the corporeal medium taking on a form that would oppose it. So a principle governing corporeal media is that they should not be similar to the forms they are meant to receive. (It is clear, upon reflection, that Timaeus has in mind generic as opposed to specific similarities.)

This principle governing corporeal media holds for the Receptacle as well. In order to receive all things, the Receptacle should not have the form of anything that appears in it. Thus, the Receptacle does not itself take on the form of anything that appears within it, nor does it possess any other form. Sensible form is not predicated of the place wherein it appears. The fiery may fleetingly appear in some place. And that place may appear hot and dry. But the Receptacle is not hot and dry, not even in that place. And so hot and dry are not predicated of the Receptacle. Timaeus' stated reason is that taking on the form of what appears in it would interfere with the Receptacle's capacity to receive new forms. Should the Receptacle take on some form, the presence of that form would resist the reception of any form opposed to it. Or it would receive a new form but only in a manner distorted by the presence of the old form.

The corporeal image is thus meant to have substantive lessons about the constant character of the Receptacle and the nature of that character. We have learned that the Receptacle is governed by a principle that governs corporeal media as well---that media should not be similar to the forms it is meant to receive. As a consequence, the constant character of the Receptacle is not among the forms that it receives. The Receptacle receives the whole of the sensible and the corporeal. So the constant character of the Receptacle is neither sensible nor corporeal. The constant character of the Receptacle is neither sensible nor corporeal but is rather the great catholic power to receive all things, a power mysteriously grounded in its very formlessness.

The corporeal image also offers a framework for understanding the relationship between the three kinds, the Paradigm, its image, and that in which it is imaged, the Receptacle. The gold is moved and marked as figures enter into it, and so appears differently at different times. Similarly, the Receptacle is moved and marked as figures enter into it, and so appears differently at different times. These figures are copies of what always is. Clearly we are meant to imagine the goldsmith displaying Demiurgic virtue in using an eternal as opposed to a generated model (discussed in chapter~\ref{sec:model_and_image}). If the goldsmith does indeed display Demiurgic virtue, this would further confirm Broadie's \citeyearpar[28--9]{Broadie:2012vl} claim that if what makes for beauty or excellence is guidance by an intelligible model, then we too, in our affairs, should endeavour to look to the intelligible. Arguably this is implicit in the thought that we are meant to imagine a good goldsmith. A good goldsmith produces beautiful work. And if this work is mimetic, then it is only beautiful if it is based upon an eternal model. In the case of the Receptacle, it is moved and marked as figures enter into it by being stamped, in a marvellous manner, hard to describe, by the Paradigm and its contents. So the Cosmos is a sensible image stamped upon the Receptacle in a marvellous manner, hard to describe, by the Paradigm and its contents. The Paradigm and its contents are represented as acting upon the Receptacle so as to generate a sensible image. This action may be hard to describe, but it is not impossible to describe. Indeed Timaeus promises to elaborate later. Unfortunately, Timaeus remains mortal and, despite being wise and good, fails to fullfil his promise.

While we may lament not knowing how the Forms stamp or impress upon the Receptacle to generate sensible images, in emphasising the activity of the eternal model, Timaeus raises a puzzle. It is the model's activity that is marvellous if hard to describe, not the goldsmith's work. It seems that we have been told something about the causal relationship (as we might anachronistically put it) between the Paradigm, its image, and that in which it is imaged, the Receptacle. But the Demiurge has been left out of account.

Here it is useful to bear in mind the distinction between the pre-cosmic chaos and the Cosmos established by the Demiurgic imposition of form and number. In the pre-cosmic chaos there are fleeting sensible powers that are traces of the Forms of the primary bodies. These traces are presumably stamped by the Forms in a marvellous manner, hard to describe, without Demiurgic intervention. That is why they display a disorderly motion. In this sense they are a wandering cause. They wander in the sense that they move without the guidance of \emph{nous}. Matters are different with respect to the Cosmos. The generation of the Cosmos required Demiurgic intervention. In order to generate the Cosmos, the Demiurge (\emph{nous} personified, on the Demiurge and \emph{nous} see \citealt{Menn:1995di}) had to impose form and number upon the chaotic contents of the Receptacle. Sensible powers in the pre-cosmic chaos, traces of the Forms in the spatial matrix of the Receptacle, receive form and number by being assigned regular polyhedra with a certain number of sides. Thus form and number is imposed upon traces of the fiery by assigning those powers complementary shapes with a determinate number of sides. Specifically traces of the fiery are assigned tetrahedra since of the regular polyhedra they are the smallest and swiftest and their sharp angles and sides cut easily. The choice of polyhedron is determined by the sensible power they are meant to capture and extend. Not only does the choice of tetrahedra capture the fiery character but it also extends this character to be sufficiently like the Form of fire so that it may be called ``fire'' with any safety.  Once form and number is imposed upon the chaotic contents of the Receptacle, their motion is consequently more orderly and suitable as auxiliary causes to the true cause as determined by the will of the benevolent and ungrudging Demiurge. 

Notice how this bears upon our puzzle. The Forms need to act independently of the Demiurge in order to produce the traces that constitute the chaotic contents of the Receptacle prior to the imposition of form and number. Moreover, the Demiurge generates the Cosmos, at least in part, by acting upon these trace powers by imposing form and number upon them. The Demiurge could not do that if there were no trace sensible powers to act upon. And there would be no trace sensible powers without the Forms stamping them, in a marvellous manner, hard to describe, upon the all-receptive Receptacle. Recall, Timeaus is describing the works of Necessity at this point in his speech. The corporeal image provides a framework for the works of Necessity. The absence of Demiurgic influence is no mystery. The framework describes the causal relationships between the Paradigm, its image, and that in which it is imaged, the Receptacle, that together constitute the workings of Necessity.

Close attention to Timaeus' language here reveals further reason to deny that the corporeal image is a strict analogy. The gold is moved and marked as figures enter into it. In the case of corporeal media, this is a process of alteration. But the Receptacle is not altered by the sensible appearing in it. After all, the sensible that appears in the Receptacle is not predicated of the Receptacle. So if the Receptacle is moved and marked, it is not moved and marked literally, as this would involve an alteration that it does not in fact undergo. Similarly, Timaeus describes it as a molding stuff. But the Receptacle is not strictly analogous to corporeal media and so not literally molding stuff. Molding stuff is a metaphor for the Receptacle's power to receive. We shall return to this issue when discussing Aristotle's criticism of the corporeal image.

% Corporeal media suffer from a material recalitrance in the way that the Receptacle does not.

At this point, Timaeus, while retaining the craft analogy, shifts the relevant craft from a goldsmithing to perfumery. The variant analogy is meant to emphasize two things. It emphasizes both the Receptacle's power to receive all things and that being without form is a precondition on possessing this power. The base of a perfume must be as odour free as the perfumer can make it so as to better receive the scents that will be combined in that base. Just as the gold having a shape of its own would interfere with its ability to receive another shape, so the base of a perfume having a scent of its own would interfere with its receiving other scents. Since the Receptacle is capable of receiving all things, this must be all the more true for it.

The material recalcitrance of corporeal media may be limiting, and craftsmen deploy great artifice to sufficiently overcome these limitations, at least as far as they can for practical purposes, but they are intelligibly media capable of receiving sensible forms for all that, even if imperfectly. And Timaeus is right that if you want a corporeal medium to take on a certain shape, then it should not have a shape of its own that it will retain through its working. However, generalizing from the power of corporeal media to receive a limited range of sensible forms to the great catholic power to receive all things may be a step too far. While the former is familiar and intelligible, the latter is strange and may not be intelligible.

% section corporeal_media (end)

\section{Aristotle's Objections} % (fold)
\label{sec:aristotle_s_objections}

Doubts about the corporeal image are usefully pursued by first considering Aristotle's objections to it. In \emph{De generatione et creatione} (2 329a13ff), Aristotle makes a series of distinct if related objections. Understanding the force of these objections require knowledge of the immediate context. Book 2 of \emph{De generatione et creatione} marks a shift of topic. Aristotle has discussed coming to be and passing away, explaining under which conditions they occur, in what subject, and owing to what cause. Aristotle has also explained what alteration is and how it differs from coming to be and passing away. As we shall see, he is keen to mark that distinction as he conceives of it. What Aristotle has not discussed, up to this point, is the elements. 

The elements, however, are related to his previous topics. Aristotle explicitly states the relation as follows: Primary materials whose change results in coming to be and passing away are rightly described as principles (\emph{archē}) or elements (\emph{stocheia}). So the elements are primary materials whose change results in coming to be and passing away. The elements are materials in the sense that they are sensible bodies out which things are generated by undergoing some appropriate change. The elements are the primary materials in that they are the most basic materials whose change results in coming to be and passing away. The elements can thus exist separably from what comes to be and passes away as result of their changing.

While amenable to the role of the elements in generation, Aristotle, however, strenuously resists a further commitment. Some thinkers postulate a further entity in addition to the sensible bodies that constitute the primary material. In addition to these bodies, these thinkers postulate a single matter (\emph{mian} \emph{hulēn}), itself conceived to be corporeal and separable. Aristotle refers to this entity as the Unlimited (\emph{apeiron}). The difficulty with the Unlimited, we are told, is its alleged corporeal nature. If the Unlimited is corporeal, then it is subject to sensible contrariety. To be subject to sensible contrariety is to be qualified by sensible oppositions, such as hot and cold, dry and wet, and so on. Frustratingly Aristotle does not make explicit his reasoning at this point, but he clearly thinks that being subject to sensible contrariety is inconsistent with being the Unlimited. Perhaps the thought is that to be qualified by sensible oppositions is to be limited in way that the Unlimited could not be. But what Aristotle's presentation fails to make explicit is why a single matter must be unlimited in a manner inconsistent with being subject to sensible contrariety. Aristotle, nonetheless, regards the claim that the Unlimited must be subject to sensible contrariety as a decisive objection against the target class of views.

At this point Aristotle considers the Receptacle and Timaeus' corporeal image of it.

Timaeus' account is said to be confused, or perhaps, ambiguous. It may be all too easy to sympathise with the first Peripatetic charge since Timaeus is struggling to provide a clear account of the nature of the Receptacle. However, Aristotle means something specific by that charge. He goes on to say that it is confused or ambiguous because Timaeus does not explain clearly whether the Receptacle can exist separably from the elements. What is confused, perhaps owing to ambiguity, is how Timaeus' account fits with Aristotle's doxographic category. Is the Receptacle a species of the Unlimited postulated by the target class of views? The Unlimited exists separably from the elements. But, according to Aristotle, it is unclear whether the Receptacle exists separably from the elements. The confusion, then, consists in the poorness of fit between Timaeus' account of the Receptacle and Aristotle's doxographic category. Once the charge is specified thus, it looses much of its force. Suppose that Aristotle is right and Timaeus has not in fact explained clearly whether the Receptacle can exist separably from the elements. Why is the poorness of fit a problem for Timaeus' account rather than for Aristotle's doxographic category and any taxonomy upon which it relies?

Timaeus postulates the Receptacle only to make no use of it. Aristotle's second charge may be pointed. \citet[40]{Gregory:2003aa} observes that Aristotle may be echoing Socrates' language in the \emph{Phaedo} (98b8--9) when he complains that while Anaxagoras postulates \emph{nous} he makes no use of it. Aristotle himself will object to Anaxagoras in similar terms (\emph{Metaphysica} A 4 985a18--21). The legitimacy of the complaint depends, in no small measure, on their being uses, overlooked by the account, to which the Receptacle may be legitimately be put. It is easy to see what such uses would be in the case of Anaxagoras' cosmology, at least by Socrates' lights. When Anaxagoras claimed that \emph{nous} is the cause of all, Socrates expected a cosmological explanation for why everthing is for the best. Anaxagoras does not use \emph{nous} to explain much, and the bulk of his explanations are indistinguishable in kind from Milesian alternatives. According to Socrates, these appeal, not to causes, but to conditions without which there would be no cause. (Timaeus, more generously, regards these as auxiliary causes, see chapter~\ref{sec:_emph_aitia_and_emph_sunaitia}.) We can see the legitimate use to which \emph{nous} might be put but is not---namely, in supplying a cause and a reason for why things are for the best in addition to conditions without which there would be no cause. Taking the \emph{Phaedo} allusion seriously requires asking what legitimate use does Aristotle have in mind that Timaeus might have put the Receptacle but fails to do and so makes no use of? If the \emph{Phaedo} allusion is genuinely Aristotelian, then I fear that the rhetoric misfires. The lesson that we are meant to take away is that the Receptacle is obscure, poorly explained, and Timaeus makes no use of it, and so we can safely ignore it since Timaeus' explanations are all cast in terms of the elemental triangles and the regular polyhedra constructed from them. But that is just to say that there is no overlooked legitimate use to which the Receptacle might be put. Aristotle's complaint about Timaeus moves in the opposite direction from Socrates' complaint about Anaxagoras.

The observation upon which the second charge rests may be true. Once the Demiurge has imposed form and number upon the chaotic contents of the Receptacle, further explanations are framed in terms of elemental triangles and the regular polyhedra constructed from them. So after the Demiurgic imposition of form and number, Timaeus does indeed make no further use of the Receptacle. But it is unclear why this is an objection. We cannot understand it on the model of Socrates' charge against Anaxagoras. So what is the force of the objection? Timaeus seems to be describing the framework that must be in place in order to allow for the Demiurgic imposition of form and number and so the generation of the Cosmos. One could only complain that Timaeus makes no further use of the Receptacle if one antecedently doubted that the Receptacle was, after all, necessary for the Demiurgic imposition of form and number. But what is the source of that doubt? Aristotle declines to say.

According to Aristotle, the Receptacle is a \emph{substratum} prior to and underlying the elements. Moreover, Timaeus is claimed to hold this on the basis of his corporeal image, an image understood by Aristotle as an analogy. Just as the gold underlies each of the figures generated by the goldsmith, the Receptacle underlies each of the figures generated by the Demiurge. Just as gold is the \emph{substratum} of the figures generated by the goldsmith, the Receptacle is the \emph{substratum} of the figures generated by the Demiurge. And yet Aristotle maintains that the comparison may be criticized.

Before considering Aristotle's reason for criticizing the corporeal analogy, several comments are in order. 

First, Timaeus would deny that the Receptacle is a \emph{substratum}. It is an invariant element in a process of continual change, but Timaeus does not further conceive of the invariant element as the \emph{substratum} for the continual change. Aristotle must be thinking of the gold as the material substrate of the figures generated by the goldsmith and then concluding by analogy that the Receptacle must itself be a \emph{substratum}. 

Second, Timaeus and Aristotle appeal to different spatial ideas. The figure may be generated out of the gold, its material substrate, but the sensible and the corporeal that appear in the Receptacle are not generated out of the Receptacle. Rather, the Receptacle is the place wherein they appear. 

Third, we have already shown how key rhetorical objectives of Timaeus' narrative may survive the failure of the corporeal analogy. The goldsmith's unceasing work encourages the thought that variation undermines eligibility for demonstration. And, more positively, the narrative may reframe our experience in such a way as to help us to acknowledge the only thing that we so much as could demonstrate, the place wherein the sensible and the corporeal appear and wherefrom they depart. Since key rhetorical objectives survive the failure of the corporeal analogy, perhaps Timaeus was not relying, in this way, and the way that Aristotle suggests, on analogical reasoning. Perhaps the gold is less an analogy, at least strictly speaking, than a framing metaphor. 

Fourth, we have seen how the Receptacle is not moved and marked the way that the gold is when figures enter into it. The gold is moved and marked in the sense of being altered, but the Receptacle is not altered by the sensible figures that enter into it. So, \emph{pace} Aristotle, the corporeal image could not be understood as an analogy. A Peripatetic might object that at 51b Timaeus explicitly claims that the Receptacle is ignified with the appearance of fire and liquefied with the appearance of water. In citing this passage in support of attributing to Timaeus a corporeal analogy, the Peripatetic misreads this passage. Timaeus has already just claimed that we should not speak of the Receptacle by the names of the ``roots'' or any aggregates thereof. If that is right, then being ignified, in whatever sense that it is, the Receptacle does not itself become fiery such that fire may be predicated of it. Recall, that only the Receptacle is the object of demonstration and that the sensible and the corporeal that appear within it can be only be described as ``such-like'' with any safety. That the Receptacle is said to be ignified is meant to capture how the appearance of fire is ``such-like'', not that the Receptacle has been altered in being made fiery in the way required if Timaeus' corporeal image is understood as an analogy. 

Finally, in describing the framework of causal relationships between the Paradigm, its image, and the Receptacle, Timaeus refers to the Receptacle as molding stuff. Again, and for similar reasons, it is clear that the Receptacle could not literally be molding stuff the way that corporeal media, such as gold and wax are. It is not moved and marked as figures enter into it the way that corporeal media are altered as figures enter into them. Rather, describing the Receptacle as molding stuff is a metaphor for the great catholic power to receive all things. The Receptacle, like molding stuff, has the power to receive, if only on a grander scale.

Aristotle criticizes the corporeal image, understood as an analogy, on the basis of his distinction between generation and alteration. When confronted with the unceasing work of the goldsmith and asked what is that, the safest possible answer, we are told, is that is gold. However, things that come to be and pass away are not named after the material out of which they came to be. Only the effects of alteration retain the name. So according to Aristotle, since the figures that appear are named after the material out of which they were fashioned, a change in figure is not a coming to be and passing away, but is rather a species of alteration. Again, Aristotle is frustratingly inexplicit about the precise charge made on the basis of this claim. But the thought is plausibly that gold at best only provides a corporeal analogy for the contents of the Receptacle altering. But the Mother and Nurse of Becoming is not merely a nurse of alteration. Mortal beings come to be and pass away in Her embrace. But if the coming to be and passing away of mortal beings are understood on the model of alteration, then Timaeus obscures the distinction between generation and alteration.

% Aristotle accepts the Timaean lesson that demonstration requires invariance or sufficient stability. Aristotle does not, however, accept Timaeus' linguistic recommendation. Moreover, he insists that the invariant element be understood as a \emph{substratum}.  Ordinarily we would take ``this fire'' to designate a particular fire, but Timaeus thinks that the only thing that the demonstrative could intelligibly refer to is the place in which the fire appeared.

One problem with understanding the corporeal image as an analogy is that Timaeus' linguistic recommendations are inconsistent with such an analogy. In the corporeal image, the ``that'' refers to the gold. But gold, being sensible and corporeal, lacks sufficient stability to be the object of demonstration, at least by strict Timaean standards. By Timaeus' lights, the gold fails to be the way it would have to be if it were analogous to the Receptacle in being a genuine object of demonstration.

Aristotle accepts the Timaean lesson that demonstration requires invariance or sufficient stability. Aristotle does not, however, accept Timaeus' linguistic recommendation. Moreover, he insists that the invariant element be understood as a \emph{substratum}. These two differences are perhaps not unrelated. By continuing to adhere to an ordinary understanding of what ``that'' could refer to, without addressing Timaeus' \emph{aporia}, Aristotle thinks that we can demonstrate sensible and corporeal beings that come to be and pass away. Since he does not accept Timaeus linguistic recommendation, Aristotle pays no heed to how it is inconsistent with the deictic reference of ``that'' being gold. In this case the object of demonstration, the gold, is sensible and corporeal, but more specifically it is a molding stuff, a corporeal medium. And since Aristotle reads the corporeal image as an analogy, he concludes that the Receptacle is itself a molding stuff, at least in the sense of being a \emph{substratum} for the sensible and the corporeal that appear in it and depart from it. So the two ways in which Aristotle differ from Timaeus are in this way related.

While Aristotle's attachment to the idea of a \emph{substratum} is profound, its really the first difference that is doing the work in his disagreement with Timaeus. Presumably Aristotle rejects Timaeus' linguistic recommendations since he thinks that we can legitimately demonstrate things other than the place in which the sensible and the corporeal appear. Prominent among them, presumably are mortal beings that come to be and pass away. And yet Aristotle accepts that demonstration requires invariance. Aristotle can understand how these mortal beings may be the invariant elements in the changes they undergo before they pass away. They can be understood as \emph{substrata} on analogy with molding stuff. What Aristotle cannot understand is how this could provide a model for generation and destruction. But in order for that complaint to gain hold, Aristotle needs to explain how mortal beings possess sufficient stability to be eligible for demonstration in the first place. Only in this way would the \emph{aporia} that motivates Timaeus' linguistic recommendation be resolved.

Aristotle's central objection that the corporeal image obscures the distinction between gerneration and alteration depends on understanding that image as an analogy. The corporeal image, however, could not be an analogy but is better understood as a framing metaphor. But if the Peripatetic charge is evaded in this way, the victory threatens to be Pyrrhic. If the corporeal image is understood as analogy, the distinction between generation and alteration may be obscured, but at least we have been given a model of the Receptacle's power to receive. The Receptacle's power to receive all things would be understood on the model of a corporeal medium. But if we give up on the idea that the corporeal image is an analogy, then we lack an understanding of the great catholic power to receive all things apart from the fact that it is meant to be grounded in the formlessness of the Receptacle. 




% It is credible that Aristotle has the resources to provide such an explanation.

% section aristotle_s_objections (end)

\section{The Receptacle} % (fold)
\label{sec:the_receptacle}



Recall the ontological distinction between the intelligible, that which always is and never becomes, and the sensible, that which becomes and never is, was explicated in the \emph{proemium} as the objects of distinct cognitive attitudes. What kinds of things are these? Well, they are the kinds of things that are understood and the object of opinion. Specifically, while that which always is and never becomes is the object of \emph{noēsei meta logou perilēpton}, that which becomes and never is is the object of \emph{doxē met’ aisthēseōs alogou doxaston}. The Paradigm is that which always is and never becomes and is apprehended in understanding with an account. The Cosmos, an image and copy of the Paradigm, is that which becomes and never is and is opinable with the aid of perception and sensation. In distinguishing the Receptacle from the Paradigm and the Cosmos, Timaeus extends the ontology first introduced in the \emph{proemium}. A novel third ontological kind is introduced. But how does this effect the epistemology of the \emph{proemium}? And how, more generally, does Timaeus' new start relate to the principles laid down in the \emph{proemium}?

Begin with the novel tripartite ontological division. The \emph{proemium} postulated what becomes and never is, the Cosmos, and wherefrom it is produced and copied, what always is and never becomes, the intelligible Paradigm and its contents. Timaeus' New Start postulates, in addition to Becoming and the intelligible source wherefrom it is produced and copied, the place wherein it becomes.

Timaeus introduces a biological simile for this tripartite division. The Receptacle of all Becoming is like the Mother, the source is like the Father, and what they engender is their Offspring.  In \emph{De generatione animalium} (A19, B1 763b30), Aristotle attributes to Anaxagoras and other students of nature the view that the Father furnishes the seed, and the Mother merely the place in which the seed develops. Perhaps, Timaeus' biological simile is drawn upon a similar medical tradition (see \citealt[187]{Cornford:1935fk}). Regardless of any specific influence, like this medical tradition, Timaeus seems to regard the Father as the agent of generation with the Mother merely providing the place wherein the generation occurs. Thus the Paradigm and its contents stamp, in a marvellous manner, the Receptacle, and the Receptacle provides the place wherein their image is generated. There may be an additional element to the biological simile. If the Mother merely provides the place in which the generation occurs, then it is the Father that the Offspring resembles. If the image is the Offspring, then the image resembles the Father, the Form that stamped its likeness, in a marvellous manner, upon the Receptacle in generating that image. And as formlessness is a precondition for the Mother's receptivity, it would seem that the Offspring could bear no family resemblance to her. For it the Offspring did, then Mother and Offspring would have some form in common, a form inconsistent with the Mother being all-receiving. 

Is the Receptacle the object of a distinct cognitive attitude? It is not clear that it is. First, recall that not only are the ontological kinds of the \emph{proemium} the objects of distinct cognitive attitudes, they are explicated as the objects of these attitudes. The idea is that if one wants to know what always is and never become one should reflect on the kinds of things that are apprehended in understanding with an account. But the Receptacle is not itself explicated as the object of a distinct kind of cognitive attitude. If anything, despite Timaeus' best efforts, the Receptacle remains elusive, and Timaeus confesses to it being baffling and obscure (about which \citealt{Derrida:1993aa} makes much) and the object of dreams and myth.

That the Receptacle differs from the two other kinds in not being explicated as the object of a distinct attitude allows for the possibility that the epistemology of the \emph{proemium} is comprehensive. If the Receptacle were explicated as the object of a distinct cognitive attitude, then since the \emph{proemium} made no mention of this attitude, then the epistemology of the \emph{proemium} would incomplete and so not comprehensive. If, however, in order to understand the Receptacle, to the extent to which we can, it need not be explicated as the object of a novel cognitive attitude, then the epistemological principles of the \emph{proemium} may prove sufficient, and there would be no threat to the comprehensiveness of the epistemology of the \emph{proemium}. At they very least, the epistemology of the \emph{proemium} need not be augmented with new principles governing new attitudes that take a new object. Indeed, the summary statement of the Receptacle, begins with a reaffirmation of the epistemology of the \emph{proemium}, along with some development of it. So, clearly, Timaeus' ``likely'' account of the Receptacle is meant to be consistent with the fundamental principles laid down in the \emph{proemium}.

The reaffirmation of the epistemology of the \emph{proemium} is prompted by a question that arises from claims about the Receptacle and its contents. Let us begin with these claims. When ignified, fire appears in the Receptacle, when liquified, water appears, and in general, the Empedoclean ``roots'' appear insofar as the Receptacle receives copies of them (51b). Again, the claim that the Receptacle is ignified should be understood only as implying that the appearance of the fiery, being sensible and corporeal, is ``such-like'', and not that fire is predicated of the Receptacle, as it would be if it were altered and made fiery.

These claims being made, Timaeus, at this point, raises a pair of questions concerning them. The pair of questions together present two alternatives that Timaeus is inviting his audience to choose between. Indeed, Timaeus is inviting his audience to make a stark choice between, two metaphysical pictures. As we shall see, Timaeus has, in effect, recast the Eleatic Stranger's presentation of the Gigantomachy. 

The first question is this: If being ignified is the Receptacle's reception of an image or copy of fire, then there must be a source from which the image was produced and copied. But is it plausible to suppose that in addition to the sensible fire that appears in some place that there exists a self-subsisting intelligible fire that could play the paradigmatic role if the fire is, in fact, an image and copy? 

% Two observations about Timaeus' question are in order.

The initial formulation of the question has a specific character. Timaeus is asking whether we should believe that fire and the other Empedoclean ``roots'' are copies or images of Forms that are self-subsisting realities. Despite the specificity of the question, its intent is general. Timaeus makes this explicit when he asks whether self-subsistent fire or any other self-subsistent objects exist. Given the generality of its intent, to answer the question in the affirmative, is to cast one's vote for the Friends of the Forms. And yet the specificity of Timaeus' initial formulation is not insignificant. Not only will Timaeus go on to affirm the existence of Forms as self-subsistent realities, but he will affirm as well the existence of a self-subsisting intelligible fire. Recall that young Socrates, when asked by Parmenides if there is a Form corresponding to every predicate or property, expresses uncertainty about whether there are Forms of fire and water and skepticism about Forms of mud and dirt (\emph{Parmenides} 130a--e). Part of what is at issue is whether there are Forms of fire, air, water, and earth. Timaeus, unlike young Socrates, maintains that there are. The sensible powers in the pre-cosmic chaos are traces of these Forms. But these traces could only be called ``such-like'' with any safety once the Demiurge has imposed form and number upon them. For it is only then that the fleeting fiery traces, in have the form of a tetrahedron imposed upon them, will be sufficiently like the Form of fire to itself be called ``fire'' with any safety.

% It is important to recognize the specificity of Timaeus' question. He is not asking whether there are self-subsisting realities. It is clear that he accepts that there are. Rather, Timaeus is asking whether we should believe that fire and the other Empedoclean ``roots'' are copies or images of Forms that are self-subsisting realities.

The second question is this: Is it only those things that we see or otherwise perceive by corporeal instruments that exist? Existence, then, would be confined to the sensible. The sensible and the corporeal exist and besides which there would exist no other kind of thing in no other manner. If so, it would be merely an idle assertion to say that there is a Form corresponding to every object. Allow me to make three observations.

The first observation concerns the characterization of the sensible. Again, despite the specificity of the initial formulation, its intent is general. Timaeus initially askes whether existence is confined to what can be seen. It is clear that being visible, here, is a stand in for being sensible more generally. Timaeus makes this explicit by generalizing his claim to whether existence is confined to what can be perceived with corporeal instruments. That the visible is, in this way, a stand in for the sensible more generally is something that we have seen Timaeus do repeatedly. He describes the intelligible as invisible to express its insensibility and describes the corporeal as visible to express its sensibility. 

A detail about Timaeus' explicit generalization raises a question. I am unsure of its answer. Timaeus does not merely generalize by asking whether existence is confined to what can be perceived, he introduces a qualification. Timaeus asks whether existence is confined to what can be perceived with corporeal instruments. Is the qualification significant? Consider two possibilities. Either all perception involves corporeal instruments or not. If all perception involves corporeal instruments then the qualification is not significant. To be perceptible would be to be perceptible with corporeal instruments. But perhaps not all perception involves corporeal instruments. (Recall this possibility is raised by Proclus, \emph{In Timaeum} 2 83.3–85.31, \citealt{Diehl:1903re}, and is discussed in chapters~\ref{sec:Being and Becoming} and \ref{sec:knowledge_and_opinion}.) While mortal beings who suffer an encosmic existence require corporeal instruments to perceive, perhaps an immortal being who does not suffer an encosmic existence may enjoy perception without corporeal instruments. If not all perception involves corporeal instruments, then the qualification is significant. Not only may the modes of perception differ between that which depends upon corporeal instruments and that which does not, not least if the relevant modes of perception derive from the nature of the corporeal instruments involved, but also, and significantly, the objects of perception may differ between that which depends upon corporeal instruments and that which does not. Perhaps perception that does not rely on corporeal instruments is not hindered or otherwise impeded by the material recalcitrance of corporeal instruments and can perceive more that what can be perceived in relying upon corporeal instruments. If Proclus is right in his speculations concerning cosmic perception, then perhaps the World Soul may perceive the elemental triangles that it encompasses in the way that any perceptual power dependent upon corporeal instruments could not.

The second observation concerns existence being confined to the sensible. If existence is confined to the sensible, then there exist only bodies and their corporeal parts and powers since the sensible is the mark of the corporeal. On this altenative, there are no other kinds of things. What Timaeus actually says is that there would exist no other kind of thing in no other manner. The envisioned alternative rules out not just incorporeal kinds of things, but also insensible modes of existence. Thus, on this alternative, there are no incorporeal Forms that enjoy an intelligible existence.

The third observation concerns the linguistic consequences of there being no incorporeal Forms that enjoy an intelligible existence. There are two such consequences. First, if there are no such Forms, then it is idle to say that there are. The conditional is undoubtedly true, on some appropriate understanding of idleness. The real question is whether the antecedent is true and so whether the consequent may be discharged. Second, if there are no such Forms, then the claim that there always exist an intelligible Form of every object is empty. Idleness and emptiness seem like different ideas. The charge of idleness is, roughly, that there is no point in speaking a certain way. What is the charge of emptiness? Perhaps the following approaches, if somewhat anachronistically, what Timaeus meant. The claim that corresponding to every object there is always an intelligible Form may be true but only on a pleonastic reading. A white thing may have ``white'' truly predicated of it. And you may say, if you like, that the whiteness of the thing is an image or copy of the intelligible Form of whiteness. But if true, this latter claim is true only pleonastically (in something like Schiffer's \citeyear{Schiffer:1987aa} sense). The truth of such a claim does not require the existence of incorporeal entities with insensible modes of existence. On this reading, the emptiness of the claim consists in its being true, at best, pleonastically. The Friends of the Forms, by contrast, understand the Forms' claim to existence to be a substantive truth not amenable to a pleonastic understanding. Not only are the charges of idleness and emptiness conceptually distinct, they may be extensionally distinct as well. Empty talk may yet have utility (a claim periodically urged by nominalists and antirealists), and if it does it is thereby not idle.

The linguistic consequences of there being no incorporeal Forms that enjoy an intelligible existence echo Parmenidean skepticism about the Forms. Here I mean the skepticism of Parmenides, the character of the eponymous dialogue, and not the historical Parmenides. Parmenides, the character from the dialogue, can be seen as offering young Socrates reasons for thinking that talk of Forms is idle and empty.

Timaeus' pair of questions presents his audience with a stark choice between two metaphysical pictures. Either we accept in addition to the sensible fire that appears in our experience of the living Cosmos that there is, besides, an intelligible fire, or we accept that existence is confined to what is sensible and corporeal and that talk of the intelligible Form of fire is idle and empty. This is the Gigantomachy recast. To accept the former is to cast one's vote for the Friends of the Forms. To accept the latter is to cast one's vote for the Giants. Timaeus, sensibly, does not propose to definitively settle the matter here. It is not clear that it could be settled given the confines of Timaeus' inquiry. Adjudicating this metaphysical dispute would undoubtedly require appealing to first principles in the way that Timaeus does not in his cosmology. Nevertheless, Timaeus provides us with an indication of the reasons why he favors one of these alternatives.

In the terms of the \emph{proemium}, the dispute is whether or not we should accept in addition to what becomes and never is what always is and never becomes. Interestingly, in addressing this dispute, Timaeus appeals to the cognitive attitudes in terms of which these ontological categories were explicated in the \emph{proemium}. Indeed, the alternatives are recast in terms of these cognitive attitudes. If understanding and true opinion (\emph{nous kai doxa alēthēs}) are two kinds, then self-subsisting Forms do exist, and these are imperceptible and the object of understanding alone. On the other hand, if true opinion does not differ in kind from understanding, then all things that we perceive with corporeal instruments must be judged most stable.

Before considering how Timaeus adjudicates between the epistemic recasting of our alternatives, let us briefly consider each.

First consider the claim that if understanding and true opinion are two kinds, then self-subsisting Forms do exist, and these are imperceptible and the object of understanding alone. 

Begin with the antecedent of the conditional. Understanding and true opinion are claimed not merely to differ but to differ in kind. That understanding and true opinion differ is consistent with understanding being true opinion if not all true opinion is understanding. If, however, understanding and true opinion differ in kind, then understanding is not merely a species of true opinion. It is a fundamentally different kind of attitude. I belabor this because it is an unfamiliar idea in contemporary epistemology. In the Anglosphere, perhaps only \citet{Cook-Wilson:1926sf} maintains the distinction in kind, though later authors may approach this.

Now consider the consequent of the conditional, that self-subsisting Forms exist. Or rather, consider why the existence of self-subsisting Forms follows from understanding differing in kind from true opinion. Recall that what always is and never becomes was explicated in terms of being the object of understanding. So the self-subsisting Forms claimed to exist are themselves the objects of understanding but only if understanding differs in kind from true opinion. How does a difference in kind between cognitive attitudes connect with a difference in their objects? Clearly, Timaeus is assuming that if understanding differs in kind from true opinion, then the objects of understanding will themselves differ from the objects of true opinion. If understanding were merely a species of true opinion, then the objects of understanding would be the objects of true opinion. 

The last clause elaborating the epistemic status of self-subsisting Forms is related to the claim that understanding and true opinion differ in kind. Self-subsisting Forms are claimed to be imperceptible and the objects of understanding alone. Recall that opinion is based on \emph{aisthēsis}. (Or at least mortal opinion is. Bracket, for the moment, the issue concerning how the World Soul can have opinion about the sensible realm of Becoming while lacking the corporeal instruments of perception, discussed in chapters~\ref{sec:Being and Becoming} and \ref{sec:knowledge_and_opinion}.) However that claim is understood, it is meant to have the consequence that imperceptible things are not themselves the objects of opinion. Imperceptible, here, must mean not merely, say, being too small to be perceptible. Sufficiently many things individually too small to be perceptible may, in aggregate, be perceptible. Intelligible things like self-subsisting Forms remain imperceptible even in aggregate. So being imperceptible rules out self-subsisting Forms from being the objects of opinion. So they are the objects of understanding if not the objects of opinion. If these are the only kinds of cognitive attitudes, then it follows that among cognitive attitudes the self-subsisting Forms are the objects of understanding alone.

Second, consider the claim that if true opinion does not differ in kind from understanding, then all things that we perceive with corporeal instruments must be judged most stable. 

This second claim introduces an idea not made explicit in the first. Specifically, according to the second claim, the objects of perception must be judged most stable, if opinion does not differ in kind from understanding. Notice that the first claim makes no explicit mention of stability. Presumably, the idea is that if there were in fact self-subsisting Forms, these would be more stable than sensible objects, and this is what would make them eligible as being the objects of an understanding that differs in kind from true opinion. But having rejected self-subsisting Forms, on this alternative, the sensible and the corporeal are the most stable.

The alternatives so recast, Timaeus votes for understanding differing in kind from true opinion, and so in favor of the existence of self-subsisting Forms. Timaeus stands with the Friends of the Forms. And he does so for two reasons.

According to Timaeus, understanding and true opinion must differ in kind for two reasons (51e). First, they come into existence separately. And second, they are unlike in condition.

The first reason for thinking that understanding and true opinion differ in kind concerns the separability of their existence. Separability must be rightly understood. Obviously, two sensible bodies can be generated at different times and so exist separately, on one reasonable understanding of separability. When Timaeus speaks of understanding and true opinion as existing separately, however, he has in mind a difference in their source or mode of generation. Thus whereas understanding arises in us as a result of teaching, opinion arises in us as a result of persuasion. The idea is that things that differ in kind differ in their characteristic source or mode of generation. So since understanding and true opinion differ in their characteristic source or mode of generation, it is plausible to suppose that they differ kind.

Moreover, understanding true opinion are unlike in condition. Understanding is always accompanied with reasoning, whereas opinion is not. Understanding is is unalterable by persuasion, while opinion is alterable by persuasion. Everyone partakes of opinion, but only the Gods and a few mortals partake of understanding. Since understanding and true opinion differ in condition in this way, again, it is plausible that they differ in kind. And the two reasons taken together naturally reinforce one another.

Since understanding differs in kind from true opinion, Timaeus concludes that the object of understanding must differ from the object of true opinion. Genuine understanding must have an object. Moreover genuine understanding exists and is at least possible for mortal beings. And since that which always is and never becomes was explicated in terms of being the object of understanding, and since self-subsisting Forms always are and never become, Timaeus concludes that self-subsisting Forms exist as the potential objects of understanding. So in addition to the sensible phenomena, there are Forms from which these images were produced and copied.

Timaeus concludes this discussion with a summary statement of the three ontological kinds. 

One kind is self-subsisting Form, ungenerated and imperishable. This kind never receives anything from anywhere else nor itself passes into any other place. This kind is invisible. And Timaeus makes it explicit that he means by this that it is in no way perceptible to sense. The insensibility of this kind is related to its epistemic status. This kind is insensible since it is the object of understanding.

The second kind is that which is named after the first and is similar to it. This second kind is produced and copied from the first, in a marvellous manner, and it is for this reason that is is named after the first and similar to it. Unlike the first, the second kind is generated and perishable. There is a place wherein it comes into being and a place wherefrom it departs. Moreover, the second kind is sensible. And again this is related to its epistemic status. The second kind is apprehensibly by opinion by means of perception. 

The first two kinds were explicitly postulated by the \emph{proemium}. The third is the ontological novelty. Here, Timaeus identifies the third kind with ever existing place. Like the first kind, the third is imperishable. Though he does not here make it explicit, later Timaeus will claim that this kind existed even before the Heavens came into existence (52d). So it is plausible that the third kind, like the first, is not only imperishable but ungenerated. Unlike the first kind, however, it receives things from elsewhere since it provides room for all Becoming.

At this point in his description of the first and second kinds, Timaeus would make claims about their sensibility or insensibility and the consequences of this for their epistemic status. And similar claims follow suit with respect to the third kind. It is worth paying careful attention to Timaeus' language here. The third kind is neither the object of understanding nor opinion and yet Timaeus does not introduce any novel epistemological attitudes. Specifically, Timaeus claims that the third kind is apprehensible by a bastard reasoning, with the aid of non-perception, hardly an object of belief. Several observations are in order here.

First, it is hard not to be struck by the self-consciously aporetic character of Timaeus' claims here. Notice how Timaeus echoes the formula for the epistemic status of the second kind even as denies this status of the third kind. The third kind is apprehensible with the aid of non-perception. (Compare Butler's description of moral conscience as the sentiment of the understanding and the perception of the heart.) 

Second, Timaeus approach to accommodating this third kind in the epistemological framework of the \emph{proemium} would seem to be an endorsement of a kind of apophatic or negative epistemology. Just as apophatic or negative theology advances only claims about what God is not, Timeaus' apophatic epistemology of the third kind would seem to advance only claims about how this third kind is not apprehended. This third kind is not apprehensible through a non-bastardized reasoning, nor is it apprehensible through perception, nor is it the object of opinion. One may well worry that apophatic epistemology is a fig leaf for the non-comprehensiveness of the epistemology of the \emph{proemium}.

The epistemology of the third kind is not completely apophatic, however. The third kind is apprehensible in dreams. Dreams, here, should not be regarded as a third cognitive attitude in addition to understanding and opinion. Dreams, to be sure, are not to be identified with understanding or true opinion. Dreams are sensory experiences and not cognitive attitudes. Understanding is non-sensory, and while opinion is linked with the sensory, it is perception and not dreams upon which opinion depends. But what Timaeus claims is that when we regard this third kind we dimly dream and affirm its existence. Timaeus is not claiming that the existence of the Receptacle came to him in a dream at night. He is not claiming that the third kind is the object of a sensory experience undergone when asleep. The Receptacle is insensible, and this not only rules it out as the object of perception but plausibly as the object of sensory experience more generally. Rather, he is claiming that its existence is apprehensible, if at all by mortal beings, in a dream-like manner. This becomes explicit when Timaeus later describes our waking condition as dream-like.

What we dimly dream and affirm is that it is somehow necessary that everything in the Cosmos must exist in some place. This is the commitment of the primordial corporeal metaphysics, dramatized by the Eleatic Stranger as the Giants opposition to the Gods in the Gigantomachy. That which is neither on Earth nor in the Heavens is nowhere, according to Timaeus, and thus nothing, the Giants further insist. Timaeus and the Giants alike maintain that nothing sensible and corporeal exists within the Cosmos without place. 

Importantly, it is we who dimly dream and affirm is that it is somehow necessary that everything in the Cosmos must exist in some place. Note the use of the the first person plural. Timaeus is reminding his audience, and by extension us, of a primordial commitment that we have already brought to the conversation. He is not trying to persuade us to take up this commitment or somehow instill it in us. It is something that we are already committed to. The source of this commitment is what is obscure, and unclarity about it partly accounts for its dream-like character. The commitment is of an uncertain or ambiguous heritage. This will affect the results of reasoning from this commitment. And so the conclusion of Timaeus' own reasoning from this and what it takes to be the object of deictic reference is itself of uncertain or ambiguous heritage and hence the product of a bastard reasoning.

It is this primordial commitment, to there being a place wherein the sensible and the corporeal come to be, that goes beyond what may be claimed in a strictly apophatic epistemology of the third kind. 

\section{The Pre-Cosmic Chaos} % (fold)
\label{sec:the_pre_cosmic_chaos}

The pre-cosmic chaos is the state of the Receptacle and its contents prior to the Demiurgic imposition of form and number. 

Ovid, in the \emph{Metamorphoses}, was the first person to describe the state of the pre-cosmos in terms of the divinity, Chaos. It is true that Hesiod in the \emph{Theogeony} claims that Chaos was first to exist, but Chaos is not, for Hesiod, the personification of the disorderly contents of the pre-cosmic Receptacle---that distinctively Timaean notion did not so much as exist---but it could not even anachronistically be taken to be such a personification. Ovid, by contrast, gives a fair approximation of Timaeus' vision:
\begin{verse}
	An age there was, before the land and sea\\
	And sky, that covers all began to be,\\
	When nature's face was blank, and what they call\\
	Chaos, a crude unsorted mass, was all---\\
	A mere dead-weight, whose atoms, which could keep\\
	No confirmation, huddled in a heap.\\
	(Ovid, \emph{Metamorphoses} 1 5--10; \citealt[1]{Watts:1980aa})
\end{verse}

The Receptacle is ignified as it receives traces of the Form of fire and so the fiery appears within its embrace. In being ignified, the Receptacle is not set ablaze, it is only that the fiery appears in it. The fiery, here, is a sensible power. It is not properly fire. Sensible and corporeal fire is the result of the Demiurge imposing the form of a tetrahedron on traces of the fiery. It is only then that what was fiery is enough like the Form of fire to be called by its name. Not only is the Receptacle ignified, but it is also liquified, as it receives traces of the Form of water. And similarly for air and earth. So Timaeus holds that there are Forms of the Empedoclean ``roots'' and that their traces constitute sensible powers in the Receptacle. The Receptacle not only receives these traces but also affections that accompany them. It is not clear to me exactly what Timaeus has in mind here. Are the affections simply those that arise as the the trace sensible powers interact? Or are these, somehow, the pre-cosmic anticipation of secondary bodies? The matter is unclear. Nevertheless, the trace sensible powers and their accompanying affections give rise to a variety of appearances. Allow me to make three observations.

First, let me emphasize again that Timaeus, unlike young Socrates in the \emph{Parmenides}, harbors no doubt or uncertainty that there are Forms of fire, air, water, and earth. 

Second, the mechanism by which these Forms leave traces in the receptacle is not specified. They are somehow stamped in a marvellous manner. But there are other questions, as well in the area. Are the sensible powers traces of only fire, air, water, and earth? The trace of no other Forms are mentioned. But if so, how is it that these Forms stamp, in a marvellous manner, the trace sensible powers but no others do? An account of how the Forms stamp impressions, in a marvellous manner, on the Receptacle would perhaps explain this.

Third, the traces of the Forms of fire, air, water, and earth are sensible powers. These powers and their accompanying affections give rise to a variety of appearances. So sensible powers can exist in the absence of perceivers. There is no question, then, of attributing to Timaeus, the Secret Doctrine that Socrates attributes to Protagoras in the \emph{Theatetus}. There, it is maintained that perception and its object are ``twin births'', (\citealt{Cornford:1935fk} claims that Timaeus accepts, in essentials, the Protagorean account, a claim sympathetically defended by \citealt{OBrien:1984ji}.) But if there can exist sensible powers without perceivers, and these powers are potential objects of perception, then perception and its object are not ``twin births''.

Not only are there sensible powers in the pre-cosmic chaos, but motion as well. The sensible powers are unlike and unbalanced. They are unlike since they are traces of different Forms of primary bodies. They are the traces of the Forms of fire, air, water, and earth. That they are unbalanced in the pre-cosmic chaos marks a potential contrast with Empedocles (DK 31B17, see \citealt[199 n2]{Cornford:1935fk}). Since the trace sensible powers are unlike and unbalanced, they are not in equipoise but are, instead, in constant motion. Allow me to make three observations.

First, the pre-cosmic chaos plausibly exists prior to, not only the generation of the body of the Cosmos, but the generation of the World Soul as well. It is hard, then, to interpret Timaeus as subscribing to the view that soul is the principle of motion (\emph{Leges}). Plutarch

Second, there is a contrast, here, as well with the \emph{Sophistes}. The Eleatic Stranger posited five greatest kinds---Being, Sameness, Difference, Rest, and Motion. Timaeus, however, seems to explain Motion in terms of Difference and Rest in terms of Sameness. If that is right, then Rest and Motion are not explanatorily basic the way they would be if they were among the greatest kinds.  

Third, that the trace sensible powers are unlike and unbalanced explains their initial motion. However, the full character of the motion of the disorderly contents of the Receptacle is explained by a reciprocal interaction between the moving contents, on the one hand, and the Receptacle itself, on the other. We shall consider the fuller explanation. But first consider Timaeus' explanation of the initial motion of the trace sensible powers. Being unlike and unbalanced, they are incapable of equipoise and so are in constant motion. Later Timaeus will make explicit the principle behind this explanation (57d--e). According to Timaeus, motion cannot exist in a state of homogeneity. Thus if the contents of the Receptacle were homogenous, then they would be at rest. Why does motion presuppose heterogeneity? Presumably, Timaeus is drawing out a consequence of Eleatic doctrine. Parmenides held that Being was homogenous and motionless. It is as if Timaeus concludes that if there is motion, then there must be heterogeneity. Heterogeneity is required since motion presupposes the distinction between mover and moved, the agent of motion and that which is subject to it. Perfect homogeneity would obliterate these differences and so is inconsistent with the possibility of motion. In this way is motion explained in terms of Difference and rest in terms of Sameness. To return to the chaotic contents of the pre-cosmic Receptacle, it is the trace sensible powers being unlike and unbalanced that is the source of heterogeneity that results in their initial motion.

The full character of the disorderly motion of the content of the Receptacle is not explained by their being unlike and unbalanced alone. This heterogeneity may explain their initial motion, but this initial motion will have further affects that partly determine the full character of the disorderly motion. THe trace sensible powers are unlike and unbalanced. So profound is their lack of equipoise, that Timaeus represents the Receptacle as swaying unevenly and being shook by the motions of its contents. Just as the heterogenous content shakes the Receptacle, the Receptacle, in turn, shakes its heterogenous content. 

% section the_pre_cosmic_chaos (end)

% section the_receptacle (end)

\section{Elemental triangles} % (fold)
\label{sec:elemental_triangles}



% section elemental_triangles (end)

\section{Primary Bodies} % (fold)
\label{sec:primary_bodies}



% section primary_bodies (end)

\section{Secondary Bodies} % (fold)
\label{sec:secondary_bodies}



% section secondary_bodies (end)

% Chapter gene (end) 