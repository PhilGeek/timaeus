%!TEX root = /Users/markelikalderon/Documents/Git/timaeus/timaeus.tex

\chapter{\emph{Genē}} % (fold)
\label{cha:gene}

\section{The Cosmological Project} % (fold)
\label{sec:the_cosmological_project}

Cosmology, at least in antiquity, is comprehensive, taking as its object the All. Cosmology aims to account for the whole of it, at least in principle. Given the sensible diversity of the All, to account for the whole of it would be it to account for the entire range of sensible phenomena. There is a challenge, here, since cosmologies tend to operate with a limited number of principles. So to establish the adequacy of a cosmology would be to show how our experience of the entire range of sensible phenomena can be explained in terms of the limited number of principles postulated by that cosmology.

It is easy to see how this challenge can degrade into conflict as a result of impotence. If the principles cannot explain our experience of the entire range of sensible phenomena, and are to that extent impotent, then a conflict arises between the scientific image of nature, encapsulated by the principles of that cosmology, and the manifest image of nature, the sensible phenomena as we experience them to be but the principles fail to explain. A cosmologist facing such a conflict, should they be more confident in their principles then the phenomena that they fail to explain, may attempt to defend their claim to comprehensiveness by dismissing the recalcitrant sensible phenomena as illusory or otherwise misleading.

Timaeus aims to meet the challenge, at least on the whole. If successful, on his own terms, then the challenge, being met, does not degenerate into conflict. It is true that Timaeus is willing to criticize common ways of talking. For example, he thinks that are temporal language is confused. Moreover, as we shall see, Timaeus is willing to criticize common ways of conceiving sensible phenomena. Thus, for example, as we shall see, while the Empedoclean ``roots'' appear to transform into one another, this is misleading. While three of the ``roots'' may transform into one another, a fourth does not. Nevertheless, while the cycle of elemental transformation is a misleading approximation of the truth, Timaeus' account explains why it should appear as if it were so. So, on the whole, Timaeus attitude toward sensible phenomena is broadly ecumenical and conciliationist. 

Timaeus thus aims to explain the sensible diversity of the All. This is strikingly exemplified in his account of the \emph{genē}. There are four primary \emph{genē}---fire, air, water, and earth. And there is variability in kind within the primary \emph{genē}. Thus, as we have seen, there are different kinds of fire---flame, a mild light that does not burn, and that which is retained in embers. Every sensible phenomena that we experience is an aggregate of these variable kinds of \emph{genē}. Timaeus posits two kinds of elemental triangles from which he derives five regular polyhedra. The simplest four of these are shapes imposed upon powers in the Receptacle by the Demiurge to constitute the primary bodies. The variable kinds within the primary \emph{genē} are explained in terms of variable sizes of triangles that compose them. So given two kinds of elemental triangles, the Demiurge imposes form and number upon the pre-cosmic Chaos, to give rise to primary and secondary bodies. Timaeus, account will only be comprehensive if the the sensible diversity of the secondary bodies (the primary bodies being too small to be visible) matches the sensible diversity of the world as we experience it to be.

In explaining the sensible diversity that we experience in terms of elemental triangles that construct regular polyhedra and the way that these aggregate, Timaeus' cosmology invites comparison to the atomism of Leucippus and Democritus. Upon reflection, the similarities are superficial and the differences striking. That there should be differences is perhaps unsurprising. Diogenes Laertius reports Aristoexenus' claim that Plato wished to burn all of the writings of Democritus (\emph{Vitae Philosophorum} 9.40). Even if apocryphal, the competitive attitude expressed by the anecdote may have been genuine. Though Plato never directly discusses Democritus, his student Aristotle does extensively, and we can be sure that Plato was familiar with the writings that Aristoexenus purports he wished to burn. One difference is that the atomists do not understand the sensible Cosmos to be governed by \emph{nous}, and so for the best, the way Timaeus does and Socrates despaired of (\emph{Phaedo} 96a6–99d2). Thus the Demiurge, associated with \emph{nous}, makes no appearance, even under a distinct avatar, in an atomist cosmology. Atoms exist within the void, but Timaues maintains that there is neither void within the Cosmos nor without. The Cosmos, is, instead, a qualified plenum. Moreover, the atomists posited nothing like the Receptacle, that in which the Paradigm is imaged, a truly Timaean innovation. Importantly, for our purposes, there is also a difference in their attitudes toward the sensible. While Timaeus' attitude toward the sensible is broadly ecumenical and conciliationist, the atomists' attitude is not (for a comparison of Platonic and atomist epistemologies see \citealt{Lee:2005qr}). Thus consider Sextus Empiricus' dramatic report of Democritus' attitude toward the sensible:
\begin{quote}
	And Democritus in some places abolishes things that appear to the senses and asserts that none of them appears in truth but only in opinion, the true fact in things existent being the existence of atoms in the void; for ``By convention,'' he says, ``is sweet, by convention bitter, by convention hot, by convention cold, by convention color; but by verity atoms and void.'' (Sextus Empiricus, \emph{Adversus Mathematicos} 7; \citealt{Bury:1997uq})
\end{quote}
Clearly, abolishing things that appear to the senses is not an ecumenical and conciliationist attitude toward the sensible, even broadly construed.

The apparent cycle of elemental transformation raises a puzzle or \emph{aporia} whose resolution requires the postulation of the Receptacle. We shall begin by discussing this \emph{aporia} and then discuss the construction of the regular polyhedra from elemental triangles and their assignment to the four primary \emph{genē}, and the sensibly diverse secondary bodies that these give rise to. 

% section the_cosmological_project (end)

\section{An \emph{Aporia}} % (fold)
\label{sec:an_emph_aporia}

The \emph{aporia} begins with the phenomena of flux. The relevant notion of flux, however, is restricted to the cycle of elemental transformation: the tendency for the four primary bodies---the Empedoclean ``roots'', fire, air, water, and earth---to transform into one another. The relevant example of flux, the cycle of elemental transformation, is of pre-Socratic provenance. It can be found in Anaximenes (Hippolytus, \emph{Refutatio omnium haeresium} 1.7.1--3 = DK 13A7), Heraclitus (DK 22B31, 22B36), Melissus (Simplicius, \emph{In Aristotelis De caelo commentaria} 558.19--559.12 = DK 30B8), and Anaxagoras (Simplicius, \emph{In physica} 179.8 = DK 59B16). The puzzle concerns the possibility of naming or linguistic reference. Specifically, if the Empedoclean ``roots'' are continually transforming into one another, then they fail to provide a sufficiently stable object for deictic reference. Concerning any primary body, it is not possible to refer to it as ``this'' (\emph{touto}) if it is continually transforming into another primary body. At best, we must describe it as ``such like'' (\emph{toiouto}). The primary bodies are less substances than recurring kinds (49e5). The primary bodies are thus not enduring objects that can be identified and re-identified but are rather recurring phases in the cycle of qualitative flux.

The puzzle has antecedents in the \emph{Cratylus} (439d--e) and the \emph{Theaetetus} (181c--183b). The \emph{Cratylus} and the \emph{Theaetetus} each has as its target the very coherence of the doctrine of total flux---that everything is changing in every respect at all times. This is a strong claim, and its rejection is consistent with Becoming being reconceived as partial flux, arguably one of the tasks of the \emph{Timaeus}. It is useful to compare, and importantly to contrast, the present puzzle with these earlier treatments. 

In the \emph{Cratylus}, Socrates presents three arguments against the doctrine of total flux. Specifically, that doctrine is said to raise linguistic, ontological, and epistemological problems. Toward the end of that dialogue (439c), Socrates shares a dream that he often has of the Forms---Beauty in itself, Goodness in itself, and each of the things that are. Whereas a human face does not always retain its beauty---whether through the ravages of time, disease, or some tragic accident---true beauty is invariably beautiful. The Form of Beauty is itself invariably beautiful. If it were not, it would not be possible to say of it that it is a ``this'' or that it is ``such-like'' (439d). Thus, if the Form of Beauty were not invariably beautiful, then having discovered true beauty, one could not say of it ``This is beautiful''.  Moreover, echoing Parmenides, Socrates maintains that that anything always changing could not properly be said to be (439e). Finally, Socrates applies to the Forms the Eleatic doctrine that knowledge requires stable objects. Nothing variable can be known by anyone, for as one approaches it as a matter of inquiry, it changes from what it was (440a-e). Thus in inquiring into true beauty only the Form of Beauty could be known because of all the beautiful things only that Form is invariable and so has the requisite stability to be a potential object of knowledge. A couple of observations are already pertinent. First, in the \emph{Cratylus}, total flux is said to be inconsistent with, not only naming, but predication as well. Second, that doctrine not only raises linguistic difficulties, but ontological and epistemological difficulties as well.

The linguistic difficulty is raised again in the \emph{Theaetetus}. Again the issue is how to coherently describe something in the process of continual change in every respect. Again Plato emphasizes that this process continues even as we speak in attempting to describe it (\emph{Cratylus} 439d10, \emph{Theaetetus} 182d7). There are, however, differences. The ontological problem is not raised in the \emph{Theaetetus}. While, in that dialogue, Eleatic monism is opposed to Ionian Heracliteanism, no Eleatic assumptions about the nature of Being are directly leveraged against the doctrine of total flux. The epistemological concern of the \emph{Cratylus}, that something continually changing in every respect could not be the object of knowledge, is not explicitly raised in the \emph{Theaetetus}. This is reasonably since, in that dialogue, what counts as knowledge is precisely what is at issue. 

However, one may wonder whether this specific epistemological concern, while not made explicit in the \emph{Theaetetus}, is, nonetheless, implicit in the overall dialectical structure of that dialogue. Notoriously, \emph{Parmenides} raises six \emph{aporiai} concerning the Forms, and the \emph{Theaetetus} makes no mention of the Forms. Perhaps the \emph{Theaetetus} may be read as beginning with the assumption that there are no Forms and drawing out the conclusion that no adequate definition of knowledge is to be had. This, at the very least, approaches Plato's Eleatic conviction that knowledge requires stable objects along with his further insistence that only the Forms have the requisite stability. \citet{Cornford:1951ei} and \citet{Kahn:2013ob} offer interpretations of this kind. This issue, however, is beyond the scope of the present essay.

In the \emph{Theaetetus}, the doctrine of total flux is expressed as everything being in motion. Socrates begins by clarifying that motion, here, cannot mean narrowly spatial motion, locomotion, but must mean change of any kind including growth and alteration (181d). Socrates goes on to emphasize this by coining a new term \emph{poiotēs}, literally what-sort-ness, for the quality that is changed in alteration (182a). (The English word ``quality'' derives from Cicero's Latin translation of Plato's Greek: ``\emph{Qualitates igitur appellavi, quas Graeci} {\sbl ποιότητας} \emph{appellant, quod ipsum apud Graecos non est vulgi verbum, sed philosophorum; atque id in multis. Dialecticorum vero verba nulla sunt publica, suis, utuntur. Et id quidem commune omnium fere est artium; aut enim nova sunt rerum novarum facienda nomina aut ex aliis transferenda}'' \emph{Academicae Quaestiones} 1.25.) If everything is in flux, then the white that we see is passing over into another color even as we see it. And this raises the difficulty of how we can call it ``white''. For even as we utter that description the color is changing (182d). The difficulty generalizes. If everything is in flux, so are perceptions such as seeing and hearing. And if language requires a stable object, then nothing may be described as seeing or not seeing, or perceiving or not perceiving more generally (182e). Like in the \emph{Cratylus}, though this is not made explicit, the difficulty seems general, applying equally to naming and predication.

The \emph{Cratylus} and the \emph{Theaetetus} reject the doctrine of total flux as incoherent. But Plato does not restrict Being to what is invariable. For example, in the \emph{Philebus} 27b8, Plato seems to extend the notion of Being to what comes to be. So it would seem that while Plato rejects the doctrine of total flux as incoherent, he is amenable to a world of partial flux (on Plato's persistent attachment to flux see Aristotle \emph{Metaphysics} A 6). However, the \emph{Cratylus} and the \emph{Theaetetus}, while clear about the incoherence of the doctrine of total flux, remain silent about any successor notion. The new metaphysical scheme of the \emph{Timaeus}, involving, the Paradigm, its image, and the Receptacle in which that image appears, is meant, in part, to provide the resources for a coherent successor notion of partial flux.

Another difference concerns the scope of the linguistic difficulty. In the \emph{Timaeus} the cycle of elemental transformations is meant to raise a difficulty about naming the primary bodies, fire, air, water, and earth. No explicit difficulty about predication is raised. This seems \emph{prima facie} odd since the relevant linguistic difficulty seems perfectly general. That is to say, if constant change suffices to make naming impossible parallel reasoning would apply equally as well to the case of predication. However, it is worth noting that Timaeus, in his discussion of the gold analogy (50a--c), strikes a cautious note even about describing what comes to be as ``such-like''. We should be content that, here, we can even call such things ``such-like'' with any safety. To protest against Timaeus' linguistic recommendations, then, would be insenstive to the danger of the difficulty generalizing as it does in the \emph{Cratylus} and the \emph{Theaetetus}. But there remains the question why, here, is there sufficient safety even to apply ``such-like''?

Timaeus does not offer an answer to our question. However, his account has the resources to offer an answer, though it is difficult to determine whether it should be credited as TImaeus' answer. Begin with the observation that the cycle of elemental transformation concerns the primary bodies. Fire as apposed to fleeting fiery traces in the pre-cosmic chaos. Later we shall learn that the primary bodies are the result of the Demiurge imposing ``form and number'' on chaotic contents of the Receptacle. Perhaps the sensible traces in the pre-cosmic chaos are too fleeting even for them to be called ``such-like'' with any safety. It is only once the Demiurge imposes ``form and number'' that the primary bodies are generated and have sufficient stability to be called ``such-like'', at least with due caution. Within the Cosmos, primary bodies, despite their variable nature, may be called ``such-like'' thanks to the Demiurgic imposition of ``form and number''.

A further difference also concerns the scope of the linguistic difficulty though along another dimension. Unlike in the \emph{Cratylus} and the \emph{Theaetetus}, Timaeus argues not that ``this'' lacks deictic reference, only that it cannot refer to any of the primary bodies. If we say ``This is fiery'', ``this'' may successfully refer to the place in which that primary body appears. Images of the Forms may come and go in the Receptacle, but the spatial matrix of the Receptacle remains invariant, and so it, or at least parts of it, possess the requisite stability to be the object of deictic reference. If ``This is fiery'' is true, then ``this'' picks out the place wherein the fiery appeared, a place wherefrom it will depart when it perishes, by being quenched or otherwise extinguished.

Timaeus explains his linguistic recommendations in terms of an analogy. We are meant to imagine a craftsman capable of molding any figure with gold constantly working and reworking a hunk of gold into different possible figures. If the craftsman were asked ``What is this?'' the safest answer would be that this is gold. ``This'' could not refer to any of the figures that the craftsman worked since these are always changing. The Receptacle is meant to be analagous to the gold. But to what extent is the analogy apt?

The gold is always called gold, even as the craftsman continually works new figures from it. Similarly, the Receptacle has its own proper name, because it has a constant nature. This nature, from which it never departs, is, Timaues tells us, the capacity to receive all things.

Timaeus begins to elaborate upon the Receptacle's capacity to receive all things, but his remarks are modeled on what a craftsman is looking for in a medium which is to receive a sensible form. The gold that the craftsman works is capable of receiving all the forms that the craftsman imposed. Of course, the Receptacle's capacity to receive is greater still. The gold may be capable of receiving a range of figures, but it is not capable of receiving all things the way that the Receptacle can. There may, for example, be forms too small to be molded in gold. (Think of the form of an elemental triangle.) Sensible media suffer from a material recalitrance in the way that the Receptacle does not. In order to receive all things, the Receptacle does not itself take on the form of anything that appears within it, nor does it possess any other form. Timaeus' stated reason is that taking on the form of what appears in it would interfere with the Receptacle's capacity to recieve the form of what subsequently appears once the initial appearance has departed. Should the Receptacle take on some form, the presence of that form would resist the reception of any form opposed to it.

At this point, Timaeus, while retaining the craft analogy, shifts the relevant craft from a goldsmithing to perfumery. The base of a perfume must be as odour free as the perfumer can make it so as to better receive the scents that will be combined in the base. Just as the gold having a shape of its own would interfere with its ability to receive another shape, so the base of a perfume having a scent of its own would interfere with its receiving other scents. Since the Receptacle is capable of receiving all things, this must be all the more true for it.

% section an_emph_aporia (end)

\section{The Receptacle} % (fold)
\label{sec:the_receptacle}



% section the_receptacle (end)

\section{Elemental triangles} % (fold)
\label{sec:elemental_triangles}



% section elemental_triangles (end)

\section{Primary Bodies} % (fold)
\label{sec:primary_bodies}



% section primary_bodies (end)

\section{Secondary Bodies} % (fold)
\label{sec:secondary_bodies}



% section secondary_bodies (end)

% Chapter gene (end) 