%!TEX root = /Users/markelikalderon/Documents/Git/timaeus/timaeus.tex

\chapter{\emph{Genē}} % (fold)
\label{cha:gene}

\section{A New Start} % (fold)
\label{sec:a_new_start}

If the first part of Timaeus' speech concerned the operation of Reason (27d--47e), the second part of Timaeus' speech concerns, instead, the operation of Necessity (47e--69a). Timaeus' insight is that Cosmos is generated as the joint product of Reasons and Necessity. In the pre-cosmic chaos, prior to the Demiurge's imposition of form and number, Necessity was a wandering cause. In imposing form and number, however, the Demiurge persuades Necessity to act for the best.

% [Why persuasion? Hoackforth suggests that persuasion marks a contrast with violence. On Hackforth's understadning, Reason is not forcing Necessity to act for the best, but nevertheless acheives this reult through discursive persuasion. I wonder, however, if this gets the emphasis of Timaeus' talk of persuasion right. Later in the same section, Timaeus will contrast understanding and opinion. Opinion is based upon perception and so concerns the sensible realm of Becoming. Opinion is subject to persuasion and this marks a contrast with understanding. Understanding may be achieved through a reasoned account but it is not subject to persuasion.]

In describing the works of Reason in the first part, Timaeus distinguished the intelligible model, the Paradigm, and its sensible copy, the Cosmos. Now, in describing the works of Necessity, Timaeus introduces a third thing, the Receptacle. The Receptacle is baffling and obscure, perhaps more obscure than the difficult to discover deity impossible to explain to all. The Receptacle is that in which the Paradigm is imaged. Reflection on elemental transformation reveals the existence of the Receptacle, as the invariant place in which the sensible fleetingly appears and so as the only invariant thing that may be the deictic reference of ``this''. We also learn that the Receptacle is receptive to all and that this requires that it be without form. And while the Receptacle is the place in which the sensible appears it is not itself sensible and so is invisible like the intelligible Paradigm.

% section a_new_start (end)

\section{The Cosmological Project} % (fold)
\label{sec:the_cosmological_project}

Cosmology, at least in antiquity, is comprehensive, taking as its object the All. Cosmology aims to account for the whole of it, at least in principle. Given the sensible diversity of the All, to account for the whole of it would be it to account for the entire range of sensible phenomena. There is a challenge, here, since cosmologies tend to operate with a limited number of principles. So to establish the adequacy of a cosmology would be to show how our experience of the entire range of sensible phenomena can be explained in terms of the limited number of principles postulated by that cosmology.

It is easy to see how this challenge can degenerate into conflict as a result of impotence. If the principles cannot explain our experience of the entire range of sensible phenomena, and are to that extent impotent, then a conflict arises between the Scientific Image of Nature, encapsulated by the principles of that cosmology, and the Manifest Image of Nature, the sensible phenomena as we experience them to be but the principles fail to explain. A cosmologist facing such a conflict, should they be more confident in their principles than the alleged phenomena that they fail to explain, may attempt to defend their claim to comprehensiveness by dismissing the recalcitrant sensible phenomena as illusory or otherwise misleading.

Timaeus aims to meet the challenge, at least on the whole. If successful, on his own terms, then the challenge, being met, does not degenerate into conflict. It is true that Timaeus is willing to criticize common ways of talking. For example, he thinks that are temporal language is confused. Moreover, as we shall see, Timaeus is willing to criticize common conceptions of sensible phenomena. Thus, for example, as we shall see, while the Empedoclean ``roots'' appear to transform into one another, this is misleading. While three of the ``roots'' may transform into one another, a fourth does not. Nevertheless, while the cycle of elemental transformation is a misleading approximation of the truth, Timaeus' account explains why it should appear as if it were so. So, on the whole, Timaeus' attitude toward sensible phenomena is broadly ecumenical and conciliationist. It is a qualified endorsement of the Manifest Image of Nature.

Timaeus thus aims to explain the sensible diversity of the All. This is strikingly exemplified in his account of the \emph{genē}. There are four primary \emph{genē}---fire, air, water, and earth. And there is variability in kind within the primary \emph{genē}. Thus, as we have seen, there are different kinds of fire---flame, a mild light that does not burn, and that which is retained in embers. Every sensible phenomena that we experience is an aggregate of these variable kinds of \emph{genē}. Timaeus posits two kinds of elemental triangles from which he derives five regular polyhedra. The simplest four of these are shapes imposed upon powers in the Receptacle by the Demiurge to constitute the primary bodies. The variable kinds within the primary \emph{genē} are explained in terms of variable sizes of triangles that compose them. So given two kinds of elemental triangles, the Demiurge imposes form and number upon the pre-cosmic chaos, to give rise to primary and secondary bodies. Timaeus, account will only be comprehensive if the the sensible diversity of the secondary bodies (the primary bodies being too small to be visible) matches the sensible diversity of the world as we experience it to be.

In explaining the sensible diversity that we experience in terms of elemental triangles that construct regular polyhedra and the way that these aggregate, Timaeus' cosmology invites comparison to the atomism of Leucippus and Democritus. Upon reflection, the similarities are superficial and the differences striking. That there should be differences is perhaps unsurprising. Diogenes Laertius reports Aristoexenus' claim that Plato wished to burn all of the writings of Democritus (\emph{Vitae Philosophorum} 9.40). Even if apocryphal, the competitive attitude expressed by the anecdote may have been genuine. Though Plato never directly discusses Democritus, his student Aristotle does extensively, and we can be sure that Plato was familiar with the writings that Aristoexenus purports he wished to burn. One difference is that the atomists do not understand the sensible Cosmos to be governed by \emph{nous}, and so for the best, the way Timaeus does and Socrates despaired of (\emph{Phaedo} 96a6–99d2). Thus the Demiurge, associated with \emph{nous}, makes no appearance, even under a distinct avatar, in an atomist cosmology. Atoms exist within the void, but Timaues maintains that there is neither void within the Cosmos nor without. The Cosmos, is, instead, a qualified plenum. Moreover, the atomists posited nothing like the Receptacle, that in which the Paradigm is imaged, a truly Timaean innovation. Importantly, for our purposes, there is also a difference in their attitudes toward the sensible. While Timaeus' attitude toward the sensible is broadly ecumenical and conciliationist, the atomists' attitude is not (for a comparison of Platonic and atomist epistemologies see \citealt{Lee:2005qr}). Thus consider Sextus Empiricus' dramatic report of Democritus' attitude toward the sensible:
\begin{quote}
	And Democritus in some places abolishes things that appear to the senses and asserts that none of them appears in truth but only in opinion, the true fact in things existent being the existence of atoms in the void; for ``By convention,'' he says, ``is sweet, by convention bitter, by convention hot, by convention cold, by convention color; but by verity atoms and void.'' (Sextus Empiricus, \emph{Adversus Mathematicos} 7; \citealt{Bury:1997uq})
\end{quote}
Clearly, abolishing things that appear to the senses is not an ecumenical and conciliationist attitude toward the sensible, even broadly construed.

The apparent cycle of elemental transformation raises a puzzle or \emph{aporia} whose resolution requires the postulation of the Receptacle. We shall begin by discussing this \emph{aporia} and then discuss the construction of the regular polyhedra from elemental triangles and their assignment to the four primary \emph{genē}, and the sensibly diverse secondary bodies that these give rise to. 

% section the_cosmological_project (end)

\section{An \emph{Aporia}} % (fold)
\label{sec:an_emph_aporia}

The \emph{aporia} begins with the phenomena of flux. The relevant notion of flux, however, is restricted to the cycle of elemental transformation: the tendency for the four primary bodies---the Empedoclean ``roots'', fire, air, water, and earth---to transform into one another. The relevant example of flux, the cycle of elemental transformation, is of pre-Socratic provenance. It can be found in Anaximenes (Hippolytus, \emph{Refutatio omnium haeresium} 1.7.1--3 = DK 13A7), Heraclitus (DK 22B31, 22B36), Melissus (Simplicius, \emph{In Aristotelis De caelo commentaria} 558.19--559.12 = DK 30B8), and Anaxagoras (Simplicius, \emph{In physica} 179.8 = DK 59B16). The puzzle concerns the possibility of naming or being the object of demonstration. Specifically, if the Empedoclean ``roots'' are continually transforming into one another, then they fail to provide a sufficiently stable object for deictic reference. Concerning any primary body, it is not possible to refer to it as ``this'' (\emph{touto}) if it is continually transforming into another primary body. At best, we must describe it as ``such like'' (\emph{toiouto}). The primary bodies are less substances than recurring kinds (49e5). The primary bodies are thus not enduring objects that can be identified and re-identified but are rather recurring phases in the cycle of qualitative flux.

Within the Platonic corpus, the puzzle has antecedents in the \emph{Cratylus} (439d--e) and the \emph{Theaetetus} (181c--183b). The \emph{Cratylus} and the \emph{Theaetetus} each has as its target the very coherence of the doctrine of total flux---that everything is changing in every respect at all times. This is a strong claim, and its rejection is consistent with Becoming being reconceived as partial flux, arguably one of the tasks of the \emph{Timaeus}. It is useful to compare, and importantly to contrast, the present puzzle with these earlier treatments. 

In the \emph{Cratylus}, Socrates presents three arguments against the doctrine of total flux. Specifically, that doctrine is said to raise linguistic, ontological, and epistemological problems. Toward the end of that dialogue (439c), Socrates shares a dream that he often has of the Forms---Beauty in itself, Goodness in itself, and each of the things that are. Whereas a human face does not always retain its beauty---whether through the ravages of time, disease, or some tragic accident---true beauty is invariably beautiful. The Form of Beauty is itself invariably beautiful. If it were not, it would not be possible to say of it that it is a ``this'' or that it is ``such-like'' (439d). Thus, if the Form of Beauty were not invariably beautiful, then having discovered true beauty, one could not say of it ``This is beautiful''.  Moreover, echoing Parmenides, Socrates maintains that that anything always changing could not properly be said to be (439e). Finally, Socrates applies to the Forms the Eleatic doctrine that knowledge requires stable objects. Nothing variable can be known by anyone, for as one approaches it as a matter of inquiry, it changes from what it was (440a-e). Thus in inquiring into true beauty only the Form of Beauty could be known because of all the beautiful things only that Form is invariable and so has the requisite stability to be a potential object of knowledge. A couple of observations are already pertinent. First, in the \emph{Cratylus}, total flux is said to be inconsistent with, not only naming, but predication as well. Second, that doctrine not only raises linguistic difficulties, but ontological and epistemological difficulties as well.

The linguistic difficulty is raised again in the \emph{Theaetetus}. Again the issue is how to coherently describe something in the process of continual change in every respect. Again Plato emphasizes that this process continues even as we speak in attempting to describe it (\emph{Cratylus} 439d10, \emph{Theaetetus} 182d7). There are, however, differences. The ontological problem is not raised in the \emph{Theaetetus}. While, in that dialogue, Eleatic monism is opposed to Ionian Heracliteanism, no Eleatic assumptions about the nature of Being are directly leveraged against the doctrine of total flux. The epistemological concern of the \emph{Cratylus}, that something continually changing in every respect could not be the object of knowledge, is not explicitly raised in the \emph{Theaetetus}. This is reasonably since, in that dialogue, what counts as knowledge is precisely what is at issue. 

% However, one may wonder whether this specific epistemological concern, while not made explicit in the \emph{Theaetetus}, is, nonetheless, implicit in the overall dialectical structure of that dialogue. Notoriously, \emph{Parmenides} raises six \emph{aporiai} concerning the Forms, and the \emph{Theaetetus} makes no mention of the Forms. Perhaps the \emph{Theaetetus} may be read as beginning with the assumption that there are no Forms and drawing out the conclusion that no adequate definition of knowledge is to be had. This, at the very least, approaches Plato's Eleatic conviction that knowledge requires stable objects along with his further insistence that only the Forms have the requisite stability. \citet{Cornford:1951ei} and \citet{Kahn:2013ob} offer interpretations of this kind. This issue, however, is beyond the scope of the present essay.

In the \emph{Theaetetus}, the doctrine of total flux is expressed as everything being in motion. Socrates begins by clarifying that motion, here, cannot mean narrowly spatial motion, locomotion, but must mean change of any kind including growth and alteration (181d). Socrates goes on to emphasize this by coining a new term \emph{poiotēs}, literally what-sort-ness, for the quality that is changed in alteration (182a). (The English word ``quality'' derives from Cicero's Latin translation of Plato's Greek: ``\emph{Qualitates igitur appellavi, quas Graeci} {\sbl ποιότητας} \emph{appellant, quod ipsum apud Graecos non est vulgi verbum, sed philosophorum; atque id in multis. Dialecticorum vero verba nulla sunt publica, suis, utuntur. Et id quidem commune omnium fere est artium; aut enim nova sunt rerum novarum facienda nomina aut ex aliis transferenda}'' \emph{Academicae Quaestiones} 1.25.) If everything is in flux, then the white that we see is passing over into another color even as we see it. And this raises the difficulty of how we can call it ``white''. For even as we utter that description the color is changing (182d). The difficulty generalizes. If everything is in flux, so are perceptions such as seeing and hearing. And if language requires a stable object, then nothing may be described as seeing or not seeing, or perceiving or not perceiving more generally (182e). Like in the \emph{Cratylus}, though this is not made explicit, the difficulty seems general, applying equally to naming and predication.

The \emph{Cratylus} and the \emph{Theaetetus} reject the doctrine of total flux as incoherent. But Plato does not restrict Being to what is invariable. For example, in the \emph{Philebus} 27b8, Plato seems to extend the notion of Being to what comes to be. So it would seem that while Plato rejects the doctrine of total flux as incoherent, he is amenable to a world of partial flux (on Plato's persistent attachment to flux see Aristotle \emph{Metaphysics} A 6). However, the \emph{Cratylus} and the \emph{Theaetetus}, while clear about the incoherence of the doctrine of total flux, remain silent about any successor notion. The new metaphysical scheme of the \emph{Timaeus}, involving, the Paradigm, its image, and the Receptacle in which that image appears, is meant, in part, to provide the resources for a coherent successor notion of partial flux.

Another difference concerns the scope of the linguistic difficulty. In the \emph{Timaeus} the cycle of elemental transformations is meant to raise a difficulty about naming the primary bodies, fire, air, water, and earth. No explicit difficulty about predication is raised. This seems \emph{prima facie} odd since the relevant linguistic difficulty seems perfectly general. That is to say, if constant change suffices to make naming impossible parallel reasoning would apply equally as well to the case of predication. However, it is worth noting that Timaeus, in his discussion of the gold analogy (50a--c), strikes a cautious note even about describing what comes to be as ``such-like''. We should be content that, here, we can even call such things ``such-like'' with any safety. A question naturally arises: why is there sufficient safety, here, even to call such things ``such-like''?

% To protest against Timaeus' linguistic recommendations, then, would be insenstive to the danger of the difficulty generalizing as it does in the \emph{Cratylus} and the \emph{Theaetetus}.

Timaeus does not offer an answer to our question. However, his account has the resources to offer an answer, though it is difficult to determine whether it would be the answer that Timaeus would give were he to explicitly consider our question. Begin with the observation that the cycle of elemental transformation concerns the primary bodies. Fire as apposed to fleeting fiery traces in the pre-cosmic chaos. Later we shall learn that the primary bodies are the result of the Demiurge imposing ``form and number'' on chaotic contents of the Receptacle. Perhaps the sensible traces in the pre-cosmic chaos are too fleeting even for them to be called ``such-like'' with any safety. It is only once the Demiurge imposes form and number that the primary bodies are generated and have sufficient stability to be called ``such-like'', at least with due caution. Within the Cosmos, primary bodies, despite their variable nature, may be called ``such-like'' thanks to the Demiurgic imposition of form and number.

A further difference also concerns the scope of the linguistic difficulty, though along another dimension. Unlike in the \emph{Cratylus} and the \emph{Theaetetus}, Timaeus argues not that ``this'' lacks deictic reference, only that it cannot refer to any of the primary bodies. If ``this'' cannot refer to any of the primary bodies, and it successfully refers, then it must refer to something other than the primary bodies. What then does ``this'' refer to? According to Timaeus, if we say ``This is fiery'', ``this'' may successfully refer to the place in which that primary body appears. Images of the Forms may come and go in the Receptacle, but the spatial matrix of the Receptacle remains invariant, and so it, or at least parts of it, possess the requisite stability to be the object of deictic reference. If ``This is fiery'' is true, then ``this'' picks out the place wherein the fiery appeared, a place wherefrom it will depart when it perishes, by being quenched or otherwise extinguished. ``This'' may refer to the place wherein the fiery appeared, but we should not suppose that the fiery appeared from some other place (like the guest of a talk-show making their appearance from the greenroom). The fiery, when quenched, departs from the place wherein it appeared. And again we should not suppose the fiery has departed to some other place (like the guest subsequently returning to the greenroom after their appearance).

Timaeus is engaged in linguistic revision here. Ordinarily, we would take ``this fire'' to refer to the demonstrated fire and not the place in which the fiery body appeared. This is not the first time in his speech that Timaeus has engaged in linguistic revisionism. Thus, for example, Timeaus thinks that our temporal language is confused. Observing that Timaeus is engaged in linguistic revision, here, is not a coy way of pressing a worry or objection. Philosophers have been known to recommend a number linguistic revisions for a number of reasons. Timaeus departs from ordinary usage for a principled reason. Timaeus sees our ordinary usage as aporetic since it insists that we refer to something that lacks the requisite stability to be the object of demonstration. That Timaeus is prescribing what is by his lights the only intelligible usage rather than describing a pre-existing usage is relevant to assessing his proposal. As we shall see, Aristotle retains the ordinary usage, and some, at least, of his criticisms of Timaeues in \emph{De generatione et corruptione} seem to presuppose this. 

Timaean linguistic revisionism, however, raises a question about the nature of his argument. We are meant to be given an argument for the existence of the Receptacle. But appealing to what a hypothetical usage could so much as refer to seems weak as existence proofs go, even by the standards of a ``likely'' account. 

Though Timaeus' linguistic recommendations depart from ordinary usage, not only is that usage subject to \emph{aporia}, but Timaeus believes that his recommended usage better conforms with a deeply held commitment. (In this regard, Timaeus perhaps deserves the Orwellian epithet, ``Conservative Revisionist''.) Without endorsing a corporealist metaphysics, Timaeus accepts a fundamental corporealist commitment. In the \emph{Sophist}, the Eleatic Stranger transforms the Gigantomachy, the battle for political supremacy over the Cosmos between the Giants, the sons of Gaia and Uranus, and the Olympian Gods, into a metaphysical dispute between corporealists and the Friends of the Forms. The Giants press their case as follows:
\begin{quote}
	One party is trying to drag everything down to earth out of heaven and the unseen, literally grasping rocks and trees in their hands, for they lay hold upon every stock and stone and strenuously affirm that real existence belongs only to that which can be handled and offers resistance to the touch. (Plato, \emph{Sophist} 246a; Cornford in \citealt{Hamilton:1989fk})
\end{quote}
Timaeus, of course, denies that existence belongs only to that which can be handled and offers resistance to touch. Nevertheless, according to Timaeus, the Giants are right at least to this extent: Only that which appears in some place, like every stock and stone, enjoys an encosmic existence. And so while Timaeus' linguistic recommendation departs from ordinary usage, it conforms to a deeply held commitment that captures part of what was right about a primordial corporeal metaphysics. It is really this commitment---that only that with a place may suffer encorsmic corporeal existence---that is being leveraged in Timaeus' case for the Receptacle.

The sensible that appears in our experience within the living Cosmos is, at best, ``such-like''. If ``this'' does in fact refer, it must refer to something insensible. And the Receptacle, like the Living Being and intelligible things more generally, is invisible and insensible. Moreover it is insensible in the way that intelligible things are as opposed to the way that elemental triangles are. Elemental triangles may be too small to be perceived, but aggregates of elemental triangles are perceptible. Elemental triangles are the object of \emph{doxa} or empirical judgment. But the Receptacle is not the object of \emph{doxa} or \emph{pistis}. Rather, it is known, if at all, through a ``bastard reasoning''.

% section an_emph_aporia (end)

\section{Corporeal Media} % (fold)
\label{sec:corporeal_media}

Timaeus introduces at this point a corporeal analogy. The corporeal analogy involves a craftsman, specifically a goldsmith, continually fashioning gold into a range of distinct figures. The analogy is meant to do three things. First, it is meant to lend credence to Timaeus' linguistic recommendations. Second, in so far as it does, it is meant to bolster Timaeus' case for the existence of the Receptacle. Third, the analogy is meant to disclose a substantive claim about the nature of the Receptacle.

We are meant to imagine a goldsmith capable of fashioning any of a range of figures constantly working and reworking a piece of gold into different possible figures. The goldsmith's work is unceasing. As soon as they have fashioned a figure they begin fashioning another. If during this continuing process, the goldsmith were asked ``What is this?'' the safest possible answer would be to say that this is gold. ``This'' could not refer to any of the figures that the goldsmith worked since these are always changing. The only invariant aspect of this process is the gold in which the figures appear.

First, consider how the corporeal analogy is meant to lend credence to Timaeus' linguistic recommendations. If the appearance of the sensible and the corporeal in the spatial matrix of the Receptacle is really analogous to the appearance of figures in the constantly reworked piece of gold, and we accept that the objects of demonstration must have the requisite stability, then we are naturally encouraged to accept the negative and positive claims behind Timaeus' recommendation. The negative claim consists in the thought that there is nothing in the sensible realm of Becoming with the requisite stability to be the object of demonstration. So nothing sensible and corporeal so much as could be the object of demonstration. The positive claim consists in the thought that since the place wherein the sensible and the corporeal appear and wherefrom they depart, like the gold, is an invariant aspect of a process of continual change with sufficient stability to be the object of demonstration. Just as ``this'' could only refer to the gold in the case of the unceasing goldsmith, ``this'' could only refer to the place in which the sensible and the corporeal appear.

There is an apparent puzzle that potentially sheds light on how the analogy is meant to work. First, the puzzle. If we accept that the corporeal analogy supports his linguistic recommendations, then ``this'' in the case of the goldsmith could not even refer to the gold. The gold may be soft and pliable and thus receptive to the various figures that the goldsmith imposes upon it, but it remains itself sensible and corporeal. And the sensible and the corporeal are meant to lack the requisite stability to be the objects of demonstration. If that is right, then not only could ``this'' not refer to the figures fashioned in the gold, but it could not refer to the gold either. We seem to be in the paradoxical position where if we accept the moral of the analogy then the case is not, after all, analogous. 

I am uncertain whether and to what extent this is a problem for Timaeus, even a pragmatic one. The narrative of the unceasing work of the goldsmith might still succeed in reinforcing the idea that instability undermines being the object of demonstration even should it not be strictly speaking analogous. Notice that it is disanalogous in the sense that, by strict Timaean standards, there is no invariant element in the process involving the gold. So any lessons not based upon this invariant element remain secure, such as the way in which variation undermines eligibility for demonstration. Moreover, even if in this way disanalogous, the narrative might still help frame our experience of life within the living Cosmos so as to facilitate recognizing the only invariant thing that we so much as could demonstrate, the place wherein the sensible and the corporeal appear and wherefrom they depart. So these negative and positive claims behind Timaeus' linguistic recommendations arguably survive the failure of his corporeal analogy. 

Second, suppose that Timaeus' corporeal analogy does, in fact, one way or another, lend credence to his linguistic recommendations. How does this bolster his case for the existence of this third kind, distinct from paradigm and image, but that in which the image of the paradigm is formed, the Receptacle? If we accept that nothing sensible and corporeal has sufficient stability to be the object of demonstration, then ``this'' could refer to nothing sensible and corporeal. And while the sensible and the corporeal come to be and pass away, the place wherein they appear and wherefrom they depart remains. As this is the only thing to which ``this'' could refer, if we accept that it does, we must accept the existence of the object of that demonstration, the Receptacle, the Nurse and Mother of all Becoming. While the reasoning here is structured around what it takes to be the object of demonstration, there is an additional element as well. Timaeus in his narrative gets us to independently attend to the place wherein the sensible and the corporeal appear. This indicates an independent positive commitment to something with the requisite stability to be the object of demonstration. Again, Timaeus is leveraging the primordial commitment that animates corporeal metaphysics---that there must be a place in which the sensible and the corporeal appear.

The Receptacle is a uniquely Timaean innovation of his cosmological account. That account is likely. There is no question, then, of demonstrating the existence of the Receptacle in the sense required for the Receptacle to the apprehended in understanding with an account. Part of what prevents Timaeus' reasoning from being a demonstration of the existence of the Receptacle, concerns the status of the commitment to there being a place in which the sensible and the corporeal appear. What is the source of this commitment? It appears to us in dreams, and in myth (the Gigantomachy, at least as retold by the Eleatic Stranger). While dreams may mislead, they are also divinatory, though it may take wisdom to distinguish mundane from the divine, the false from the true. Similarly myths may mislead (think of the banishment of the mimetic poets in \emph{Republic} 10), but they may also succeed in meeting the positive epistemic standard of being ``likely'' in the sense of fitting, fair, natural, or reasonable, even if this standard falls short of what is required for understanding with an account. Some myths may even be divinely inspired. Like divinatory dreams, they are the divine disclosure of some truth. Like divinatory dreams, myth requires interpretation. And interpretation requires wisdom, not only to distinguish the divinatory from the mundane, but to understand the divinely disclosed truth as well. So appearing in dream and myth may be consistent with being true or at least being a reasonable representation of the truth. That the place in which the sensible and the corporeal appear is made explicit as the object of dream and myth introduces an element of ambiguity. Are the dreams in which the Receptacle presents itself mundane or divinatory? Do the myths present divine truths or dangerous falsehoods? While the ambiguity may prompt the mind to wander, genuine wisdom knows the path and can discern the true from the false. The existence of the Receptacle is only revealed in a ``bastard'' reasoning. The source of the conviction that the sensible and the corporeal must appear in some place is ambiguous. If bastards are of ambiguous heritage, then perhaps the reasoning from this conviction with an ambiguous source is at least part of what makes it a ``bastard'' reasoning.

Third, not only is the corporeal analogy meant to lend credence to Timaeus' linguistic recommendations and bolster his case for the existence of the Receptacle, but in terms of it, Timaeus, in addition, makes some substantive claims about the nature of the Receptacle. 

The gold is always called gold, even as the craftsman continually works new figures from it. Similarly, the Receptacle has its own proper name, because it has a constant character. So we learn two things. First, that the Receptacle, like the gold, has its own proper name. It is, at the very least, the object of demonstration. Second, we learn why the Receptacle has its own name. The Receptacle can have a name proper to it because it retains its character through the coming and going of corporeal appearances. That the Receptacle has a constant character and the nature of that character are the substantive claims about the Receptacle introduced by the corporeal analogy. If the analogy is meant to hold, then the gold too has its proper name because of its constant character. Presumably the constant character of the gold consists in its softness and pliability since these explain its receptivity to the figures imposed upon it by the goldsmith. The constant character of the Receptacle is similar if crucially, and puzzlingly, different.

This character, from which it never departs, is, Timaues tells us, the catholic power to receive all things. This is similar to the constant character of the gold since each is a power to receive. In the case of the gold, the gold has the power to receive the figures that the goldsmith could impose upon it. And this because of its corporeal characteristics, its softness and pliability. Throughout being worked upon, the gold retains its softness and pliability, and so retains its ability to receive potential figures imposed upon it by the goldsmith. The Receptacle, we are told, has its proper name because of its character, from which it never departs, the power to receive all things.

The gold's power to receive potential figures imposed upon it by the goldsmith is quotidian, familiar from life within the living Cosmos (at least to the urbanites that were the mostly likely readers of Plato's dialogue). And it is also, importantly limited, as the contrast with the Receptacle makes clear. The gold may have the power to receive a range of potential figures, but it manifestly lacks the great catholic power to receive all things. A corporeal medium's power to receive is inexorably limited by the material recalcitrance of the corporeal medium itself. 

This specific difference with the great catholic power of the Receptacle reveals something general about how Timaeus conceives of powers to receive. Clearly Timaeus conceives of such powers as arrayed from Great to Small, in order of the range of things that they may receive. So understood, the catholic power to receive all things is Great and the power to receive a limited range of potential figures is Small (at least in relation to the catholic power). Nevertheless, Timaeus thinks that by reflecting upon the way that some bodies have the power to receive a limited range of sensible forms will help explicate the Receptacle's great catholic power to receive all things.

 % The Receptacle is meant to be analogous to the gold. But to what extent is the analogy apt?

Timaeus begins to elaborate upon the Receptacle's power to receive all things, but his remarks are modeled on what a craftsman would look for in a corporeal medium which is to receive a sensible form. The gold that the craftsman works is capable of receiving all the figures that the craftsman imposed. Of course, the Receptacle's capacity to receive is greater still. The gold may be capable of receiving a range of figures, but it is not capable of receiving all things the way that the Receptacle can. There may, for example, be forms too small to be molded in gold. (Think of the form of an elemental triangle.) This is a manifestation of the material recalcitrance of gold. Timaeus, however, draws our attention to a different potential limitation. In order for the gold to receive a range of figures, it must not retain any of the figures that were imposed upon it. Were the gold to retain the figure, no new figure could be imposed, and even if it could, it would be distorted by the presence of the retained figure. Not only are the powers to receive arrayed from Great to Small, in proportion with the range of things that they may receive, but this constrained by the range of sensible forms retained by the body in which that power inheres. Any retained sensible form is an obstacle to the corporeal medium taking on a form that would oppose it. So a principle governing corporeal media is that they should not be similar to the forms they are meant to receive.

This principle governing corporeal media holds for the Receptacle as well. In order to receive all things, the Receptacle should not have the form of anything that appears in it. Thus, the Receptacle does not itself take on the form of anything that appears within it, nor does it possess any other form. Sensible form is not predicated of the place wherein it appears. The fiery may fleetingly appear in some place. And that place may appear hot and dry. But the Receptacle is not hot and dry, not even in that place. And so hot and dry are not predicated of the Receptacle. Timaeus' stated reason is that taking on the form of what appears in it would interfere with the Receptacle's capacity to receive new forms. Should the Receptacle take on some form, the presence of that form would resist the reception of any form opposed to it. Or it would receive a new form but only in a manner distorted by the presence of the old form.

The corporeal analogy is thus meant to have substantive lessons about the constant character of the Receptacle and the nature of that character. We have learned that the Receptacle is governed by a principle that governs corporeal media as well---that media should not be similar to the forms it is meant to receive. As a consequence, the constant character of the Receptacle is not among the forms that it receives. The Receptacle receives the whole of the sensible and the corporeal. So the constant character of the Receptacle is neither sensible nor corporeal. The constant character of the Receptacle is neither sensible nor corporeal but is rather the great catholic power to receive all things, a power mysteriously grounded in its very formlessness.

The corporeal analogy also offers a framework for understanding the relationship between the three kinds, the Paradigm, its image, and that in which it is imaged, the Receptacle. The gold is moved and marked as figures enter into it, and so appears differently at different times. Similarly, the Receptacle is moved and marked as figures enter into it, and so appears differently at different times. These figures are copies of what always is. Clearly we are meant to imagine the goldsmith displaying Demiurgic virtue in using an eternal as opposed to a generated model. Arguably this is implicit in the thought that we are meant to imagine a good goldsmith. A good goldsmith produces beautiful work. And if this work is mimetic, then it is only beautiful if it is based upon an eternal model. In the case of the Receptacle, it is moved and marked as figures enter into it by being stamped, in a marvellous manner, hard to describe, by the Paradigm and its contents. So the Cosmos is a sensible image stamped upon the Receptacle in a marvellous manner, hard to describe, by the Paradigm and its contents. The Paradigm and its contents are represented as acting upon the Receptacle so as to generate a sensible image. This action may be hard to describe, but it is not impossible to describe. Indeed Timaeus promises to elaborate later. Unfortunately, Timaeus remains mortal and, despite being wise and good, fails to fullfil his promise.

While we may lament not knowing how the Forms stamp or impress upon the Receptacle to generate sensible images, in emphasising the activity of the eternal model, Timaeus raises a puzzle. It is the model's activity that is marvellous if hard to describe, not the goldsmith's work. It seems that we have been told something about the causal relationship between the Paradigm, its image, and that in which it is imaged, the Receptacle, but the Demiurge has been left out of account.

Here it is useful to bear in mind the distinction between the pre-cosmic chaos and the Cosmos established by the Demiurgic imposition of form and number. In the pre-cosmic chaos there are fleeting sensible powers that are traces of the Forms of the primary bodies. These traces are presumably stamped by the Forms in a marvellous manner, hard to describe, without Demiurgic intervention. That is why they display a disorderly motion. In this sense they are a wandering cause. They wander in the sense that they move without the guidance of \emph{nous}. Matters are different with respect to the Cosmos. The generation of the Cosmos required Demiurgic intervention. In order to generate the Cosmos, the Demiurge had to impose form and number upon the chaotic contents of the Receptacle. Sensible powers in the pre-cosmic chaos, traces of the Forms in the spatial matrix of the Receptacle, receive form and number by being assigned regular polyhedra with a certain number of sides. Thus form and number is imposed upon traces of the fiery by assigning those powers complementary shapes with a determinate number of sides. Specifically traces of the fiery are assigned tetrahedra since of the regular polyhedra they are the smallest and swiftest and their sharp angles cut easily. The choice of polyhedron is determined by the sensible power they are meant to capture and extend. Not only does the choice of tetrahedra capture the fiery character but it also extends this character to be sufficiently like fire so that it may be said to be ``such like'' with any safety.  Once form and number is imposed upon the chaotic contents of the Receptacle, their motion is consequently more orderly and suitable as auxiliary causes to the true cause as determined by the will of the benevolent and ungrudging Demiurge. 

Notice how this bears upon our puzzle. The Forms need to act independently of the Demiruge in order to produce the traces that constitute the chaotic contents of the Receptacle prior to the imposition of form and number. Moreover, the Demiurge generates the Cosmos, at least in part, by acting upon these trace powers by imposing form and number upon them. The Demiurge could not do that if there were no trace sensible powers to act upon. And there would be no trace sensible powers without the Forms stamping them, in a marvellous manner, hard to describe, upon the all-receptive Receptacle.

% Corporeal media suffer from a material recalitrance in the way that the Receptacle does not.

At this point, Timaeus, while retaining the craft analogy, shifts the relevant craft from a goldsmithing to perfumery. The base of a perfume must be as odour free as the perfumer can make it so as to better receive the scents that will be combined in that base. Just as the gold having a shape of its own would interfere with its ability to receive another shape, so the base of a perfume having a scent of its own would interfere with its receiving other scents. Since the Receptacle is capable of receiving all things, this must be all the more true for it.

The material recalictrance of corporeal media may be limiting, and craftsmen deploy great artifice to sufficiently overcome these limitations, at least as far as they can for practical purposes, but they are intelligibly media capable of receiving sensible forms for all that, even if imperfectly. And Timaeus is right that if you want a corporeal medium to take on a certain shape it should not have a shape of its own that it will retain through its working. However, generalizing from the capacity of corporeal media to receive a limited range of sensible forms to the capacity to receive all things may be a step too far. Perhaps while the former is intelligible, an abstract general capacity to receive all things is not.

Consider the puzzling nature of the Receptacle. It has a name proper to it because it has a constant nature. The constant nature of the Receptacle is to receive all things. In order to receive all things the Receptacle must not take on any form. But what kind of constant nature does something have if it is without form? As we shall see, Timaeus deploys a variety of images in struggling to explain the Receptacle. It might be thought that Peripatetics have the resources to provide an answer, but though Timaeus speaks of powers, he does not have Aristotle's distinction between \emph{dunamis} and \emph{energeia}. Moreover, questions remain, even on a Peripatetic approach. Consider the interpretative controversies surrounding \emph{nous} as Aristotle understands it and the puzzles that attend it (on the latter see especially \citealt{Rosen:1961aa}). 

% section corporeal_media (end)

\section{The Receptacle} % (fold)
\label{sec:the_receptacle}



Recall the ontological distinction between the intelligible, that which always is and never becomes, and the sensible, that which becomes and never is, was explication in the \emph{proemium} as the objects of distint cognitive attitudes. While that which always is and never becomes is the object of \emph{noēsei meta logou perilēpton}, that which becomes and never is is the object of \emph{doxē met’ aisthēseōs alogou doxaston}. The Paradigm is that which always is and never becomes and is apprehended in understanding with and account. The Cosmos, a copy of the Paradigm, is that which becomes and never is and is oponable with the aid of perception and sensation. In distinguishing the Receptacle from the Paradigm and the Cosmos, Timaeus extends the ontology first introduced in the \emph{proemium}. But how does this effect the epistemology of the \emph{proemium}? Is the Receptacle the object of a distinct cognitive attitude?

It is not clear that it is. First, not only are the ontological kinds of the \emph{proemium} the objects of distinct cognitive attitudes, they are explicated as the objects of these attitudes. The idea is that if one wants to know what always is and never become one should reflect on the kinds of things that are apprehended in understanding with an account. But the Receptacle is not itself explicated as the object of a disinct kind of cognitive attitude. If anything, despite Timaeus' best efforts, the Receptacle remains elusive, and Timaeus confesses to it being baffling and obscure (about which \citealt{Derrida:1993aa} makes much).

% section the_receptacle (end)

\section{Elemental triangles} % (fold)
\label{sec:elemental_triangles}



% section elemental_triangles (end)

\section{Primary Bodies} % (fold)
\label{sec:primary_bodies}



% section primary_bodies (end)

\section{Secondary Bodies} % (fold)
\label{sec:secondary_bodies}



% section secondary_bodies (end)

% Chapter gene (end) 