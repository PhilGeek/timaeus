%!TEX root = /Users/markelikalderon/Documents/Git/timaeus/timaeus.tex

\chapter{The Bonds of Life} % (fold)
\label{cha:the_bonds_of_life}

\section{The Soul--Body Union for Mortals} % (fold)
\label{sec:the_soul_body_union_in_mortals}

The soul-body union for mortals differs from the union of the World Soul and the body of the Cosmos. Mortal beings are not only embodied but embedded in an environment. The Cosmos may be an embodied living being, but it is not embedded in an environment, it is the environment. This is related to a further difference. Whereas the World Soul encompasses the body of the Cosmos, the bodies of mortal beings encompasses their soul. The soul of mortal beings is bound to their body and this bond must be protected from the effects of strong powers in the environment. Thus the Young Gods when weaving the immortal and mortal parts together position these bonds within. In this manner, the body encompasses the soul of a mortal being. The soul of mortal beings is bound to marrow, and this marrow is surrounded by hard bone to protect the bonds from external impact. Thus the skull is the primary fortification of the sovereign acropolis. By contrast, there are no strong powers external to the Cosmos that may age or weaken it. And hence no need of corporeal encompassment to protect what binds the World Soul to the body of the Cosmos. So the World Soul may, instead, safely encompass the body of the Cosmos. The souls of mortal beings are bound within since they are embedded in an environment with strong powers.

Two features of mortal anatomy are relevant to the soul-body union, the marrow and the bone that encases it. The marrow is not the bond of the soul but that to which the soul is bound. It has the character that it has for just this purpose. So Timaeus' account of the marrow will be informative about the soul-body union for mortals. The bone that encases the marrow is to protect the bond from the effects of strong powers in the environment. The soul of mortal beings has an immortal part and a mortal part. The immortal soul is bound to the marrow in the skull, called the ``brain''. The Young Gods made the skull spherical, in imitation of the shape of the Cosmos and the circular activity of cognition. The mortal soul is bound to marrow as well. Its roots are in the marrow of the spine. When contrasted with the circular skull, we are perhaps struck first by the linear character of the spine. But if we recall that the spine encompasses the marrow that it contains, we recognize that the spine is, more specifically, cylindrical. Recall that motions are either circular or linear. The skull takes its shape from the circular motion of cognition. The spine combines the circular and the linear in its shape. What may we infer about the mortal soul's bond with the marrow of the spine?

The linear aspect of its cylindrical shape is explicable given our previous discussion. Setting aside the possibility of cosmic perception (Proclus, \emph{In Timaeum} 2 83.3–85.31, \citealt{Diehl:1903re}, discussed in chapter~\ref{sec:knowledge_and_opinion}), \emph{aisthēsis} is a power of the mortal soul. \emph{Aisthēsis} involves a corporeal instrument that receives affections from strong powers in the perceiver's environment. Timaeus describes the action of these strong powers as linear. \emph{Aisthēsis} does not merely involve the reception of an affection from without, but importantly, it is the cognizance of the power of the agent that caused that affection. Perhaps the circular aspect of the spine's cylindrical shape reflects the circular activity that constitutes the soul's cognizance of the object of \emph{aisthēsis}. Later Platonists (such as Priscian, \emph{Metaphrasis in Theoprastum}, and Pseudo-Simplicius, \emph{In de anima}) will explicitly characterize perceptual apprehension as circular activity.

% section the_soul_body_union_in_mortals (end)

\section{Marrow} % (fold)
\label{sec:marrow}

God generated marrow from the finest elemental triangles. He chose only those elemental triangles that were unwarped and smooth. The triangles here, could not be the triangles of planar geometry \citep[293 n2]{Cornford:1935fk}. A warped triangle is not a planar figure. We have encountered this kind of issue before (chapter~\ref{sec:the_circles_of_the_same_and_the_different}). Thus, for example, the Demiurge formed two lines from the soul mixture and bent them back to form two circles. But a geometrical line cannot become a circle. The triangles should be understood on a material model. They are, after all, composed of sensible powers, traces of the Forms in the Receptacle, upon which the Demiurge has imposed form and number. In choosing only the finest elemental triangles, God ensures that they could exactly compose the purest fire, water, air, and earth. (I am uncertain why Timaeus departs, here, from the more usual ordering of fire, air, water, and earth.) God separated these triangles into kinds (presumably their shape) and mixed them in proportions that Timaeus does not reveal (perhaps for fear of impiety). He then fashioned the marrow from this mixture which Timaeus describes as the seed of all mortal kind. 

It is unclear what Timaeus means by this last remark. It is worth pausing over since talk of seeds and associated agrarian imagery predominates in Timaeus' account of marrow. It would be good to know in what sense marrow is a seed since Timaeus goes on to describe marrow as field that receives a divine seed. So is the marrow a seed or a seed bed? And in what sense or senses?

Later (91b1) Timaeus claims that marrow is the source of semen. So the marrow is the seed of all mortal kind since it is at the very least the source of the semen from which subsequent generations of mortals will spring. Notice, as well, the generality of Timaeus. Marrow is not merely the seed of human kind \citep[73]{Waterfield:2008lx} but of mortal kind. This encourages \citet[295]{Cornford:1935fk} to claim that ``the marrow is the fundamental life-substance in all animals and the same substance in all''. As we shall see, that marrow is described as universal seed since it is the fount of semen in mortals is not inconsistent with Timaeus subsequently describing marrow as a field or seed bed.

Though Timaeus speaks of the God in the singular here, the generation of the marrow is clearly a task that the Demiurge assigned to the Young Gods. The Young Gods, in generating the mortal parts, imitate the Demiurge's generation of the immortal part. Here, particulary, the \emph{mimēsis} is striking. The Demiurge generates the immortal soul by taking indivisible and divisible Being, Sameness, and Difference, mixing these, and using this mixture to fashion the immortal soul. Similarly, God generates marrow by selecting its components, mixing them, and fashioning the marrow. God's procedure is the corporeal analogue of the Demiurge's generation of the incorporeal soul.

God implanted in the marrow various kinds of soul. Allow me to make two sets of remarks about this deceptively simple claim.

First, \emph{Phuteuō} can mean implant, beget or engender, or to produce or bring about more generally. I translate (with \citealt[271]{Archer-Hind:1888qd}, \citealt[293]{Cornford:1935fk}, \citealt[70]{Lee:2008ca}, \citealt[77]{Taylor:1929ov} \citealt[73]{Waterfield:2003gs}, and \citealt[67]{Zeyl:2000cs}) \emph{phuteuōn} as implanted. Even if \emph{phuteuōn} could mean engendered in this context (as \citealt[191--3]{Bury:1929jb} translates), given the prominence of the agrarian imagery, the association of implanting carried by the Greek is salient. Recall that the immortal soul, in its celestial descent was sown in the instruments of time, and this signaled a greater involvement in corporeality than when it was merely a passenger in a stellar vehicle (chapter~\ref{sec:the_laws_of_destiny}). When it is implanted in the marrow that involvement is greater still for it is now embodied and embedded in an environment with strong powers.

Second, God implants in the marrow various kinds of soul. What kinds of soul does Timaeus have in mind? There are two alternatives. The kinds in question may simply be the immortal and mortal kinds. Or, given that the mortal soul has spirited and appetitive parts, the kinds may be reason, spirit, and appetite. On the first alternative there are two kinds on the second there are three. This may not seem like much of a difference given that the second kind of the first alternative, the mortal soul, divides exhaustively into the second and third kinds of the second alternative, spirit and appetite. But indifference between these alternatives betrays an insensitivity to a substantive explanatory difference. Are there two fundamentally different kinds of soul or three? Cast in these terms, it would seem that there are two, the immortal and mortal kind. The former does not perish upon death though the later does (though later Platonists envision the possibility of life after death for even the mortal soul, to animate the powers of shades in Hades, say). The Demiurge generates the immortal soul. The Young Gods generate the mortal soul. For recall, the Demiurge is incapable of generating anything mortal. And while spirit and appetite differ in their powers, locations and corporeal instruments, there does not seem to be a fundamental difference between them the way there is between sovereign reason and the broader \emph{polis}. This was manifest in the political toplogy of the soul. Reason occupies the acropolis, raised on a hill, surrounded by water and joined to the mainland and the broader \emph{polis} by a narrow isthmus. The social distance between the occupants of the acropolis and the occupants of the broader \emph{polis} is manifest in their spatial distance. And the splendor and fortification of the acropolis emphasize this. 

Having implanted the various kinds of soul in the marrow, God now divides that marrow. The divisions of the marrow correspond to the number and shape of the kinds of the souls implanted there. Again, this is a deceptively simple claim.

First, note the oddity of God's procedure. The kinds of soul are first implanted in the marrow and then the marrow is separated and formed into shapes that match the kinds. One might have expected, instead, that the marrow was divided and shaped first and then the various kinds of souls implanted in the appropriate divisions of marrow. Why does God proceed otherwise? Given His intellect and benevolence, it must be for the best. Perhaps God implants the various kinds of soul in undivided marrow because of the unity of the souls of mortal beings. It is one thing for Timaeus to distinguish parts of the soul. It is another to maintain that there are distinct rational, spirited, and appetitive souls. Even reason persisting when spirit and appetite perish, does not, by itself, establish this. Perhaps, then, God's implanting soul in undivided marrow dramatically emphasizes the unity of the soul of mortal beings (though, of course, questions remain).

Second, the number and kinds of shapes of the divided marrow correspond with the number and kinds of shapes of the souls implanted there. What is the number of souls? If the number of souls is one in each case there is not much point in making number explicit. This really only makes sense if the number could be more than one. So, for example, we know that there is only one part of the soul implanted in the head, the immortal part. What about the mortal part? The mortal part itself divides into spirit and appetite. if there are separate divisions of marrow for spirit and appetite, then the number in each case is one. But that is the trivial case. But if the mortal soul is implanted in its own marrow, then the number of souls (strictly parts of soul) implanted there are two. Thus the fundamental division of the soul into the immortal and mortal part is preserved, consistent with the further subdivision of the mortal soul required for tripartition.

Third, Timaeus' language here remains neutral concerning an important topic. What is the relationship between the shape of the marrow and the shape of the soul implanted there? If the soul is extended, then perhaps the marrow must be so shaped otherwise the soul would not fit. Or, if the soul is inextended, perhaps the shape of the marrow is modeled on an inextended paradigm. Timaeus does not specify the explanatory relationship, such as efficient causation or paradigmatic causation, if any, between the shape of the marrow and the shape of the soul implanted in it. Timaeus merely claims that they match.

God first divides the marrow that receives the immortal soul. This He makes into a sphere. Since the sphere of marrow received the immortal soul, and given its divine nature, Timaeus likens the marrow to a plough-field receiving divine seed. God calls this sphere of marrow ``brain'' (\emph{engkephalon}).

First, while Timaeus remains silent on the explanatory relationship between the shape of the soul and the shape of the marrow, details of the narrative would seem to rule out one natural alternative. Notice that the kind of soul---strictly, a part of a unified soul---is implanted in the undivided marrow. The marrow is then divided. And only then is it shaped into a sphere. That means that the immortal soul was implanted in the marrow before it was spherical. If that is right, then it is not the case that the marrow has its shape so that the relevant kind of soul will fit. The immortal soul fit fine when implanted in the undivided marrow.

Second, the agrarian imagery continues. The soul, and not the marrow, is now portrayed as a divine seed. The immortal soul is divine. It can apprehend in understanding divine intelligible things, and so assimilate to them. Timaeus has already described the soul as sown in the instruments of time and subsequently implanted in the undivided marrow. Since seeds are what one plants, this is a natural elaboration of the agrarian imagery. But it is an elaboration. Timaeus highlights the way in which seeds contain within themselves the power of growth and vital activity. Perhaps, the immortal soul is the divine seed since it is a divine principle of vital activity embedded within corporeal material designed to receive it. The marrow, of course, in being generated from the finest elemental triangles in the right proportions, was divinely prepared to receive the divine seed and so is likened to a plough field.

Describing the soul as divine seed implanted in the field of marrow is consistent with Timaeus' earlier description of marrow as universal seed. Indeed, the descriptions may be complementary. The marrow is like a field since it receives the divine seed and this activates vital activity in the newly incarnate mortal. The marrow is itself universal seed since it is the fount of semen. But as this is vital activity, the marrow is only the fount of semen insofar as the divine seed has been implanted there by God.

Third, a name is bestowed upon this spherical division of marrow. God calls it ``brain''  (\emph{engkephalon}). As \citet[77 n2]{Taylor:1929ov} and \citet[293 n3]{Cornford:1935fk} observe, this is most likely derived from \emph{en kephalē}, in the head. Given its gross observable features, it is perhaps natural to first regard the brain as the marrow of the skull. Timaeus claims that every living thing has the head as the container for the brain.

If the immortal soul is assigned to one division of marrow, the mortal soul is assigned to a plurality of divisions of marrow. God forms these into shapes that were both circular and linear. They have cylindrical shapes. Think, for example, of the shape of the marrow in the femur, or a bone in a finger. Upon these God bestowed the name ``marrow''.

The marrow is a divided seed bed. Nevertheless, from these, as from anchors, God bound the whole soul.

First, concerning the plural reference of ``these'', I take Timaeus to mean not only what God calls ``marrow'' but also what God calls ``brain''. For what is bound is the whole soul. And the whole soul includes the immortal part implanted in the brain.

Second, Timaeus shifts from agrarian to nautical imagery. The divided marrow anchors the whole soul. If the marrow is an anchor, then the soul is a ship tethered to this anchor by ropes or chains. The buoyancy of a ship keeps it afloat and so up from the floor of the sea. The image suggests that the corporeal is weighing down the whole soul which, if left to its own devices, might move upward in a celestial ascent (as in the \emph{Phaedrus} myth).

Third, Timaeus is unfortunately reticent about the nature of the bonds that anchor the whole soul to the divided marrow. Whatever their nature, they must be so as to bind the incorporeal to the corporeal. If we think of corporeal bonds such as ropes and chains this can seem puzzling. But not all bonds are corporeal. Timaeus, for example, has argued that proportion is a bond, but proportion is not a body. Moreover, we have seen how this incoporeal bond may bind the corporeal. The Empedoclean ``roots'' are bound together by proportion in the body of the Cosmos. Timaeus does not say that the bonds that anchor the whole soul in the divided marrow are proportions. Moreover, unlike the bonds that bind the Cosmos (32c3--5, chapter~\ref{sec:the_elemental_composition_of_the_corporeal}), the Young Gods (41a7--b6, chapter~\ref{sec:the_demiurge_addressing_the_gods}), and the immortal part of the soul (43d, chapter~\ref{sec:the_shock_of_embodiment}), the bonds that anchor the whole soul to the body that it animates may be severed even by an agent other than the one who bound them. Nevertheless, what reflection on proportions as bonds reveals is that there is room within Timaeus' framework for a bond that binds the corporeal and the incorporeal.

% section marrow (end)

\section{The Skull} % (fold)
\label{sec:the_skull}



% section the_skull (end)

\section{The Spine} % (fold)
\label{sec:the_spine}


% section the_spine (end)

\section{The Location of the Mortal Soul} % (fold)
\label{sec:the_location_of_the_mortal_soul}



% section the_location_of_the_mortal_soul (end)

\section{The Unity of the Soul} % (fold)
\label{sec:the_unity_of_the_soul}

% section the_unity_of_the_soul (end)

% Chapter the_bonds_of_life (end) 