%!TEX root = /Users/markelikalderon/Documents/Git/timaeus/timaeus.tex

\chapter{The Flesh and the Mortal Soul} % (fold)
\label{cha:the_flesh_and_the_mortal_soul}

\section{The Methodological Dilemma Revisited} % (fold)
\label{sec:the_methodological_dilemma_revisited}

Recall that Timaeus inaugurates his discussion of the \emph{pathēmata} with a methodological dilemma (61c4–d5). The \emph{pathēmata}, affections of the corporeal instruments of \emph{aisthēsis}, are contrasted with two other things. Specifically they are contrasted with the origin of the flesh and things pertaining to it, on the one hand, and the mortal parts of the soul, on the other. The problem is twofold. First, one cannot adequately account for the \emph{pathēmata} having to do with \emph{aisthēsis} without accounting for the origin of the flesh and things pertaining to it and the mortal part of the soul. But second, one cannot account for the origin of the flesh and things pertaining to it and the mortal parts of the soul without accounting for the \emph{pathēmata} having to do with \emph{aisthēsis}. The problem is not one of mutual entailment or that these subject matters somehow presuppose one another directly. The problem is rather that an account of either subject matter each presupposes \emph{aisthēsis} with the consequence that an account of each must make reference to the other. Hence the methodological dilemma: Timaeus cannot account for either without reference to the other, but it is impossible to give an account of both subjects at once. Timaeus solution is to first account for the \emph{pathēmata} (61d–68d) and then subsequently account for the origin of the flesh and things pertaining to it (73b1ff) and the mortal parts of the soul (69c5ff). 

We have already discussed Timaeus' account of the (chapters~\ref{cha:common_pathemata} and \ref{cha:peculiar_pathemata}). In the present chapter we shall discuss the origin of the flesh and things pertaining to it and the mortal parts of the soul. Unfortunately not the whole of it. Our approach will be selective. The selection is subservient to two goals. First, I want to learn what I can about how the methodological dilemma is resolved from this discussion. As audition is the clearest case where further insight is shed on a perceptual power, it is natural, as well, to focus on this. Second, I want to learn what the mortal soul is and how it acts as a principle of vital activity.

Earlier, when I said it was unfortunate that we are not in a position to discuss the whole of Timaeus' account of the flesh and the mortal soul, some may be forgiven for hearing a hint of irony there. None was intended. The significance of Timaean anatomy is underappreciated. For just consider the context. Socrates the day before, on the occasion of the Panathenaea, has given a speech on the topic of the \emph{Republic} and requests that the return speeches he is now owed portray Socrates' ideal city in action. Critias proposes that a secret history, passed down in his family, of ancient Athenians fighting a war with Atlantis over the Mediterranean would serve this purpose. This history will be delivered only after Timaeus' cosmology. What is the connection? The Cosmos is only completely perfected when there are mortal living beings embodied and embedded within it. Anthropogony is required to perfect the Cosmos by making it better resemble the intelligible Living Being upon which it was modeled by the Demiurge. The intelligible Living Being is comprehensive. It contains within it all other intelligible living beings. In order for the sensible Cosmos to be comprehensive it must contain within it all other sensible living beings. As a consequence of divinely ordained embodiment, mortal souls not only have an immortal part, which the Demiurge addressed in delivering the Laws of Destiny, but now have a mortal part as well. There are two parts of the mortal soul, the spirited part, and the appetitive part. So Timaeus seems to subscribe to a tripartite psychology, though there are differences between the Timaean account and the accounts found in the \emph{Republic} and the \emph{Phaedo}. Nevertheless, Timaeus retains the city--soul analogy. Moreover, the moral psychology determined by his anthropogony, required for the perfection of the Cosmos, is the moral psychology of the Athenian heroes who staved off the enslavement of the Mediterranean by Atlantis. It is only if the actions of the Athenians are sufficiently like the actions of Socrates' ideal city will the return \emph{logoi} meet with his approval. And in order for Timaeus' speech to contribute to this collective effort, the actions of the Athenian heroes must be explicable in terms of the moral psychology dictated by his anthropogony. And these are bound up with commitments concerning the flesh and the mortal soul.

A vivid example of the ethical significance of Timaean anatomy involves audition. We are providentially provided ears to hear with to correct the revolutions of the immortal soul disrupted as a result of being embodied and embedded in an environment with strong powers and so subject to the irrational linear motion of \emph{aisthēsis}. In listening to music or rational speech with rational attentiveness, the revolutions of the soul have a tendency to align with the revolutions of our elder sister, the World Soul. While we have been told that sound impacts upon the brain and the blood through the ears and causes a motion from the head to the region of the liver, the rational effects of rational attentive listening have yet to be explained. These, however, become explicable once we understand the ethical function of the liver. So our knowledge of Divine Providence is incomplete without detailed anatomical knowledge of the liver (a lowly, ignoble organ confined with the other beasts of appetite to a manger far enough for their din to be unable to reach the acropolis from which the rational part of the soul rules and issues sovereign commands).

% section the_methodological_dilemma_revisited (end)

\section{Tripartition in the \emph{Timaeus}} % (fold)
\label{sec:tripartition_in_the_emph_timaeus}

The three parts of the soul consist in the immortal part, the rational part of the soul, and the two parts of the mortal part, the spirited and appetitive parts. These share the body as a vehicle, as drivers rather than passengers, and are said to cohabit that body and so come to share a common household or perhaps more aptly a common \emph{polis}. The three parts of the soul are associated with different regions of the body. For each part of the mortal soul, Timaeus will single out a primary and secondary organ. His anatomical descriptions are rough if accurate (especially given the state of anatomical knowledge in the 5th century B.C.E). What is striking is the language he uses to describe their relative locations and the functions he assigns them. 

First, the language that Timaeus uses to describe the locations of the parts of the soul is striking. Taken collectively they describe a \emph{polis}. 

The immortal part of the soul, the rational part, is in the head since the roundness of the skull resembles its revolutions. In the \emph{Republic}, the rational part was described as the acropolis of the soul, but in Timaeus' hands the rational part is an occupant of the acropolis. An acropolis is a fortified city within a city, usually on a hill. (The Greek from which ``acro'' derives means high, so acropolis suggests a city on high.) Just as the Circle of the Same has sovereignty over the Circle of the Different, the rational part of the soul, comprised of these, has sovereignty over the mortal parts of the soul and the body that it animates. From its shining acropolis, the rational part of the soul issues rational commands that may be heard and heeded by the mortal soul.

The mortal part of the soul is separated from the immortal part by an isthmus. While separated in this way, this is consistent with their being some line of communication between the acropolis and the rest of the \emph{polis}, for verbal commands are issued from the acropolis and heard and heed by the mortal soul.

The spirited part of the soul is housed in the barracks of the guardians, the heart. It is just beyond the isthmus that separates it from the immortal soul, since it is receptive to reason. Spirit is portrayed as fulfilling the role of the guardians, both attacking external enemies and policing internal ones.

The appetitive part is portrayed as unruly beasts confined to a manger, far enough away for their din not to reach the acropolis. Perhaps the mirror-like liver is a still pool adjacent to the manger.

Timaues has provided us with a political topology of the soul. Each part of the soul being located where the appropriate class of people are located in Socrates' ideal city.

It is not just the political topology of his descriptions that is striking, but the functions that he assigns to the primary and secondary organs associated with the two mortal parts of the soul. While Timaeus provides reasonably accurate anatomical descriptions of these organs, he is not at all concerned with their primary biological function. Instead he assigns them ethical functions that correspond to he functions of the different classes in the ideal city. The rational part, of course, rules. The heart is the primary organ associated with the spirited part, but Timaeus does not here mention the circulation of blood, but focuses, instead, on the boiling of the blood in anger when reason communicates some evil. From the biological perspective, one might have expected the stomach to be the primary organ associated with the appetitive part given its role in nutrition, but Timaeus does not here mention nutrition. Instead, the liver is the primary organ associated with the appetitive part. Since appetite cannot understand the discursive demands of reason, the liver is providentially provide to compensate for this. The liver may not be able to receive discursive content but thanks to its smooth surface it may receive visual content in the form of images reflected on its surface. Reason uses these images to control the mortal part of the soul when it otherwise refuses to heed its command.


% section tripartition_in_the_emph_timaeus (end)

% \section{Aporia} % (fold)
% \label{sec:aporia}
%
%
%
% % section aporia (end)

\section{Two Parts of the Mortal Soul} % (fold)
\label{sec:two_parts_of_the_mortal_soul}



\section{Spirit} % (fold)
\label{sec:the_spirited_part}



% section the_spirited_part (end)

\section{Appetite} % (fold)
\label{sec:the_appetitive_part}



% section the_appetitive_part (end)

% section two_parts_of_the_mortal_soul (end)

\section{Marrow and Bone} % (fold)
\label{sec:marrow_and_bone}



% section marrow_and_bone (end)

% Chapter the_flesh_and_the_mortal_soul (end) 