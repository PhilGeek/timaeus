%!TEX root = /Users/markelikalderon/Documents/Git/timaeus/timaeus.tex

\chapter{The Flesh and the Mortal Soul} % (fold)
\label{cha:the_flesh_and_the_mortal_soul}

\section{The Methodological Dilemma Revisited} % (fold)
\label{sec:the_methodological_dilemma_revisited}

Recall that Timaeus inaugurates his discussion of the \emph{pathēmata} with a methodological dilemma (61c4–d5). The \emph{pathēmata}, affections of the corporeal instruments of \emph{aisthēsis}, are contrasted with two other things. Specifically they are contrasted with the origin of the flesh and things pertaining to it, on the one hand, and the mortal parts of the soul, on the other. The problem is twofold. First, one cannot adequately account for the \emph{pathēmata} having to do with \emph{aisthēsis} without accounting for the origin of the flesh and things pertaining to it and the mortal part of the soul. But second, one cannot account for the origin of the flesh and things pertaining to it and the mortal parts of the soul without accounting for the \emph{pathēmata} having to do with \emph{aisthēsis}. The problem is not one of mutual entailment or that these subject matters somehow presuppose one another directly. The problem is rather that an account of either subject matter each presupposes \emph{aisthēsis} with the consequence that an account of each must make reference to the other. Hence the methodological dilemma: Timaeus cannot account for either without reference to the other, but it is impossible to give an account of both subjects at once. Timaeus solution is to first account for the \emph{pathēmata} (61d–68d) and then subsequently account for the origin of the flesh and things pertaining to it (73b1ff) and the mortal parts of the soul (69c5ff). 

We have already discussed Timaeus' account of the (chapters~\ref{cha:common_pathemata} and \ref{cha:peculiar_pathemata}). In the present chapter we shall discuss the origin of the flesh and things pertaining to it and the mortal parts of the soul. Unfortunately not the whole of it. Our approach will be selective. The selection is subservient to two goals. First, I want to learn what I can about how the methodological dilemma is resolved from this discussion. As audition is the clearest case where further insight is shed on a perceptual power, it is natural, as well, to focus on this. Second, I want to learn what the mortal soul is and how it acts as a principle of vital activity.

Earlier, when I said it was unfortunate that we are not in a position to discuss the whole of Timaeus' account of the flesh and the mortal soul, some may be forgiven for hearing a hint of irony there. None was intended. The significance of Timaean anatomy is underappreciated. For just consider the context. Socrates the day before, on the occasion of the Panathenaea, has given a speech on the topic of the \emph{Republic} and requests that the return speeches he is now owed portray Socrates' ideal city in action. Critias proposes that a secret history, passed down in his family, of ancient Athenians fighting a war with Atlantis over the Mediterranean would serve this purpose. This history will be delivered only after Timaeus' cosmology. What is the connection? The Cosmos is only completely perfected when there are mortal living beings embodied and embedded within it. Anthropogony is required to perfect the Cosmos by making it better resemble the intelligible Living Being upon which it was modeled by the Demiurge. The intelligible Living Being is comprehensive. It contains within it all other intelligible living beings. In order for the sensible Cosmos to be comprehensive it must contain within it all other sensible living beings. As a consequence of divinely ordained embodiment, mortal souls not only have an immortal part, which the Demiurge addressed in delivering the Laws of Destiny, but now have a mortal part as well. There are two parts of the mortal soul, the spirited part, and the appetitive part. So Timaeus seems to subscribe to a tripartite psychology, though there are differences between the Timaean account and the accounts found in the \emph{Republic} and the \emph{Phaedo}. Nevertheless, Timaeus retains the city--soul analogy. Moreover, the moral psychology determined by his anthropogony, required for the perfection of the Cosmos, is the moral psychology of the Athenian heroes who staved off the enslavement of the Mediterranean by Atlantis. It is only if the actions of the Athenians are sufficiently like the actions of Socrates' ideal city will the return \emph{logoi} meet with his approval. And in order for Timaeus' speech to contribute to this collective effort, the actions of the Athenian heroes must be explicable in terms of the moral psychology dictated by his anthropogony. And these are bound up with commitments concerning the flesh and the mortal soul.

A vivid example of the ethical significance of Timaean anatomy involves audition. We are providentially provided ears to hear with to correct the revolutions of the immortal soul disrupted as a result of being embodied and embedded in an environment with strong powers and so subject to the irrational linear motion of \emph{aisthēsis}. In listening to music or rational speech with rational attentiveness, the revolutions of the soul have a tendency to align with the revolutions of our elder sister, the World Soul. While we have been told that sound impacts upon the brain and the blood through the ears and causes a motion from the head to the region of the liver, the rational effects of rational attentive listening have yet to be explained. These, however, become explicable once we understand the ethical function of the liver. So our knowledge of Divine Providence is incomplete without detailed anatomical knowledge of the liver (a lowly, ignoble organ confined with the other beasts of appetite to a manger far enough for their din to be unable to reach the acropolis from which the rational part of the soul rules and issues sovereign commands).

% section the_methodological_dilemma_revisited (end)

\section{Tripartition in the \emph{Timaeus}} % (fold)
\label{sec:tripartition_in_the_emph_timaeus}

The three parts of the soul consist in the immortal part, the rational part of the soul, and the two parts of the mortal part, the spirited and appetitive parts. These share the body as a vehicle, as drivers rather than passengers, and are said to cohabit that body and so come to share a common household or perhaps more aptly a common \emph{polis}. The three parts of the soul are associated with different regions of the body. For each part of the mortal soul, Timaeus will single out a primary and secondary organ. His anatomical descriptions are rough if accurate (especially given the state of anatomical knowledge in the 5th century B.C.E). What is striking is the language he uses to describe their relative locations and the functions he assigns them. 

First, the language that Timaeus uses to describe the locations of the parts of the soul is striking. Taken collectively they describe a \emph{polis}. 

The immortal part of the soul, the rational part, is in the head since the roundness of the skull resembles its revolutions. In the \emph{Republic}, the rational part was described as the acropolis of the soul, but in Timaeus' hands the rational part is an occupant of the acropolis. An acropolis is a fortified city within a city, usually on a hill. (The Greek from which ``acro'' derives means high, so acropolis suggests a city on high.) Just as the Circle of the Same has sovereignty over the Circle of the Different, the rational part of the soul, comprised of these, has sovereignty over the mortal parts of the soul and the body that it animates. From its shining acropolis, the rational part of the soul issues rational commands that may be heard and heeded by the mortal soul.

The mortal part of the soul is separated from the immortal part by an isthmus. While separated in this way, this is consistent with their being some line of communication between the acropolis and the rest of the \emph{polis}, for verbal commands are issued from the acropolis and heard and heed by the mortal soul.

The spirited part of the soul is housed in the barracks of the guardians, the heart. It is just beyond the isthmus that separates it from the immortal soul, since it is receptive to reason. Spirit is portrayed as fulfilling the role of the guardians, both attacking external enemies and policing internal ones.

The appetitive part is portrayed as unruly beasts confined to a manger, far enough away for their din not to reach the acropolis. Perhaps the mirror-like liver is a still pool adjacent to the manger.

Timaues has provided us with a political topology of the soul. Each part of the soul being located where the appropriate class of people are located in Socrates' ideal city.

It is not just the political topology of his descriptions that is striking, but the functions that he assigns to the primary and secondary organs associated with the two mortal parts of the soul. While Timaeus provides reasonably accurate anatomical descriptions of these organs, he is not at all concerned with their primary biological function. Instead he assigns them ethical functions that correspond to the functions of the different classes in the ideal city. Or more specifically, the ethical functions are assigned to the primary organs and the secondary organs have an auxiliary function in aiding the primary organ. The rational part, of course, rules. The heart is the primary organ associated with the spirited part, but Timaeus does not here mention the circulation of blood, but focuses, instead, on the boiling of the blood in anger when reason communicates some evil. The lungs, the secondary organs associated with spirit, serve to cool the heart when it becomes overheated. Though Timaeus has an elaborate theory of respiration, no mention of respiration is made here. Instead, the lungs play the regulatory role that they do in order to maintain the heart and so that it may continue its primary ethical function. From the biological perspective, one might have expected the stomach to be the primary organ associated with the appetitive part given its role in nutrition, and though Timaeus provides an account of the stomach and digestion more generally, Timaeus does not here mention nutrition. Instead, the liver is the primary organ associated with the appetitive part. Since appetite cannot understand the discursive demands of reason, the liver is providentially provided to compensate for this. The liver may not be able to receive discursive content but thanks to its smooth surface it may receive visual content in the form of images reflected on its surface. Reason uses these images to control the mortal part of the soul when it otherwise refuses to heed its command. Moreover, the appetitive part of the soul is further providentially accommodated by the liver being the source of divination.


% section tripartition_in_the_emph_timaeus (end)

\section{Reason} % (fold)
\label{sec:reason}

Recall Timaeus' account of the generation of the head. The Young Gods bound the divine revolutions (\emph{theias periodous}), the revolutions of the Circles of the Same and the Different, within a spherical body. This spherical body is an imitation of the spherical body of the Cosmos. There is, however, a notable difference with respect to the soul–body union. Whereas the body of the Cosmos is bound within the World Soul, the soul of mortal beings is bound within the spherical body. This spherical body is called the “head”. Like the body of the visible god, it is divine, and given its divine status it reigns over the other parts of the mortal being which serve it.

The theme of containment continues in the present passage in a number of guises. The immortal soul is in the body as in a vehicle, as a driver not a passanger (though this contrast has yet to be explained). This is a corporeal image of containment. Think of the way in which a carriage or chariot contains its occupants. Moreover, the immortal soul is contained within the head. As we shall see it is bound to the marrow made of the finest triangles within the circular head. The imortal soul is represented as the occupant of an acropolis and so as contained within it. 

The image of the acropolis is worth dwelling on. Again, an acropolis is a fortified city within a city, usually on a hill or otherwise elevated, that is the seat of political power. The divine revolutions are sovereign over the mortal soul and the body that it animates. The immortal soul issues discursive commands that are heard and heeded by the mortal soul. It is apt then that the immortal soul, given its divine status and sovereignty over the mortal, should be seated in an acropolis. Two further features of an acropolis are worth dwelling on---that they are fortified and that they are elevated.

First, an acropolis is fortified. As the seat of political power it is prudential that it is. The sovereign divine revolutions are occupants of an acropolis and so are occupants of a fortified position. The skull of the head is the fortification of the immortal soul that is bound within. The thickness of the skull, and so the effectiveness of the fortification, is carefully balanced to optimize the intelligence of mortal beings. (The thicker the skull, the better fortified, but the less room there is for marrow with which to bind the divine revolutions.)

Second, an acropolis is typically located in an elevated position. There may be a number of reasons in play. In being elevated, the acropolis is closer to the divine and the intelligible, the source of its occupants' political authority. Being elevated is also a visible symbol of the political authority invested in the occupants of the acropolis. Finally, being elevated is an aspect of its fortification, in that an elevated position is easier to defend. At least the first two reasons are relevant to Timaeus' speech and possibly the third. Let us consider these in turn.

Perhaps the head is elevated so that it may be closer to the divine and the intelligible. The head, the acropolis of the immortal soul, is elevated in humans as opposed to beasts with four legs whose heads are directed downwards, towards the earth. By contrast, human heads, in being elevated, may look, instead, to the Heavens and the intelligible. The Sun illuminates the wanderers so that we may see their motion and learn the art of number and perhaps even philosophy. The revolutions of the Heavens are divine thought made visibly manifest. In rationally attending to these in the right sort of way, we may perfect ourselves by assimilating to the divine. We are drawn to the divine and the intelligible through the Sun's illumination and this thanks, in part, to the Young Gods providentially providing the head with an elevated position in the human body. 

Timaeus hails from Locri, but Plato having Timaeus describe the head in these terms may be part of a broader Athenian rationalism. Consider the following dramatic incident from Euripides' \emph{Heracles}. Acting on Hera's orders, Iris has Madness drive Heracles mad who kills his wife and children in that state. Theseus, who Heracles had rescued from Hades, encounters Heracles grieving with his father, Amphitryon. Amphitryon entreats Heracles to throw the mantle from his eyes and look to the sun (\emph{Heracles} 1204--5). And Theseus, in a moving gesture of friendship, declares that he has no fear of pollution, and demands that Heracles unveil himself and look up (\emph{Heracles} 1225). Unveiling and looking up is to rationally attend to one's circumstance, and this is ethically significant. To fail to do so is ignoble even should one's circumstance be terrible as is the tragic circumstance that Theseus finds Heracles in. Rationally attending to one's circumstance is a kind of acceptance, and the noble soul endures divinely ordained misfortune and does not refuse them (\emph{Heracles} 1225). Euripides, like Timaeus, emphasizes that the head should be elevated, and there is a similar play with the divine, the intelligible, and the Sun's illumination. Plato, in having Timaeus place the divine revolutions in the shining acropolis, gives specific expression to a central trope of a broader Athenian rationalism.

Perhaps the head is elevated as the expression of the political authority of its occupant. The immortal soul is divine and so suited to rule over the mortal soul and the body that it animates, and its being seated in the acropolis is an expression of its political authority. In demanding that Heracles raise his head, Theseus can be understood as exhorting Heracles to exercise rational control over his grief and regain his noble authority overthrown by grievous tragedy. The divine revolutions have political authority over the mortal soul and the body that it animates, and that its fortification is elevated is a visible expression of that political authority. 

Perhaps the head is elevated since an elevated position is easier to defend. If the third reason is in play as well, then elevation is not distinct from fortification. The head may be elevated to make it easier to defend from, say, low lying attacks from beasts on all fours. Thus, a dog may bite your hand or leg but, as long as you are standing, it cannot bite your head.

These three reasons are distinct, if complementary. I am inclined to believe that at least the first two are in play in Timaeus' speech.  I hesitate to endorse this third reason since Timaeus does not make it explicit. Like the utility of sight in navigating a complex environment with strong powers, the benefit of an elevated positon being easier to defend, if genuine, goes unmentioned.

There are two further aspects of the acropolis' fortification. Both may be described as parts of the extended fortification of the acropolis.

First, not only is the immortal soul bound in marrow encased in a hard skull in an elevated position, but it is separated from the region of the body occupied by the mortal soul. The head is separated from the trunk of the body by a narrow isthmus, the neck. If the hardness of skull and elevated position are meant to fortify against external attacks, the separation by means of a narrow isthmus is meant to defend against internal disruptions. Think of the disruptions of the cognitive revolutions occasioned by the \emph{pathēmata} of linear \emph{aisthēsis} in the shock of embodiment. While the immortal soul is divine, the source of these disruptions is corporeal, and Timaeus represent these as a form of pollution. (A potential point of contrast with Theseus who claims the divine does not fear pollution from mortals, but this is not the usual Greek view, and perhaps this is nothing more than a comforting lie directed towards the still grieving Heracles, \emph{Heracles} 1230.)

Second, just past the isthmus is stationed a barracks of guardians, the heart. The spirited part of the soul is the most receptive to the discursive rational commands of the divine revolutions. The heart is located just beyond the isthmus, in part, to better hearken to sovereign command. The heart may become inflamed because of an external evil perceived by reason. But importantly the heart plays a policing role, as well, in keeping the unruly aspects of the appetitive soul in line so that they may not disrupt the divine revolutions of the soul. In these respects, in guarding against external and internal disruptions of the divine revolutions, the heart is part of the extended fortification of the acropolis.

% section reason (end)

\section{Spirit} % (fold)
\label{sec:spirit}

Just as the immortal part of the soul is separated from the mortal part by means of a narrow isthmus, the mortal part is itself divided. The mortal part of the soul is located in the thorax. The better part is separated from the worst part. The architectural imagery persists but has shifted to a more domestic setting. Timaeus asks us to envision the better part separated from the worst part as a division separating the men's chambers from the women's chambers, perhaps by means of a screen. The imagery of containment persists. The better and worse parts of the mortal soul are contained as occupants in divided chambers. This division is the midriff or perhaps the diaphragm more specifically. But Timaeus' description of this biological detail in terms of common feature of a domestic setting signals, by way of contrast with the divine splendor of the sovereign acropolis, that we have moved from the ruler to the ruled.

Spirit is housed in the better chambers, the men's chambers. These are nearer to the isthmus bordering the grounds of the acropolis. So the men's chambers in which the better part of the mortal soul is housed is the upper thorax, above the midriff or diaphragm.

Spirit animates social passion that aim at competitive advantage. Timaeus describes it as a love of victory. The spirited part of the soul is placed in the upper thorax so that it may better hearken to reason. Despite being fundamentally other regarding, the primary ethical function assigned to spirit involves maintaining order in the psychic \emph{polis}. In the virtuous, spirit hearkens to reason and so joins with it in putting down the unruly tribe of desires, the worse, appetitive part of the soul, located at a safe distance from the sovereign acropolis, in the lower thorax, below the midriff or diaphragm.

Within the men's chambers, that houses the better part of the mortal soul, the upper thorax, the Young God's place the heart. The heart is the primary organ associated with spirit. The heart is described as a knot of veins and a fount of blood that circulates through the limbs. This is a reasonably accurate if curiously generic description. Thus, for example, Timaeus seems not to mark the distinct roles of veins and arteries. However, Timaeus is not, here, primarily interested in the heart's role in the circulatory system. Rather, Timaeus is more concerned to place the heart in the political topology of the mortal soul, and explain how it may be the corporeal instrument for the rational oppression of the unruly tribes of desire when they fail to heed the commands of the sovereign acropolis.

The Young Gods place the heart in the upper thorax. The men's chambers must really be understood as a region of the psychic \emph{polis}, adjacent to the isthmus separating the acropolis. Timaeus describes the heart as the barracks for guardians. So these barracks are in the upper region adjacent to the isthmus but are not their soul occupants. The lungs, the secondary organs of spirit, are also located in the men's chambers. 

The heart's being described as a barracks highlights its ethical function. The barracks are located near the isthmus so that the discursively rational commands of the immortal soul are better heard. Again, the isthmus may separate, but it is so designed that there is a line of communication between the acropolis and the barracks of the guardians. The guardian's reception of discursively rational commands is described in terms of audition. And not any form of audition, but one of the special forms of audition involved in Timaeus' sensory soteriology. We may better harmonize with the revolutions of the World Soul by listening with rational attention to music or rational speech. The guardians housed in the barracks listen with rational attention to the rational commands of the immortal soul housed in the acropolis.

How are we to understand the communication between the immortal and mortal parts of the soul? Does the occupant of the acropolis merely issue sovereign commands or does it also receive reports on what transpires within the broader \emph{polis}? It is clear that the commands of the immortal part of the soul are discursively articulated. In understanding these commands, in whatever sense that they do, do the mortal parts of the soul receive discursively articulated contents that they can undersand? As we shall see, at least for the worse part of the mortal soul, it seems the answer is no. From the commands of the immortal soul being discursively articulated we may not conclude that their reception is itself discursively articulated. Even someone who denies that dogs have the power of speech will readily concede that there is some sense in which they understand and heed their master's discursively articulated commands. Understanding the nature of such communication is important for the topic of the present essay, the nature of perception. Perceptual powers are powers of the mortal soul and perception only occurs when it is reported to the \emph{phronimon}, a part of the rational, immortal soul. So the nature of the report or channel of communication bears on whether or not the content of perception is best conceived as discursively articulated. We shall return to this important issue after discussing the worse part of the mortal soul, appetite.

% section spirit (end)

\section{Appetite} % (fold)
\label{sec:appetite}



% section appetite (end)

\section{Aporia} % (fold)
\label{sec:aporia}



% section aporia (end)

\section{Marrow and Bone} % (fold)
\label{sec:marrow_and_bone}



% section marrow_and_bone (end)

% Chapter the_flesh_and_the_mortal_soul (end) 