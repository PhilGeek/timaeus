%!TEX root = /Users/markelikalderon/Documents/Git/timaeus/timaeus.tex

\chapter{\emph{Proemium}} % (fold)
\label{cha:proemium}

\section{A Double Significance} % (fold)
\label{sec:a_double_significance}

The first part of Timaeus's speech is traditionally called the \emph{proemium}, echoing Socrates' own description of it at 29d5 as \emph{prooimion} or prelude. Entitling the present chapter ``\emph{Proemium}'' thus has a double significance. It at once announces that the present chapter will take Timaeus' \emph{proemium} as its subject matter and that the chapter will itself serve as prelude to the essay as a whole.

% section a_double_significance (end)

\section{The Process of Perception} % (fold)
\label{sec:the_process_of_perception}



% section the_process_of_perception (end)

\section{\emph{Proemium}} % (fold)
\label{sec:proemium}



% section proemium (end)

% Chapter proemium (end) 