%!TEX root = /Users/markelikalderon/Documents/Git/timaeus/timaeus.tex

\chapter{\emph{Proemium}} % (fold)
\label{cha:proemium}

% \section{A Double Significance} % (fold)
% \label{sec:a_double_significance}
%
% Entitling the present chapter ``\emph{Proemium}'' thus has a double significance. It at once announces that the present chapter will take Timaeus' \emph{proemium} as its subject matter and that the chapter will itself serve as prelude to the essay as a whole.
%
% % section a_double_significance (end)


\section{\emph{Proemium}} % (fold)
\label{sec:proemium}

The first part of Timaeus's speech is traditionally called the \emph{proemium}, echoing Socrates' own description of it at 29d5 as \emph{prooimion} or prelude. The \emph{proemium} is bounded by the last two remarks that Socrates will make in the dialogue. In the first, Socrates praises the feast of \emph{logoi} that has been proposed as a return gift for his speech from the previous day on the topic of the \emph{Republic} and prompts Timaeus to begin by duly invoking the gods. In the second, Socrates praises Timaeus, accepting his prelude, and encouraging him to proceed. In the \emph{proemium}, Timaeus begins by invoking the gods. After which Timaeus states some general principles, which he uses to circumscribe his subject matter and describe the kind of account that he proposes to give of it. 

The principles include:
\begin{enumerate}[(1)]
	\item The distinction between that which always is and has no becoming and that which becomes and never is as explicated in terms of the the manner in which they are cognized
	\item The principle that everything that comes to be has a cause
	\item The principle that the cause of what has come to be is a maker that looks to a model
	\item The principle that if that which becomes and never is is beautiful (\emph{kalon}), then it has been modeled on what uniformly is by a benevolent maker
\end{enumerate}

That which always is and never becomes is the object of understanding (\emph{noēsis}) with an account (\emph{logos}) since it uniformly is what it is. That which becomes and never is is the object of opinion (\emph{doxa}) based on perception (\emph{aisthēsis}). That which always is and has no becoming is intelligible, and that which becomes and never is sensible. The Cosmos is sensible and thus must belong to the realm of Becoming. That which becomes and never is has a cause. This cause is a maker that looks to a model. Since the Cosmos is beautiful, then, by the final principle, Timaeus concludes that the Cosmos has an intelligible model and a good maker. It is the generation of the Cosmos, conceived as a beautiful likeness of an intelligible model, that will be the subject matter of Timaeus' speech. Accounts should be appropriate to their subject matter, and Timaeus concludes that since his subject matter is a likeness his account shall be ``likely''.

While the general outline of the \emph{proemium} is relatively clear, each element of this tightly constructed account requires comment as does their interrelation. 

% section proemium (end)

\section{The Invocation of the Gods} % (fold)
\label{sec:the_invocation_of_the_gods}

Timaeus, at Socrates' prompting, begins by invoking the Gods, and his prayer is addressed to the Gods invoked. Without dwelling overmuch on Timaean theology, at least three remarks are worth making about Timae\-us' invocation of the Gods. The first is an observation about who is not invoked, the second is a puzzle about who is invoked, and the third is an observation about the significance of a plural self-invocation with which Timaeus' prayer ends.

It is reasonable to call upon God at the outset of an undertaking (27c1--3). This is presented as a general practical precept. Later Timaeus explains that we invoke the Gods so as to pray for their approval of our undertaking (27c6--7). It would be unreasonable to fail to do so. And this remains so whether the undertaking is great or small. Timaues, however, proposes to undertake an account of the All (\emph{toū pantos}---understood, here, as the Cosmos without the explicit suggestion that it is ordered) and whether and how it was generated. This is a grand topic and a great undertaking, and so it is especially appropriate that Timaeus call upon the Gods at the outset.

What it is to pray for such a great undertaking is elaborated in later prayers. Timaeus briefly invokes an unnamed God and prays for salvation (48d4). The nature of such salvation is made clear in Timaeus' final prayer (\emph{Critias} 106a1–b6). There Timaeus prays that his words may endure if they are spoken truly and meet with divine approval. Timaeus further prays that should his speech be false that a just penalty be imposed, the just penalty in the case of error being correction. Finally, Timaeus prays for knowledge, the most perfect among medicines, so that he may speak truly in the future. Given his present undertaking, to account for the nature and generation of the All, to pray for salvation, is to pray that the account may endure if it is true and meets with divine approval. And if the account fails to do so, Timaeus prays that  justice may prevail, where justice involves both correction and reform. That is, falsehoods should be replaced by truths, and Timaeus should become knowledgeable so that he is less likely to reoffend. The salvation, then, is a kind of rational salvation.

It is striking who is not called upon by Timaeus. Notable in His absence is the Demiurge, the Maker and Father of the All. This divinity is difficult to discover and impossible to explain to all (28c3–5). He only begins to emerge in reasoning from principles that Timaeus only states after the invocation. Timaeus is invoking the Gods and addressing his prayers to Them, in part, in aid of this difficult discovery. If so, then the object of his search was not among the Gods that were invoked even it is itself divine.

There is a related reason for thinking that the Demiurge was not among the Gods invoked and prayed to. Even if it were possible to privately invoke the Gods and pray for Their approval of some solitary undertaking, this is not the context that Timaeus finds himself in. The invocation of the Gods urged by Socrates was, importantly, a public act of piety. And Timaues complies in the first person plural. It is we who are about to converse that have need to invoke the Gods and the Goddesses. A divinity difficult to discover and impossible to explain to all is perhaps not the most apt choice to invoke in a public act of piety. One's piety toward the difficult to discover divinity may be genuine, but it might not be publicly manifest. One's piety is only guaranteed to be publicly manifest if directed toward recognized divinities. And a difficult to discover divinity impossible to explain to all is not a recognized divinity. The obscurity of the Maker and Father of All is inconsistent with the public nature of the invocation.

Perhaps, in the present instance, the public is restricted to the divided reference of ``we''---Timaeus, Socrates, Critias, and Hermocrates. The restriction may be significant. A divinity not widely known to many may yet be known to the present few. However, while Timaeus might rely on Socrates to already know of this difficult to discover divinity, is it really reasonable that Timaeus should similarly rely on Critias? Even among the present dignitaries, the obscurity of the divinity is in tension with the publicity of the act.

As \citet[13--4]{Broadie:2012vl} observes, the Demiurge is not invoked in any of the prayers offered by Timaeus. Broadie's observation coheres with the present explanation. Since all of the prayers in the dialogue are public acts, the obscurity of the divinity would preclude His public invocation.

Not only is it notable that Timaeus does not invoke the Demiurge, but it is obscure who, exactly, is being invoked. Specifically, is Timaeus invoking one God or many? Notice he first calls upon God in the singular and then upon Gods and Goddesses in the plural. When Timaeus first calls upon God in the singular, it is in stating the general practical precept to call upon God and pray for approval of one's undertaking. When Timaeus calls upon Gods and Goddesses in the plural, this is not in the context of stating a general practical precept, but rather in the context of proposing to undertake an account of the All and its generation. Perhaps it is the Young Gods who are being invoked when Timaeus invokes the many. The Young Gods are divine, immortal beings generated by the Demiurge who are the instruments of Demiurgic activity. Specifically, as we shall see (chapters~\ref{cha:incarnation} and \ref{cha:the_flesh_and_the_mortal_soul}), they play an indispensable role in the generation of mortal beings. If the Young Gods are the many, who could the one be if not the Demiurge? One candidate is the All itself. Timaeus conceives of the Cosmos as a visible God (34a8, 34b8--9, 92a5--9), defers to it concerning its proper name in act of ritual piety (28b3--5, for discussion see \citealt[66]{Taylor:1928qb}), and addresses his final prayer to that God (\emph{Critias} 106a1--b6). By contrast, as \citet[35]{Cornford:1935fk} observes, ``Neither in the \emph{Timaeus} nor anywhere else is it suggested that the Demiurge should be an object of worship: he is not a religious figure.''

There are, of course, alternatives. Perhaps the God in the singular being invoked is a generic divinity of which the Gods and the Goddesses in the plural are species. On this alternative, an appeal to a generic divinity is appropriate in the statement of the content of a general practical percept. Similarly, it is appropriate to invoke the species of this generic divinity given the specific undertaking to account for the All. That all species of divinity should be invoked and their approval prayed for is perhaps only required for their specific undertaking of discoursing on the All and its generation. Perhaps a single divinity would do if the undertaking were small enough. But the collective undertaking that Timaeus proposes is suitably grand as to require the invocation of every species of divinity.

A deflationary take on the identity of the Gods invoked might be motivated by the perfunctory nature of the invocation. Proclus describes the worry as follows:
\begin{quote}
	But how can it be, they say, that Timaeus has announced with a grand flourish that one should pray and call on the gods and goddesses, but that he fails to do this himself and immediately turns to proposed accounts without praying? (Proclus, \emph{In Timaeum} 1 221.9--12, \citealt{Diehl:1903re}; \citealt[58]{Runia:2008aa})
\end{quote}
If Timaeus has not in fact prayed, then there is no point in inquiring into to whom Timaeus has addressed his prayer. (That the invocation of the Gods should strike some as a grand flourish is worth remarking on. As \citealt[104]{Runia:1997vz}, observers, ``Only a minority of \emph{prooemia} begin with an \emph{invocation of the gods}, continuing in the tradition set by Hesiod in his \emph{Theogony}.'') Proclus' reply is ingenious in that he, in effect, proposes that we understand Timaeus' speech as performative in something like Austin's \citeyearpar{Austin:1975nx} sense. In saying that we must pray to the Gods, Timaeus has done just that. While one may decide to pray and pray later, in true prayer, the decision to pray and prayer are one and the same thing. Timaeus's public prayer on the occasion of the Panathenaea is a true prayer. And so in saying that we must pray, he does.

Still, even if a deflationary take on the Gods invoked may be resisted in this way, exactly who is being invoked by Timaues may not be so neatly resolvable. Perhaps there is a kind of semantic slippage here, from singular to plural, the one resolving into the many. If so, this will be echoed in the third part of Timaeus' speech which concerns the interaction of Reason and Nececssity, especially in his account of the generation of the flesh and the mortal soul. There, Timeaus slides freely between singular and plural in describing what is officially meant to be rational cooperative activity, the Demiurge generating the immortal part of the soul and the Young Gods weaving it together with its mortal body that they generated from corporeal material borrowed from the body of the Cosmos. However, even for tasks that are specifically meant for the Young Gods, their executions are sometimes described in the singular. We shall discuss the significance of this when we discuss the flesh and the mortal soul (chapter~\ref{cha:the_flesh_and_the_mortal_soul}). For now, I merely note a potential anticipatory trace of this puzzling usage. 

One last observation. Not only are the Gods invoked, but Timaeus also invokes themselves. The plural self-invocation is for the audience---Socrates, Critias, and Herm\-ocrates---to more easily learn and for the speaker---Timaeus---to more clearly expound his account of the All and the cause of its generation. (On the self-invocation see Proclus, \emph{In Timaeum} 1 222.11--223.2, \citealt{Diehl:1903re}.) Not only is Timaeus praying for the approval of the Gods concerning their present collective undertaking---his conversing with them about the nature and generation of the All---but he is importantly also praying that they themselves approve of their discourse, if only after being granted divine approval. While for the most part, Timaeus delivers an uninterrupted speech, in the invocation he describes it as a collective activity with the audience as co-participants. Perhaps the self-invocation is meant to get the participants of the discourse in a position where they may engage in it in such a way that they come to themselves approve of that discourse should it merit their approval. This would involve the audience rationally attending to the speaker's account so that they may more easily learn his views about the nature and generation of the All. And this would involve the speaker clearly expounding his views to this audience. Even if largely a monologue, Timaeues' speech remains directed to its audience. And Timaeus must bear them in mind if he is to clearly expound his views to them in such a way as warrant their collective approval. Only in this way could the return speech discharge Timaeus' obligation to Socrates. Notice that it is his views that Timaeus wishes his audience to learn. Contrast learning the truth about the All and its generation. Timaeus' account may be true but he is pointedly not proposing to demonstrate such truths. Timaeus, here, is quietly allowing room for what will emerge as accounts with a specific epistemic status, ``likely'' accounts.

If the plural self-invocation still seems odd or unfamiliar, it really should not. You have doubtless encountered it in contemporary speech. Think of someone inaugurating their address with ``Let's do this!'' Think of a motivational speaker or, better yet, a Hip Hop artist. ``This'', that which is to be done (dropping bars for the Hip Hop artist, cosmology for Timaeus, both are \emph{logoi}), is explicitly represented as a collective activity---``Let \emph{us} do this'', where ``us'' includes both speaker and audience. In saying ``Let's do this!'', the speaker draws attention to themselves, making their audience more receptive to their address, and the speaker expresses their determination in a way that draws the speaker's own attention to their audience, in an attempt to thereby more effectively deliver their address. Should the speaker prove genuinely effective, they may even garner approval for that address. It even has, in line with Proclus' suggestion, a performative aspect. In saying ``Let's do this!'', that which is to be done has already begun.

% section the_invocation_of_the_gods (end)

\section{Being and Becoming} % (fold)
\label{sec:Being and Becoming}

Having duly invoked the Gods and Goddesses, Timaeus begins by stating that, in his opinion, a distinction should be drawn between: 
\begin{enumerate}[(1)]
	\item that which always is and has no becoming (\emph{to on aei, genesin de ouk echon}) and
	\item that which becomes and never is (\emph{to gignomenon men, on de oudepote}).
\end{enumerate}
Three observations are immediately relevant. 

First, the ontological distinction is presented as Timaeus' opinion, or perhaps judgment more broadly, and not as self-evident or subject to demonstration. Perhaps the distinction is, in fact, subject to demonstration from first principles, but if it is, then the demonstration is beyond the scope of Timaeus' present undertaking (53d6--7). Timaeus has undertaken to account for the nature and generation of the All, and not to account for first principles from which the principles of his account may be derived. Speaking anachronistically, Timaues has invoked the Gods and Goddesses in aid of his discourse on cosmology, not metaphysics (on this theme see \citealt{Broadie:2012vl}). Again, Timaeus is quietly make room for an account with a specific epistemic status, ``likely'' accounts.

Second, the distinction provides us with two cases and each case is characterized by positive and negative claims. On the one hand, there is that which always is (positive) and has no becoming (negative). On the other hand, there is that which becomes (positive) and never is (negative). The abstract form of the distinction echoes the distinction given by the unnamed Goddess of Parmenides' poem (DK 28B2)---that which is (positive) and is impossible not to be (negative) (\emph{estin te kai ōs ouk esti mē einai}) and that which is not (positive) and is needful not to be (negative) (\emph{ōs ouk estin te kai ōs chreōn esti mē einai}). Both distinctions, Timaeus' and the unnamed Goddesses', concern Being, and though they conceptually differ, they are, nonetheless, in this, way formally similar. Perhaps the echo is deliberate. At any rate, as we shall see, Timaeus' approach to elucidating this distinction follows the unnamed Goddess' lead (on the historical background to Timaeus' \emph{proemium} see \citealt{Naddaf:1997jt} and \citealt{Runia:1997vz}, on Plato and Parmneides more generally see \citealt{Palmer:1999uq}).

Third, there is a minor textual issue here. Timaeus begins by describing that which always is (\emph{to on aei}). Perhaps for stylistic reasons, it is natural to expect a parallel description of the contrasting case. This would involve a second \emph{aei}: \emph{to gignomenon men aei}---that which always becomes. But as \citet{Hackforth:1959dj} observed, \emph{aei} is omitted in manuscripts F and Y, by neo-Platonist commentators such as Proclus and Simplicius, and in the Latin translations of Cicero and Calcidius. Subsequent scholarship supports Hackforth's suggestion that the second \emph{aei} is an unwarranted emendation (\citealt{Whittaker:1969mq,Whittaker:1973nz}, \citealt{Robinson:1979wk}, and \citealt{Dillon:1989hc}). 

(1) \emph{That which always is and has no becoming}. On a flat-footed reading, tempting to moderns, that which always is is that which exists always. However, the Greek verb \emph{enai} and its cognates differ semantically from the corresponding English verbs (see \citealt{Kahn:2009kx}, \citealt{Brown:1994aa}, and \citealt{Leigh:2008aa}). Both in Greek and in English such a verb may occur with and without a grammatical complement. But whereas in English an occurrence of the verb ``to be'' without a grammatical complement merely means that the subject exists, in Greek it need not merely mean this. 

While the semantics of the Greek verb may allow for a broader meaning than in English, there is an independent motive for such a reading. That which always is never becomes, and the meaning of the latter potentially constrains the meaning of the former. 

Just as we should bear in mind the range of meanings of the Greek verb \emph{enai}, similarly we should bear in mind the range of meanings associated with the Greek verb \emph{gegonen}, ``to come to be.'' To come to be can be a matter of coming into being, to have a beginning, but, importantly, to come to be can also be a matter of coming to be some way there is for things to be \citep[24--5]{Cornford:1935fk}. If we do not restrict the range of available meanings, then in claiming it never comes to be, Timaeus claims that what always is not only had no beginning, but that it never changes and never perishes.

That which always is may always exist but, importantly, it also always is what it is. If it always is what it is and thus retains a stable character, then this character is not subject to change and in that sense that which always is has no becoming. That which always is has no becoming since it always already is what it is thus leaving no scope for it to come to be some way there is for things to be. 

However exactly we are to understand that which always is, that understanding should coordinate with our understanding of the corresponding contrast, that which never is, and as we shall see, the negative existential reading of the latter is implausible.


(2) \emph{That which becomes and never is}. Again, if we do not restrict the range of available meanings, that which becomes includes that which comes into being, changes, and perishes. At least initially, I am inclined to read what has come to be (\emph{to gignomenon}) widely enough to include anything that has at least one of these features. Even something that has always been without beginning but is subject to change would count as what becomes.

One difficulty in understanding this contrasting case is understanding in what sense what comes to be never is. If the complete use of \emph{enai} is existential, then Timaeus would be claiming that what becomes does not exist. While Melissus might embrace such a conclusion (DK 30B1--3), Timaeus' sympathies, and his author's, lie in a different direction. Plato, for example seems to extend the notion of Being to what comes to be in the \emph{Philebus} 27b8. As for Timaeus, he will subsequently describe what comes to be as clinging to existence (52c), and one cannot cling to what one does not have. What, then, could the denial mean if it is not simply a negative existential?

Bear in mind that what comes to be includes not only what comes into being but what comes to be some way there is for things to be. Perhaps coming to be some way there is for things to be is never fully realized. While there is movement towards being some way there is for things to be this aspiration is never completely fulfilled. Perhaps, we can understand the contrast here between a sensible being and the Form of that sensible being. The sensible being may be F by participating in the Form but it is not fully F the way the Form is. The idea is that the denial is not that what becomes never is, existentially, but rather never really is, for some appropriate understanding of ``really''.

So much for preliminaries. Timaeus, himself, does not take the distinction as evident. Indeed Timaeus first raises the distinction in the context of a question. What is that which always is and has no becoming and that which becomes and never is? The corresponding epistemological distinction that Timaeus now introduces is meant to be an answer.  What kind of character do these things have? The kind of character exhibited by the objects of understanding (\emph{noesis}) and opinion (\emph{doxa}). Specifically: 
\begin{enumerate}[(1)]
	\item that which always is and has no becoming is the object of \emph{noēsei meta logou perilēpton}, and
	\item that which becomes and never is the object of \emph{doxē met' aisthēseōs alogou doxaston}.
\end{enumerate}
Just as the unnamed Goddess explicates that which is in terms of that which can be known should one follow the Way of Truth, Timaeus explicates that which always is in terms of that which can be apprehended in understanding \emph{meta logou}. In general, Timaeus explicates the ontological distinction in terms of a distinction between the objects of distinct cognitive attitudes.

Before we can understand how the ontological categories are explicated as the objects of distinct cognitive attitudes, we must first get clearer on the nature of these attitudes. Specifically, we need to get clearer about the qualifications on understanding and opinion that we have so far left untranslated. 

(1) \emph{Noēsei meta logou perilēpton}. That which always is and has no becoming is the object of \emph{noēsei meta logou perilēpton}. The attitude is a kind of understanding (\emph{noēsei}). The object of this understanding is apprehended (\emph{perilēpton}). Interestingly, while Timaeus could have said \emph{lepton} here, he says, instead, \emph{perilēpton}. This potentially highlights, I think, the significance of \emph{peri}-. Timaeus is pointedly claiming noetic apprehension circumscribes its object. This is an anticipatory trace of an important theme in Timaean psychology, the circular motion of cognitive activity (discussed in chapter~\ref{cha:cognitive_revolution}).

The understanding whose apprehension circumscribes its object is \emph{meta logou}. How are we to understand the occurrence of \emph{logos}? Given the breadth of semantic field, there are options (though some may not be relevant, such as \emph{logos} understood as ratio). 

Perhaps \emph{logos} means reason or rational, in which case \emph{noēsei meta logou perilēpton} means something like apprehended in understanding with reason, or rational understanding, and perception would be described as without reason or irrational (\citealt[87]{Archer-Hind:1888qd}). In support of this reading, one might cite the cognitive disruption occasioned by the linear impact of \emph{aisthēsis} in the shock of embodiment (43a--44c, discussed in chapter~\ref{cha:incarnation}). But if \emph{aisthēsis} occasions cognitive disruption, it also providentially provided to resolve any such disruption as it may occasion. The Young Gods provide us with eyes to see with and ears to hear with in order correct the revolutions in our soul disrupted by our initial encounter with the sensible (46e--47e, discussed in chapter~\ref{cha:the_end_of_vision_and_audition}). Given Timaeus' sensory soteriology, it is misleading at best to describe perception as irrational. Another potential source of support is the distinction that Timaeus draws between rational an irrational motion. Circular motion is rational and the six linear motions (forward, backward, left, right, up, and down) are irrational. Timaeus associates circular motion with cognitive activity and explicitly describes \emph{aisthēsis} as linear motion.

Perhaps, though, \emph{logos} may mean reasoning, in which case \emph{noēsei meta logou perilēpton} means something like apprehended in understanding based upon reasoning. And since perception does not involve reasoning, describing it as unreasoning seems both apt and true. (\citealt{Bury:1929jb} gives this interpretation in his translation.) 

\emph{Logos}, however may mean account, in which case \emph{noēsei meta logou perilēpton} means something like apprehended in understanding with an account. And since perception does not involve an account of its objects, describing it as without an account also seems both apt and true. (\citealt[61]{Taylor:1928qb}, \citealt[22]{Cornford:1935fk}, and \citealt[16]{Waterfield:2008lx}, give interpretations of this kind, though this is obscured in Taylor's, \citeyear[25]{Taylor:1929ov}, translation.) These last two readings are distinct, if not unrelated. For mortals at least, coming to have an account of an object of understanding involves discursive reasoning.

Since the present occurrence of \emph{logos} contrasts with the way that perception is \emph{alogos}, these should receive coordinated readings. (Not all adhere to this precept, however; see, for example, \citealt[13]{Zeyl:2000cs}.) As such, we should hold off on how to interpret the present occurence of \emph{logos} until we understand in what sense perception is \emph{alogos}.

Another issue concerns the force of \emph{meta}. Suppose, for the sake of argument, that \emph{logos} means a discursive account. Understanding with an account might be understood as the conjunction of distinct conditions. There is the noetic apprehension that circumscribes its object and there is the possession of an account of what is thus apprehended. Understanding with an account occurs when both of these conditions are met. Understanding with an account might, however, be understood in another way, not as understanding that meets a distinct discursive condition, but rather where the account is the discursive means by which the object is understood. One comes to understand an object by engaging in the discursive reasoning required to come to possess an account of it, and one noetically apprehends that object when one comes to possess an account of that object. So conceived, understanding and possessing an account are not distinct conditions, and so \emph{noēsei meta logou perilēpton} is not the conjunction of distinct conditions. 

While the immediate passage in which \emph{noēsei meta logou perilēpton} occurs cannot, by itself, determine which of these two interpretations are correct, I am nevertheless inclined to favor the latter. There is, however, a complication facing any such interpretation. An account is discursively articulated into conceptually distinct parts. An account, in this way, admits of decomposition (in something like Dummett's \citeyear{dummett73} sense). It is plausible that the process of decomposition cannot carry on indefinitely, and that this is a condition on at least the mortal intelligibility of an account. This means that an account will have as parts things which are not themselves subject to account. How are the conceptually primitive parts of an account cognized? Certainly not by understanding with an account. Since Timaeus is explicating what always is in terms of what can be understood with an account, are we to conclude that the conceptually primitive parts of an account never are but come to be? That seems implausible. Suppose one held that the conceptually primitive parts of the account were cognized in some other way, by means of some other cognitive attitude. Then since possession of an account would involve the apprehension of its primitive parts, this attitude would be more fundamental than \emph{noēsei meta logou perilēpton}. But there is no explicit mention of such an attitude in the \emph{proemium}. Perhaps the Timaean formula is meant to be understood broadly enough so as to include noetically apprehending an object through a discursive account, as well as everything that is required to have that attitude such as the apprehension of the conceptually primitive parts of that account.

(2) \emph{Doxē met' aisthēseōs alogou doxaston}. That which comes to be and never is is the object of \emph{doxē met' aisthēseōs alogou doxaston}. Two questions arise. What is the relationship between opinion and perception? And in what sense is perception \emph{alogos}? 

Perhaps while understanding, at least for mortals, is based upon reasoning, opinion is based upon perception. So understood, \emph{doxa} is conceived as something like a perceptual judgment, a judgment formed on the basis of perception. Perhaps Timaeus has in mind a generalized notion of perceptual judgment, call it empirical judgment. The subject matter of an empirical judgment may not be perceived, but it may yet count as empirical if it is ultimately based upon perception. (Thus Timaeus will maintain that the sensible and corporeal are composed out of elemental triangles too small to be seen. Judgments about these are nevertheless empirical in the sense adumbrated.)

Even on this generalized conception of perceptual judgment---empirical judgment---perception is a necessary condition on opinion. One potential difficulty with this reading is that the World-Soul has opinion but lacks the instruments of perception. The Cosmos has no eyes to see with nor ears to hear with (33c). Does this mean that the World-Soul has opinion without perception? If so, then perception is not a necessary condition on opinion, and so could be neither perceptual nor empirical judgment. 

This difficulty might be avoided by claiming that the epistemological principles that Timaeus provides us are principles of mortal cognition. Timaeus and his invoked audience are mortals conversing on the nature and generation of the All---as are Plato and the readers of his dialogue. The principles given in the \emph{proemium} are principles of human cosmology and so make no claim about cosmic cognition. But the epistemological principles of the \emph{proemium} do not seem to be restricted to mortal cognition in this way. When Timaeus discusses the cognitive activity of the World Soul, he does so in the terms set out in the \emph{proemium} (chapter~\ref{cha:cognitive_revolution}).

Another response to this difficulty is to claim that only mortal perception requires corporeal instruments (Proclus, \emph{In Timaeum} 2 83.3–85.31, \citealt{Diehl:1903re}). Perception is a necessary condition on opinion. The World Soul opines and so must perceive. The sensible and the corporeal are, for the most part, exogenous to mortal perceivers and so mortals require corporeal instruments in order to receive their affections from without and so come to perceive them. The sensible and the corporeal, however, are generated within the World Soul and so are endogenous to it. Since the sensible and the corporeal are endogenous to the World Soul, the World Soul does not require corporeal instruments to perceive them. Opinion may require perception, but cosmic perception does not require instruments. In this way cosmic perception is akin to bodily self-awareness. On this response, the objection goes wrong in inferring a lack of cosmic perception from a lack of cosmic instruments of perception. (The Proclean interpretation is discussed further in chapter~\ref{sec:knowledge_and_opinion}).

In what sense is perception \emph{alogos}? This narrow question of detail may nonetheless be significant given the theme of the present essay for it bears on the nature of perception. Timaeus' claim, here, seems consistent with each of the canvassed readings of \emph{logos}, at least with qualification. The object of perception is not perceived on the basis of reasoning. Nor does perception provide and account of its object. Perception may even be irrational, if you like, so long as a power providentially provided to promote reason and virtue may be understood to be irrational in some suitable sense. 

Earlier we noted that the occurrence of \emph{logos} in \emph{noēsei meta logou perilēpton} could not be interpreted as ratio. It is also worth observing that perception being \emph{alogos} could not mean that perception is without ratio or somehow disproportionate. As we shall see (chapters~\ref{cha:common_pathemata} and \ref{cha:peculiar_pathemata}), Timaeus understands perception on the model of measurement (a model taken up by Aristotle in \emph{Metaphysica} I 1053a31). Bodily affections are measures of the powers of the agents that caused them, and perception is the cognizance of what is thus measured. At the very least, then, the objects of perception and the \emph{pathēmata} are proportional if the latter can measure the former. Notice that ratios are not restricted to \emph{arithmoi}, and so powers and \emph{pathēmata} may be proportional. Thus, for example, Timaeus will go on to claim that the four primary bodies---fire, air, water, and earth---are bound together in the body of the Cosmos by proportion (31b--32b, see also Aristotle, \emph{Physica} 3 204b14–19).

Perhaps, \emph{alogos}, here, means something like non-discursive. Doing so pro\-mises to at least capture what was right about each of the alternatives. If perception is non-discursive it is not based on discursive reasoning, nor does it provide a discursive account of its object, nor is it discursively rational. The object of perception may be discursively articulated in perceptual judgment but it is reason, not perception, that does so, The object of perception is not discursively articulated the way reasons are. Understanding \emph{alogos} as non-discursive seems to capture what was right about each of the alternatives since each of the alternatives involved discursive commitments.

A worry may be raised about \emph{alogos} understood as irrational being glossed as not discursively rational. On that gloss, the object of perception is not a discursive reason, and so in that sense irrational, since it is not discursively articulated the way discursive reasons are. However, one may reasonably wonder whether this is, in fact, Timaeus' view. As we shall see (chapters~\ref{cha:common_pathemata} and \ref{cha:peculiar_pathemata}), sensible qualities are powers of agents that affect the corporeal instruments of perception and these qualities are perceived when these powers are ``reported'' to the \emph{phronimon}, the seat of cognizance (64b4). Since perception only occurs with the report, does this not mean that the object of perception is discursively articulated?

Whether or not it has this implication very much depends on how the report is itself understood. If it is understood as speech act, then the report is discursively articulated, and since it captures the content of perception, then it is plausible that that content is itself discursively articulated. At the very least, the content of perception would be discursively articulable. If the report is a speech act, then there is a speaker, the agent of that speech act. But who is the speaker? Timaeus will distinguish immortal and mortal parts of the soul, maintaining that its rational powers are invested in the immortal part. Is it the mortal soul that animates the body that is reporting the power of the agent that affected that body to the immortal soul? But is the mortal soul sufficiently personal to intelligibly ascribe speech acts to? Perhaps the report should not be understood as a speech act but is merely a means of transmission of information. If the report is not understood as speech act, but merely as a means of transmission, then the implication that the content of perception is discursively articulated is avoided. So understood, information about the presence of a power in the perceiver's environment is transmitted to the \emph{phronimon}. Even on this reading, however, the \emph{phronimon}'s uptake of the information transmitted, its cognizance of the power, might itself be discursively articulated. We shall return to this interpretive issue.

Now that we have a better sense of the relevant cognitive attitudes, how do they explicate the ontological status of their objects? Very roughly, these objects have to be the way they are in order to be the objects of these attitudes. In gaining insight into what it takes to be the object of these attitudes we gain insight into the character of these objects. This need not be understood in terms of the measure doctrine that Socrates attributes to Protagoras in the \emph{Theaetetus}. For all that has been said, the objects of cognition can be the way they are independently of being cognized and yet their being that way ensures that they are cognizable. And if they are the way they are independently of being cognized, and the way they are ensures that they are cognizable, then reflecting on what it takes to be cognizable will reveal how these objects are independent of our cognition of them.

Begin with \emph{noēsei meta logou perilēpton}. What must something be like to be the object of this attitude? Timaeus offers an explicit answer. What always is and has no becoming is the object of \emph{noēsei meta logou perilepton} because it always uniformly is (\emph{aei kata tauta on}). Recall, what always is has no becoming not only in the sense that it lacks a beginning, and so does not come into being, but also in the sense that it never comes to be some way for things to be. What always is is always already what it is. It fully is what it is leaving no further scope for it coming to be some way there is for things to be. 

What always is is some way there is for things to be. What always is is the way it is eternally, and so does not vary over time. Now we learn that neither does it vary across its parts, in some suitable sense ``part'' (on Platonic mereology see \citealt{Harte:2002tl}). So conceived, that which always is and has no becoming uniformly is what it is. This is a deliberate Eleatic echo. Parmenides conceived of the One Being as spatial and finitely extended in every direction thus forming a voluminous sphere. When Parmenides claims that the One Being is uniform, he is attributing to it a kind of self-similarity. Every part of the voluminous sphere is like every other part. It will turn out that Timaeus, unlike Parmenides, denies that what always is is spatially extended. Nevertheless, what always is is uniform in the sense of being self-similar. In not being spatially extended, what always is lacks spatial parts. But so long as parts are not restricted to spatial parts, it may yet be true that every part of what always is is just like every other part. Recall what always is is apprehended with an account. An account is discursively articulable, so perhaps what always is has conceptually distinct parts, the parts articulated in the associated account. And even if what always is lacks parts distinct from the whole even in a non-spatial sense, it will still be trivially true that it is uniform in the sense of being self-similar. In general, if something is self-similar, in the relevant sense, then if x an y are parts of the thing distinct from the whole of it, then x an y are not dissimilar (and, hence, exactly similar). If what always is lacks parts distinct from the whole of it, then the antecedent of the embedded conditional is false, and that conditional trivially true.

On the present interpretation, uniformity of being is a condition on being noetically apprehensible. As we shall see (chapter~\ref{cha:cognitive_revolution}), this is confirmed in Timaeus account of the cognitive activity of the World Soul. And again, Timaeus transforms the Eleatic framework that he builds upon. 

Parmenides maintains that only Being is intelligible. And so only Being is the object of thought. Parmenides conceives of the One Being as spatially extended and finite, in the shape of a voluminous sphere. If the thought that thinks the One Being involves spatial movement, it could only go around that sphere. Thought is about its object by going around it. Thus the unnamed Goddess that reveals the Way of Truth to Parmenides proclaims that her thought is \emph{amphis alētheis}.

Timaeus accepts the the Eleatic pun where what the motion is about (\emph{amphis}, \emph{peri}), the center, is what the thought is about (\emph{amphis}, \emph{peri}), its object (\citealt[191--3]{Mourelatos:2008ve}). As we shall see (chapter~\ref{sec:the_cognitive_significance_of_circular_motion}), however, the Eleatic pun is transformed in his hands. It is at least open to claim that, for Parmendies, thought literally moves around its object. However, Timaeus thinks that the objects of noetic apprehension lack extension. So noetic apprehension could not be spatial movement around the contours of its object. And yet Timaeus retains the Eleatic pun despite rejecting the Eleatic claims upon which it rests. There is a shift of sense here. And not just in what is meant by the pun, but also in the meaning of it. For Timaeus, the Eleatic pun could only be a metaphor.

The circular motion around the object of noetic apprehension is the incorporeal activity of a cognitive act that encompasses its object. In order for this incorporeal activity to encompass its object in an act of noetic apprehension, the circular motion must be uniform (34a3, 36c2, 40a7). If as Aristotle claims (\emph{De anima} 1.2 404b16--18, 1.3 406b28--31, see also Alcinous \emph{Didaskalikos} 14.4), Timaeus accepts the principle that like is known by like, on some suitable understanding of that principle (\citealt{Corcilius:2018bd}, chapter~\ref{cha:cognitive_revolution}), then we can understand uniformity as a condition on being noetically apprehensible as follows. What always is can be apprehended in understanding with an account. Understanding with an account involves uniform incorporeal activity that circumscribes its object. Like is known by like. Only that which is uniform may be apprehended by an incorporeal activity that is itself uniform. Therefore, what always is must itself be uniform.

Consider now \emph{doxē met’ aisthēseōs alogou doxaston}. What must something be like in order to be the object of this attitude?

First, notice that what becomes is not understood with an account. From this we may conclude that it is not uniform. Indeed, what becomes is not uniform or self-similar. What becomes comes into and goes out of being and changes over time. Moreover, what becomes may have parts distinct from the whole that vary qualitatively. Thus the pre-cosmic chaos may be fleetingly fiery over here, and fleetingly earthy over there, such that not every part of the pre-cosmic chaos is alike, a general feature that the Cosmos itself retains despite its orderliness. Not every part of the extended sensible Cosmos is alike.

Second, while non-uniform and so not the object of understanding with an account, what comes to be and never is not utterly uncognizable. Not only may we opine about what comes to be and never is, it is possible to have true opinion about the realm of Becoming. If the unnamed Goddess is really understood to be claiming that all opinion about what comes to be is false, then Timaeus differs, in this way, from Her. Like the unnamed Goddess, Timaeus accepts that opinion differs from understanding and knowledge. They disagree about how to understand this difference. According to Timaeus, opinion differs from understanding not because it is systematically false or misleading, but because a successful account of the sensible realm of Becoming has an epistemic status distinct from an account involved in understanding. As we shall see, such an account is ``likely''. (On Parmenides and Timaeus on ``likely'' accounts see \citealt{Bryan:2012bt}.)

Third, uniformity is a positive feature that explains, at least in part, why something is the object of uniform cognitive activity that constitutes understanding with an account. However, Timaeus does not correspondingly cite a positive feature that explains, even in part, why something is the object of opinion. Instead, Timaeus draws our attention to the fact that what becomes comes into and goes out of being and is subject to change. Far from being features that explain the cognition of these objects, these features seem, at least initially, to be obstacles to their cognition (compare \emph{Cratylus} 439d–e and \emph{Theaetetus} 181c–183b; this asymmetry is obscured in Zeyl's \citeyear[13]{Zeyl:2000cs}, translation.) Timaeus' thought is that the objects of opinion are cognizable despite coming to be because they are sensible. Here the occurrence of \emph{meta} must mean something like on the basis of. Opinion is about what comes to be and never is on the basis of perception. Instead of a positive feature that explains why something is the object of of opinion, we have an obstacle to cognition that could only be overcome by perception.

This interpretation again faces the issue that the World Soul truly opines about the sensible realm of Becoming but lacks the instruments of perception. If we may infer a lack of perception from the lack of an instrument of perception, then the World Soul's opinions are not based on perception, and \emph{meta} in \emph{doxē met’ aisthēseōs alogou} must be understood in some other way. Or perhaps Proclus is right (\emph{In Timaeum} 2 83.3–85.31, \citealt{Diehl:1903re}), and we should not infer a lack of perception from a lack of an instrument of perception. Perhaps lacking an instrument of perception is just a matter of how cosmic perception differs from mortal perception. If the World Soul may perceive, then there is no obstacle to the World Soul truly opining on the basis of what it perceives. We shall return to this issue (chapter~\ref{sec:knowledge_and_opinion}).

What always is and has no becoming is apprehended in understanding with an account. Since it is uniform in its being, it may be encompassed by uniform intellectual activity. It is thus intelligible. What comes to be and never is opinable with perception. Though non-uniform in coming to be, it is nonetheless cognizable since it is sensible and so opinable. It is thus sensible.

This last observation, though quotidian, is nevertheless worth emphasizing, given the topic of the present essay. Recall the ontological categories are being explicated in terms of the objects of distinct cognitive attitudes. We learn something about their character by learning what about them enables them to be the objects of these attitudes. What becomes and never is is sensible and so cognizable by opinion. The realm of Becoming is, by nature, sensible. Only if it were sensible could it so much as be the object of opinion. There is no room, here, for conflict between a scientific image of nature, provided by Timaeus' ``likely'' account, and the manifest image of nature. That nature is manifest in perceptual experience is what makes it subject to opinion and, hence, to the kind of ``likely'' account that Timaeus unfolds in his speech after the \emph{proemium}.

% section Being and Becoming (end)

\section{The Cause of All} % (fold)
\label{sec:the_cause_of_all}

Having distinguished that which always is and has no becoming from that which becomes and never is, and having explained this distinction in terms of the objects of distinct cognitive attitudes, Timaeus now states a causal principle:
\begin{enumerate}[(1)]
	\item Everything that comes to be of necessity comes to be as the result of some cause.
\end{enumerate}
Timaeus' next claim expands on the cause of the All. Timaeus claims that when a maker looks to a model that is uniform, the product is necessarily beautiful. This is really two claims in one:
\begin{enumerate}[(1)]\addtocounter{enumi}{1}
	\item The cause of that which comes to be is a maker that looks to a model, and
	\item If the model that the maker looks to is uniform, then the product is necessarily beautiful,
\end{enumerate}
Let us examine these in turn.

(1) \emph{Everything that comes to be of necessity comes to be as the result of some cause}. It is unclear, yet, what, for Timaeus, counts as a cause (\emph{aitia}). The second principle promises to shed light on this. However Timaeus understands this explanatory determination, all that comes to be is so determined. This is a strong claim. It rules out the possibility of spontaneous generation and that the realm of Becoming is fundamentally chancy. (Indeed it seems to be an early occurrence of the Principle of Sufficient Reason, if not the earliest. Thus Aristotle seems to ascribe such a principle to Anaximander, \emph{De Caelo} 295b10--16, a principle that Parmenides can be read as accepting as well, DK 28B8 9--10.) And yet Timaeus provides no justification for it. (Though, as Leibniz quips to Clarke, ``Has not everybody made use of this principle upon a thousand occasions?'', \citealt[346]{Ariew:1989la}.) What seems like a justification is really just a restatement of the principle. Thus Timaeus claims that everything that comes to be of necessity results of some cause since if there were no cause, it would be impossible for anything to come to be. This is less a justification or explanation than a contrapositive equivalence. Perhaps a non-circular justification could only be provided by appeal to first principles and so would be beyond the scope of Timaeus' speech (53d6--7).

Timaeus needs this strong principle, however, since the existence of the difficult to discover divinity, the Demiurge, Maker and Father of the All, depends upon it. Timaeus will reason that since the All is generated, then, by our principle, the All has a cause, and the Demiurge is identified with the cause of the All. 

Besides being necessary to discover this obscure divinity, impossible to explain to all, the strength of the principle has another significance. If everything that comes to be of necessity has a cause, then the sensible Cosmos is intelligible in the minimal sense that it is subject to causal explanation. Whatever occurs in the sensible Cosmos does so for a reason.

(2) \emph{The cause of that which comes to be is a maker that looks to a model}. Timaeus conceives of the cause of the All as a maker. This involves three separable elements. There is the maker. As we shall see, this is the difficult to discover divinity, the Demiurge, Maker and Father of the All. (On the importance of the separability of the Demiurge see \citealt[chapter 1]{Broadie:2012vl}.) Then there is the product, the All, the sensible Cosmos. Finally, and implicitly, there are the materials out of which the product is made. As we shall see, these are the contents of the Receptacle. 

Importantly, with this premise, Timaeus understands, or perhaps reconceives, the explanation of the generation of the All as an action explanations. The model which the maker looks to figures, as it were, in the content of the intention in the making of the product. The maker, in looking to the model, uses the model to guide their productive activity. (On the connection between this and Timaeus' teleology see \citealt{Johansen:2004dx}.)

Earlier I observed that the sensible Cosmis is intelligible in the minimal sense that it is subject to causal explanation. Since Timaeus conceives of the relevant causal explanation as an action explanation, the sensible Cosmos is intelligible in a richer sense still. The Maker of the Cosmos has a reason for engaging in His productive activity and so there is a divine reason for the sensible Cosmos coming to be.

(3) \emph{If the model that the maker looks to is uniform, then the product is necessarily beautiful}. This hearkens back to the ontological distinction between what always is and has no becoming, and that which becomes and never is. Recall that that which always is and has no becoming always uniformly is (\emph{aei kata tauta on}). What becomes and never is, by contrast, is non-uniform. So we have here a claim about the consequence of the maker looking to a certain kind of model, a uniform model consisting of that which always is and has no becoming. 

When looking to such a model the product is necessarily beautiful. And only when looking to such a model. If the model is non-uniform, if the model consists in what becomes and never is, the result is not beautiful. 

There is an additional element missing from Timaeus' explicit formulation at this point. It explicitly emerges when Timaeus applies the principle. Specifically, the maker must be benevolent. This might tie in with Timaeus' conception of the relevant causal explanation as an action explanation since, according to Timaeus, one always acts under the guise of the good. Thus in a choice situation it is rational to choose the best option. That, by itself, is too weak, however, if even the vicious act under the guise of the good. Perhaps, as Socrates maintains (for a particularly clear statement see Xenophon \emph{Memorabilia} 3.9.5), the vicious are simply ignorant of the good. If the maker is somehow under a misapprehension about what is good, then the result will not be beautiful. Benevolence involves not only acting under the guise of the good but knowing what is in fact good. So the principle really should be:
\begin{enumerate}[(3*)]
	\item If the maker is benevolent and looks to a uniform model, then the product is necessarily beautiful
\end{enumerate}

% section the_cause_of_all (end)

\section{Maker and Father} % (fold)
\label{sec:maker_and_father}

The Demiurge, the Maker and Father of the All, is difficult to discover and impossible to explain to all. Nevertheless, a case can be made for His existence using Timaues' causal principles. The argument can be schematically represented as follows:
\begin{enumerate}[(1)]
	\item The Cosmos is generated (by hypothesis, Timaeus' explicit argument for this shall be discussed in the next section)
	\item Everything that has come to be of necessity comes to be as the result of some cause
	\item The cause of what comes to be is a maker that looks to a model
	\item So there is a maker of the Cosmos who looks to a model in its generation (from 1--3)
	\item The Demiurge just is this maker (by definition)
	\item So the Demiurge exists (from 4 and 5)
\end{enumerate}

Despite the simplicity and clarity of the argument, three questions remains. First, the very simplicity and clarity of the argument raises a question. If this difficult to discover divinity can be established by means of so simple an argument, in what sense is this obscure divinity difficult to discover? Second, while the argument makes clear in what sense the Demiurge is the Maker of the All, in what sense is He its Father? After all, it is not generally true that makers of products are the fathers of these. Third, why is this obscure divinity impossible to explain to all? Again, the simplicity and clarity of the argument presents an obstacle to understanding this claim. Why cannot the existence of the Demiurge be explained to all by means of this simple argument? Let us consider these in turn.

First, given the simplicity of the argument, it can be hard to understand how this obscure divinity is difficult to discover. Perhaps, in asking this question, we are blinded by hindsight. While there are creationist tendencies in Anaxagoras and Empedocles (on these see \citealt[chapters 1 and 2]{Sedley:2007pi}), that the Cosmos may have an intelligent creator is really only explicitly entertained by Socrates (\emph{Phaedo} 96a6–99d2). I say entertained advisedly since Socrates avowedly comes to despair of cosmology. It is really only Timaeus who forthrightly and explicitly argues for an intelligent maker of the Cosmos. Such a divinity is difficult to discover since He is largely absent in the \emph{peri phuseōs} (On Nature) tradition, despite the remarkable advances since its inception (on the \emph{Timaeus} and the \emph{peri phuseōs} tradition see \citealt{Naddaf:1997jt} and \citealt{Runia:1997vz}).

Second, while the argument makes explicit that the Demiurge is the Maker of the All, it remains unclear why He is its Father. Again, not all makers are fathers of their products. Indeed, it is awkward, to say the least, to describe a father's offspring as his product. So in what sense is the Demiurge Father of the All? As we shall see, Timaeus importantly conceives of the Cosmos as a living being. Moreover, it is plausible, at least in classical antiquity, to think of the cause of a living being as its father. So perhaps Timaeus, in describing the Demiurge as the Father of the All is anticipating this aspect of his cosmological thinking. (This is Proclus' solution, \emph{In Timaeum} 1 300, \citealt{Diehl:1903re}.)

Third, why is the Maker and Father of the All impossible to explain to all? After all, Timaeus' argument for His existence seems simple enough. Proclus cites the Eleatic Stranger of the \emph{Sophist} as claiming that the eyes of the many do not have the strength to look upon the truth. Proclus appears to be (mis)quoting from memory. What the Eleatic Stranger in fact claims is that the eyes of the soul of the many are not strong enough to look upon the divine (\emph{Sophist} 254a10--b1). However, that claim is unsuited for Timaeus' purposes. It is only the Demiurge that is obscure, other divinities are not. Indeed, as we shall see, we are providentially provided with eyes to see with to look upon the visible Gods in the Heavens. One thought might be that the worry misunderstands the operative modality. Perhaps Timaeus is not claiming that all are incapable of coming to understand that the Maker and Father of the All exists but rather that one should not explain His existence to all. Perhaps, then, the operative modality is deonitc. While Timaeus' is willing to share his theology with the privileged few, the existence of the Demiurge is an esoteric doctrine not to be publicized. Iamblichus reports that this was common practice among the Pythagoreans (\emph{De vita Pythagorica} 32 226.8--227.9, \citealt{Deubner:1937ys}). While elements of Pythagorean thought are discernible in Timaeus' speech, Taylor's \citeyearpar{Taylor:1928qb} contention that Timaeus was himself a Pythagorean has been effectively refuted by \citet{Cornford:1935fk}. And if Timaeus is not himself a Pyhtagorean, he is not bound by the prohibition on publicising esoteric doctrine. Another thought might be that the radical novelty of Timaeus claim that the Cosmos has an intelligent creator was an impediment to its widespread acceptance in classical antiquity. All may, in principle, be capable of understanding this, just not in the present historical circumstances.


% section maker_and_father (end)

\section{The Generation of the Cosmos} % (fold)
\label{sec:the_generation_of_the_cosmos}

So far Timaeus' discussion has been fairly abstract. He has made an ontological distinction and explicated it in terms of a corresponding epistemological distinction, and given us some causal principles. It is at this point that this material is applied to cosmology. Timaeus begins with the question whether the Cosmos is generated or not and provides an argument that the Cosmos is, in fact, generated that appeals to the abstract principles he has so far adumbrated.

The argument begins with an empirical premise. The Cosmos is visible, tangible, and corporeal. I describe it as empirical since it is not derived from the abstract principles and is evident from our lived experience within the Cosmos. I want to make two remarks about this premise, both of which will be developed further as we go along.

First, the Cosmos is described as visible and tangible. Why begin with these two species, the visible and the tangible, rather than the genus, the sensible? And why these two species, rather than the tastable, the smellable, or the audible? 

With respect to the second question, while we may see and feel the Cosmos, perhaps we cannot taste it, even if the Cosmos has parts that we can taste. The suggestion is that while the Cosmos as a whole may be visible and tangible, the Cosmos as a whole is not tastable, smellable, or audible (Proclus, \emph{In Timaeam} 2 25--6, \citealt{Diehl:1903re}). If that is right, then the visible and the tangible are the only two species of the sensible that qualify the Cosmos as a whole.

Proclus also makes a helpful suggestion concerning our first question, why begin with these two species, the visible and the tangible, rather than the genus, the sensible? Proclus suggests that the visible and the tangible are the extreme terms of the sensible (\emph{In Timaeum} 2 6--7, \citealt{Diehl:1903re}, see also Calcidius' translation of 31c as well as his \emph{In Timaeum} 21). So understood, the visible and the tangible are opposed with the other sensibles arrayed as intermediaries. This coheres well with the order in which Timaeus discusses the special sensibles when discussing the affections of the body that are liable to give rise to perception and sensation (61d--68e). Timaeus begins with the tangible and ends with the visible, with the tastable, the smellable, and the audible arrayed as intermediaries. On the Proclean interpretation, the specification involves no loss of generality. One may speak of the sensible in general in terms of its opposed extremes. Thus one may speak of temperature in general by speaking of the hot and the cold since the hot and the cold are the opposed extremes of temperature and all other temperatures are arrayed between them as proportionate intermediaries. (The Proclean interpretation is discussed further in chapters~\ref{sec:the_elemental_composition_of_the_corporeal} and \ref{sec:common_versus_peculiar_pathemata}.)

Second, while Timaeus claims only that the Cosmos is visible, tangible, and corporeal, this obscures an implication. Timaues differs from Descartes in that being sensible, as opposed to being extended, is the characteristic mark of the corporeal. This emerges in Timaeus' derivation of the elemental composition of the Cosmos (31b--32b, discussed in chapter~\ref{sec:the_elemental_composition_of_the_corporeal}). Thus, the  Cosmos being sensible implies that it is corporeal.

The next two premises involve the principle that what comes to be and never is is the object of \emph{doxē met’ aisthēseōs alogou doxaston}. Moreover, the manner of the involvement makes clear, in the way that it may not have been when first presented, the strength of this principle.

Since the Cosmos is sensible, it is apprehended by perception and opinion. The sensible character of the Cosmos thus guarantees that it is the object of \emph{doxē met’ aisthēseōs alogou doxaston}. 

The next premise draws out a consequence of this. What is the object of \emph{doxē met’ aisthēseōs alogou doxaston} has come to be and never is. So not only is what comes to be and never is apprehended by perception and opinion, but only what comes to be and never is is apprehended by perception and opinion. \emph{Doxē met’ aisthēseōs alogou doxaston} can have no other object. Presumably this holds as well for the principle that what always is and has no becoming is the object of \emph{noēsei meta logou perilēpton}. So not only is what always is and has no becoming apprehended in understanding with an account, but only what always is and has no becoming is apprehended in understanding with an account. \emph{Noēsei meta logou perilēpton} can have no other object.

Timaeus' argument can thus be schematically represented as follows:
\begin{enumerate}[(1)]
	\item The Cosmos is visible, tangible, and corporeal (empirical premise)
	\item The Cosmos is sensible since only sensible things are visible, tangible, and corporeal (from 1)
	\item Since the Cosmos is sensible, it is apprehended by perception and opinion
	\item What is apprehended by perception and opinion has come to be and never is
	\item Therefore, the Cosmos has come to be and never is (from 3 and 4)
\end{enumerate}

% section the_generation_of_the_cosmos (end)

\section{Model and Image} % (fold)
\label{sec:model_and_image}

The sensible Cosmos has come to be, and everything that has come to be has a cause. Timaeus conceives of the cause of the All as a maker that looks to a model. If the product is sensible and so has come to be and never is, what is the nature of the model? Given the ontological distinction, there are two options. The model either always is and has no becoming or the model becomes and never is. Which kind of model did the Demiurge look to?

This is not, however, the terms in which Timaeus poses this question. Specifically, he asks whether the model was the kind that does not change and uniformly is or the kind that comes to be. As we have seen, however, that which always is and has no becoming does not change (trivially, since then it would have becoming) and uniformly is what it is. And that which becomes and never is comes to be (again, trivially). Beyond the elision, the main difference from the official formulation of the ontological distinction is the uniformity condition that emerged from the epistemological explication of the first ontological category. It is in virtue of being uniformly what it is that that which always is and has no becoming is intelligible, and so may be apprehended in understanding with an account. Timaues, then, is emphasizing the intelligibility of one kind of model. (This is made explicit at 29a6--b1.)

Before we turn to Timaeus answer it is worth remarking about the oddity of his question. Notice that at this stage of the narrative, nothing has come to be. There is no generated model that the Demiurge could choose to look to in generating the Cosmos. Indeed, Timaeus claims that it is impious to even suppose that the Demiurge looked to a generated model. The Demiurge is not really in a choice situation faced with superior and inferior alternatives where He wisely chooses the superior. The question is really posed for our benefit. In understanding why the Demiurge must look to an intelligible model, we learn that the Cosmos is an image of the intelligible. Indeed, it is only if this is so is cosmology so much as possible. If the Cosmos is generated by the Demiurge looking to an intelligible model, then the Cosmos is like that model, insofar as it can be. The Cosmos is thus a sensible image or likeness of the intelligible. We have already seen that what comes to be and never is comes to be from some cause and so for a reason. It is unsurprising that the Cosmos is in this sense minimally intelligible since it is a sensible image of the intelligible, like it insofar a it can be. The very possibility of cosmology, of which Socrates despaired (\emph{Phaedo} 96a6–99d2), is vouchsafed by the Cosmos being in this way a sensible image and likeness of the intelligible. \citet[28--9]{Broadie:2012vl} suggests that there may be an additional benefit for us. If what makes for beauty or excellence is guidance by an intelligible model, then we too, in our affairs, should endeavour to look to the intelligible.

If the reasoning that established the existence of the difficult to discover divinity impossible to explain to all appealed to the first two causal principles (described in section~\ref{sec:the_cause_of_all}), the nature of the model is established by appeal to the third causal principle in its modified form. Timaeus' argument can be schematically represented as follows:
\begin{enumerate}[(1)]
	\item If the maker is benevolent and looks to a uniform model, then the product is necessarily beautiful (\emph{kalon})
	\item The Cosmos is beautiful (empirical premise)
	\item The maker of the Cosmos is benevolent
	\item So the model that the maker looks to is uniform (from 1--3)
\end{enumerate}

The relevant kind of model is officially described as unchanging and uniform. When the relevant causal principle was first introduced (28b6--b2), the model was solely described in terms of uniformity. But when Timaeus argues that the model must be unchanging and uniform, it is described not in terms of uniformity the way that I have schematically represented his argument, but as unchanging. However, the model must be uniform since Timaeus will go on to claim that it is apprehended in reasoning (\emph{phronēsi perilepton} 29a7). Timaeus thus freely switches between describing the relevant kind of model as uniform or unchanging.

The schematic representation of Timaeus' argument differs, as well, in the bluntness of the empirical premise. Timaeus does not straightforwardly assert that the Cosmos is beautiful (\emph{kalon}) but rather claims that it is more beautiful than anything that has come to be. Strictly speaking this is weaker than required. The Cosmos may be more beautiful than anything that has come to be and yet not itself be beautiful, at least by some reasonable standard of beauty. Perhaps Timaeus is implicitly engaged in enthymematic reasoning here. Some of what comes to be is beautiful. The Cosmos is more beautiful than anything that has come to be. Therefore, the Cosmos is itself beautiful. Despite the weakness of Timaeus' explicit claim, the beauty of the Cosmos is never really in doubt here. 

A similar issue affects the premise that the maker of the Cosmos is benevolent. Again, Timaeus does not straightforwardly assert this. Rather, Timaeus claims that there is no better cause than the cause of the Cosmos. And, again, this is, strictly speaking, too weak. The cause of the Cosmos may be better than any other cause consistent with no cause being good. However, just as the beauty of the Cosmos is never really in doubt, neither is the benevolence of its maker. Indeed, it would be impious to suppose otherwise.

% section model_and_image (end)

\section{``Likely'' Account} % (fold)
\label{sec:likely_account}

In arguing that the Cosmos that we inhabit and experience is generated, Timae\-us has circumscribed the subject matter of his speech, the nature and generation of the All. Having thus circumscribed the subject matter of his speech, in the final part of the \emph{proemium}, Timaeus explains the epistemic status of his cosmological account.

Timaues begins by drawing out a consequence of the foregoing. If the sensible Cosmos is generated by the Demiurge looking to a model, then the Cosmos is an image of that model. Timaeus, has, of course, already argued that the model in question is intelligible, so it is telling that he states this claim in the general terms that he does. Timaeus is emphasizing a metaphysical aspect of the object of his account, that it is an image and likeness of something, for this will bear on its epistemic status.

That the epistemic status of a cosmological account depends upon the metaphysical status of the Cosmos turns on the following principle:
\begin{quote}
	Accounts (\emph{logoi}) are akin (\emph{suggeneis}) to things of which they are \emph{exēgētai}
\end{quote}
Some remarks are in order.

First, I flag the occurrence of \emph{logoi} since sometimes Timaeus refers to his ensuing account as a \emph{logos} and sometimes as a \emph{muthos}. We need to get clearer about these terms and their relationship. With respect to \emph{muthos}, \citet[144--5]{Burnyeat:2005it} argues that it is not merely an account cast in narrative terms as \citet[71--3]{Vlastos:1939td} claimed. A \emph{muthos} for Timaeus is, in fact, a narrative, but Burnyeat emphasizes the divine character of its subject matter. Timaeus maintains that the Cosmos is a living, visible God, and so his cosmogony is literally a theogony. As such, Burnyeat prefers translating \emph{muthos} as myth, as opposed to story or narrative. Burnyeat suggests that \emph{logos} is a general term and that \emph{muthos} is a species of \emph{logos}. Consistent with this, \emph{logos} sometimes seems to be used in a more specific sense to contrast with \emph{muthos}, as when Timaeus articulates the reasons behind the Demiurgic activity that he narrates. So, for example, in this more specific sense, the derivation of the elemental composition of the Cosmos is a \emph{logos} while the Demiruge mixing Being, Sameness, and Difference in a hypercosmic \emph{kratēr} is a \emph{muthos}. As Burnyeat himself observes, Timaeus has the tendency to use the full range of the meanings of his terms.

Accounts are akin (\emph{suggeneis}) to their subject matters. Timaeus really means successful accounts, here. Being akin to its subject matter is something that accounts should aspire to. Thus, a false account of a subject matter need not be very much akin to its subject matter. The Greek word \emph{suggenēs} primarily means kinship, as if successful accounts were somehow the offspring of their subject matters. Why offspring rather than parents or siblings, say? Prior to stating the above principle, Timaeus stresses the importance of starting from the natural beginning. If the natural beginning of an account is its subject matter, then Timaeus is exhorting us to look to the subject matter of our inquiry so that we may produce an account of that subject matter that is akin to it \citep[120--1]{Bryan:2012bt}. Since subject matters and successful accounts of them are bound by kinship, they share a family resemblance. Proclus emphasizes this aspect of the metaphor, claiming that accounts are of the same kind (\emph{homion}) as their subject matters (\emph{In Timaeum} 1 341, \citealt{Diehl:1903re}). However, it is unclear whether the metaphor is dispensable in this way (see \citealt[122]{Bryan:2012bt}). One reason for this is that in speaking of the kinship of accounts and their subject matters the metaphor personifies accounts, and this personification persists in Timaeus claim that they are \emph{exēgētai}.

We have so far left \emph{exēgētai} untranslated. Most translators miss out on the personification. Thus, for example, \citet[53]{Bury:1929jb} translates the passage as ``we must affirm that accounts given will themselves be akin to the diverse objects which they serve to explain''. \citet[]{Burnyeat:2005it} opposes this interpretative tradition by emphasizing that \emph{exēgētai} is a noun, with the result that Timaeus has personified accounts as interpreters of their subject matters. And not just any interpreters, but ``exegetes who expound or explain the unobvious significance of an object like a dream, ritual or oracle which does not bear on its meaning on its surface, because it comes from, or has some important connection with, the divine'' \citep[149]{Burnyeat:2005it}. Burnyeat's reading has an important consequence for our understanding of the above principle, for it clearly does not concern any old account. Just as not every interpreter is an exegete in the intended sense, so, even granting the personification, not every account that interprets its subject matter is an exegete. Timaeus thus signals a restriction on the principle in deploying this solemn word. Presumably, Timaeus' account satisfies this restriction by having a divine subject matter. Cosmology takes as its subject matter the Cosmos, a visible God. (Without denying its religious connotations, \citealt[216--7]{Betegh:2010aa} observes that \emph{exēgētēs} has secular uses as well.) 

That successful accounts are exegetes of their subject matters, or at least those that are, explains their kinship. Succesful accounts exegetic of their subject matter are the offspring of their subject matter. And since those that are bound by kinship share a family of resemblance, successful accounts that are exegetic of their subject matters will be like them in certain respects. Timaeus, at this point, articulates some of the respects in which successful accounts are like their kindred subject matters. There are two cases. The two cases are related though distinct from Timaeus' initial ontological distinction. The first case consists in successful accounts that take the intelligible as their subject matter and thus straightforwardly correspond to that which always is and has no becoming. The second case, however, does not straightforwardly correspond to what becomes and never is. As \citet{Betegh:2010aa} observes, the second case is a special case of this broader class. Recall, what becomes comes to be as a result of some cause, and Timaeus conceives of this cause as a maker looking to a model. Moreover, there are two types of models corresponding to the initial ontological distinction. The second case corresponds to what comes to be as a result of a maker looking to an intelligible model. Let us consider, then, how these two cases resemble their subject matters.

First, Timaeus considers successful accounts that are exegetic of intelligible subject matters. Specifically, Timaeus considers accounts that are exegetic of what is permanent and stable and manifest in understanding (\emph{meta nou kataphanous}). It is clear that that which always is and has no becoming should be permanent and stable. Moreover, we have seen Timaeus explicate this ontological category as what is apprehended in understanding with an account (\emph{noēsei meta logou perilēpton}). That which is apprehended in understanding with an account is manifest (\emph{kataphanous}) to that understanding. \emph{Kataphanous} primarily means clearly seen, but is used, as well, to mean evident or manifest. In this regard, the Greek is like English. Indeed, most languages are such that perceptual verbs have epistemic uses. Thus one can see the validity of a proof and this despite the fact that validity is, strictly speaking, invisible. Just as intelligible objects are permanent and stable (\emph{monimou kai bebaiou}), successful accounts exegetic of these are themselves permanent and unchangeable (\emph{monimous kai ametaptōtous}). I am uncertain why Timaeus varies his vocabulary here in substituting \emph{ametaptōtous} for \emph{bebaiou}. But it is plausible that the intelligible is permanent, stable, and unchangeable, and that successful accounts exegetic of the intelligible should share these features or, at the very least, possess analogues of them.

Another variation should be observed. Earlier, Timaeus told us that what always is and has no becoming is the object of \emph{noēsei meta logou perilepton} because it always uniformly is (\emph{aei kata tauta on}). Recall, what always is has no becoming not only in the sense that it lacks a beginning, and so does not come into being, but also in the sense that it never comes to be some way for things to be. What always is is always already what it is. It fully is what it is leaving no further scope for it coming to be some way there is for things to be. Permanence, stability, and unchangeability are at least necessary conditions for Being to be uniform in this sense.

That successful accounts exegetic of the intelligible are like their objects in being permanent, stable, and unchangeable, and this affects their epistemic status. Such accounts are irrefutable and invincible. In portraying irrefutable accounts as invincible, Timaeus continues with the personification of accounts. In describing irrefutable accounts as invincible, Timaeus imagines such accounts as always victorious in a contest of reasons. But, of course, only persons may engage in such a contest. Being irrefutable and invincible is not merely an epistemic status of successful accounts exegetic of the intelligible, this is a standard that all accounts that purport to be exegetical of the intelligible should aspire to. If it is possible and fitting that such accounts be irrefutable and invincible then there must be no falling short of this. \citet[150]{Burnyeat:2005it} follows \citet[88]{Archer-Hind:1888qd} and \citet{Burnet:1905oq} in reading \emph{de} as \emph{deī}, and this reading is accepted as well by \citet[74]{Taylor:1928qb} and \citet[52]{Bury:1929jb}. However, as \citet[121 n22]{Bryan:2012bt} observes, that this is a status that such accounts ought to aspire to is anyway clear from there being a necessary connection between subject matters and successful accounts. 

Second, Timaeus considers successful accounts that are exegetic of what is made to be like an intelligible model. Such accounts take as their subject matter an image and likeness. And, as \citet[]{Betegh:2010aa} observed, not just any image and likeness but an image and likeness made by a maker looking to an intelligible model. Moreover, just as we saw in the first case, where successful accounts share features of their subject matter, here too we should expect that pattern to continue. And, indeed, it does. Successful accounts of things that are made as a likeness of that which is permanent and stable and manifest in understanding are themselves \emph{eikotas} and have a share in likelihood. Such accounts have, as a consequence, a certain epistemic status. They are ``likely''. 

I translate \emph{eikōs} as ``likely'' to capture the verbal link between \emph{eikōs} and \emph{eikōn}. However, I insist upon the scare quotes since, while the translation is traditional and grammatically natural, it is problematic. In English, ``likeness'' and ``likely'' are no longer semantically related the way that ``brightness'' and ``brightly'' are. Moreover, describing an account as ``likely'' carries the strong presumption that it is not definitely true and even sometimes that it is probably not true. Thus, describing an account or narrative as a ``likely story'' is naturally understood as a retort. There is a tradition of endorsing what \citet[214]{Betegh:2010aa} describes as a limited or restrictive interpretation where a ``likely'' accounts fall short of some standard, such as the certainty of demonstration. \citet[59--60, 73--4, 440--1]{Taylor:1928qb} and \citet[28--30]{Cornford:1935fk} offer different interpretations of this kind. Against traditional restrictive readings, \citet{Burnyeat:2005it} has argued that ``likely'' accounts positively succeed in meeting some standard appropriate to accounts that take as their subject matter a likenesses of the intelligible. \citet[146--7]{Burnyeat:2005it} observes that there are many contexts where the Greek word \emph{eikōs} could not aptly be translated as ``likely'' or ``probable'' but would be better translated as ``fitting'', ``fair'', ``natural'', or ``reasonable''. Burnyeat is right to insist that \emph{eikōs} signals the achievement of some standard appropriate to an account's subject matter, but it is important to recognize that an account being \emph{eikōs} may merit praise in meeting some appropriate standard and still fall short of another standard, appropriate to accounts with some other subject matter. That is to say we are not faced with a stark choice between the traditional restrictive readings and the more positive reading endorsed by Burnyeat. An account may at once be positively lauded as \emph{eikōs} thus meeting a standard appropriate to its subject matter while at the same time lacking the certainty of demonstration (see \citealt{Betegh:2010aa}, \citealt{Mourelatos:2010bz}, \citealt[chapter 3]{Bryan:2012bt}, \citealt[33-4 n15]{Broadie:2012vl}).

Retaining the verbal link between \emph{eikōs} and \emph{eikōn} is no mere stylistic preference. It has philosophical significance as well. As \citealt[chapter 3]{Bryan:2012bt}, argues cosmology is ``likely'' not because its subject matter consists in what comes to be an never is, but rather because its subject matter is a likeness. Describing a successful cosmological account as fitting or reasonable would obscure this (which is not say that Burnyeat was under any such misapprehension). Moreover, this is important to bear in mind if we are to understand Timaeus claim that successful accounts exegetic of the intelligible stand in proportion to successful accounts exegetic of a likeness of the intelligible.

Timaeus claims that successful accounts exegetic of the intelligible stand in proportion to successful accounts exegetic of a likeness of the intelligible, Indeed, they stand in the following proportion: As being is to becoming, so truth is to conviction (\emph{pistis}). Again, it is important to bear in mind that, for Timaeus, proportion or ratio is not restricted to \emph{arithmoi}. And none of the terms of the two proportions are specifically \emph{arithmoi}. The terms in the second proportion, as truth is to conviction, require some clarification. Let us consider them in turn.

\emph{Alēthia}, here, should not narrowly be understood as the semantic notion of truth. So accustomed are Anglophone philosophers to speaking what Quine described as Loglish, an amalgam of English and logic, that it can be hard for them to appreciate that ``truth'' can have anything other than semantic significance. Ordinary English as well as Greek are not so constrained, however. With respect to English, ``truth'' can mean genuineness, loyalty, or faithfulness. Importantly, it can even refer to cognitive attitudes. Thus Milton, in \emph{Of Education}, writes of ``assertions, the knowledge and the use of which, cannot but be a great furtherance both to to the enlargement of truth and honest living'' \citep[51]{Ainsworth:1928cy}. Truth, here, must be understood as the body of human knowledge, as known truths. This is important since \citet[]{Burnyeat:2005it} has argued that \emph{alētehia}, here, refers to a cognitive attitude, the apprehension of the intelligible in understanding (\emph{nous}), reasoning (\emph{phronēsis}), and perhaps knowledge (\emph{epistēme}) as well. \emph{Alēthia} would have to be a cognitive attitude to stand in proportion to the cognitive attitude of \emph{pistis}. Recall the ontological distinction was explicated in terms of a distinction between the objects of distinct cognitive attitudes. Timaeus' general principle seems to be just as the objects of these attitudes stand to one another, so these cognitive attitudes stand to one another. This could only be so if \emph{alēthia} were itself a cognitive attitude. Specifically, we know that \emph{alēthia} is a cognitive attitude that takes intelligible being as its object and stands to \emph{pistis} as Being stands to Becoming.

\emph{Pistis}, conviction, is often tied with \emph{doxa}, opinion. Recall that only the sensible and what comes to be is the object of \emph{doxa} and \emph{pistis}. Timaeus seems to have in mind, here, the epistemic genus rather than the specific way in which \emph{pistis} may differ from \emph{doxa} or other doxastic states. The object of the relevant \emph{pistis}, generically understood, is, however, restricted. Timaeus seeks to account for the sensible image and likeness of the intelligible. Since it is sensible, it belongs to the realm of Becoming. But not all that comes to be, even if it comes to be on the basis of a maker looking to a model, comes to be on the basis of a maker looking to an intelligible model. Cosmological \emph{doxa} or \emph{pistis} concerns the image and likeness of an intelligile model. As such, the apprehension of the intelligible stands to it, as Being stands to Becoming.

The ontological distinction between what always is and has no becoming and what comes to be and never is is explicated in terms of being the objects of distinct cognitive attitudes. It is natural or fitting then that the cognitive attitudes stand in proportion to their objects. So just as intelligible being is, so too is the apprehension of intelligible being in understanding with an account. And just as a sensible image and likeness of an intelligible model is, so too is the apprehension of that image and likeness in \emph{doxa} and \emph{pistis}. 

Intelligible Being always already is what it is and, in this sense, it is uniform or uniformly is. Since it uniformly is what it is, it is apprehended in understanding with an account. Being permanent and stable is at the very least necessary conditions for being uniform. So being permanent and stable are necessary conditions for being apprehended in understanding with an account. Understanding with an account is itself a uniform cognitive activity. At every moment that it occurs it is the apprehension of its object and so uniformly is what it is for long as it occurs. Thus like is know by like, at least on some suitable understanding of that principle. What comes to be by a maker looking to a uniform intelligible model is not itself uniform. As a consequence it is not itself apprehended in understanding with an account. But it is not beyond our cognitive apprehension for all that. What comes to be and never is may not be uniformly circumscribed by uniform intelligible activity, but may, instead, be apprehended in \emph{doxa} or \emph{pistis} based on perception or sensation (\emph{aisthēsis}). What is sensible comes to be. And what comes to be is apprehended in \emph{doxa} or \emph{pistis} because it is sensible and so the object of \emph{aisthēsis}. Mortal beings have corporeal instruments that receive affections from without, and \emph{aisthēsis} is an awareness of the powers of the agents that caused such affections. In this way, at least mortal cognitive activity may compensate for the lack of uniformity in the sensible realm of Becoming. What we sense is a likeness of the intelligible. Successful accounts, based on perception, exegetic of their object, the visible God, must have a share in this likeness of the intelligible. They are thus ``likely'', in the sense that Burnyeat urges, as apt, or fitting, or reasonable. Accounts meet such a standard (\emph{eikōs}) by being successful accounts exegetic of their subject matter, a sensible image  (\emph{eikōn}) and likeness of the intelligible.

% section likely_account (end)

\section{Concluding Observations} % (fold)
\label{sec:concluding_observations}

Allow me to highlight some of the issues that have arisen that are directly relevant to the topic of the present essay, the philosophy of perception.

The epistemological section of Timaeus' \emph{proemium} is generally relevant to understanding perception. Perception involves the cognizance of its object, and in the epistemological section of the \emph{proemium}, Timaeus lays down the principles that fundamentally govern his epistemology. They are not, strictly speaking, first principles since Timaeus is engaged in cosmology and not metaphysics. But within the account that Timaeus proposes to give, they function as first principles. In this sense they fundamentally govern Timaean epistemology.

The epistemological section of the \emph{proemium} is thus generally relevant to Timaeus' philosophy of perception. But it is not merely of general relevance. Timaeus has already made some substantive claims.

First, Timaeus has described perception as \emph{alogos}. There is no consensus among modern Anglophone commentators about how to translate this occurrence of \emph{alogos}. I argued that we capture what is right about the alternatives if we understand \emph{alogos} as non-discursive, since the main relevant alternatives all had discursive commitments. If perception is \emph{alogos} then its content is not discursively articulated. 

Here we should observe the weakness of Timaeus' denial. He denies only that the content of perception discursively articulated. He does not further deny that the content of perception is discursively articulable. For reasons that have only begun to emerge, I believe that Timaeus does not, even implicitly, make the further, stronger denial. The power of the agent that caused the affection of the instrument of perception is reported to the \emph{phronimon}, the seat of cognizance, and perception supervenes. Even if the ``report'' is not literally a speech act, the \emph{phronimon}'s cognitive uptake of what is reported plausibly involves discursive articulation. This could only be so if what is reported to the \emph{phronimon} is at the very least discursively articulable.

Second, Timaeus has described a fundamental epistemological role for perception. The Cosmos is a sensible image and likeness of an intelligible model. Since it is sensible, it has come to be. But what comes to be is not uniform in the way that is required to apprehended in understanding with an account. The non-uniformity of the Cosmos is an obstacle to understanding. Despite this, the Cosmos is, nonetheless, subject to cognitive apprehension in \emph{doxa} or \emph{pistis} based upon \emph{aisthēsis}. The Cosmos is the object of opinion because it is sensible. Perception is what allows mortal beings to apprehend their sensible environment despite its non-uniformity. One could only opine, even about the nature and generation of the All, because perception overcomes the cognitive obstacle posed by cosmic non-uniformity.

Third, as we have see, this raises an issue that potentially turns on the difference between cosmic and mortal perception. The World Soul opines but the Cosmos lacks instruments of perception. Instruments of perception are corporeal and receive an affection from the object perceived. May we infer a lack of perception from a lack of the instruments of perception? If so, then there is a puzzle about how the World Soul may so much as opine, given the fundamental epistemological role that Timaeus assigns to perception in the \emph{proemium}. Or perhaps, as Proclus urges, we should not infer a lack of perception from a lack of the instruments of perception. If Proclus is right, then there is a fundamental difference between cosmic and mortal perception. While the perception of mortal beings requires instruments of perception to receive affections from strong powers from without, cosmic perception has no need of such instruments. The reasons for this emerge in the next chapter. As we shall see, they turn on a large and important thesis about the nature of perception. Is perception environmental, does it only occur in beings embedded in an environment and taking as its object aspects of that environment, or is only mortal perception environmental?

Fourth, not only have we learned some substantive lessons about sense, but so too we have learned substantive lessons about \emph{sensibilia}. The Cosmos is a living visible God ensouled and engaged in rational activity. Moreover, the Cosmos is a sensible image and likeness of an intelligible model. The sensible character of the Cosmos is in no way in tension with its intelligible underpinnings. There is no conflict here, between the Manifest Image of Nature and its Scientific Image. Timaeus is unlike the ancient atomists, such as Democritus in this regard. This is not an instance of premodern naivety. It is easy to see how there might be a reading of Parmenides poem that articulates such a conflict. (This may be granted even if one thinks that this tendency is exaggerated in Melissus.) And Timaeus has drawn upon the terms laid down by the unnamed Goddess in Parmenides' poem. The verbal echoes are no mere happenstance but an attempt to revise the Parmenidean cosmology while retaining some of its metaphysical and epistemological insights. Whether or not Timaeus is successful in this regard, his cosmology is not developed in ignorance of the possibility of a conflict between the Manifest and Scientific Images of Nature. 

Finally, in the course of discussing the sense in which perception is \emph{alogos}, I had to appeal to material later in the dialogue. Specifically, I had to appeal to the general conception of perception that Timaeus operates with. On that general conception, perception is understood on the model of measurement. Mortal perceivers are equipped with instruments of perception, such as eyes to see with and ears to hear with, and these receive affections from objects in the natural environment. These affections are the measure of the powers of the agents that caused them, and these powers are reported to the \emph{phronimon}, the seat of cognizance. It is only with such a report does perception occur. On this basis I argued that \emph{alogos} could not mean without ratio or proportion since then perception could not be understood on the model of measurement. This general conception only emerges in Timaeus discussion of the \emph{pathēmata} common to the body as a whole and peculiar to particular parts of the body (61d–68e, discussed in chapters~\ref{cha:common_pathemata} and \ref{cha:peculiar_pathemata}).

% section concluding_observations (end)

% Chapter proemium (end) 