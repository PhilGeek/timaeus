%!TEX root = /Users/markelikalderon/Documents/Git/timaeus/timaeus.tex

\chapter{\emph{Proemium}} % (fold)
\label{cha:proemium}

\section{A Double Significance} % (fold)
\label{sec:a_double_significance}

The first part of Timaeus's speech is traditionally called the \emph{proemium}, echoing Socrates' own description of it at 29d5 as \emph{prooimion} or prelude. Entitling the present chapter ``\emph{Proemium}'' thus has a double significance. It at once announces that the present chapter will take Timaeus' \emph{proemium} as its subject matter and that the chapter will itself serve as prelude to the essay as a whole.

% section a_double_significance (end)


\section{\emph{Proemium}} % (fold)
\label{sec:proemium}

The \emph{proemium} is bounded by the last two remarks that Socrates will make in the dialogue. In the first, Socrates praises the feast of \emph{logoi} that has been proposed as a return gift for his speech the previous day on the topic of the \emph{Republic} and prompts Timaeus to begin by duly invoking the gods. In the second, Socrates praises Timaeus, accepting his prelude, and encouraging him to proceed. In the \emph{proemium}, Timaeus begins by invoking the gods. After which Timaeus states some general principles, which he uses to circumscribe his subject matter and describe the kind of account that he proposes to give of it. 

The principles include:
\begin{enumerate}[(1)]
	\item The distinction between that which always is and never becomes and that which always becomes and never is and the manner in which they are cognized
	\item The principle that everything that has come to be and never is has a cause
	\item The principle that the cause of that which has come to be and never is a maker that looks to a model
	\item The principle that if that which has come to be and never is is beautiful then it has been modeled on that which always is and never becomes by a good maker
\end{enumerate}

That which always is and never becomes is the object of understanding (\emph{noēsis}) based on reasoning (\emph{logos}). That which becomes and never is is the object of opinion (\emph{doxa}) based on unreasoning perception (\emph{aisthēsis}). That which always is and never becomes is intelligible, and that which becomes and never is sensible. The Cosmos is sensible and thus must belong to the realm of Becoming. That which becomes and never is has a cause. This causes is a maker that looks to a model. Since the Cosmos is beautiful then by the final principle, Timaeus concludes that the Cosmos has an intelligible model and a good maker. It is the generation of the Cosmos, conceived as a beautiful likeness of an intelligible model, that will be the subject matter of Timaeus' speech. Accounts should be appropriate to their subject matter, and Timaeus concludes that since his subject matter is a likeness his account shall be ``likely''.

While the general outline of the \emph{proemium} is relatively clear, each element of this tightly constructed account requires comment as does their interrelation. 

% section proemium (end)

\section{The Invocation of the Gods} % (fold)
\label{sec:the_invocation_of_the_gods}

Without dwelling on Timaean theology, two observations are worth making about Timaeus' invocation of the Gods. The first is an observation about who is not invoked, and the second is an observation about who is invoked.

It is reasonable to call upon God at the outset of an undertaking. This is presented as a general practical precept. Later Timaeus explains that we invoke the Gods so as to pray for their approval of our undertaking. It would be unreasonable to fail to do so. And this remains so whether the undertaking is great or small. Timaues, however, proposes to undertake an account of the All (\emph{toū pantos}---understood, here, as the Cosmos without the explicit suggestion that it is ordered) and whether and how it was generated. This is a grand topic and a great undertaking so it is especially appropriate that Timaeus call upon the Gods at the outset. 

It is striking who is not called upon by Timaeus. Notable in His absence is the Demiurge, the Maker and Father of the All. The Maker and Father of the All is difficult to discover and impossible to explain to all (28c3–5). He only begins to emerge in reasoning from principles that Timaeus only states after the invocation. Timaeus is invoking the Gods, in part, in aid of this difficult discovery. If so, then the object of his search was not among the Gods that were invoked even it is itself divine.

There is a related reason for thinking that the Demiurge was not among the Gods invoked and prayed to. Even if it were possible to privately invoke the Gods and pray for their approval of some solitary undertaking, this is not the context that Timaeus finds himself in. The invocation of the Gods urged by Socrates was, importantly, a public act of piety. And Timaues complies in the first person plural. It is we who are about to discourse that have need to invoke the Gods and the Goddesses. A divinity difficult to discover and impossible to explain to all is perhaps not the most apt choice to invoke in a public act of piety. One's piety toward the difficult to discover divinity may be genuine but it might not be publicly manifest. One's piety is only guaranteed to be publicly manifest if directed toward recognized divinities. And a difficult to discover divinity impossible to explain to all is not a recognized divinity. The obscurity of the Maker and Father of the All is inconsistent with the public nature of the invocation.

Perhaps, in the present instance, the public is restricted to the divided reference of ``we''---Timaeus, Socrates, Critias, and Hermocrates. The restriction may be significant. A divinity not widely known to many may yet be known to the present few. However, while Timaeus might rely on Socrates to already know of this difficult to discover divinity, is it really reasonable that Timaeus should similarly rely on Critias? Even among the present dignitaries, the obscurity of the divinity is in tension with the publicity of the act.

As \citet[13--4]{Broadie:2012vl} observes, the Demiurge is not invoked in any of the prayers offered by Timaeus. Broadie's observation coheres with the present explanation. Since all of the prayers in the dialogue are public acts, the obscurity of the divinity would preclude His public invocation.

Not only is it notable that Timaeus does not invoke the Demiurge, but it is obscure who is being invoked. Specifically, is Timaeus invoking one God or many? Notice he first calls upon God in the singular and then upon Gods and Goddesses in the plural. When Timaeus first calls upon God in the singular, it is in stating the general practical precept to call upon God and pray for approval of one's undertaking. When Timaeus calls upon Gods and Goddesses in the plural, this is not in the context of stating a general practical precept, but rather in the context of proposing to undertake an account of the All and its generation. Perhaps it is the Young Gods who are being invoked when Timaeus invokes the many. The Young Gods are divine, immortal beings generated by the Demiurge who are the instruments of Demiurgic activity. Specifically, as we shall see, they play an indispensable role in the generation of mortal beings. If the Young Gods are the many, who could the one be if not the Demiurge? One candidate is the All itself. Timaeus conceives of the Cosmos as a visible God (34a8, 34b8--9, 92a5--9), defers to it concerning its proper name in act of ritual piety (28b3--5), and addresses his final prayer to that God (\emph{Critias} 106a1--b6). By contrast, as \citet[35]{Cornford:1935fk} observes, ``Neither in the \emph{Timaeus} nor anywhere else is it suggested that the Demiurge should be an object of worship: he is not a religious figure.''

There are, of course, alternatives. Perhaps the God in the singular being invoked is a generic divinity of which the Gods and the Goddesses in the plural are species. On this alternative, an appeal to a generic divinity is appropriate in the statement of the content of a general practical percept. Similarly, it is appropriate to invoke the species of this generic divinity given the specific undertaking to account for the All. That all species of divinity should be invoked and their approval prayed for is perhaps only required for their specific undertaking of discoursing on the All and its generation. Perhaps a single divinity would do if the undertaking were small enough. But the collective undertaking that Timaeus proposes is suitably grand as to require the invocation of every species of divinity.

A deflationary take on the identity of the Gods invoked might might be motivated by the perfunctory nature of the invocation. Proclus describes the worry as follows:
\begin{quote}
	But how can it be, they say, that Timaeus has announced with a grand flourish that one should pray and call on the gods and goddesses, but that he fails to do this himself and immediately turns to proposed accounts without praying? (Proclus, \emph{In Timaeum} 1 221.9--12, \citealt{Diehl:1903re}; \citealt[58]{Runia:2008aa})
\end{quote}
If Timaeus has not in fact prayed, then there is no point in inquiring into to whom Timaeus has addressed his prayer. Proclus' reply is ingenious in that he, in effect, proposes that we understand Timaeus' speech as performative in something like Austin's \citeyearpar{Austin:1975nx} sense. In saying that we must pray to the Gods, Timaeus has done just that. While one may decide to pray and pray later, in true prayer, the decision to pray and prayer are one and the same thing. Timaeus's public prayer on the occasion of the Panathenaea is a true prayer. And so in saying that we must pray, he does.

Still, exactly who is being invoked by Timaues may not be so neatly resolvable. Perhaps there is a kind of semantic slippage here, from singular to plural, the one resolving into the many. If so, this will be echoed in the third part of Timaeus' speech which concerns the interaction of Reason and Nececssity, especially in his account of the generation of the flesh and the mortal soul. There, Timeaus slides freely between singular and plural in describing what is officially meant to be rational cooperative activity, the Demiurge generating the immortal part of the soul and the Young Gods weaving it together with its mortal body that they generated from corporeal material borrowed from the body of the Cosmos. However, even for tasks that are meant for the Young Gods, their executions are sometimes described in the singular. We shall discuss the significance of this when we discuss the flesh and the mortal soul (chapter~\ref{cha:the_flesh_and_the_mortal_soul}). For now, I merely note a potential anticipatory trace of this puzzling usage. 

One last observation. Not only are the Gods invoked, but Timaeus also invokes themselves. The plural self-invocation is for the audience---Socrates, Critias, and Herm\-ocrates---to more easily learn and for the speaker---Timaeus---to more clearly expound his account of the All and the cause of its generation. (On the self-invocation see Proclus, \emph{In Timaeum} 1 222.11--223.2, \citealt{Diehl:1903re}.) Not only is Timaeus praying for the approval of the Gods concerning their present collective undertaking---his discoursing with them the nature and generation of the All---but he is importantly also praying that they themselves approve of their discourse, if only after being granted divine approval. While for the most part, Timaeus delivers an uninterrupted speech, in the invocation he describes it as a collective activity with the audience as co-participants. Perhaps the self-invocation is meant to get the participants of the discourse in a position where they may engage in it in such a way that they come to themselves approve of that discourse. This would involve the audience rationally attending to the speaker's account so that they may more easily learn his views about the nature and generation of the All. And this would involve the speaker clearly expounding his views to this audience. (Notice that it is his views that Timaeus wishes them to learn. Contrast learning the truth about the All and its generation. Timaeus is pointedly not proposing to demonstrate such truths. Timaeus, here, is quietly allowing room for what will emerge as accounts with a specific epistemic status, ``likely'' accounts.) Even if largely a monologue, Timaeues' speech remains directed to its audience. And Timaeus must bear them in mind if he is to clearly expound his views to them in such a way as warrant their collective approval. Only in this way could the return speech discharge Timaeus' obligation to Socrates. 

If the plural self-invocation still seems odd or unfamiliar, it really should not. You have doubtless encountered it in contemporary speech. Think of someone inaugurating their address with ``Let's do this!'' Think of a motivational speaker or, better yet, a Hip Hop artist. ``This'', that which is to be done (dropping bars for the Hip Hop artist, cosmology for Timaeus, both are \emph{logoi}), is explicitly represented as a collective activity---``Let \emph{us} do this'', where ``us'' includes both speaker and audience. In saying ``Let's do this!'', the speaker draws attention to themselves, making their audience more receptive to their address, and the speaker expresses their determination in a way that draws the speaker's own attention to their audience, in an attempt to thereby more effectively deliver their address. Should the speaker prove genuinely effective, they may even garner approval for that address. It even has, in line with Proclus' suggestion, a performative aspect. In saying ``Let's do this!'', that which is to be done has already begun.

% section the_invocation_of_the_gods (end)

\section{Being and Becoming} % (fold)
\label{sec:Being and Becoming}

Timaeus begins by stating that, in his opinion, a distinction should be drawn between that which always is and has no becoming (\emph{to on aei, genesin de ouk echon}) and that which becomes and never is (\emph{to gignomenon men, on de oudepote}). Three observations are immediately relevant. 

First, the ontological distinction is presented as Timaeus' opinion, or perhaps judgment more broadly, and not as self-evident or subject to demonstration. Perhaps the distinction is, in fact, subject to demonstration from first principles, but if it is, then the demonstration is beyond the scope of Timaeus' present undertaking. Timaeus has undertaken to account for the nature and generation of the All, and not to account for first principles from which the principles of his cosmology may be derived. Timaues has invoked the Gods and Goddesses in aid of his discourse on cosmology, not metaphysics. Again, Timaeus is quietly make room for an account with a specific epistemic status, ``likely'' accounts.

Second, the distinction provides us with two cases and each case is characterized by a pair of claims, an assertion and a denial. On the one hand, there is that which always is (assertion) and has no becoming (denial). On the other hand, there is that which becomes (assertion) and never is (denial). The abstract form of the distinction echoes the distinction given by the unnamed Goddess of Parmenides' poem (DK 28B2)---that which is and is impossible not to be (\emph{estin te kai ōs ouk esti mē einai}) and that which is not and is needful not be (\emph{ōs ouk estin te kai ōs chreōn esti mē einai}). Both distinctions, Timaeus' and the unnamed Goddesses', concern Being, and though they materially differ, they are in this way formally similar. Perhaps the echo is deliberate. At any rate, as we shall see, Timaeus' approach to elucidating this distinction follows the unnamed Goddess' lead.

Third, there is a minor textual issue here. Timaeus begins by describing that which always is (\emph{to on aei}). Perhaps for stylistic reasons, it is natural to expect a parallel description of the contrasting case. This would involve a second \emph{aei}: \emph{to gignomenon men aei}---that which always becomes. But as \citet{Hackforth:1959dj} observed, \emph{aei} is omitted in manuscripts F and Y, by Neoplatonist commentators such as Proclus and Simplicius, and in the Latin translations of Cicero and Calcidius. Subsequent scholarship supports Hackforth's suggestion that the second \emph{aei} is an unwarranted emendation (\citealt{Whittaker:1969mq,Whittaker:1973nz} and \citealt{Dillon:1989hc}). 

\emph{That which always is and has no becoming}. On a flat-footed reading, tempting to moderns, that which always is is that which exists always. However, the Greek verb ``to be'', \emph{enai}, and its cognates differ semantically from the corresponding English verbs (see \citealt{Kahn:2009kx}, \citealt{Brown:1994aa}, and \citealt{Leigh:2008aa}). Both in Greek and in English such a verb may occur with and without a grammatical complement. But whereas in English an occurrence of the verb ``to be'' without a grammatical complement merely means that the subject exists, in Greek it need not merely mean this. While the semantics of the Greek verb may allow for a broader meaning than in English, there is an independent motive for such a reading. That which always is never becomes, and the meaning of the latter potentially constrains the meaning of the former. That which always is may always exist but, importantly, it also always is what it is. If it always is what it is and thus retains a stable character, then this character is not subject to change and in that sense that which always is has no becoming. That which always is has no becoming since it always already is what it is thus leaving no scope for it to come to be. However we are to understand that which always is, that understanding should coordinate with our understanding of the corresponding contrast, that which never is.

Just as we should bear in mind the range of meanings of the Greek verb \emph{enai}, similarly we should bear in mind the range of meanings associated with the Greek verb \emph{gegonen}, ``to come to be.'' To come to be can be a matter of coming into existence, to have a beginning, but, importantly, to come to be can also be a matter of coming to be some way there is for things to be \citep[24--5]{Cornford:1935fk}. If we do not restrict the range of available meanings, then in claiming it never comes to be, Timaeus claims that what always is not only had no beginning, but that it never alters and never perishes.

\emph{That which becomes and never is}. Again, if we do not restrict the range of available meanings, that which comes to be includes that which has comes into existence, alters, and perishes. At least initially, I am inclined to read what has come to be (\emph{to gignomenon}) widely enough to include anything that has at least one of these features. Even something that has always been without beginning but is subject to change would count as what becomes.

One difficulty in understanding this contrasting case is understanding in what sense what comes to be never is. If the complete use of \emph{enai} is existential, then Timaeus would be claiming that what becomes does not exist. While Melissus might embrace such a conclusion, Timaeus' sympathies, and his author's, lie in a different direction. Plato, for example seems to extend the notion of Being to what comes to be in the \emph{Philebus} 27b8. As for Timaeus, he will subsequently describe what comes to be as clinging to existence (52c), and one cannot cling to what one does not have. What, then, could the denial mean if it is not simply a negative existential?

Bear in mind that what comes to be includes not only what comes into existence but what comes to be some way there is for things to be. Perhaps coming to be some way there is for things to be is never fully realized. While there is movement towards being some way there is for things to be this aspiration is never completely fulfilled. Perhaps, we can understand the contrast here between a sensible being and the Form of that sensible being. The sensible being may be F by participating in the form but it is not fully F the way the Form is. The idea is that the denial is not that what becomes never is, existentially, but rather never really is, for some appropriate understanding of ``really''.

So much for preliminaries. Timaeus, himself, does not take the distinction as evident. Indeed Timaeus first raises the distinction in the context of a question. What is that which always is and has no becoming and that which becomes and never is? The corresponding epistemological distinction that Timaeus now introduces is meant to be an answer.  What kind of character do these things have? The kind of character exhibited by the objects of understanding (\emph{noesis}) and opinion (\emph{doxa}). That which always is and has no becoming is the object of \emph{noēsei meta logou}. That which becomes and never is the object of opinion \emph{doxē met' aisthēseōs alogou}. Just as the unnamed Goddess explicates that which is in terms of that which can be known should one follow the Path of Truth, Timaeus explicates that which always is in terms of that which can be understood \emph{meta logou}. In general, Timaeus explicates the ontological distinction in terms of a distinction between the objects of distinct cognitive attitudes.

Before we can understand how the ontological pair are explicated as the objects of distinct cognitive attitudes, we must first get clearer on the nature of these attitudes. Specifically, we need to get clearer about the qualifications on understanding and opinion that we have so far left untranslated. 

\emph{Understanding \emph{meta logou}}. That which always is and has no becoming is the object of \emph{noēsei meta logou}. How are we to understand the occurrence of \emph{logou}? Given the breadth of semantic field, there are options. Since it contrasts with the way that perception is \emph{alogou}, these should receive coordinated readings. (Not all adhere to this precept, however; see, for example, \citealt[13]{Zeyl:2000cs}.) Perhaps \emph{logos} means reason or rational, in which case \emph{noēsei meta logou} means something like understanding with reason, or rational understanding, and perception would be described as without reason or irrational (\citealt[87]{Archer-Hind:1888qd}). In support of this reading, one might cite the cognitive disruption occasioned by the linear impact of \emph{aisthēsis} in the shock of embodiment (43a--44c, discussed in chapter~\ref{cha:incarnation}). But if \emph{aisthēsis} occasions cognitive disruption, it also providentially provided to resolve any such disruption as it may occasion. The Young Gods provide us with eyes to see with and ears to hear with in order correct the revolutions in our soul disrupted by our initial encounter with the sensible (46e--47e, discussed in chapter~\ref{cha:the_end_of_vision_and_audition}). Given Timaeus' sensory soteriology, it is misleading at best to describe perception as irrational. Perhaps, though, \emph{logos} may mean reasoning, in which case \emph{noēsei meta logou} means something like understanding based on reasoning. And since perception does not involve reasoning, describing it as unreasoning seems both apt and true. (\citealt{Bury:1929jb}) \emph{Logos}, however may mean account, in which case \emph{noēsei meta logou} means something like understanding with an account. And since perception does not involve an account of its objects, describing it as without an account also seems both apt and true. (\citealt[61]{Taylor:1928qb} and  \citealt{Cornford:1935fk} give interpretations of this kind, though this is obscured in Taylor's, \citeyear[25]{Taylor:1929ov} translation.) These last two readings are distinct, if not unrelated. For mortals at least, coming to have an account of an object of understanding involves discursive reasoning.

\emph{Opinion \emph{met' aisthēseōs alogou}}. That which comes to be and never is is the object of \emph{doxē met' aisthēseōs alogou}. Two questions arise. What is the relationship between opinion and perception? And in what sense is perception \emph{alogou}? 

Perhaps while understanding, at least for mortals, is based upon reasoning, opinion is based upon perception. So understood, \emph{doxa} is conceived as something like a perceptual judgment, a judgment formed on the basis of perception. Perhaps Timaeus has in mind a generalized notion of perceptual judgment, call it empirical judgment. The subject matter of an empirical judgment may not be perceived, but it may yet count as empirical if it is ultimately based upon perception. 

Even on this generalized conception of perceptual judgment, empirical judgment, perception is a necessary condition on opinion. One potential difficulty with this reading is that the World-Soul has opinion but lacks the instruments of perception. The Cosmos has no eyes to see with nor ears to hear with. Does this mean that the World-Soul has opinion without perception? If so, then perception is not a necessary condition on opinion, and so could not be neither perceptual nor empirical judgment.

This difficulty might be avoided by claiming that the epistemological principles that Timaeus provides us are principles of mortal cognition. Timaeus and his invoked audience are mortals discoursing on the nature and generation of the All. As are Plato and the readers of his dialogue. The principles given in the \emph{proemium} are principles of human cosmology and so make no claim about cosmic cognition. But the epistemological principles of the \emph{proemium} do not seem to be restricted to mortal cognition in this way. When Timaeus discusses the cognitive activity of the World Soul, he does so in the terms set out in the \emph{proemium} (chapter~\ref{cha:cognitive_revolution}).

Another response to this difficulty is to claim that only mortal perception requires corporeal instruments (Proclus, \emph{In Timaeum} 2 83.3–85.31, \citealt{Diehl:1903re}). Perception is a necessary condition on opinion. The World Soul opines and so must perceive. The sensible and the corporeal are, for the most part, exogenous to mortal perceivers and so mortals require corporeal instruments in order to receive their affections from wihtout and so come to perceive them. The sensible and the corporeal, however, are generated within the World Soul and so are endogenous to it. Since the sensible and the corporeal are endogenous to the World Soul, the World Soul does not require corporeal instruments to perceive them. Opinion may require perception, but cosmic perception does not require instruments. In this way cosmic perception is akin to bodily self-awareness. On this response, the objection goes wrong in inferring a lack of cosmic perception from a lack of cosmic instruments of perception (chapter~\ref{sec:knowledge_and_opinion}).

In what sense is perception \emph{alogou}? This narrow question of detail may nonetheless be significant given the theme of the present essay for it bears on the nature of perception. However, Timaeus' claim, here, seems consistent with each of the canvassed readings of \emph{logos}, at least with qualification. The object of perception is not perceived on the basis of reasoning. Nor does perception provide and account of its object. Perception may even be irrational, if you like, so long as a power providentially provided to promote reason and virtue may be understood to be irrational in some suitable sense. Perhaps, though, \emph{alogou} means, here, something like non-discursive. Doing so promises to at least capture what was right about each of the alternatives. If perception is non-discursive it is not based on discursive reasoning, nor does it provide a discursive account of its object, nor is it discursively rational---the object of perception may be discursively articulated in perceptual judgment but it is reason, not perception, that does so. Understanding \emph{alogou} as non-discursive seems to capture what was right about each of the alternatives since each of the alternatives involved discursive commitments.

A worry may be raised about the non-discursive gloss on the interpretation of \emph{alogou} as irrational. On that gloss, the object of perception is not a discursive reason, and so in that sense irrational, since it is not discursively articulated the way discursive reasons are. However, one may reasonably wonder whether this is, in fact, Timaeus' view of perception. As we shall see (chapters~\ref{cha:common_pathemata} and \ref{cha:peculiar_pathemata}), sensible qualities are powers of agents that affect the corporeal instruments of perception and that these qualities are perceived when these powers (and not their effects) are ``reported'' to the \emph{phronimon}, the seat of cognizance (64b4). Perception occurs either with the report or with its receipt and understanding by the \emph{phronimon}. Since perception only occurs with the report, does this not mean that the object of perception is discursively articulated?

Whether or not it has this implication very much depends on how the report is itself understood. If it is understood as speech act, the the report is discursively articulated, and since it captures the content of perception, then plausibly that content is itself discursively articulated. If, however, the report is not understood as speech act, but merely as a means of transmission, then this implication is avoided. So understood, information about the presence of a power in the perceiver's environment is transmitted to the \emph{phronimon}.


Now that we have a better sense of the relevant cognitive attitudes, how do they explicate the ontological status of their objects? Very roughly, these objects have to be the way they are in order to be the objects of these attitudes. In gaining insight into what it takes to be the object of these attitudes we gain insight into the character of these objects. This need not be understood in terms of the measure doctrine that Socrates attributes to Protagoras in the \emph{Theaetetus}. For all that has been said, the objects of cognition can be the way they are independently of being cognized and yet their being that way ensures that they are cognizable. And if they are the way they are independently of being cognized, and the way they are ensures that they are cognizable, then reflecting on what it takes to be cognizable will reveal how these objects are independent of our cognition of them.

Begin with understanding \emph{meta logou}.

% section Being and Becoming (end)

\section{Maker and Father} % (fold)
\label{sec:maker_and_father}



% section maker_and_father (end)

\section{Model and Image} % (fold)
\label{sec:model_and_image}



% section model_and_image (end)

\section{Likely Account} % (fold)
\label{sec:likely_account}



% section likely_account (end)

\section{Concluding Observations} % (fold)
\label{sec:concluding_observations}



% section concluding_observations (end)

% Chapter proemium (end) 