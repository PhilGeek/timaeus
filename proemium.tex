%!TEX root = /Users/markelikalderon/Documents/Git/timaeus/timaeus.tex

\chapter{\emph{Proemium}} % (fold)
\label{cha:proemium}

\section{A Double Significance} % (fold)
\label{sec:a_double_significance}

The first part of Timaeus's speech is traditionally called the \emph{proemium}, echoing Socrates' own description of it at 29d5 as \emph{prooimion} or prelude. Entitling the present chapter ``\emph{Proemium}'' thus has a double significance. It at once announces that the present chapter will take Timaeus' \emph{proemium} as its subject matter and that the chapter will itself serve as prelude to the essay as a whole.

% section a_double_significance (end)


\section{\emph{Proemium}} % (fold)
\label{sec:proemium}

The \emph{proemium} is bounded by the last two remarks that Socrates will make in the dialogue. In the first, just prior to Timaeus' speaking, Socrates praises the feast of \emph{logoi} that has been proposed as a return gift for his speech the previous day on the topic of the \emph{Republic} and prompts Timaeus to begin by duly invoking the gods. In the second, Socrates praises Timaeus, accepting his prelude, and encouraging him to proceed. In the \emph{proemium}, Timaeus begins by invoking the gods. After which Timaeus states some general principles, which he uses to circumscribe his subject matter and describe the kind of account that he proposes to give of it. 

The principles include:
\begin{enumerate}[(1)]
	\item The distinction between that which always is and has no becoming and that which becomes and never is and the manner in which they are cognized
	\item The principle that everything that has come to be has a cause
	\item The principle that the cause of that which has come to be is a maker that looks to a model
	\item The principle that if that which has come to be and never is is beautiful then it has been modeled on that which always is and has no becoming by a good maker
\end{enumerate}

That which always is and never becomes is the object of understanding (\emph{noēsis}) based on reasoning (\emph{logos}). That which becomes and never is is the object of opinion (\emph{doxa}) based on unreasoning perception (\emph{aisthēsis}). That which always is and has no becoming is intelligible, and that which becomes and never is sensible. The Cosmos is sensible and thus must belong to the realm of Becoming. That which becomes and never is has a cause. This causes is a maker that looks to a model. Since the Cosmos is beautiful then by the final principle, Timaeus concludes that the Cosmos has an intelligible model and a good maker. It is the generation of the Cosmos, conceived as a beautiful likeness of an intelligible model, that will be the subject matter of Timaeus' speech. Accounts should be appropriate to their subject matter, and Timaeus concludes that since his subject matter is a likeness his account shall be ``likely''.

While the general outline of the \emph{proemium} is relatively clear, each element of this tightly constructed account requires comment as does their interrelation. 

% section proemium (end)

\section{The Invocation of the Gods} % (fold)
\label{sec:the_invocation_of_the_gods}

Timaeus invokes the Gods and addresses his prayer to the Gods invoked. Without dwelling over much on Timaean theology, two remarks are worth making about Timae\-us' invocation of the Gods. The first is an observation about who is not invoked, and the second is a puzzle about who is invoked.

It is reasonable to call upon God at the outset of an undertaking. This is presented as a general practical precept. Later Timaeus explains that we invoke the Gods so as to pray for their approval of our undertaking. It would be unreasonable to fail to do so. And this remains so whether the undertaking is great or small. Timaues, however, proposes to undertake an account of the All (\emph{toū pantos}---understood, here, as the Cosmos without the explicit suggestion that it is ordered) and whether and how it was generated. This is a grand topic and a great undertaking so it is especially appropriate that Timaeus call upon the Gods at the outset.

What it is to pray for such a great undertaking is elaborated in Timaeus' final prayer (\emph{Critias} 106a1–b6). There Timaeus prays that his words may endure if they are spoken truly and meet with divine approval. Timaeus further prays that should his speech be false that a just penalty be imposed, the just penalty in the case of error being correction. Finally, Timaeus prays for knowledge, the most perfect among medicines, so that he may speak truly in the future.

It is striking who is not called upon by Timaeus. Notable in His absence is the Demiurge, the Maker and Father of the All. The Maker and Father of the All is difficult to discover and impossible to explain to everyone (28c3–5). He only begins to emerge in reasoning from principles that Timaeus only states after the invocation. Timaeus is invoking the Gods and addressing his prayers to Them, in part, in aid of this difficult discovery. If so, then the object of his search was not among the Gods that were invoked even it is itself divine.

There is a related reason for thinking that the Demiurge was not among the Gods invoked and prayed to. Even if it were possible to privately invoke the Gods and pray for their approval of some solitary undertaking, this is not the context that Timaeus finds himself in. The invocation of the Gods urged by Socrates was, importantly, a public act of piety. And Timaues complies in the first person plural. It is we who are about to converse that have need to invoke the Gods and the Goddesses. A divinity difficult to discover and impossible to explain to all is perhaps not the most apt choice to invoke in a public act of piety. One's piety toward the difficult to discover divinity may be genuine but it might not be publicly manifest. One's piety is only guaranteed to be publicly manifest if directed toward recognized divinities. And a difficult to discover divinity impossible to explain to all is not a recognized divinity. The obscurity of the Maker and Father of All is inconsistent with the public nature of the invocation.

Perhaps, in the present instance, the public is restricted to the divided reference of ``we''---Timaeus, Socrates, Critias, and Hermocrates. The restriction may be significant. A divinity not widely known to many may yet be known to the present few. However, while Timaeus might rely on Socrates to already know of this difficult to discover divinity, is it really reasonable that Timaeus should similarly rely on Critias? Even among the present dignitaries, the obscurity of the divinity is in tension with the publicity of the act.

As \citet[13--4]{Broadie:2012vl} observes, the Demiurge is not invoked in any of the prayers offered by Timaeus. Broadie's observation coheres with the present explanation. Since all of the prayers in the dialogue are public acts, the obscurity of the divinity would preclude His public invocation.

Not only is it notable that Timaeus does not invoke the Demiurge, but it is obscure who is being invoked. Specifically, is Timaeus invoking one God or many? Notice he first calls upon God in the singular and then upon Gods and Goddesses in the plural. When Timaeus first calls upon God in the singular, it is in stating the general practical precept to call upon God and pray for approval of one's undertaking. When Timaeus calls upon Gods and Goddesses in the plural, this is not in the context of stating a general practical precept, but rather in the context of proposing to undertake an account of the All and its generation. Perhaps it is the Young Gods who are being invoked when Timaeus invokes the many. The Young Gods are divine, immortal beings generated by the Demiurge who are the instruments of Demiurgic activity. Specifically, as we shall see (chapters~\ref{cha:incarnation} and \ref{cha:the_flesh_and_the_mortal_soul}), they play an indispensable role in the generation of mortal beings. If the Young Gods are the many, who could the one be if not the Demiurge? One candidate is the All itself. Timaeus conceives of the Cosmos as a visible God (34a8, 34b8--9, 92a5--9), defers to it concerning its proper name in act of ritual piety (28b3--5, for discussion see \citealt[66]{Taylor:1928qb}), and addresses his final prayer to that God (\emph{Critias} 106a1--b6). By contrast, as \citet[35]{Cornford:1935fk} observes, ``Neither in the \emph{Timaeus} nor anywhere else is it suggested that the Demiurge should be an object of worship: he is not a religious figure.''

There are, of course, alternatives. Perhaps the God in the singular being invoked is a generic divinity of which the Gods and the Goddesses in the plural are species. On this alternative, an appeal to a generic divinity is appropriate in the statement of the content of a general practical percept. Similarly, it is appropriate to invoke the species of this generic divinity given the specific undertaking to account for the All. That all species of divinity should be invoked and their approval prayed for is perhaps only required for their specific undertaking of discoursing on the All and its generation. Perhaps a single divinity would do if the undertaking were small enough. But the collective undertaking that Timaeus proposes is suitably grand as to require the invocation of every species of divinity.

A deflationary take on the identity of the Gods invoked might might be motivated by the perfunctory nature of the invocation. Proclus describes the worry as follows:
\begin{quote}
	But how can it be, they say, that Timaeus has announced with a grand flourish that one should pray and call on the gods and goddesses, but that he fails to do this himself and immediately turns to proposed accounts without praying? (Proclus, \emph{In Timaeum} 1 221.9--12, \citealt{Diehl:1903re}; \citealt[58]{Runia:2008aa})
\end{quote}
If Timaeus has not in fact prayed, then there is no point in inquiring into to whom Timaeus has addressed his prayer. (That the invocation of the Gods should strike some as a grand flourish is worth remarking on. As \citealt[104]{Runia:1997vz}, observers, ``Only a minority of \emph{prooemia} begin with an \emph{invocation of the gods}, continuing in the tradition set by Hesiod in his \emph{Theogony}.'') Proclus' reply is ingenious in that he, in effect, proposes that we understand Timaeus' speech as performative in something like Austin's \citeyearpar{Austin:1975nx} sense. In saying that we must pray to the Gods, Timaeus has done just that. While one may decide to pray and pray later, in true prayer, the decision to pray and prayer are one and the same thing. Timaeus's public prayer on the occasion of the Panathenaea is a true prayer. And so in saying that we must pray, he does.

Still, exactly who is being invoked by Timaues may not be so neatly resolvable. Perhaps there is a kind of semantic slippage here, from singular to plural, the one resolving into the many. If so, this will be echoed in the third part of Timaeus' speech which concerns the interaction of Reason and Nececssity, especially in his account of the generation of the flesh and the mortal soul. There, Timeaus slides freely between singular and plural in describing what is officially meant to be rational cooperative activity, the Demiurge generating the immortal part of the soul and the Young Gods weaving it together with its mortal body that they generated from corporeal material borrowed from the body of the Cosmos. However, even for tasks that are meant for the Young Gods, their executions are sometimes described in the singular. We shall discuss the significance of this when we discuss the flesh and the mortal soul (chapter~\ref{cha:the_flesh_and_the_mortal_soul}). For now, I merely note a potential anticipatory trace of this puzzling usage. 

One last observation. Not only are the Gods invoked, but Timaeus also invokes themselves. The plural self-invocation is for the audience---Socrates, Critias, and Herm\-ocrates---to more easily learn and for the speaker---Timaeus---to more clearly expound his account of the All and the cause of its generation. (On the self-invocation see Proclus, \emph{In Timaeum} 1 222.11--223.2, \citealt{Diehl:1903re}.) Not only is Timaeus praying for the approval of the Gods concerning their present collective undertaking---his conversing with them about the nature and generation of the All---but he is importantly also praying that they themselves approve of their discourse, if only after being granted divine approval. While for the most part, Timaeus delivers an uninterrupted speech, in the invocation he describes it as a collective activity with the audience as co-participants. Perhaps the self-invocation is meant to get the participants of the discourse in a position where they may engage in it in such a way that they come to themselves approve of that discourse. This would involve the audience rationally attending to the speaker's account so that they may more easily learn his views about the nature and generation of the All. And this would involve the speaker clearly expounding his views to this audience. Even if largely a monologue, Timaeues' speech remains directed to its audience. And Timaeus must bear them in mind if he is to clearly expound his views to them in such a way as warrant their collective approval. Only in this way could the return speech discharge Timaeus' obligation to Socrates. Notice that it is his views that Timaeus wishes his audience to learn. Contrast learning the truth about the All and its generation. Timaeus is pointedly not proposing to demonstrate such truths. Timaeus, here, is quietly allowing room for what will emerge as accounts with a specific epistemic status, ``likely'' accounts.

If the plural self-invocation still seems odd or unfamiliar, it really should not. You have doubtless encountered it in contemporary speech. Think of someone inaugurating their address with ``Let's do this!'' Think of a motivational speaker or, better yet, a Hip Hop artist. ``This'', that which is to be done (dropping bars for the Hip Hop artist, cosmology for Timaeus, both are \emph{logoi}), is explicitly represented as a collective activity---``Let \emph{us} do this'', where ``us'' includes both speaker and audience. In saying ``Let's do this!'', the speaker draws attention to themselves, making their audience more receptive to their address, and the speaker expresses their determination in a way that draws the speaker's own attention to their audience, in an attempt to thereby more effectively deliver their address. Should the speaker prove genuinely effective, they may even garner approval for that address. It even has, in line with Proclus' suggestion, a performative aspect. In saying ``Let's do this!'', that which is to be done has already begun.

% section the_invocation_of_the_gods (end)

\section{Being and Becoming} % (fold)
\label{sec:Being and Becoming}

Timaeus begins by stating that, in his opinion, a distinction should be drawn between: 
\begin{enumerate}[(1)]
	\item that which always is and has no becoming (\emph{to on aei, genesin de ouk echon}) and
	\item that which becomes and never is (\emph{to gignomenon men, on de oudepote}).
\end{enumerate}
Three observations are immediately relevant. 

First, the ontological distinction is presented as Timaeus' opinion, or perhaps judgment more broadly, and not as self-evident or subject to demonstration. Perhaps the distinction is, in fact, subject to demonstration from first principles, but if it is, then the demonstration is beyond the scope of Timaeus' present undertaking. Timaeus has undertaken to account for the nature and generation of the All, and not to account for first principles from which the principles of his account may be derived. Speaking anachronistically, Timaues has invoked the Gods and Goddesses in aid of his discourse on cosmology, not metaphysics. Again, Timaeus is quietly make room for an account with a specific epistemic status, ``likely'' accounts.

Second, the distinction provides us with two cases and each case is characterized by positive and negative claims. On the one hand, there is that which always is (positive) and has no becoming (negative). On the other hand, there is that which becomes (positive) and never is (negative). The abstract form of the distinction echoes the distinction given by the unnamed Goddess of Parmenides' poem (DK 28B2)---that which is (positive) and is impossible not to be (negative) (\emph{estin te kai ōs ouk esti mē einai}) and that which is not (positive) and is needful not to be (negative) (\emph{ōs ouk estin te kai ōs chreōn esti mē einai}). Both distinctions, Timaeus' and the unnamed Goddesses', concern Being, and though they conceptually differ, they are, nonetheless, in this, way formally similar. Perhaps the echo is deliberate. At any rate, as we shall see, Timaeus' approach to elucidating this distinction follows the unnamed Goddess' lead (on the historical background to Timaeus' \emph{proemium} see \citealt{Runia:1997vz}).

Third, there is a minor textual issue here. Timaeus begins by describing that which always is (\emph{to on aei}). Perhaps for stylistic reasons, it is natural to expect a parallel description of the contrasting case. This would involve a second \emph{aei}: \emph{to gignomenon men aei}---that which always becomes. But as \citet{Hackforth:1959dj} observed, \emph{aei} is omitted in manuscripts F and Y, by Neoplatonist commentators such as Proclus and Simplicius, and in the Latin translations of Cicero and Calcidius. Subsequent scholarship supports Hackforth's suggestion that the second \emph{aei} is an unwarranted emendation (\citealt{Whittaker:1969mq,Whittaker:1973nz} and \citealt{Dillon:1989hc}). 

(1) \emph{That which always is and has no becoming}. On a flat-footed reading, tempting to moderns, that which always is is that which exists always. However, the Greek verb \emph{enai} and its cognates differ semantically from the corresponding English verbs (see \citealt{Kahn:2009kx}, \citealt{Brown:1994aa}, and \citealt{Leigh:2008aa}). Both in Greek and in English such a verb may occur with and without a grammatical complement. But whereas in English an occurrence of the verb ``to be'' without a grammatical complement merely means that the subject exists, in Greek it need not merely mean this. While the semantics of the Greek verb may allow for a broader meaning than in English, there is an independent motive for such a reading. That which always is never becomes, and the meaning of the latter potentially constrains the meaning of the former. That which always is may always exist but, importantly, it also always is what it is. If it always is what it is and thus retains a stable character, then this character is not subject to change and in that sense that which always is has no becoming. That which always is has no becoming since it always already is what it is thus leaving no scope for it to come to be some way there is for things to be. However we are to understand that which always is, that understanding should coordinate with our understanding of the corresponding contrast, that which never is, and as we shall see, the negative existential reading of the latter is implausible.

Just as we should bear in mind the range of meanings of the Greek verb \emph{enai}, similarly we should bear in mind the range of meanings associated with the Greek verb \emph{gegonen}, ``to come to be.'' To come to be can be a matter of coming into being, to have a beginning, but, importantly, to come to be can also be a matter of coming to be some way there is for things to be \citep[24--5]{Cornford:1935fk}. If we do not restrict the range of available meanings, then in claiming it never comes to be, Timaeus claims that what always is not only had no beginning, but that it never changes and never perishes.

(2) \emph{That which becomes and never is}. Again, if we do not restrict the range of available meanings, that which becomes includes that which comes into being, changes, and perishes. At least initially, I am inclined to read what has come to be (\emph{to gignomenon}) widely enough to include anything that has at least one of these features. Even something that has always been without beginning but is subject to change would count as what becomes.

One difficulty in understanding this contrasting case is understanding in what sense what comes to be never is. If the complete use of \emph{enai} is existential, then Timaeus would be claiming that what becomes does not exist. While Melissus might embrace such a conclusion (DK 30B1--3), Timaeus' sympathies, and his author's, lie in a different direction. Plato, for example seems to extend the notion of Being to what comes to be in the \emph{Philebus} 27b8. As for Timaeus, he will subsequently describe what comes to be as clinging to existence (52c), and one cannot cling to what one does not have. What, then, could the denial mean if it is not simply a negative existential?

Bear in mind that what comes to be includes not only what comes into being but what comes to be some way there is for things to be. Perhaps coming to be some way there is for things to be is never fully realized. While there is movement towards being some way there is for things to be this aspiration is never completely fulfilled. Perhaps, we can understand the contrast here between a sensible being and the Form of that sensible being. The sensible being may be F by participating in the Form but it is not fully F the way the Form is. The idea is that the denial is not that what becomes never is, existentially, but rather never really is, for some appropriate understanding of ``really''.

So much for preliminaries. Timaeus, himself, does not take the distinction as evident. Indeed Timaeus first raises the distinction in the context of a question. What is that which always is and has no becoming and that which becomes and never is? The corresponding epistemological distinction that Timaeus now introduces is meant to be an answer.  What kind of character do these things have? The kind of character exhibited by the objects of understanding (\emph{noesis}) and opinion (\emph{doxa}). Specifically: 
\begin{enumerate}[(1)]
	\item that which always is and has no becoming is the object of \emph{noēsei meta logou perilēpton}, and
	\item that which becomes and never is the object of opinion \emph{doxē met' aisthēseōs alogou doxaston}.
\end{enumerate}
Just as the unnamed Goddess explicates that which is in terms of that which can be known should one follow the Way of Truth, Timaeus explicates that which always is in terms of that which can be understood \emph{meta logou}. In general, Timaeus explicates the ontological distinction in terms of a distinction between the objects of distinct cognitive attitudes.

Before we can understand how the ontological pair are explicated as the objects of distinct cognitive attitudes, we must first get clearer on the nature of these attitudes. Specifically, we need to get clearer about the qualifications on understanding and opinion that we have so far left untranslated. 

(1) \emph{Noēsei meta logou perilēpton}. That which always is and has no becoming is the object of \emph{noēsei meta logou perilēpton}. The attitude is a kind of understanding (\emph{noēsei}). The object of this understanding is apprehended (\emph{perilēpton}). Interestingly, while Timaeus could have said \emph{lepton} here, he says, instead, \emph{perilēpton}. This potentially highlights, I think, the significance of \emph{peri}-. Timaeus is pointedly claiming noetic apprehension circumscribes its object. This is an anticipatory trace of an important theme in Timaean psychology, the circular motion of cognitive activity (chapter~\ref{cha:cognitive_revolution}).

The understanding whose apprehension circumscribes its object is \emph{meta logou}. How are we to understand the occurrence of \emph{logos}? Given the breadth of semantic field, there are options (though some may not be relevant, such as \emph{logos} understood as ratio). Perhaps \emph{logos} means reason or rational, in which case \emph{noēsei meta logou perilēpton} means something like understanding with reason, or rational understanding, and perception would be described as without reason or irrational (\citealt[87]{Archer-Hind:1888qd}). In support of this reading, one might cite the cognitive disruption occasioned by the linear impact of \emph{aisthēsis} in the shock of embodiment (43a--44c, discussed in chapter~\ref{cha:incarnation}). But if \emph{aisthēsis} occasions cognitive disruption, it also providentially provided to resolve any such disruption as it may occasion. The Young Gods provide us with eyes to see with and ears to hear with in order correct the revolutions in our soul disrupted by our initial encounter with the sensible (46e--47e, discussed in chapter~\ref{cha:the_end_of_vision_and_audition}). Given Timaeus' sensory soteriology, it is misleading at best to describe perception as irrational. Perhaps, though, \emph{logos} may mean reasoning, in which case \emph{noēsei meta logou perilēpton} means something like understanding based on reasoning. And since perception does not involve reasoning, describing it as unreasoning seems both apt and true. (\citealt{Bury:1929jb}) \emph{Logos}, however may mean account, in which case \emph{noēsei meta logou perilēpton} means something like understanding with an account. And since perception does not involve an account of its objects, describing it as without an account also seems both apt and true. (\citealt[61]{Taylor:1928qb}, \citealt[22]{Cornford:1935fk}, and \citealt[16]{Waterfield:2008lx}, give interpretations of this kind, though this is obscured in Taylor's, \citeyear[25]{Taylor:1929ov}, translation.) These last two readings are distinct, if not unrelated. For mortals at least, coming to have an account of an object of understanding involves discursive reasoning.

Since the present occurrence of \emph{logos} contrasts with the way that perception is \emph{alogos}, these should receive coordinated readings. (Not all adhere to this precept, however; see, for example, \citealt[13]{Zeyl:2000cs}.) As such, we should hold off on how to interpret the present occurence of \emph{logos} until we understand in what sense perception is \emph{alogos}.

Another issue concerns the force of \emph{meta}. Suppose, for the sake of argument, that \emph{logos} means a discursive account. Understanding with an account might be understood as the conjunction of distinct conditions. There is the noetic apprehension that circumscribes its object and there is the possession of an account of what is thus apprehended. Understanding with an account occurs when both of these conditions are met. Understanding with an account might, however, be understood in another way, not as understanding that meets a further discursive condition, but rather where the account is the discursive means by which the object is understood. One comes to understand an object by engaging in the discursive reasoning required to come to possess an account of it, and one noetically apprehends that object when one comes to possess an account of that object. So conceived, understanding and possessing an account are not distinct conditions, and so \emph{noēsei meta logou perilēpton} is not the conjunction of distinct conditions. 

While the immediate passage in which \emph{noēsei meta logou perilēpton} occurs cannot by itself determine which of these two interpretations are correct, I am nevertheless inclined to favor the latter. There is, however, a complication facing any such interpretation. An account is discursively articulated into conceptually distinct parts. An account, in this way, admits of decomposition (in something like Dummett's \citeyear{dummett73} sense). It is plausible that process of decomposition cannot carry on indefinitely, and that this is a condition on at least the mortal intelligibility of an account. This means that an account will have as a part things which are not themselves subject to account. How are the conceptually primitive parts of an account cognized? Certainly not by understanding with an account. Since Timaeus is explicating what always is in terms of what can be understood with an account, are we to conclude that the conceptually primitive parts of an account never are but come to be? That seems implausible. Suppose one held that the conceptually primitive parts of the account were cognized in some other way, by means of some other cognitive attitude. Then since possession of an account would involve the apprehension of its primitive parts, this attitude would be more fundamental than \emph{noēsei meta logou perilēpton}. But there is no explicit mention of such an attitude in the \emph{proemium}. Perhaps the Timean formula is meant to be understood broadly so as to include noetically apprehending an object through a discursive account, as well as everything that is required to have that attitude such as the apprehension of the conceptually primitive parts of the account.

(2) \emph{Doxē met' aisthēseōs alogou doxaston}. That which comes to be and never is is the object of \emph{doxē met' aisthēseōs alogou doxaston}. Two questions arise. What is the relationship between opinion and perception? And in what sense is perception \emph{alogos}? 

Perhaps while understanding, at least for mortals, is based upon reasoning, opinion is based upon perception. So understood, \emph{doxa} is conceived as something like a perceptual judgment, a judgment formed on the basis of perception. Perhaps Timaeus has in mind a generalized notion of perceptual judgment, call it empirical judgment. The subject matter of an empirical judgment may not be perceived, but it may yet count as empirical if it is ultimately based upon perception. (Thus Timaeus will maintain that the sensible and corporeal are composed out of elemental triangles too small to be seen. Judgments about these are nevertheless empirical in the sense adumbrated.)

Even on this generalized conception of perceptual judgment, empirical judgment, perception is a necessary condition on opinion. One potential difficulty with this reading is that the World-Soul has opinion but lacks the instruments of perception. The Cosmos has no eyes to see with nor ears to hear with (33c). Does this mean that the World-Soul has opinion without perception? If so, then perception is not a necessary condition on opinion, and so could not be neither perceptual nor empirical judgment. 

This difficulty might be avoided by claiming that the epistemological principles that Timaeus provides us are principles of mortal cognition. Timaeus and his invoked audience are mortals conversing on the nature and generation of the All. As are Plato and the readers of his dialogue. The principles given in the \emph{proemium} are principles of human cosmology and so make no claim about cosmic cognition. But the epistemological principles of the \emph{proemium} do not seem to be restricted to mortal cognition in this way. When Timaeus discusses the cognitive activity of the World Soul, he does so in the terms set out in the \emph{proemium} (chapter~\ref{cha:cognitive_revolution}).

Another response to this difficulty is to claim that only mortal perception requires corporeal instruments (Proclus, \emph{In Timaeum} 2 83.3–85.31, \citealt{Diehl:1903re}). Perception is a necessary condition on opinion. The World Soul opines and so must perceive. The sensible and the corporeal are, for the most part, exogenous to mortal perceivers and so mortals require corporeal instruments in order to receive their affections from wihtout and so come to perceive them. The sensible and the corporeal, however, are generated within the World Soul and so are endogenous to it. Since the sensible and the corporeal are endogenous to the World Soul, the World Soul does not require corporeal instruments to perceive them. Opinion may require perception, but cosmic perception does not require instruments. In this way cosmic perception is akin to bodily self-awareness. On this response, the objection goes wrong in inferring a lack of cosmic perception from a lack of cosmic instruments of perception (chapter~\ref{sec:knowledge_and_opinion}).

In what sense is perception \emph{alogos}? This narrow question of detail may nonetheless be significant given the theme of the present essay for it bears on the nature of perception. Timaeus' claim, here, seems consistent with each of the canvassed readings of \emph{logos}, at least with qualification. The object of perception is not perceived on the basis of reasoning. Nor does perception provide and account of its object. Perception may even be irrational, if you like, so long as a power providentially provided to promote reason and virtue may be understood to be irrational in some suitable sense. 

Earlier we noted that the occurrence of \emph{logos} in \emph{noēsei meta logou perilēpton} could not be interpreted as ratio. It is also worth observing that perception being \emph{alogos} could not mean that perception is without ratio or somehow disproportionate. As we shall see (chapters~\ref{cha:common_pathemata} and \ref{cha:peculiar_pathemata}), Timaeus understands perception on the model of measurement (a model taken up by Aristotle in \emph{Metaphysica} I 1053a31). Bodily affections are measures of the powers of the agents that caused them, and perception is the cognizance of what is thus measured. At the very least, then, the objects of perception and the \emph{pathēmata} are proportional if the latter can measure the former. Moreover, ratios are not restricted to \emph{arithmoi}, and so powers and \emph{pathēmata} may be proportional. Thus, for example, Timaeus will go on to claim that the four primary bodies---fire, air, water, and earth---are bound together in the Cosmos by proportion (31b--32b, see also Aristotle, \emph{Physica} 3 204b14–19).

Perhaps, though, \emph{alogos}, here, means something like non-discursive. Doing so pro\-mises to at least capture what was right about each of the alternatives. If perception is non-discursive it is not based on discursive reasoning, nor does it provide a discursive account of its object, nor is it discursively rational. The object of perception may be discursively articulated in perceptual judgment but it is reason, not perception, that does so, The object of perception is not discursively articulated the way reasons are. Understanding \emph{alogos} as non-discursive seems to capture what was right about each of the alternatives since each of the alternatives involved discursive commitments.

A worry may be raised about \emph{alogos} understood as irrational being glossed as not discursively rational. On that gloss, the object of perception is not a discursive reason, and so in that sense irrational, since it is not discursively articulated the way discursive reasons are. However, one may reasonably wonder whether this is, in fact, Timaeus' view. As we shall see (chapters~\ref{cha:common_pathemata} and \ref{cha:peculiar_pathemata}), sensible qualities are powers of agents that affect the corporeal instruments of perception and these qualities are perceived when these powers are ``reported'' to the \emph{phronimon}, the seat of cognizance (64b4). Since perception only occurs with the report, does this not mean that the object of perception is discursively articulated?

Whether or not it has this implication very much depends on how the report is itself understood. If it is understood as speech act, the the report is discursively articulated, and since it captures the content of perception, then it is plausible that that content is itself discursively articulated. At the very least, the content of perception would be discursively articulable. If the report is a speech act, then there is a speaker, the agent of that speech act. But who is the speaker? Timaeus will distinguish immortal and mortal parts of the soul, maintianing that its rational powers are invested in the immortal part. Is it the mortal soul that animates the body that is reporting the power of the agent that affected that body to the immortal soul? But is the mortal soul sufficiently personal to intelligibly ascribe speech acts to? Perhaps the report should not be understood as a speech act but is merely a means of transmission of information. If the report is not understood as speech act, but merely as a means of transmission, then the implication that the content of perception is discursively articulated is avoided. So understood, information about the presence of a power in the perceiver's environment is transmitted to the \emph{phronimon}. Even on this reading, however, the \emph{phronimon}'s uptake of the information transmitted, its cognizance of the power, might itself be discursively articulated. We shall return to this interpretive issue.

Now that we have a better sense of the relevant cognitive attitudes, how do they explicate the ontological status of their objects? Very roughly, these objects have to be the way they are in order to be the objects of these attitudes. In gaining insight into what it takes to be the object of these attitudes we gain insight into the character of these objects. This need not be understood in terms of the measure doctrine that Socrates attributes to Protagoras in the \emph{Theaetetus}. For all that has been said, the objects of cognition can be the way they are independently of being cognized and yet their being that way ensures that they are cognizable. And if they are the way they are independently of being cognized, and the way they are ensures that they are cognizable, then reflecting on what it takes to be cognizable will reveal how these objects are independent of our cognition of them.

Begin with \emph{noēsei meta logou perilēpton}. What must something be like to be the object of this attitude? Timaeus offers an explicit answer. What always is and has no becoming is the object of \emph{noēsei meta logou perilepton} because it always uniformly is (\emph{aei kata tauta on}). Recall, what always is has no becoming not only in the sense that it lacks a beginning, and so does not come into being, but also in the sense that it never comes to be some way for things to be. What always is is always already what it is. It fully is what it is leaving no further scope for it coming to be some way there is for things to be. 

What always is is some way there is for things to be. What always is is the way it is eternally, and so does not vary over time. Now we learn that neither does it vary across its parts, in some suitable sense ``part'' (on Platonic mereology see \citealt{Harte:2002tl}). So conceived, that which always is and has no becoming uniformly is what it is. This is a deliberate Eleatic echo. Parmenides conceived of the One Being as spatial and finitely extended in every direction thus forming a voluminous sphere. When Parmenides claims that the One Being is uniform, he is attributing to it a kind of self-similarity. Every part of the voluminous sphere is like every other part. It will turn out that Timaeus, unlike Parmenides, denies that what always is is spatially extended. Nevertheless, what always is is uniform in the sense of being self-similar. In not being spatially extended, what always is lacks spatial parts. But so long as parts are not restricted to spatial parts, it may yet be true that every part of what always is is just like every other part. Recall what always is is apprehended with an account. An account is discursively articulable, so perhaps what always is has conceptually distinct parts, the parts articulated in the associated account. And even if what always is lacks parts distinct from the whole even in a non-spatial sense, it will still be trivially true that it is uniform in the sense of being self-similar. In general, if something is self-similar, in the relevant sense, then if x an y are parts of the thing distinct from the whole of it, then x an y are similar. If what always is lacks parts distinct from the whole of it, then the antecedent of the embedded conditional is false, and that conditional trivially true.

On the present interpretation, uniformity of being is an condition on being noetically apprehensible. As we shall see (chapter~\ref{cha:cognitive_revolution}), this is confirmed in Timaeus account of the cognitive activity of the World Soul. And again, Timaeus transforms the Eleatic framework that he builds upon. Parmenides maintains that only Being is intelligible. And so only Being is the object of thought. Parmenides conceives of the One Being as spatially extended and finite, in the shape of a voluminous sphere. If the thought that thinks the One Being involves spatial movement, it could only go around that sphere. Thought is about its object by going around it. Thus the unnamed Goddess that reveals the Ways of Truth and Mortal Opinion to Parmenides proclaims that her thought is \emph{amphis alētheis}.

Timaeus accepts the the Eleatic pun where what the motion is about (\emph{amphis}, \emph{peri}), the center, is what the thought is about (\emph{amphis}, \emph{peri}), its object (\citealt[191--3]{Mourelatos:2008ve}). As we shall see (chapter~\ref{sec:the_cognitive_significance_of_circular_motion}), however, the Eleatic pun is transformed in his hands. It is at least open to claim that, for Parmendies, thought literally moves around its object. However, Timaeus thinks that the objects of noetic apprehension lack extension. So noetic apprehension could not be spatial movement around the contours of its object. And yet Timaeus retains the Eleatic pun despite rejecting the Eleatic claims upon which it rests. There is a shift of sense here. And not just in what is meant by the pun, but also in the meaning of it. For Timaeus, the Eleatic pun could only be a metaphor.

The circular motion around the object of noetic apprehension is the incorporeal activity of a cognitive act that encompasses its object. In order for this incorporeal activity to encompass its object in an act of noetic apprehension, the circular motion must be uniform (34a3, 36c2, 40a7). If as Aristotle claims (\emph{De anima} 1.2 404b16--18, 1.3 406b28--31, see also Alcinous \emph{Didaskalikos} 14.4), Timaeus accepts the principle that like is known by like, on some suitable understanding of that principle (\citealt{Corcilius:2018bd}, chapter~\ref{cha:cognitive_revolution}), then we can understand uniformity as a condition on being noetically apprehensible as follows. What always is can be understood with an account. Understanding with an account involves uniform incorporeal activity that circumscribes its object. Like is known by like. Only that which is uniform may be apprehended by an incorporeal activity that is itself uniform. Therefore, what always is must itself be uniform.

Consider now \emph{doxē met’ aisthēseōs alogou doxaston}. What must something be like in order to be the object of this attitude?

First, notice that what becomes is not understood with an account. From this we may conclude that it is not uniform. Indeed, what becomes is not uniform or self-similar. What becomes comes into and goes out of being and changes over time. Moreover, what becomes may have parts distinct from the whole that vary qualitatively. Thus the pre-cosmic chaos may be fleetingly fiery over here, and fleetingly earthy over there, such that not every part of the pre-cosmic chaos is alike. A general feature that the Cosmos itself retains. Not every part of the extended sensible Cosmos is alike.

Second, while non-uniform and so not the object of understanding with an account, what becomes and never is not utterly uncognizable. Not only may we opine about comes to be and never is, it is possible to have true opinion about the realm of Becoming. If the unnamed Goddess is really understood to be claiming that all opinion about what comes to be is false, then Timaeus differs, in this way, from Her. Like the unnamed Goddess, Timaeus accepts that opinion differs from understanding and knowledge. They disagree about how to understand this difference. According to Timaeus, opinion differs from understanding not because it is systematically false, but because a systematic account of the sensible realm of Becoming has an epistemic status distinct from an account involved in understanding. As we shall see, it is merely ``likely''. (On Parmenides and Timaeus on ``likely'' accounts see \citealt{Bryan:2012bt}.)

Third, uniformity is a positive feature that explains, at least in part, why something is the object of uniform cognitive activity that constitutes understanding with an account. However, Timaeus does not correspondingly cite a positive feature that explains, even in part, why something is the object of opinion. Instead, Timaeus draws our attention to the fact that what becomes comes into and goes out of being and never really is. Far from being features that explain the cognition of these objects, these features seem, at least initially, to be obstacles to their cognition. (This asymmetry is obscured in Zeyl's \citeyear[13]{Zeyl:2000cs}, translation.) Timaeus' thought is that the objects of opinion are cognizable despite coming to be because they are sensible. Here the occurrence of \emph{meta} must mean something like on the basis of. Opinion is about what comes to be and never is on the basis of perception. Instead of a positive feature that explains why something is the object of of opinion, we have an obstacle to cognition that could only be overcome by perception.

This interpretation again faces the issue that the World Soul truly opines about the sensible realm of Becoming but lacks the instruments of perception. If we may infer a lack of perception from the lack of an instrument of perception, then the World Soul's opinions are not based on perception, and \emph{meta} in \emph{doxē met’ aisthēseōs alogou} must be understood in some other way. Or perhaps Proclus is right (\emph{In Timaeum} 2 83.3–85.31, \citealt{Diehl:1903re}), and we should not infer a lack of perception from a lack of an instrument of perception. Perhaps lacking an instrument of perception is just a matter of how cosmic perception differs from mortal perception. If the World Soul may perceive, then there is no obstacle to the World Soul truly opining on the basis of what it perceives. We shall return to this issue (chapter~\ref{sec:knowledge_and_opinion}).

What always is and has no becoming is understandable with an account. Since it is uniform in its being, it may be encompassed by uniform intellectual activity. It is thus intelligible. What comes to be and never is opinable with perception. Though non-uniform in coming to be, it is nonetheless cognizable since it is sensible and so opinable. It is thus sensible.

This last observation, though quotidian, is nevertheless worth emphasizing, given the topic of the present essay. Recall the ontological pair is being explicated in terms of the objects of distinct cognitive attitudes. We learn something about their character by learning what about them enables them to be the objects of these attitudes. What becomes and never is is sensible and so cognizable by opinion. The realm of Becoming is, by nature, sensible. Only if it were sensible could it so much as be the object of opinion. There is no room, here, for conflict between a scientific image of nature, provided by Timaeus' ``likely'' account, and the manifest image of nature. That nature is manifest in perceptual experience is what makes it subject to opinion and, hence, to the kind of ``likely'' account that Timaeus unfolds in his speech after the \emph{proemium}.

% section Being and Becoming (end)

\section{Maker and Father} % (fold)
\label{sec:maker_and_father}

Having distinguished that which always is and has no becoming from that which becomes and never is, and having explained this distinction in terms of the objects of distinct cognitive attitudes, Timaeus now states a pair of causal principles:
\begin{enumerate}[(1)]
	\item Everything that comes to be of necessity comes to be as the result of some cause
	\item The cause of that which comes to be is a maker that looks to a model.
\end{enumerate}
Let us examine these in turn.

(1) \emph{Everything that comes to be of necessity comes to be as the result of some cause}. It is unclear, yet, what, for Timaeus, counts as a cause (\emph{aitia}). The second principle promises to shed light on this. However Timaeus understands this explanatory determination, all that comes to be is so determined. This is a strong claim. It rules out the possibility of spontaneous generation and that the realm of Becoming is fundamentally chancy. Any yet Timaeus provides no justification for it. What seems like a justification is really just a restatement of the principle: Everything that comes to be of necessity results of some cause since if there were no cause, it would be impossible for anything to come to be. This is less to provide a justification or explanation than to provide a contrapositive equivalent. Perhaps a non-circular justification could only be provided by appeal to first principles and so would be beyond the scope of Timaeus's speech.

Timaeus needs this strong principle, however, since the existence of the difficult to discover divinity, the Demiurge, Maker and Father of the All, depends upon it. Timaeus will reason that since the All is generated, by our principle, the All has a cause, and the Demiurge is identified with the cause of the All. 

Besides being necessary to discover this obscure divinity, impossible to explain to all, the strength of the principle has another significance. If everything that comes to be of necessity has a cause, then the sensible realm of Becoming is intelligible in the minimal sense that it is subject to causal explanation. Whatever occurs in the realm of Becoming does so for a reason.

(2) \emph{The cause of that which comes to be is a maker that looks to a model}. Timaeus second principle is informative about his first. The cause of what comes to be is conceived as maker that looked to a model to guide their production. 

% section maker_and_father (end)

\section{Model and Image} % (fold)
\label{sec:model_and_image}



% section model_and_image (end)

\section{Likely Account} % (fold)
\label{sec:likely_account}



% section likely_account (end)

\section{Concluding Observations} % (fold)
\label{sec:concluding_observations}



% section concluding_observations (end)

% Chapter proemium (end) 