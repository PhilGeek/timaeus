%!TEX root = /Users/markelikalderon/Documents/Git/timaeus/timaeus.tex

\chapter{\emph{Proemium}} % (fold)
\label{cha:proemium}

\section{A Double Significance} % (fold)
\label{sec:a_double_significance}

The first part of Timaeus's speech is traditionally called the \emph{proemium}, echoing Socrates' own description of it at 29d5 as \emph{prooimion} or prelude. Entitling the present chapter ``\emph{Proemium}'' thus has a double significance. It at once announces that the present chapter will take Timaeus' \emph{proemium} as its subject matter and that the chapter will itself serve as prelude to the essay as a whole.

% section a_double_significance (end)


\section{\emph{Proemium}} % (fold)
\label{sec:proemium}

The \emph{proemium} is bounded by the last two remarks that Socrates will make in the dialogue. In the first, Socrates praises the feast of \emph{logoi} that has been proposed as a return gift for his speech the previous day on the topic of the \emph{Republic} and prompts Timaeus to begin by duly invoking the gods. In the second, Socrates praises Timaeus, accepting his prelude, and encouraging him to proceed. In the \emph{proemium}, Timaeus begins by invoking the gods. After which Timaeus states some general principles, which he uses to circumscribe his subject matter and describe the kind of account that he proposes to give of it. 

The principles include:
\begin{enumerate}[(1)]
	\item The distinction between that which always is and never becomes and that which always becomes and never is and the manner in which they are cognized
	\item The principle that everything that has come to be and never is has a cause
	\item The principle that the cause of that which has come to be and never is a maker that looks to a model
	\item The principle that if that which has come to be and never is is beautiful then it has been modeled on that which always is and never becomes by a good maker
\end{enumerate}

That which always is and never becomes is the object of understanding (\emph{noēsis}) based on reasoning (\emph{logos}). That which becomes and never is is the object of opinion (\emph{doxa}) based on unreasoning perception (\emph{aisthēsis}). That which always is and never becomes is intelligible, and that which becomes and never is sensible. The Cosmos is sensible and thus must belong to the realm of Becoming. That which becomes and never is has a cause. This causes is a maker that looks to a model. Since the Cosmos is beautiful then by the final principle, Timaeus concludes that the Cosmos has an intelligible model and a good maker. It is the generation of the Cosmos, conceived as a beautiful likeness of an intelligible model, that will be the subject matter of Timaeus' speech. Accounts should be appropriate to their subject matter, and Timaeus concludes that since his subject matter is a likeness his account shall be ``likely''.

While the general outline of the \emph{proemium} is relatively clear, each element of this tightly constructed account requires comment as does their interrelation. 

% section proemium (end)

\section{The Invocation of the Gods} % (fold)
\label{sec:the_invocation_of_the_gods}

Without dwelling on Timaean theology, two observations are worth making about Timaeus' invocation of the Gods. The first is an observation about who is not invoked, and the second is an observation about who is invoked.

It is reasonable to call upon God at the outset of an undertaking. This is presented as a general practical precept. Later Timaeus explains that we invoke the Gods so as to pray for their approval of our undertaking. It would be unreasonable to fail to do so. And this remains so whether the undertaking is great or small. Timaues, however, proposes to undertake an account of the All (\emph{toū pantos}---understood, here, as the Cosmos without the explicit suggestion that it is ordered) and whether and how it was generated. This is a grand topic and a great undertaking so it is especially appropriate that Timaeus call upon the Gods at the outset. 

It is striking who is not called upon by Timaeus. Notable in His absence is the Demiurge, the Maker and Father of the All. The Maker and Father of the All is difficult to discover and impossible to explain to all (28c3–5). He only begins to emerge in reasoning from principles that Timaeus only states after the invocation. Timaeus is invoking the Gods, in part, in aid of this difficult discovery. If so, then the object of his search was not among the Gods that were invoked even it is itself divine.

There is a related reason for thinking that the Demiurge was not among the Gods invoked and prayed to. Even if it were possible to privately invoke the Gods and pray for their approval of some solitary undertaking, this is not the context that Timaeus finds himself in. The invocation of the Gods urged by Socrates was, importantly, a public act of piety. And Timaues complies in the first person plural. It is we who are about to discourse that have need to invoke the Gods and the Goddesses. A divinity difficult to discover and impossible to explain to all is perhaps not the most apt choice to invoke in a public act of piety. One's piety toward the difficult to discover divinity may be genuine but it might not be publicly manifest. One's piety is only guaranteed to be publicly manifest if directed toward recognized divinities. And a difficult to discover divinity impossible to explain to all is not a recognized divinity. The obscurity of the Maker and Father of the All is inconsistent with the public nature of the invocation.

Perhaps, in the present instance, the public is restricted to the divided reference of ``we''---Timaeus, Socrates, Critias, and Hermocrates. The restriction may be significant. A divinity not widely known to many may yet be known to the present few. However, while Timaeus might rely on Socrates to already know of this difficult to discover divinity, is it really reasonable that Timaeus should similarly rely on Critias? Even among the present dignitaries, the obscurity of the divinity is in tension with the publicity of the act.

As \citet[]{Broadie:2012vl} observes, the Demiurge is not invoked in any of the prayers offered by Timaeus. Broadie's observation coheres with the present explanation. Since all of the prayers in the dialogue are public acts, the obscurity of the divinity would preclude His public invocation.

Not only is it notable that Timaeus does not invoke the Demiurge, but it is obscure who is being invoked. Specifically, is Timaeus invoking one God or many? Notice he first calls upon God in the singular and then upon Gods and Goddesses in the plural. When Timaeus first calls upon God in the singular, it is in stating the general practical precept to call upon God and pray for approval of one's undertaking. When Timaeus calls upon Gods and Goddesses in the plural, this is not in the context of stating a general practical precept, but rather in the context of proposing to undertake an account of the All and its generation. Perhaps it is the Young Gods who are being invoked when Timaeus invokes the many. The Young Gods are immortal beings generated by the Demiurge and are themselves the instruments of Demiurgic activity. Specifically, as we shall see, they play an indispensable role in the generation of mortal beings. If the Young Gods are the many, who could the one be if not the Demiurge? One candidate is the All itself. Timaeus conceives of the Cosmos as a visible God. 

There are, of course, alternatives. Perhaps the God in the singular being invoked is a generic divinity of which the Gods and the Goddesses in the plural are species. On this alternative, an appeal to a generic divinity is appropriate in the statement of the content of a general practical percept. Similarly, it is appropriate to invoke the species of this generic divinity given the specific undertaking to account for the All. That all species of divinity should be invoked and their approval prayed for is perhaps only required for their specific undertaking of discoursing on the All and its generation. Perhaps a single divinity would do if the undertaking were small enough. But the collective undertaking that Timaeus proposes is suitably grand as to require the invocation of every species of divinity.

A deflationary take on the identity of the Gods invoked might might be motivated by the perfunctory nature of the invocation. Proclus describes the worry as follows:
\begin{quote}
	But how can it be, they say, that Timaeus has announced with a grand flourish that one should pray and call on the gods and goddesses, but that he fails to do this himself and immediately turns to proposed accounts without praying? (Proclus, \emph{In Timaeum} 1 221.9--12, \citealt{Diehl:1903re}; \citealt[58]{Runia:2008aa})
\end{quote}
If Timaeus has not in fact prayed, then there is no point in inquiring into who Timaeus has prayed to. Proclus' reply is ingenious in that he, in effect, proposes that we understand Timaeus' speech as performative. In saying that we must pray to the Gods, Timaeus has done just that. While one may decide to pray and pray later, in true prayer, the decision to pray and prayer are one and the same thing. Timaeus's public prayer on the occasion of the Panathenaea is a true prayer. And so in saying that we must pray, he does.

Still, exactly who is being invoked by Timaues may not be so neatly resolvable. Perhaps there is a kind of semantic slippage here, from singular to plural, the one resolving into the many. If so, this will be echoed in the third part of Timaeus' speech which concerns the interaction of Reason and Nececssity, especially in his account of the generation of the flesh and the mortal soul. There, Timeaus slides freely between singular and plural in describing what is officially meant to be rational cooperative activity, the Demiurge generating the immortal part of the soul and the Young Gods weaving it together with its mortal body that they generated from corporeal material borrowed from the body of the Cosmos. However, even for tasks that are meant for the Young Gods, their executions are sometimes described in the singular. We shall discuss the significance of this when we discuss the flesh and the mortal soul (chapter~\ref{cha:the_flesh_and_the_mortal_soul}). For now, I merely note a potential anticipatory trace of this puzzling usage. 

One last observation. Not only are the Gods invoked, but Timaeus also invokes themselves. The plural self-invocation is for the audience---Socrates, Critias, and Herm\-ocrates---to more easily learn and for the speaker---Timaeus---to more clearly expound his account of the All and the cause of its generation. (On the self-invocation see Proclus, \emph{In Timaeum} 1 222.11--223.2, \citealt{Diehl:1903re}.) Not only is Timaeus praying for the approval of the Gods concerning their present collective undertaking---his discoursing with them the nature and generation of the All---but he is importantly also praying that they themselves approve of their discourse, if only after being granted divine approval. While for the most part, Timaeus delivers an uninterrupted speech, in the invocation he describes it as a collective activity with the audience as co-participants. Perhaps the self-invocation is meant to get the participants of the discourse in a position where they may engage in it in such a way that they come to themselves approve of that discourse. This would involve the audience rationally attending to the speaker's account so that they may more easily learn his views about the nature and generation of the All. And this would involve the speaker clearly expounding his views to this audience. (Notice that it is his views that Timaeus wishes them to learn. Contrast learning the truth about the All and its generation. Timaeus is pointedly not proposing to demonstrate such truths. Timaeus, here, is quietly allowing room for what will emerge as accounts with a specific epistemic status, ``likely'' accounts.) Even if largely a monologue, Timaeues' speech remains directed to its audience. And Timaeus must bear them in mind if he is to clearly expound his views to them in such a way as warrant their collective approval. Only in this way could the return speech discharge Timaeus' obligation to Socrates. 

If the plural self-invocation still seems odd or unfamiliar, it really should not. You have doubtless encountered it in contemporary speech. Think of someone inaugurating their address with ``Let's do this!'' Think of a motivational speaker or, better yet, a Hip Hop artist. ``This'', that which is to be done (dropping bars for the Hip Hop artist, cosmology for Timaeus, both are \emph{logoi}), is explicitly represented as a collective activity---``Let \emph{us} do this'', where ``us'' includes both speaker and audience. In saying ``Let's do this!'', the speaker draws attention to themselves, making their audience more receptive to their address, and the speaker expresses their determination in a way that draws the speaker's own attention to their audience, in an attempt to thereby more effectively deliver their address. Should the speaker prove genuinely effective, they may even garner approval for that address. It even has, in line with Proclus' suggestion, a performative aspect. In saying ``Let's do this!'', that which is to be done has already begun.

% section the_invocation_of_the_gods (end)

\section{Being and Becoming} % (fold)
\label{sec:Being and Becoming}



% section Being and Becoming (end)

\section{Maker and Father} % (fold)
\label{sec:maker_and_father}



% section maker_and_father (end)

\section{Model and Image} % (fold)
\label{sec:model_and_image}



% section model_and_image (end)

\section{Likely Account} % (fold)
\label{sec:likely_account}



% section likely_account (end)

\section{Concluding Observations} % (fold)
\label{sec:concluding_observations}



% section concluding_observations (end)

% Chapter proemium (end) 