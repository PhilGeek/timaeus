%!TEX root = /Users/markelikalderon/Documents/Git/timaeus/timaeus.tex

\chapter{\emph{Proemium}} % (fold)
\label{cha:proemium}

\section{A Double Significance} % (fold)
\label{sec:a_double_significance}

The first part of Timaeus's speech is traditionally called the \emph{proemium}, echoing Socrates' own description of it at 29d5 as \emph{prooimion} or prelude. Entitling the present chapter ``\emph{Proemium}'' thus has a double significance. It at once announces that the present chapter will take Timaeus' \emph{proemium} as its subject matter and that the chapter will itself serve as prelude to the essay as a whole.

% section a_double_significance (end)


\section{\emph{Proemium}} % (fold)
\label{sec:proemium}

The \emph{proemium} is bounded by the last two remarks that Socrates will make in the dialogue. In the first, Socrates praises the feast of \emph{logoi} that has been proposed as a return gift for his speech the previous day on the topic of the \emph{Republic} and prompts Timaeus to begin by duly invoking the gods. In the second, Socrates praises Timaeus, accepting his prelude, and encouraging him to proceed. In the \emph{proemium}, Timaeus begins by invoking the gods. After which Timaeus states some general principles, which he uses to circumscribe his subject matter and describe the kind of account that he proposes to give of it. 

The principles include:
\begin{enumerate}[(1)]
	\item The distinction between that which always is and never becomes and that which always becomes and never is and the manner in which they are cognized
	\item The principle that everything that has come to be and never is has a cause
	\item The principle that the cause of that which has come to be and never is a maker that looks to a model
	\item The principle that if that which has come to be and never is is beautiful then it has been modeled on that which always is and never becomes by a good maker
\end{enumerate}

That which always is and never becomes is the object of understanding (\emph{noēsis}) based on reasoning (\emph{logos}). That which becomes and never is is the object of opinion (\emph{doxa}) based on unreasoning perception (\emph{aisthēsis}). That which always is and never becomes is intelligible, and that which becomes and never is sensible. The Cosmos is sensible and thus must belong to the realm of Becoming. That which becomes and never is has a cause. This causes is a maker that looks to a model. Since the Cosmos is beautiful then by the final principle, Timaeus concludes that the Cosmos has an intelligible model and a good maker. It is the generation of the Cosmos, conceived as a beautiful likeness of an intelligible model, that will be the subject matter of Timaeus' speech. Accounts should be appropriate to their subject matter, and Timaeus concludes that since his subject matter is a likeness his account shall be ``likely''.

While the general outline of the \emph{proemium} is relatively clear, each element of this tightly constructed account requires comment as does their interrelation. 

% section proemium (end)

\section{The Invocation of the Gods} % (fold)
\label{sec:the_invocation_of_the_gods}



% section the_invocation_of_the_gods (end)

\section{Being and Becoming} % (fold)
\label{sec:Being and Becoming}



% section Being and Becoming (end)

\section{Maker and Father} % (fold)
\label{sec:maker_and_father}



% section maker_and_father (end)

\section{Model and Image} % (fold)
\label{sec:model_and_image}



% section model_and_image (end)

\section{Likely Account} % (fold)
\label{sec:likely_account}



% section likely_account (end)

\section{Concluding Observations} % (fold)
\label{sec:concluding_observations}



% section concluding_observations (end)

% Chapter proemium (end) 