%!TEX root = /Users/markelikalderon/Documents/Git/timaeus/timaeus.tex

\chapter{\emph{Pathemata}} % (fold)
\label{cha:pathemata}

\section{\emph{Pathemata}} % (fold)
\label{sec:pathemata}

\emph{Pathemata} or \emph{pathe} (Plato, in the \emph{Timaeus} at least, uses these terms interchangeably) play an important role in the process eventuating in \emph{aisthēsis}. ``\emph{Pathemata}'' is perhaps most naturally translated in English as affections. In a not quite archaic usage, affections are conditions or characteristics of a thing brought about by its being affected in some way. But the \emph{pathemata} that play a role in the causal process eventuating in perception or sensation on Timaeus' account are a specification of the more generic notion provided by the natural English translation. It is surprisingly controversial, however, what, exactly, ``\emph{pathemata}'' means in this context. Thus \citet[429-31]{Taylor:1928qb} understands the \emph{pathemata}, in this context, as sensible qualities that a thing may have independently of the sensibility of a percipient being affected in any way. They are ``characters of the various bodies themselves'' as opposed to ``the effects produced by the bodies on a percipient''. \citet[258-9]{Cornford:1935fk}, by contrast, wants to assimilate Timaeus' account of perception to the account given in the \emph{Theaetetus}. As a result, he contrasts the \emph{pathemata} with those ``properties which bodies are supposed to possess in the absence of any sentient being, such as the shapes of the microscopic particles, which are never perceived'' \citep[259]{Cornford:1935fk}; for, according to the Secret Doctrine that Socrates attributes to Protagoras in the \emph{Theaetetus} 156a-b, a sensible quality and a perceiver's perception of it are ``twin births'' with the consequence that nothing has a sensible quality without a perception of it. Since it is controversial how to understand \emph{pathemata} in this context, it is worth discussing this explicitly before discussing Timaeus' specific examples of \emph{pathemata} that eventuate in perception or sensation.

Timaeus distinguishes two classes of \emph{pathemata}. There are:
\begin{enumerate}
	\item \emph{pathemata} that are common to the body as a whole (61d--65b)
	\item \emph{pathemata} that are peculiar to particular parts of the body (65b--68e)
\end{enumerate}
The body, here, is the human body or, perhaps, the body of a sentient being more generally. This is inconsistent with Taylor's \citeyearpar[431]{Taylor:1928qb} denial that the \emph{pathemata} are ``the effects produced by the bodies on a percipient''. In describing the common \emph{pathemata}, Timaeus' explicit task is to describe the affections of the body as a whole and the names of the agents that produce these affections (65b). Timaeus undertakes a similar task with the peculiar \emph{pathemata}, that is, he is meant to describe the affections of particular parts of the body and the names of the agents that produce these affections (65b6-c1). Since the agents that produce these affections, whether common or peculiar, are the objects of perception, this is inconsistent with Taylor and Cornford's common insistence that the \emph{pathemata} are sensible qualities. And this remains so whether or not sensible qualities are best conceived, in Timaeus' account, as characteristics of bodies had independently of their effects on perceivers or as twinned with the appropriate perception. 

Common \emph{pathemata}, affections of the body as a whole, are involved in the perception or sensation of:
\begin{enumerate}
 	\item hot and cold (61d5--62a5)
 	\item hard and soft (62b6--c3)
 	\item heavy and light (62c3--63e8)
 	\item smooth and rough (63e8--64a1)
 	\item pleasure and pain (64a2--65b3)
\end{enumerate}

Peculiar \emph{pathemata} are not affection of the body as a whole, but are rather affections of particular parts of the body that are liable to give rise to perception or sensation. Timaeus describes four such parts:
\begin{enumerate}
	\item the tongue (65b--66c)
	\item the nostrils (66c--67a)
	\item the ears (or perhaps the brain and the blood) (46c--47e, 67a--c, 80a)
	\item the eyes (45b--46c, 67c--68d)
\end{enumerate}
	
\emph{Pathemata} peculiar to the tongue are involved in the perception or sensation of:
\begin{enumerate}
	\item astringent and harsh (65c6--d4)
	\item acrid and agreeable (65d4--65e4)
	\item pungent (65e4--66a2)
	\item acid (66a2--b7)
	\item sweet (66b7--c7)
\end{enumerate}
	

\emph{Pathemata} peculiar to the nostrils are involved in the perception or sensation of:
\begin{enumerate}
	\item odours of water transforming into air or air into water (66c-67a)
	\item pleasant and unpleasant odours (66d1--67a6)
\end{enumerate}
	
\emph{Pathemata} peculiar to the ears (or perhaps the brain and the blood) are involved in the perception of:
\begin{enumerate}
	\item high and low (67b6)
	\item smooth and harsh (67b6-7)
	\item loud and soft (67c1)
\end{enumerate}
	
Finally, \emph{pathemata} peculiar to the eyes are involved in the perception of:
\begin{enumerate}
	\item colors (Timaeus 45b--46c, 67c--68d)
\end{enumerate}
	
If common \emph{pathemata} are the \emph{pathemata} involved in tactile perception and sensation, then the peculiar \emph{pathemata} are the \emph{pathemata} involved in the remaining four special senses: taste, smell, hearing, and vision. Here I am helping myself, in full knowledge of its anachronism, to the vocabulary of Aristotle in \emph{De Anima} and \emph{De Sensu}. Undoubtedly the \emph{Timaeus} account of perception was an influence on Aristotle's own account in these works. However, for now at least, I want to bracket any suggestion that talk of the special senses may carry that the particular parts of the body affected are sense organs. Whether or not the particular parts of the body affected, as Timaeus conceives of them, are properly deemed sense organs shall be assessed as we proceed.

\emph{Pathemata} are not the objects of perception or sensation as Taylor and Cornford maintain but, rather, they are a causal intermediary between these objects and the perceptions or sensations that they are liable to give rise to (\citealt[138]{OBrien:1984ji}; \citealt{Brisson:1997qr}). The named agents of these affections, explicitly contrasted with the \emph{pathemata} themselves, are the objects of perception. Moreover, perceptions are themselves explicitly contrasted with the \emph{pathemata}. \emph{Pathemata}, understood as affections of a sentient body, while necessary for the production of perception or sensation, are not sufficient. And if \emph{pathemata} may obtain without perception, then perceptions are distinct from \emph{pathemata}. Let's consider these claims in turn.

First, Timaeus contrasts the agents that cause the \emph{pathemata} with the \emph{pathemata} themselves. Moreover, these agents are the objects of perception or sensation (with the possible exception of pleasure and pain). Touch, for example, affords us with thermal perception. We can feel that a body is hot or cold by touching it. When our flesh is in contact with a hot or cold body it is affected in a certain way, and this affection is liable to give rise to thermal perception. When we touch something hot, for example, we don't merely feel the consequent warmth of our flesh, we feel the warmth of the body that we are in contact with. And both the affection, the \emph{pathema}, and the agent that causes it are both called by the same name, ``hot'' or ``cold'' as the case may be (61e--62b6). This is true not only for the common \emph{pathemata} (with the possible exception of pleasure and pain), but it is true, as well, of the peculiar \emph{pathemata}. Timaeus makes this general claim explicit. \emph{Aisthēsis} occurs when the power of the agent that caused the \emph{pathos} is reported to the \emph{phronimon}, the seat of intelligence or consciousness (65b4). Note well, what is reported is not the affection of the body but the power of the agent that caused that affection. And if that report determines the content of the perception then what is perceived is not the \emph{pathos}, but its external cause. But if the \emph{pathemata} are contrasted with the agents that cause them and these latter are, on the whole at least, the objects of perception and sensation, then the \emph{pathemata} are not sensible qualities as Taylor and Cornford maintain.

Second, \emph{pathemata} while necessary for the production of \emph{aisthēsis} are not sufficient. \emph{Pathemata} may obtain without eliciting perception or sensation. When an affection is violent, intense, and contrary to the natural state of the body, it is painful. However, when an affection is mild and gradual, it is insensible (64d4). And earlier (64a--d) Timaeus tells us that when the \emph{pathos} falls upon the body, whether or not it produces \emph{aisthēsis} depends upon the nature of the body receiving that \emph{pathos}. Parts of the body may be more or less mobile, depending upon the character of the particles that compose them. If the affected part of the body is mobile it passes the affection around to other mobile parts until it reaches the \emph{phronimon}, the seat of intelligence or consciousness, where it reports the power of the agent that caused the affection. Whereas if the part of the body upon which the \emph{pathos} falls is immobile, as in the case of hair and bones, though it is affected, it does not pass this affection around with the consequence that the power of the agent that caused the affection is not reported to the \emph{phronimon}. So \emph{pathemata} may obtain without eventuating in \emph{aisthēsis}. Thus, \emph{pathemata} while necessary for the production of \emph{aisthēsis} are not sufficient. If that is right, then the \emph{pathemata} are not twinned with a perception in the way that Cornford maintains.

Timaeus inaugurates the discussion of the \emph{pathemata} with a methodological dilemma (61c4--d5). Timaeus distinguishes between the \emph{pathemata}, on the one hand, and the origin of the flesh and things pertaining to it and the mortal parts of the soul, on the other. The problem is twofold. First, one cannot adequately account for the \emph{pathemata} without accounting for the origin of the flesh and things pertaining to it and the mortal part of the soul. But second, one cannot account for the origin of the flesh and things pertaining to it and the mortal parts of the soul without accounting for the \emph{pathemata}. The problem is not one of mutual entailment or that these subject matters somehow presuppose one another directly. The problem is rather that an account of either subject matter each presupposes \emph{aisthēsis} with the consequence that an account of each must make reference to the other. Hence the methodological dilemma: Timaeus cannot account for either without reference to the other, but it is impossible to give an account of both subjects at once. Timaeus solution is to first account for the \emph{pathemata} (61d--68d) and then subsequently account for the origin of the flesh and things pertaining to it (73b1ff) and the mortal parts of the soul (69c5ff). 

Two questions arise, only the first of which we are in position to answer. First how exactly and in what sense does an account of the \emph{pathemata} presuppose \emph{aisthēsis}? Second how exactly and in what sense does an account of the flesh and the mortal soul presuppose \emph{aisthēsis}? Consider, then, the first question. \emph{Pathemata} are affections to be sure. And some at least of the common \emph{pathemata} may obtain not only in animate bodies capable of sentience but inanimate bodies as well. Thus in Timaeus' discussion of hard and soft (62b6--c3) we are told that we call ``hard'' that which the flesh yields to and ``soft'' what yields to the flesh and then immediately claims that these terms also apply more generally to bodies in relation to one another (see O'Brien 1984, 110). Thus, for example, a body composed of earth, such as a stone, is harder than water since water yields to earth. So yielding is the \emph{pathema} and not only does the flesh yield to a stone, say, but do does water. Still, the \emph{pathemata}, whether common or peculiar, that Timaeus discusses are \emph{pathemata} of sentient animate bodies. The relevant \emph{pathemata} are \emph{pathemata} that when they fall upon a sentient body are liable to produce \emph{aisthēsis} (should they be intense enough and fall upon a part of the body composed of sufficiently mobile particles). 

% section pathemata (end)

\section{Naming} % (fold)
\label{sec:naming}

Timaeus, in accounting for the \emph{pathemata}, seeks to describe the agents that cause them and to explain how these are named. This linguistic interest is picking up from an earlier discussion of of the impossibility of naming in the pre-cosmic chaos and is undoubtedly linked to Plato's discussion of how the doctrine of total flux, that everything changes in every respect at all times, is incompatible with the practice of naming in the \emph{Cratylus} and the \emph{Theaetetus}.

% section naming (end)

\section{Common \emph{Pathemata}} % (fold)
\label{sec:common_emph_pathemata}

\subsection{Hot and Cold} % (fold)
\label{sub:hot_and_cold}



% subsection hot_and_cold (end)

\subsection{Hard and Soft} % (fold)
\label{sub:hard_and_soft}



% subsection hard_and_soft (end)

\subsection{Heavy and Light} % (fold)
\label{sub:heavy_and_light}



% subsection heavy_and_light (end)

\subsection{Smooth and Rough} % (fold)
\label{sub:smooth_and_rough}



% subsection smooth_and_rough (end)

\subsection{Pleasure and Pain} % (fold)
\label{sub:pleasure_and_pain}



% subsection pleasure_and_pain (end)

% section common_emph_pathemata (end)


\section{Peculiar \emph{Pathemata}} % (fold)
\label{sec:peculiar_emph_pathemata}

\subsection{The Tongue} % (fold)
\label{sub:the_tongue}



% subsection the_tongue (end)

\subsection{The Nostrils} % (fold)
\label{sub:the_nose}



% subsection the_nose (end)

\subsection{The Ears} % (fold)
\label{sub:the_ears}



% subsection the_ears (end)

\subsection{The Eyes} % (fold)
\label{sub:the_eyes}



% subsection the_eyes (end)

% section peculiar_emph_pathemata (end)

% Chapter pathemata (end) 