%!TEX root = /Users/markelikalderon/Documents/Git/timaeus/timaeus.tex

\chapter{\emph{Pathēmata}} % (fold)
\label{cha:pathemata}

\section{\emph{Pathēmata}} % (fold)
\label{sec:pathemata}

\emph{Pathēmata} or \emph{pathē} (Plato, in the \emph{Timaeus} at least, uses these terms interchangeably) play an important role in the process eventuating in \emph{aisthēsis}. ``\emph{Pathēmata}'' is perhaps most naturally translated in English as affections. In a not quite archaic usage, affections are conditions or characteristics of a thing brought about by its being affected in some way. But the \emph{pathēmata} that play a role in the causal process eventuating in perception or sensation in Timaeus' account are a specification of the more generic conception provided by the natural English translation. It is surprisingly controversial, however, what, exactly, ``\emph{pathēmata}'' means in this context. Thus \citet[429-31]{Taylor:1928qb} understands the \emph{pathēmata}, in this context, as sensible qualities that a thing may have independently of the sensibility of a percipient being affected in any way. They are ``characters of the various bodies themselves'' as opposed to ``the effects produced by the bodies on a percipient''. \citet[258-9]{Cornford:1935fk}, by contrast, wants to assimilate Timaeus' account of perception to the account given in the \emph{Theaetetus}. As a result, he contrasts the \emph{pathēmata} with those ``properties which bodies are supposed to possess in the absence of any sentient being, such as the shapes of the microscopic particles, which are never perceived'' \citep[259]{Cornford:1935fk}, for, according to the Secret Doctrine that Socrates attributes to Protagoras in the \emph{Theaetetus} (156a-b), a sensible quality and a perceiver's perception of it are ``twin births'' with the consequence that nothing has a sensible quality without a perception of it. Since it is controversial how to understand \emph{pathēmata} in this context, it is worth discussing this explicitly before discussing Timaeus' specific examples of \emph{pathēmata} that eventuate in perception or sensation.

Timaeus distinguishes two classes of \emph{pathēmata}. There are:
\begin{enumerate}
	\item \emph{pathēmata} that are common to the body as a whole (61d--65b)
	\item \emph{pathēmata} that are peculiar to particular parts of the body (65b--68e)
\end{enumerate}
The body, here, is the human body or, perhaps, the body of a sentient being more generally. Thus \citet[431]{Taylor:1928qb} is wrong to deny that the \emph{pathēmata} are ``the effects produced by the bodies on a percipient''. In describing the common \emph{pathēmata}, Timaeus' explicit task is to describe the affections of the body as a whole and the names of the agents that produce these affections (65b). Timaeus undertakes a similar task with the peculiar \emph{pathēmata}, that is, he is meant to describe the affections of particular parts of the body and the names of the agents that produce these affections (65b6-c1). Since the agents that produce these affections, whether common or peculiar, are the objects of perception, this is inconsistent with Taylor and Cornford's common insistence that the \emph{pathēmata} are sensible qualities \citep[see][225, n8]{Archer-Hind:1888qd}. And this remains so whether or not sensible qualities are best conceived, in Timaeus' account, as characteristics of bodies had independently of their effects on perceivers or as twinned with the appropriate perception. 

Common \emph{pathēmata}, affections of the body as a whole (discussed in section \ref{sec:common_emph_pathemata}), are involved in the perception or sensation of:
\begin{enumerate}
 	\item hot and cold (61d5--62a5) (section \ref{sub:hot_and_cold})
 	\item hard and soft (62b6--c3) (section \ref{sub:hard_and_soft})
 	\item heavy and light (62c3--63e8) (section \ref{sub:heavy_and_light})
 	\item smooth and rough (63e8--64a1) (section \ref{sub:smooth_and_rough})
 	\item pleasure and pain (64a2--65b3) (section \ref{sub:pleasure_and_pain})
\end{enumerate}

Peculiar \emph{pathēmata} are not affection of the body as a whole, but are rather affections of particular parts of the body that are liable to give rise to perception or sensation (discussed in section \ref{sec:peculiar_emph_pathemata}). Timaeus describes four such parts:
\begin{enumerate}
	\item the tongue (65b--66c) (section \ref{sub:the_tongue})
	\item the nostrils (66c--67a) (section \ref{sub:the_nostrils})
	\item the ears (or perhaps the brain and the blood) (46c--47e, 67a--c, 80a) (section \ref{sub:the_ears})
	\item the eyes (45b--46c, 67c--68d) (section \ref{sub:the_eyes})
\end{enumerate}

\emph{Pathēmata} peculiar to the tongue are involved in the perception or sensation of:
\begin{enumerate}
	\item astringent and harsh (65c6--d4)
	\item acrid and agreeable (65d4--65e4)
	\item pungent (65e4--66a2)
	\item acid (66a2--b7)
	\item sweet (66b7--c7)
\end{enumerate}

\emph{Pathēmata} peculiar to the nostrils are involved in the perception or sensation of:
\begin{enumerate}
	\item odours of water transforming into air or air into water (66c-67a)
	\item pleasant and unpleasant odours (66d1--67a6)
\end{enumerate}

\emph{Pathēmata} peculiar to the ears (or perhaps the brain and the blood) are involved in the perception of:
\begin{enumerate}
	\item high and low (67b6)
	\item smooth and harsh (67b6-7)
	\item loud and soft (67c1)
\end{enumerate}

Finally, \emph{pathēmata} peculiar to the eyes are involved in the perception of:
\begin{enumerate}
	\item colors (45b--46c, 67c--68d)
\end{enumerate}

If common \emph{pathēmata} are the \emph{pathēmata} involved in tactile perception and sensation, then the peculiar \emph{pathēmata} are the \emph{pathēmata} involved in the remaining four special senses: taste, smell, hearing, and vision. Here I am helping myself, in full knowledge of its anachronism, to the vocabulary of Aristotle in \emph{De Anima} and \emph{De Sensu}. Undoubtedly the \emph{Timaeus} account of perception was an influence on Aristotle's own account in these works. However, for now at least, I want to bracket any suggestion that talk of the special senses may carry that the particular parts of the body affected are sense organs. Whether or not the particular parts of the body affected, as Timaeus conceives of them, are properly deemed sense organs shall be assessed as we proceed.

\emph{Pathēmata} are not the objects of perception as Taylor and Cornford maintain but, rather, they are a causal intermediary between the objects of perception and the perceptions or sensations that they are liable to give rise to (\citealt[138]{OBrien:1984ji}; \citealt{Brisson:1997qr}). The named agents of these affections, explicitly contrasted with the \emph{pathēmata} themselves, are the objects of perception. Moreover, perceptions are themselves explicitly contrasted with the \emph{pathēmata}. \emph{Pathēmata}, understood as affections of a sentient body, while necessary for the production of perception or sensation, are not sufficient. And if \emph{pathēmata} may obtain without perception, then perceptions are distinct from \emph{pathēmata}. Let's consider these claims in turn.

First, Timaeus contrasts the agents that cause the \emph{pathēmata} with the \emph{pathēmata} themselves. Moreover, these agents are the objects of perception or sensation (as we shall see in section~\ref{sub:pleasure_and_pain}, pleasure and pain are an exceptional case). Touch, for example, affords us with thermal perception. We can feel that a body is hot or cold by touching it. When our flesh is in contact with a hot or cold body it is affected in a certain way, and this affection is liable to give rise to thermal perception. When we touch something hot, for example, we don't merely feel the consequent warmth of our flesh, we feel the warmth of the body that we are in contact with. And both the affection, the \emph{pathēma}, and the agent that causes it are called by the same name, ``hot'' or ``cold'' as the case may be (61e--62b6). This is true not only for the common \emph{pathēmata} (with the possible exception of pleasure and pain), but it is true, as well, of the peculiar \emph{pathēmata}. Timaeus makes this general claim explicit. \emph{Aisthēsis} occurs when the power of the agent that caused the \emph{pathos} is reported to the \emph{phronimon}, the seat of intelligence or consciousness (65b4). Note well, what is reported is not the affection of the body but the power of the agent that caused that affection. And if that report determines the content of the perception then what is perceived is not the \emph{pathos}, but its external cause. But if the \emph{pathēmata} are contrasted with the agents that cause them and these latter are, on the whole at least, the objects of perception and sensation, then the \emph{pathēmata} are not sensible qualities as Taylor and Cornford maintain.

Second, \emph{pathēmata} while necessary for the production of \emph{aisthēsis} are not sufficient. \emph{Pathēmata} may obtain without eliciting perception or sensation. When an affection is violent, intense, and contrary to the natural state of the body, it is painful. However, when an affection is mild and gradual, it is insensible (64d4). An insensible affection is not the object of perception or sensation. And earlier (64a--d) Timaeus tells us that when the \emph{pathos} falls upon the body, whether or not it produces \emph{aisthēsis} depends upon the nature of the body receiving that \emph{pathos}. Parts of the body may be more or less mobile, depending upon the character of the particles that compose them. If the affected part of the body is mobile it passes the affection around to other mobile parts until it reaches the \emph{phronimon}, the seat of intelligence or consciousness, where it reports the power of the agent that caused the affection. Whereas if the part of the body upon which the \emph{pathos} falls is immobile, as in the case of hair and nails, though it is affected, it does not pass this affection around with the consequence that the power of the agent that caused that affection is not reported to the \emph{phronimon}. So \emph{pathēmata} may obtain without eventuating in \emph{aisthēsis} (see also \emph{Philebus} 33d). Thus, \emph{pathēmata} while necessary for the production of \emph{aisthēsis} are not sufficient. If that is right, then the \emph{pathēmata} are not twinned with a perception in the way that Cornford maintains.

Timaeus inaugurates the discussion of the \emph{pathēmata} with a methodological dilemma (61c4--d5). Timaeus distinguishes between the \emph{pathēmata}, on the one hand, and the origin of the flesh and things pertaining to it and the mortal parts of the soul, on the other. The problem is twofold. First, one cannot adequately account for the \emph{pathēmata} having to do with \emph{aisthēsis} without accounting for the origin of the flesh and things pertaining to it and the mortal part of the soul. But second, one cannot account for the origin of the flesh and things pertaining to it and the mortal parts of the soul without accounting for the \emph{pathēmata}. The problem is not one of mutual entailment or that these subject matters somehow presuppose one another directly. The problem is rather that an account of either subject matter each presupposes \emph{aisthēsis} with the consequence that an account of each must make reference to the other. Hence the methodological dilemma: Timaeus cannot account for either without reference to the other, but it is impossible to give an account of both subjects at once. Timaeus solution is to first account for the \emph{pathēmata} (61d--68d) and then subsequently account for the origin of the flesh and things pertaining to it (73b1ff) and the mortal parts of the soul (69c5ff). In part, Timaeus' decision, here, is dictated by the structure of his speech. Discussion of the \emph{pathēmata} more naturally follows on from his discussion of the \emph{genē} or \emph{eidē} than would discussion of the origin of the flesh and things pertaining to the flesh or discussion of the mortal parts of the soul. If the \emph{pathēmata} presuppose \emph{aisthēsis}, and if an account of \emph{aisthēsis} makes essential reference to the flesh and the mortal soul, then Timaeus in accounting for the \emph{pathēmata} will have to make assumptions about the flesh and mortal soul, assumptions that will only later receive a full account.

Two questions arise, only the first of which we are in position to answer. First how exactly and in what sense does an account of the \emph{pathēmata} presuppose \emph{aisthēsis}? Second how exactly and in what sense does an account of the flesh and the mortal soul presuppose \emph{aisthēsis}? Consider, then, the first question. \emph{Pathēmata} are affections to be sure. And some at least of the common \emph{pathēmata} may obtain not only in animate bodies capable of sentience but inanimate bodies as well. Thus in Timaeus' discussion of hard and soft (62b6--c3) we are told that we call ``hard'' that which the flesh yields to and ``soft'' what yields to the flesh and then immediately claims that these terms also apply more generally to bodies in relation to one another (see \citealt[228, n6]{Archer-Hind:1888qd} and \citealt[110]{OBrien:1984ji}). Thus, for example, a body composed of earth, such as a stone, is harder than water since water yields to earth. So yielding is the \emph{pathēma} and not only does the flesh yield to a stone, say, but do does water. Still, the \emph{pathēmata}, whether common or peculiar, that Timaeus discusses are \emph{pathēmata} of sentient animate bodies. The relevant \emph{pathēmata} are \emph{pathēmata} that when they fall upon a sentient body are liable to produce \emph{aisthēsis} (should they be intense enough and fall upon a part of the body composed of sufficiently mobile particles). The connection with \emph{aisthēsis}, then, is this: The \emph{pathēmata}, whether common or peculiar, are such that when they fall upon a sentient body, should the circumstances be propitious, they are liable to produce \emph{aisthēsis}.

% section pathemata (end)

\section{Naming} % (fold)
\label{sec:naming}

Timaeus, in accounting for the \emph{pathēmata}, seeks to describe the agents that cause them and to explain how these are named. Allow me to make some preliminary remarks about this second task before discussing Timaeus' specific examples of \emph{pathēmata} that eventuate in perception or sensation. This linguistic interest is picking up from an earlier discussion of of the impossibility of naming in the pre-cosmic chaos and is undoubtedly linked to Plato's discussion of how the doctrine of total flux, that everything changes in every respect at all times, is incompatible with the possibility of naming in the \emph{Cratylus} and the \emph{Theaetetus}.

% section naming (end)

\section{Common \emph{Pathēmata}} % (fold)
\label{sec:common_emph_pathemata}

\subsection{Hot and Cold} % (fold)
\label{sub:hot_and_cold}

Timaeus provides an account of how it is that we call fire ``hot'' (61d5--62a5). He begins with the way in which fire affects our body, by dividing or cutting. This much is evident from our experience. A hot flame, should we be in contact with it, gives rise to a sharp sensation. Notice that it is our body, the body of a sentient human being, whose affection is presently being discussed, and not an inanimate body, though fire will divide and cut wood as well, say. So the \emph{pathēma} relevant to what we call ``hot'' is the division of the sentient body and the sharp sensation associated with extreme heat is the \emph{aisthēsis} that the \emph{pathēma} is liable to give rise to. 

Timaeus explains the \emph{pathēma} of fire in terms of its geometrical and kinematic properties. When the tetrahedron or pyramid is assigned to fire Timaeus proclaims it to be the sharpest of the regular solids that constitute the \emph{genē} (56a5). A body is consumed when burnt, since when a body is surrounded by a large mass of tetrahedrons, it is cut up or divided owing to the sharpness of the angles and edges of the particles of fire (56e8--57a2). Notice that the \emph{pathēma} associated with the heat of fire, division, obtains not only when it falls upon an animate sentient body, but when it falls upon inanimate bodies altogether lacking in sentience. However, when explaining how it is that we call fire ``hot'', Timeaus mentions two further factors, the relative smallness of the tetrahedrons and their rapid motion. Thus four factors are relevant to fire dividing or cutting the body in a manner that is liable to give rise to sensation. Specifically, tetrahedrons of fire have the power to cut or divide the body owing to:
\begin{enumerate}
	\item the acuteness of its angles,
	\item the sharpness of its edges,
	\item the smallness of the tetrahedrons, and
	\item the rapidity of their motion.
\end{enumerate}
These geometrical and kinematic properties of the tetrahedrons that compose sensible fire explain its power to divide or cut. Moreover, we call both this power of fire and its affect on us ``hot''. So ``hot'' names not just a common \emph{pathēma}, but a power of the agent that produced this affection. 

Notice that these geometrical properties were providentially imposed upon the pre-cosmic chaos by the benevolent Demiurge. What about the relevant kinematic property, the swiftness of the fire particles? While the disorderly motion in the pre-cosmic chaos was due to the pre-Socratic principle that like attracts like, this motion is channelled, so to speak, by the Demiurge's providential imposition of ``shape and number''. Thus, the swiftness of the tetrahedrons of fire, for example, is due, as well, to the Demiurge making them small. Moreover, making the tetrahedrons of fire small is for the best and hence what Reason requires. In this way is the swiftness of fire particles due to Reason's persuasion of Necessity. Since the geometrical and kinematic properties explain the power of the agent to produce the relevant \emph{pathēma}, then we are only able to name this \emph{pathēma} and power thanks to the Demiurge providentially imposing ``shape and number'' upon the disorderly powers in the pre-cosmic chaos.

Implicit in these remarks is a double rebuke to pre-Socratic cosmologists. Not only is the principle that like attracts like insufficient for the production of a cosmos, but if it were the sole principle of motion, the sensible realm of Becoming would lack sufficient stability for its inconstant and inharmonious powers to be named.

It is phenomenologically important that ``hot'' names not just a common \emph{pathēma} but a power of the agent that produced this affection. Specifically, it provides Timaeus with the resources to distinguish thermal perception from sensation (a fact that is perhaps obscured by \emph{aisthēsis} applying to both). When we feel the heat of a fire, when its power to divide or cut our flesh is reported to the \emph{phronimon}, its the fire whose heat we feel. This is what makes it a case of thermal perception. Its object is exogenous. Contrast feeling hot, not because of the action of an external agent, but because we are suffering a fever, say. In this case, it is our own bodies' heat that we feel. This is less a case of thermal perception than bodily sensation (see \citealt{Yrjonsuuri:2008aa}). In this case, the heat that we feel is not exogenous but endogenous. While the object of bodily sensation is the affection of the body, the object of thermal perception is not the affection of the body but the power of the agent that produced it.  It is becase ``hot'' names not only the \emph{pathēma} of division but the power of the agent that produced this affection, and because this power is reported to the \emph{phronimon}, that we can suffer the ravaging heat of the fire in our experience of it.

If fire is called ``hot'' due to its power to divide, then things are called ``cold'' not due to division but compaction. Like the case of heat, the relevant \emph{pathēma} obtains not only when it falls upon an animate sentient body but when it falls upon inanimate bodies altogether lacking in sentience as well. For example, in his discussion of the \emph{genē}, Timaeus describes the cooling and solidification of melted metal (59a1--8) and the freezing of the fluid kind of water (59d4--e5) in terms of the operation of compaction. 

A natural fact central to the art of metallurgy is that metal melts when sufficiently heated. For example, casting would not be possible if metal did not melt. Another natural fact central to metallurgy is that melted metal solidifies as it cools. Only in this way could casting melted metal produce hard solid objects. Consider Timaeus' explanation of this latter fact, that cooling melted metal solidifies it. According to the first natural fact, metal melts when sufficiently heated. Thus metal, in its melted state, contains large quantities of fire particles. The melted metal cools as these fire particles escape. However, the fire particles, when escaping from the melted metal, do not move into a void but immediately meet the air. As a result the air is compacted. And the heated compacted air presses down on the metal forcing its particles to fill in the spaces vacated by the fire particles. Cornford speculates that another force is at work here, namely the reconstruction of uniformly sized octahedra from smaller octahedra. If he is right then this is another force at work in what is essentially a process of compaction. Melted metal sets when cool because cooling and solidification are a common process of compaction.

Compaction is also at work in the production of ice, snow, hail and frost. The fluid kind of water is subject to freezing when sufficiently cooled. In this case, fire and air particles escape making the water more uniform, presumably in the sense of being more consistently composed of water particles alone. Due to the force of the fire and air particles escaping the water particles close in on one another filling in the space vacated by the escaped particles. As a result of this process the mass of water particles become compacted. Water freezes when sufficiently cooled because, once again, cooling and solidification are a common process of compaction.

Like the setting of metal, and the freezing of water, the cooling of the body is a process of compaction. Like the setting of metal and freezing of water, the cooling of the human body involves the escaping of particles. And though Timaeus does not explicitly say that the particles that escape in the cooling of the human body are fire particles, he does describe them as small, and fire particles are small. If they are indeed fire particles, then like the setting of metal and freezing of water, the escaping particles at least involve fire particles. Like setting metal and freezing water, the human body becomes more uniform as it cools. And like setting metal and freezing, the more uniform mass, as spaces vacated by the escaping particles are filled in, becomes more compact and less mobile. So the \emph{pathēma}, the body being cold, is the bodies compaction.

Timaeus mentions, in addition, a related \emph{pathēma}. The trembling or shivering that results when the human body is cold. 

In \emph{De Sensibus}, Theophrastus objects as follows:
\begin{quote}
	First of all, he gives no uniform account of all <our sensory objects>, not even those that belong to the same class. For he describes heat in terms of figures, but he has not given a like account of cold. (Theophrastus, \emph{De Sensibus} 87; \citealt[147]{Stratton:1917vn})
\end{quote}

% subsection hot_and_cold (end)

\subsection{Hard and Soft} % (fold)
\label{sub:hard_and_soft}



% subsection hard_and_soft (end)

\subsection{Heavy and Light} % (fold)
\label{sub:heavy_and_light}



% subsection heavy_and_light (end)

\subsection{Smooth and Rough} % (fold)
\label{sub:smooth_and_rough}



% subsection smooth_and_rough (end)

\subsection{Pleasure and Pain} % (fold)
\label{sub:pleasure_and_pain}



% subsection pleasure_and_pain (end)

% section common_emph_pathemata (end)


\section{Peculiar \emph{Pathēmata}} % (fold)
\label{sec:peculiar_emph_pathemata}

\subsection{The Tongue} % (fold)
\label{sub:the_tongue}



% subsection the_tongue (end)

\subsection{The Nostrils} % (fold)
\label{sub:the_nostrils}



% subsection the_notrils (end)

\subsection{The Ears} % (fold)
\label{sub:the_ears}



% subsection the_ears (end)

\subsection{The Eyes} % (fold)
\label{sub:the_eyes}



% subsection the_eyes (end)

% section peculiar_emph_pathemata (end)

% Chapter pathemata (end) 