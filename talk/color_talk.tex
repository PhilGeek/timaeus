%!TEX TS-program = xelatex 
%!TEX TS-options = -synctex=1 -output-driver="xdvipdfmx -q -E"
%!TEX encoding = UTF-8 Unicode
%
%  timaeus_on_color_mixture
%
%  Created by Mark Eli Kalderon on 2018-12-10.
%  Copyright (c) 2016. All rights reserved.
%

\documentclass[12pt]{article} 

% Definitions
\newcommand\mykeywords{Priscian, perception}
\newcommand\myauthor{Mark Eli Kalderon}
\newcommand\mytitle{Timaeus on Color Mixture}

% Packages
\usepackage{geometry} \geometry{a4paper} 
\usepackage{url}
% \usepackage{txfonts}
\usepackage{color}
\usepackage{enumerate}
\definecolor{gray}{rgb}{0.459,0.438,0.471}
% \usepackage{setspace}
% \doublespace % Uncomment for doublespacing if necessary
% \usepackage{epigraph} % optional

% XeTeX
\usepackage[cm-default]{fontspec}
\usepackage{xltxtra,xunicode}
\defaultfontfeatures{Scale=MatchLowercase,Mapping=tex-text}
\setmainfont{Hoefler Text}
\newfontfamily{\sbl}{SBL BibLit}

% Bibliography
% \usepackage[natbibapa]{apacite}

% Title Information
\title{\mytitle}
\author{\myauthor}
\date{} % Leave blank for no date, comment out for most recent date

% PDF Stuff
\usepackage[plainpages=false, pdfpagelabels, bookmarksnumbered, pdftitle={\mytitle}, pdfauthor={\myauthor}, pdfkeywords={\mykeywords}, xetex, unicode=true]{hyperref} 
% colorlinks=true, citecolor=gray, linkcolor=gray, urlcolor=gray, backref, 

%%% BEGIN DOCUMENT
\begin{document}

% Title Page
\maketitle
\begin{abstract} % optional
	This talk consists in a trick and a potential insight. The trick consists in a minimalist interpretation of color mixture. The account of color mixture is minimalist in the sense that, given certain background assumptions, there is no more to Timaeus' account of color mixture than the list of the chromatic \emph{pathēmata} and the list of how these combine to elicit perceptions of all the colors. The only potential controversial elements of the minimalist interpretation are the relevant background assumptions and the interpretation of the chromatic \emph{pathēmata}. The potential insight concerns a motive that Plato, in the guise of Timaeus, may have for presenting an account of color mixture. Specifically, I shall argue that on the minimalist interpretation, Plato may be read as reconciling the Democritrean four color scheme with an older tradition where white and black are the fundamental chromatic opposition. As we shall see, this bears on the interpretation of the chromatic \emph{pathēmata}.
\end{abstract}
% \vskip 2em \hrule height 0.4pt \vskip 2em
% \epigraph{} % optional; make sure to uncomment \usepackage{epigraph}

% Layout Settings
\setlength{\parindent}{1em}

% Main Content

Perception, for Timaeus, is a kind of measurement. \emph{Pathēmata}---affections of the body as a whole or in part that are liable to produce perceptions or sensations---are the measure of the powers of the agents that produced them, and perception is the cognizance of what is measured, not the affections, but the powers that produced these affections. Sensible qualities, for Timaeus, then, are corporeal powers, powers to produce affections in sentient animate bodies. In the earlier discussion of vision (46c-47e), Timaeus says nothing specific about the powers of the agents that produce visual affections, nor does he say anything specific about the nature of these affections. With respect to the \emph{sunaitia}, the auxiliary causes of vision, Timaeus largely confines himself to the construction of the visual body, \emph{opsis}, that is the primary recipient of affection in the causal process of visual perception. Timaeus claims that the eye contains a fire within that is emitted through the pupil owing to its fineness. Timaeus claims, in addition, that this fire combines with daylight, akin to it, in the direction that the perceiver is looking, to form a body that may be acted upon as a homogenous unity. The present passage completes Timaeus' account of the \emph{suniaitia} of vision. For here we are told about the powers of the agents that cause visual affections, namely the colors, and we are told about how the visual body is affected by these powers.

Colors consist in a kind of flame, and since the visual body consists in the fire emitted from the eye combined with the fire that constitutes daylight, each a mild light that does not burn, three fires of two kinds are involved in seeing the colors of things. Timaeus makes at least three distinguishable claims about the colors:
\begin{enumerate}[(1)]
	\item colors consist in kind of fire, a flame
	\item colors consist in effluences that emanate from bodies
	\item colors are perceptible if the effluences are proportional with the visual body
\end{enumerate}

First, in his discussion of the \emph{genē}, Timaeus claims that there are many kinds of fire but only offers three examples (58c5--d1). First, there is flame, \emph{phlox}. Second, there is the kind that issues from flame but does not burn but supplies light to the eyes. And third, there is the kind that is left behind in embers after the flame is quenched. The fire emitted by the eye and that constitutes the light of day both are of the second kind. Presumably, their compound, the visual body, itself belongs to the second kind. However, colors are said to consist in flame.

Second, colors consist in effluences that emanate from bodies. This is a clear allusion to the account that Socrates attributes to Empedocles in the \emph{Meno} 76a--d, an attribution seconded by Theophrastus in \emph{De sensibus}. Colors may consist in effluences that emanate from bodies, but I do not think that colors are identified with effluences. First, Plato in the \emph{Theaetetus} distinguished alteration from locomotion (181d), going so far as to introduce a neologism for quality, \emph{poiotēs} (182a), that which changes in alteration. Given this, the identification of color, a quality, with effluences, bodies in motion, can seem like a category mistake, and one that Plato is in position to recognize. There are also phenomenological difficulties with the identification of colors with effluences. Typically we take ourselves to see colored particulars. But can we really see colored particulars? Or are they occluded from view by the fiery effluences they give off? Moreover, colors seem confined to the remote bounded region in which they appear, but effluences are not so confined. Besides these difficulties, there is, within the \emph{Timaeus} itself, grounds to resist the identification colors with effluences. The effluences are fire particles, primary bodies, but only secondary bodies compounded of primary bodies are sensible. But colors are sensible if they are colors at all. Moreover, the following pattern may be observed in Timaeus' discussion of the \emph{pathēmata}, that the objects of perception are powers of secondary bodies as determined by the primary bodies that compose them. The chromatic powers would be the powers of secondary bodies to divide or compact the visual body as determined by their emission of fiery effluences. If we adhere to this pattern in interpreting Timaeus' account of color, then it is not susceptible to the three difficulties described above. Colors would not be effluences but powers to emit these and so a kind of \emph{poiotēs} thus avoiding the charge of category mistake. These powers inhere in bounded surfaces of particular bodies, thus respecting the phenomenology of vision. Moreover, they would not be occluded from view, since the effect of the fiery effluences is the measure of these powers whose congizance vision consists in. If that is right, then Timaeus' account is an advance over Empedocles' since Empedocles' account seems susceptible to these difficulties.

Third, colors are perceptible only if the chromatic effluences are proportional with the fire particles that constitute the visual body. Though clearly inspired by Empedocles, Timaeus' position is really the converse of Empedocles'. While Timaeus retains Empedocles' Eleatic conviction that sense objects are proportional to sense, they differ in how the relevant proportion is implemented in the sense object's interaction with sense. According to Empedocles, only effluences that fit the \emph{poroi} are perceptible. If an effluence is too large or too small to fit, then it fails to excite perception. However, according to Timaeus, if the fire particles are larger than the fire particles that constitute the visual body, then they compact that body so as to give rise to a perception of black. And if the particles are smaller than the particles that constitute the visual body, then they divide it to give rise to a perception of white. Like-sized particles, on the other hand, do not affect the visual body and so are insensible and are called ``transparent''. So while Empedocles claims that only like-sized particles excite perception, Timaeus' view is that only particles that are not like-sized excite perception. Rather than reduplicating the Empedoclean scheme, Timaeus' account of vision is its inversion.

% There is another notable difference from Empedocles' account of vision. White and black are distinguished by the elemental composition of these effluences. White effluences are composed of fire and black are composed of water. Timaeus, on the other hand, distinguishes white from black, not by a difference in their elemental composition, but by the size of the fire particles and their different effects on the visual body. Though, as we shall see, the dark waters of the eye do play a role in Timaeus' account of color perception, as well, but not the role that it plays in Empedocles' account.

% That the effluences must be proportional with the visual body for the colors to be perceptible is further evidence for Timaeus' measurement model of perception. The affection of the visual body by proportional effluences is a measure of the power of the secondary bodies that emitted them, and vision consists in a cognizance of what's measured.

Consider now not colors but their effects, the chromatic \emph{pathēmata}. These are not unrelated as colors are powers of the agents that produce these affections, and these affections are the measure of these powers whose cognizance vision consists in. The chromatic \emph{pathēmata} are affections of the visual body. There are two fundamental kinds of affections of the visual body, \emph{diakrisis} and \emph{sugkrisis}. Timaeus explicitly tells us that the affections at work in our experience of white and black respectively are the same affections at work in our experience of hot and cold and astringent and harsh. Though these affections are identical, they give rise to distinct perceptions and sensations because they differ in their causes (and also, presumably, because these affections are received in different parts of the body). These terms, as the occur in the present passage, are usually translated as dilation and contraction, respectively. However, as applied to aggregates of natural bodies, \emph{diakrisis} is more naturally understood not as dilation but as a kind of division or dispersal, and as applied to aggregates of natural bodies, \emph{sugkrisis} is naturally understood as compaction. As we have observed, the size of the fiery effluences determine their effect on the visual body. If the fire particles are smaller than the fire particles that compose the visual body, then they divide the visual body. And if the fire particles are larger than the fire particles that compose the visual body, then they compact the visual body. These affections, division and compaction, are naturally opposed as are the powers to divide and to compact the visual body. Timaeus calls ``white'' (\emph{leukon}) the power to divide the visual body and ``black'' (\emph{melan}) the power to compact the visual body. This is why white and black are the fundamental chromatic opposition.

The transparent (\emph{diaphanē}) is a special case. If the fire particles are like-sized with the particles that compose the visual body, then they neither divide nor compact it but simply have no effect and hence are insensible. This is an instance of the more general principle that like does not affect like (57a3--5). It is worth pausing to think about the lack of effect and its visual consequences. It is implausible to suppose that the visual body extends no further than the like-sized effluences that it confronts. In that case, we could see no further, and these effluences would not properly be understood to be transparent. We would reach a perceptually impenetrable boundary but the transparent is perceptually penetrable. Perhaps, then, the like-sized fire particles are assimilated by the visual body in the way that daylight is and so extends its reach, so to speak. Timaeus might be playing with a potential ambiguity in the Greek term \emph{diaphanē}. \emph{Dia} means through, but the occurrence of \emph{phanē} in \emph{diaphanē} is potentially ambiguous. It might mean light or sight. The word \emph{phanē} means torch, and \emph{phanos} means light or bright, whereas \emph{phaneros} means open to sight, visible, or manifest. So while it is natural for moderns to understand \emph{diaphanē} as that which may be seen through, perhaps it also means that which may be shone through. Interestingly, in the context of Timaeus' account, these readings are not unrelated. It is only if the visual body shines through does one see through. In the context of Timaeus' account, there might also be a further play on this ambiguity, for, recall, the visual body only shines in the direction in which the perceiver looks.

% Timaeus' account of transparency marks a departure from his predecessors. Earlier thinkers both acknowledged the phenomenon of transparency and accounted for it in terms of unobstructed movement from the perceived object. Democritus and Empedocles, in different ways, at the very least have the resources to provide such an account, Democritus by positing a void through which \emph{eidola} may travel, Empedocles by the circular replacement of effluences in a plenum (DK 31B13, 31B100). Timaeus, by contrast, explains transparency in terms of movement from the perceiver, the visual body extending from the perceiver and assimilating like-size particles of fire. Timaeus' account thus emphasizes an aspect of the the phenomenology of the transparent, its perceptual penetrability.

% There is an important limitation of Timaeus' account, however. Timaeus explains only the transparency of emitted fire particles. Strictly speaking, he provides no explicit account of transparent bodies such as glass, water, ice, and the shaved horns of animals from which lamps were made. There are at least two options for extending Timaeus' account. First, just as there are passages in the eye through which interior fire may pass, so perhaps there are passages in transparent bodies through which the visual body may pass. Second, just as the visual body assimilates like-sized fire particles, so perhaps transparent bodies are at least partially composed of like-sized fire particles that the visual body may assimilate and so extend through. Both suggestions are speculative as Timaeus remains silent on this issue.

So depending upon the size of the fire particles, they either divide, or compact, or have no effect on the visual body. Division and compaction are the fundamental affections involved in vision. Moreover, they are naturally opposed. Since white is the power to divide the visual body and black is the power to compact it, these powers are themselves fundamentally and naturally opposed and so were reckoned to be the primary colors. Taylor concurs. This is controversial, however. Thus Brisson and Ierodiakonou claim that there are four primary colors, not only white and black but red and what Timaeus calls ``bright'' (\emph{lampron}) or ``brilliant'' (\emph{stilbon}). (Democritus also posits four primary colors: white, black, red, and green, Theophrastus, \emph{De sensibus} 73--5.)

The philosophical reason for counting white and black in the Timaean color scheme as the primary colors is that they are the fundamental chromatic opposition. There are, however, additional historical reasons for this claim. This interpretation of the Timaean color scheme is in line with an ancient tradition that includes Homer, Parmenides, and Empedocles before Plato, and Aristotle and Goethe after. Moreover, there is a general tendency in Greek color vocabulary to emphasize brightness over hue. As supporting evidence this is pretty weak, admittedly. However, there is a further bit of historical evidence that is puzzling, at the very least. Theophrastus objects that Democritus posits four primary colors but earlier thinkers posit only two, white and black (\emph{De sensibus} 79). What is puzzling is that Theophrastus makes no similar complaint about Plato. Thus, Theophrastus, like Taylor after him, must hold that there are only two primary colors in the \emph{Timaeus}.

I suspect that these seemingly competing interpretations may be reconciled. Indeed, they are driven by distinct if potentially complementary ideas. Theophrastus and Taylor are moved by the idea that white and black are the fundamental chromatic opposition. Brisson and Ierodiakonou are moved by the idea that white, black, red, and brilliant are mixed to generate all the other colors. So perhaps there is no real inconsistency here. Indeed, Timaeus account of color mixture can be read as an attempt to accommodate the Democritean four color scheme within an older tradition that takes white and black as the fundamental chromatic opposition. To get this into view, let us begin by discussing the two other candidate primary colors, brilliant and red. As we shall see, the \emph{pathēmata} that are the exercise of these chromatic powers involve both division and compaction.

Like white bodies, brilliant bodies have the power to emit fire particles that divide the visual body. However, the fire particles emitted by brilliant bodies differ in kind and are more rapid than the fire particles emitted by white bodies. Timaeus does not specify what this difference in kind consists in, but presumably it is a difference in the size of the tetrahedra. And since the tetrahedra emitted by brilliant bodies are more rapid than the tetrahedra emitted by white bodies, presumably they are smaller, since the smallness of polyhedra contribute to the speed of their motion. The fire particles emitted by brilliant bodies also differ in their effects. Unlike the fire particles emitted by white bodies, the fire particles emitted by brilliant bodies divide the visual body all the way up to the eye where they produce secondary effects.

% Before considering the secondary effects of the fire particles emitted by brilliant bodies, let me pause to make an observation since it signals a further difference between Timaeus' account and Empedocles'. For Empedocles, the effluences must enter passages in the eyes in order to excite perception. But the fire particles emitted by white bodies do not divide the visual body all the way up to the eyes. They thus do not enter the passages in the eyes, the way, as we shall see, that the fire particles emitted by brilliant bodies do. In the case of seeing white things, the \emph{pathēmata} are affections of the visual body that is external to the eyes. Though, of course, the visual body as a whole passes on this affection through the passages of the eyes so that the power of the agent that produced it may be reported to the \emph{phronimon}, but this is not the reception of an effluence but the reception of its effect.

The fire particles emitted by brilliant bodies enter the eye. As they do, they pass the fire particles that the eye is itself emitting and penetrate and dissolve the very passages from which these are emitted. This would only be possible if there is no exact fit between the fire particles and the passages in the eye, just as there is no exact fit between the odorous particles and the blood vessels in the nostrils as Vlastos long ago observed. In the eye, the incoming fire particles encounter water. This has two effects, only the second of which is relevant to visual phenomenology:
\begin{enumerate}[(1)]
	\item the incoming fire particles cause a volume of water and fire to pour from the eye
	\item the incoming fire particles are mixed with the water in the eye causing all kinds of colors to appear
\end{enumerate}
The first secondary effect, the pouring forth of a volume of water and fire, is called ``tears''. Though a vital affection, unlike trembling's contribution to our sense of coldness, the production of tears does not contribute to the dazzling visual experience. It is epiphenomenal to the causal process that elicits that experience. In this way it contrasts with the second effect. When the incoming fire is mixed with the eye's moisture, this mixture causes all kinds of colors to appear. Apparently, in a striking anticipation of Newton's discovery (\emph{Opticks} 1), the stream of incoming fire is refracted in the water of the eye (compare Aristotle's claim that a weak light shining through a dense medium will cause all kinds of colors to appear, \emph{Meterologica} 1.5 342b5--8). The resulting visual experience is called ``dazzling'' (\emph{marmarugē}) and the power of the agent that produced it is called ``bright''  (\emph{lampron}) or ``brilliant'' (\emph{stilbon}).

The incoming fire particles dissolve and so destroy, at least in part, the passages in the eyes. Though Timaeus does not make this explicit, this is a departure from the natural state of the eye and is presumably painful (think of staring at the sun). The disruption of the visual body may not be painful since no violent effort is required to do so, but what is dissolved, here, is not the visual body issuing from the eyes but the passages of the eyes from which the visual body issues. These are composed of particles greater than the particles of fire that compose the visual body and so require violent effort to dissolve. Recall, too, that while the fire in the eyes is of a kind that gives light but does not burn, the incoming fire is not of this kind but is, rather, a kind of flame and so does indeed burn. This explains the damage inflicted by seeing brilliant bodies. The dazzling experience of brilliant bodies thus approaches the traumatic experience undergone by the Cyclops as Odysseus blinds him by inserting fire in his eye. Not only are both experiences destructive and painful, but each is a kind of blinding since when dazzled it is difficult to see the details of the scene before one.

% Cornford pronounces brilliance to be puzzling. All kinds of colors are said to appear in a dazzling visual experience and yet in the next breathe Timaeus describes bright or brilliant as a single color. Cornford's puzzle is a chromatic version of the \emph{aporia} about the one and the many. Cornford's puzzle is mitigated somewhat if we turn our attention away from flat opaque colors presented against neutral backgrounds such as how they are presented on Munsell color chips, and consider the metallic green of beetle. Such colors have a metallic sheen that varies with the angle of incidence of the light and yet are a distinctive unitary color. How this is so may remain puzzling, but that this is so is not. (Modern colorimetrists posit extra dimensions of color similarity to explain such phenomena. Fairchild, for example, lists five dimensions of color similarity.)

Both white and brilliant are powers to divide the visual body, but what distinguishes brilliance from white is not this affection but a secondary effect, the mixing of fire and water in the eye. In the case of the dazzling, the affection as a whole includes not only division but this mixture as well. Seeing white involves an affection produced by fire alone, a dazzling experience involves an affection produced by fire and water. While the fire that elicits a dazzling experience may be more intensely bright, understood as the amount of light, its mixture with water involves a reduction of chromatic brightness, understood as a dimension of color similarity, with the effect that all sorts of colors appear and not pure white. Mixing with the eye's water darkens. Perhaps it does so by compacting the incoming stream of fire particles that mixes with it, in which case the water of the eye is black. There is an ancient tradition that assigns black as the color of water. Thus, Anaxagoras, Empedocles, and later Aristotle all held that water is black. If Timaeus subscribes to this tradition, then the \emph{pathēmata} caused by brilliant bodies involves both division and compaction.

Let us now consider Timaeus account of red. Just as brilliant bodies emit a distinctive kind of fire particle, so red things emit a distinctive kind of fire particle. Again, it is plausible to assume that this difference in kind has to do with the size of the tetrahedra. Concerning this, Timaeus says of this kind that it is between ``these'', but there is unclarity about this anaphoric reference. Since Timaeus has just been discussing brilliant, it is plausible that the fire particles emitted by brilliant bodies is one of the kinds of fire to which the fire particles emitted by red bodies is being compared. But what is the other kind of fire? Archer-Hind claims that it is the fire that is emitted by white bodies while Bury claims that it is the fire that is emitted by black bodies. Archer-Hind's reading, that Brisson shares, may be justified in the following manner: Like the fire emitted from brilliant bodies, the fire emitted by red bodies enters they eye and mixes with the water therein to produce a sanguine appearance that we call ``red'' (\emph{eruthron}). It thus must be more like the fire emitted from brilliant bodies than the fire emitted from white bodies. The matter is unclear. I suspect, however, that Archer-Hind's interpretation is correct. If he is, then, given the assumption that kinds of fire are associated with sizes of fire particles, the fire particles emitted by red bodies are smaller than the fire particles emitted by white bodies and larger than the fire particles emitted by brilliant bodies. Nevertheless, they are small enough and rapid enough to divide the visual body all the way up to the eyes and so mix with their waters.

The fire emitted by red bodies, like the fire emitted by brilliant bodies, divides the visual body all the way up to the eye. And like the fire emitted by brilliant bodies, it mixes with the water of the eye to produce a secondary effect. This secondary effect differs from the secondary effect involved in dazzling visual experiences in two ways. First, no tearing is involved. Second, instead of causing all kinds of colors to appear, it causes, instead, a sanguineous appearance that we call ``red''. How are we to understand this? There is an ancient tradition that assigns black as the color of water. If Timaeus subscribes to this tradition, then the water in the eye is itself black, and so compacts the incoming stream of fire. And the fact that the stream of fire that passes through the dark waters in the eye produces an experience of red is an expression of red being intermediary between white and black. Moreover, if red is not merely intermediate but midway, then Aristotle's claim about red would then turn out to be of Timaean provenance.

Like the perception of white, the perception of brilliant and red each involves the division of the visual body. What distinguishes the perception of brilliant and red from the perception of white is not the division of the visual body but the secondary effects, involving the mixture of fire and water, induced by this division. The mixture of fire and water in the eye differ in the case of brilliant and red because of the kind of fire involved, plausibly understood in terms of relative size and a difference in the rapidity of their motion. And in each case the mixture of fire and water involves a reduction of brightness due to water compacting the incoming fire. In the case of brilliant, the mixture results in the appearance of all kinds of colors, many, at least, are not white. And in the case of red, the mixture results in the appearance of an even darker color. And in each case the reduction of brilliance is due to the water being black and so compacting the incoming fire. If the fundamental opposition in the Timaean color scheme derives from the opposition of \emph{diakrisis} and \emph{sugkrisis}, then brilliant and red are intermediary colors between white and black, yielding the series: white, brilliant, red, and black, with white and black being the primary opposition.

Timaeus prefaces his discussion of color mixture with a warning. It would be foolish to state the exact proportions involved in such mixtures, even if one knew these, since no demonstrative nor even probable reason could be given (68b6--8). And he concludes his discussion of color mixture by charging any experimental inquiry into these proportions with impiety. Only God is wise enough and powerful enough to blend the many into one and dissolve the one into many (68c7--d7). So while we can know which colors need to be mixed to generate another color, it is impossible for mortals to determine by test the exact proportions involved. Even if we somehow came to know these proportions, by the testimony of a god or oracular revelation, say, we would still not understand how mixing colors in these proportions generate the colors that they do, for no demonstrative or even probable reason would be available to us.

The full significance of these enigmatic remarks is unclear. However they are to be understood, I think that we should resist any cynical interpretation where Timaeus is preemptively silencing his critics, or providing a license to engage in an Athenian parlour game. Rather than manifesting a lack of seriousness, these remarks are the expression of a profound humility. For in making them, Timaeus concedes that the visible may not be fully understood by the intellect of mortals, thus confessing to the limitation of his cosmological project.

Moreover, and importantly, these remarks turn on an already stated principle, that proportionate bonds generated by the Demiurge can only be dissolved by the Demiurge. This principle is used in the argument that the body of the cosmos is composed of the four primary bodies, and in the assurance the Demiurge gives to the young gods. So if the Demiurge overcomes the opposition of the powers of division and compaction by positing proportionate intermediate powers that unite all in chromatic amity, only He may rend asunder what is thus bound.

Timaeus lists nine colors that are the result of mixing the four colors, white, black, bright or brilliant, and red or the further results of such mixtures (68b5--c7). These are:
\begin{enumerate}[(1)]
	\item golden (\emph{xanthon}) = bright (\emph{lampron}) + red (\emph{eruthron}) + white (\emph{leukon})
	\item purple (\emph{alourgon}) = red (\emph{eruthron}) + black (\emph{melan}) + white (\emph{leukon})
	\item violet (\emph{orphinon}) = black (\emph{melan}) + purple (\emph{alourgon})
	\item tawny (\emph{purron}) = golden (\emph{xanthon}) + grey (\emph{phaion})
	\item gray (\emph{phaion}) = white (\emph{leukon}) + black (\emph{melan})
	\item yellow (\emph{ochron}) = white (\emph{leukon}) + golden (\emph{xanthon})
	\item dark blue (\emph{kuanoun}) = white (\emph{leukon}) + bright (\emph{lampron}) + black (\emph{melas})
	\item light blue (\emph{glaukon}) = dark blue (\emph{kuanoun}) + white (\emph{leukon})
	\item leek green (\emph{prasinon}) = tawny (\emph{purron}) + black (\emph{melan})
\end{enumerate}
The precise sense of Greek color terms is notoriously difficult to capture in translation, and so the translations provided here are only rough equivalents and do not capture precise hue boundaries.

If we resolve the combinations with mixed colors into the colors from which they are themselves mixed we get:
\begin{enumerate}[(1)]
	\item golden (\emph{xanthon}) = bright (\emph{lampron}) + red (\emph{eruthron}) + white (\emph{leukon})
	\item purple (\emph{alourgon}) = red (\emph{eruthron}) + black (\emph{melan}) + white (\emph{leukon})
	\item violet (\emph{orphinon}) = black (\emph{melan}) + (red (\emph{eruthron}) + black (\emph{melan}) + white (\emph{leukon}))
	\item tawny (\emph{purron}) = (bright (\emph{lampron}) + red (\emph{eruthron}) + white (\emph{leukon})) + (white (\emph{leukon}) + black (\emph{melan}))
	\item gray (\emph{phaion}) = white (\emph{leukon}) + black (\emph{melan})
	\item yellow (\emph{ochron}) = white (\emph{leukon}) + (bright (\emph{lampron}) + red (\emph{eruthron}) + white (\emph{leukon}))
	\item dark blue (\emph{kuanoun}) = white (\emph{leukon}) + bright (\emph{lampron}) + black (\emph{melan})
	\item light blue (\emph{glaukon}) = (white (\emph{leukon}) + bright (\emph{lampron}) + black (\emph{melan})) + white (\emph{leukon})
	\item leek green (\emph{prasinon}) = (bright (\emph{lampron}) + red (\emph{eruthron}) + white (\emph{leukon})) + (white (\emph{leukon}) + black (\emph{melan})) + black (\emph{melan})
\end{enumerate}

Ordering all thirteen colors from light to dark, we get:
\begin{enumerate}[(1)]
	\item white (\emph{leukon}) 
	\item bright (\emph{lampron}) or brilliant (\emph{stilbon})
	\item yellow (\emph{ochron})
	\item golden (\emph{xanthon})
	\item tawny (\emph{purron})
	\item leek green (\emph{prasinon})
	\item red (\emph{eruthron})
	\item light blue (\emph{glaukon})
	\item dark blue (\emph{kuanoun})
	\item purple (\emph{alourgon})
	\item violet (\emph{orphinon})
	\item grey (\emph{phaion})
	\item black (\emph{melan})
\end{enumerate}
The list was generated given the following assumptions. First, that the four unmixed colors ordered in terms of chromatic brightness are white, brilliant, red, and black. Second that brilliant is close to white in chromatic brightness. Third, like the Aristotelian color scheme, red is midway between white and black. These assumptions were then applied to Timaeus' color combinations to yield the brightness ordering. As the unknowable proportions can make a difference, the ordering could only be an estimate, but it conforms reasonably well to intuitive judgments of relative brightness. There are difficulties in understanding the colors as intermediaries in the opposition between white and black. However, if we bracket these difficulties, this is a reasonably comprehensible ordering and less mysterious than some commentators have made it out to be.

How are we to understand color mixture as Timaeus understands it? Attempts have been made to understand mixture, here, on the model of the painter's practice of pigment mixture. The painter's practice of pigment mixture may have influenced Plato's thinking, at least indirectly, but, in the first instance, we should try to understand Timaeus' account of color mixture in terms of the powers of colored bodies to emit fiery effluences. There are historical, textual, and philosophical reasons for this. 

The historical reason concerns the observed general tendency for the Greeks in antiquity to understand colors in terms of relative brightness. This occurs in both literary and philosophical sources such as Homer, Parmenides, Empedocles, and, later, Aristotle. Plato knew these sources and was in many ways influenced by them, it would be unsurprising should a commitment to this ancient tradition be manifest in Timaeus' account.

The textual reason concerns a consequence of the Timaean account of color. According to Timaeus, colors are powers of bodies to emit fiery effluences. If colors are powers of bodies to emit fiery effluences, then it would be natural to suppose that color combinations are combinations of these powers. Moreover, in arguing that the body of the cosmos is composed of the four primary bodies, Timaeus held that powers may stand in proportionate ratios. Of course, more needs to be said about what the combination of powers amounts to. But given that colors just are these powers, their combination must be combinations of these powers. A mixture of anything else would not be a color mixture, at least by Timaeus' lights.

The philosophical reason concerns the distinction between color and pigment mixture. The colors are not pigments, and so color mixture is not pigment mixture. It is important to distinguish the mixture of colors that generate different colors from the mixture of bodies that generate differently colored bodies. Pigment mixture is a mixture of bodies that generate differently colored bodies. So we should try to understand color mixture in terms other than pigment mixture, especially in light of the material recalcitrance of the latter. Thus, for example, the color mixing that results from mixing pigments is a subtractive process, and Helmholtz showed that not every color determined by an additive process can be matched with a color determined by a subtractive process. Helmholtz's demonstration was unavailable to the ancients, but it is not anachronistic to suppose that they could mark the distinction between color and pigment mixture. Thus Aristotle criticizes attempts to understand color mixture in terms of pigment mixture in \emph{De sensu} 3. And the Peripatetic author of \emph{De coloribus} writes: 
\begin{quote}
	But we must make our investigation into these things not by mixing colours as painters do, but by comparing the rays which are reflected from those to which we have already referred. For one could especially consider the mixing of rays in nature. (\emph{De coloribus} 2 792b17--21)
\end{quote}

Katerina Ierodiakonou proposes an account of this kind (as does Liz James). Ierodiakonou begins by observing that the terms \emph{diakrisis} and \emph{sugkris} appear in Timaeus' explanation of the cycle of elemental transformation (58b7). In certain circumstances, smaller primary bodies divide larger ones with the result that, for example, we get two tetrahedra of fire from one octahedron of air. Similarly, in certain circumstances, larger primary bodies may compress smaller ones so as to combine them such that we get one octahedron of air from two tetrahedra of fire. Her suggestion, then, is to apply this model to the division and compaction of the visual body.

If the fire particles emitted from a colored body are smaller than the fire particles that compose the visual body, then these are broken down into smaller tetrahedra. And if the fire particles emitted from a colored body are larger than the fire particles that compose the visual body, then these are compressed and so combined into larger tetrahedra. So Ierodiakonou is advancing a specific interpretation of division and compaction here. The visual body is divided by dividing the tetrahedra that compose it and is compacted by combining smaller tetrahedra into larger tetrahedra.

It is now open to Ierodiakonou to understand color mixture in terms of the mixture different-sized tetrahedra in the visual body as determined by the operations of division and compaction:
\begin{quote}
	Let us take, for example, the simple case of the colour grey which is said to be a mixture of white and black. This, I suggest, is to be understood in the following way: a grey body emits fire-particles of two different sizes; namely they are pyramids which, separated according to size, are of the kind emitted by white and by black bodies, respectively. The pyramids of these two different sizes emitted by the grey body interact with and transform the particles of the visual body into smaller and larger particles so that the visual body ends up containing the same proportion of pyramids of these two sizes as the grey body emits. 
\end{quote}

The power of white and black bodies to divide and compact the visual body may produce in it smaller and larger tetrahedra, but is there reason to think that the power of red and brilliant bodies themselves produce tetrahedra of distinctive sizes? After all, their eliciting the visual experiences they do is not due solely to their effect on the visual body but is due as well to their secondary effect on the water of the eye. But this is downstream from the visual body (relative to the motion of the fire particles emitted from the colored body), no matter the sizes of tetrahedra from which the visual body is composed. Moreover, since the secondary effect is downstream, division cannot merely mean the division of constituent tetrahedra, it must mean, as well, the division of the visual body as a whole. Otherwise, how would the small and swift fiery effluences reach the water of the eye? And while it is true that in the cycle of elemental transformation smaller primary bodies may divide larger primary bodies and larger primary bodies may compact and so combine smaller primary bodies, these are all cases of primary bodies of different kinds. Timaeus never explicitly says that that smaller fire particles may divide larger fire particles into their constituent triangles. So while Ierodiakonou's account of Timaean color mixture is both ingenious and of the right kind, I am hesitant to endorse it. 

There are, of course, alternatives. Consider only the most obvious. Associated with white, brilliant, red, and black are distinct affections of the visual body or the eyes from which it issues depending upon the size and rapidity of the fire particles emitted from the colored body. Perhaps what are combined are powers to produce these chromatic \emph{pathēmata}. Recall that there are four of them:
\begin{enumerate}[(1)]
	\item White bodies divide the visual body with small and rapid fire particles
	\item Brilliant bodies divide the visual body with smaller and more rapid fire particles that mix with the waters of the eye that compact them 
	\item Red bodies divide the visual body with larger and less rapid fire particles (though smaller and more rapid than the fire particles emitted by white bodes) that mix with the waters of the eye that compact them
	\item Black bodies compact the visual body with the largest and slowest of the fire particles
\end{enumerate}

The color combinations that Timaeus describes can be understood as combinations of powers to produce these \emph{pathēmata}. So, for example, purple bodies will emit small and large fire particles but where there are two grades of small particles. The large particles, the slowest, will compact the visual body while the small particles will divide it. The smallest and fastest of these, however, divide the visual body all the way up to the eye where they mix with the eye's water and are compacted by it. Thus, purple bodies will act upon the visual body and the eye from which it issues in the ways that white, black, and red bodies do and thus combine these powers in some unknowable proportion. And so on for all the other color combinations that Timaeus describes. Combining the list of the four chromatic \emph{pathēmata}, the exercise of the powers of white, brilliant, red, and black, with the list of Timaeus' color combinations where the combinations with mixed colors are resolved into combinations of colors from which they were mixed, yields a reasonably comprehensible and comprehensive account of color mixture. It is at least as comprehensible as the ordering of the colors from light to dark that it generates, and as we have seen, it does indeed generate a reasonably comprehensible brightness ordering. And with respect to comprehensiveness, all that is missing is what Timaeus claims mortal intellects could never have, knowledge and understanding of the proportions involved in such combinations. Moreover, this account, far from being inconsistent with Ierodiakonou's, may be combined with it, at least to a certain extent. On the hybrid account, an effect of the combined actions on the visual body is that it is composed of differently sized tetrahedra, whether or not brilliant and red are best understood as corresponding to distinctive sizes of tetrahedra.

On the present interpretation, Timaeus can be understood as accommodating Democritus' claim that the mixture of four colors suffices for the generation of the colors within an older chromatic tradition that sees white and black as the fundamental chromatic opposition. Timaeus departs from the four color scheme in substituting brilliant for green (\emph{chlōron}). But, importantly, the substitution preserves the relative order of brightness. Brilliant, like green, is brighter than red and darker than white. And whereas Democritus only posits seven generated colors, Timaeus posits nine. In this way, it is an elaboration. And while some of the Democritean color combinations are preserved in the Timaean color scheme, there are adjustments given the substitution of brilliant for green, but in a way that preserves the overall structure. Perhaps this accommodation of the Democritean color scheme within the older tradition is an instance of Timaeus' tacit rivalry with Democritean natural philosophy. Democritus had written an entire book on color, and his claim that the mixture of four colors suffice to generate the colors was shared with medical and artistic traditions. Thus the medical tradition associated four colors with blood, phlegm, and black and yellow bile. And there is literary and archeological evidence that fifth century BCE painters restricted their palette to four colors. Moreover, accommodating Democritus' four color scheme in a way that preserves the fundamental opposition between white and black would explain why Timaeus is not subject to Theophrastus' criticism of Democritus (\emph{De sensibus} 79). All that was needed was to postulate, in addition to the the power of black bodies, two further powers to reduce chromatic brightness that, while distinct, themselves depend upon blackness. Brightness is reduced, not the way black bodies do, by emitting large particles that compact the visual body, but by emitting small particles that are themselves compacted in their mixing with the eyes' waters. These powers differ in the proportion of fire and water in the mixture that they elicit in the eyes. These powers may differ in this way, but they are alike in that neither are wholly independent of the power of black bodies. The water's compaction of the incoming fire particles just is the exercise of its blackness. What are distinguished are degrees of departure from whiteness, a reduction in chromatic brightness, where black alone remains in opposition to white and is presupposed by the two other departures from whiteness. That brilliant and red cannot be combined by themselves to generate a color but must be combined with either white or black to do so is further evidence of their derivative nature.

 % The smaller and more rapid fire particles emitted by brilliant bodies would result in a greater mixture of fire with the eyes' water than would result from the larger and less rapid fire particles emitted by red bodies.

The way that Timaeus' account diverges from Empedocles' would also be explained, at least in part. Theophrastus objects to Empedocles' account that while he has explained the perception of white in terms of the reception of fiery effluences and the perception of black in terms of the reception of watery effluences, he has not explained the perception of any other color (\emph{De Sensibus} 17). However, Empedocles may be understood as claiming that the perception of the rest of the colors involve different proportions of fire and water received through the eyes' passages. Timaeus accepts the basic idea that white and black are the fundamental chromatic opposition. Timaeus also accepts the Eleatic idea that the intermediary colors should be understood in terms of mixtures of various proportions (even if knowledge of these proportions would require a god's knowledge that Empedocles' impiously claims for himself). However, accommodating Democritus' four color scheme within the older tradition that Timaeus shares with Empedocles required postulating two further ways of reducing brightness beyond the way that black bodies do, and this, in turn, required corresponding adjustments to the Empedoclean scheme. For example, that only some chromatic effluences enter the eye, those emitted by brilliant and red bodies, is a departure from Empedocles that results from such adjustment, as is the revised role of the eye's water. Thus Taylor was wrong to see Timaeus account of color and its perception as simply an application of Empedocles', just as Archer-Hind was wrong to deny any Empedoclean influence.

% Finally, the persistent allure of interpreting Timaeus' color combinations in terms of the painter's practice of pigment mixture would be explicable, if mistaken. For that which Timaeus seeks to accommodate, Democritus' four color scheme, was shared to an extent with the fifth century four color painters.

% This is speculative, of course. But some account along these lines must be right given the historical, textual, and philosophical reasons to resist understanding Timaeus' account on the model of the painter's practice of pigment mixture. And this would remain so even should it turn out, in the end, that Timaeus' account is overly influenced by this practice. Compare, for example, Aristotle's criticism of understanding color mixture in terms of pigment mixture in \emph{De sensu} 3 (see \citealt[chapter 6]{Kalderon:2015fr}, for discussion).

Let me end by considering one of Theophrastus' objections to Timaeus' account of color. Specifically, Theophrastus doubts whether color could invariably be understood in terms of flame:
\begin{quote}
	It is absurd \ldots\ to say without exception that colour is a flame. For while in some respects the colour white resembles flame, black would seem to be flame's opposite. (Theophrastus, \emph{De sensibus} 91)
\end{quote}
How could the perception of black result from flame? While the worry can seem intuitively compelling, a reply may be made on behalf of Timaeus. Consider a black image on a television or computer screen. Should one turn the screen off, it will be gray and not black. And yet when on, it was emitting light which resulted in a perception of black and not gray. Understanding how this may be so may be puzzling in the way that Theophrastus suggests, but it is demonstratively so nonetheless. What is needed to resolve this puzzlement is the distinction between brightness as amount of light and brightness as a dimension of color similarity. And while the means to empirically establish this requires some method, unavailable in antiquity, of measuring the amount of light, the distinction can be found, in embryonic form, in Timaeus' account. Brilliant bodies, in their rapid emission of small tetrahedra, are very bright, but they elicit in us the appearance of all kinds of colors not all of which are white.


\end{document}
